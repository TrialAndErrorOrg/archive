\documentclass[authordate, empirical]{jote-new-article}

\usepackage{caption}

\usepackage{tabularx}

\usepackage{graphicx}

\usepackage{hyperref}

\usepackage[backend=biber,style=apa]{biblatex}

\addbibresource{bibliography.bib}

\jotetitle{Serendipity in \mbox{Scientific Research}}
\keywordsabstract{Serendipity, History of Science, Stories of Science, Discovery}
\abstracttext{Serendipity refers to the combination of “accident” and “sagacity”; an unexpected and unpredicted event which is noticed by an agent with the right skills to make the most of it. Famous examples include Jocelyn Bell’s discovery of pulsars which was made after she noticed an unusual output from a radio telescope \parencites{Arfini2023}. Bell noticed and unpredicted output on the graphical trace and followed it up, eventually discovering the existence of pulsars. The rate of serendipitous discovery in science is unclear, although it has been estimated to be high \parencites{Thagard2012}. This series is meant not only to add to the repertoire of serendipity stories, but to begin treating these tales as members in a growing archive, in which we attend to the role of chance and the unexpected in our rational pursuits of knowledge. Scientists here will share how accidents and reason intertwined in their practice, and researchers of serendipity will unpack how that happens.}
\runningauthor{Ross et al.}
\jname{Journal of Trial \& Error}
\jyear{2024}
\paperdoi{10.36850/v91j-7541}
\paperreceived{October 23, 2023}
\author[1]{\mbox{Wendy Ross}}
\affil[1]{London Metropolitan \mbox{University}}
\corremail{\href{mailto:w.ross@londonmet.ac.uk}{w.ross@londonmet.ac.uk}}
\corraddress{London Metropolitan \mbox{University}}
\runningauthor{Ross, Copeland, \& Firestein}
\author[2]{\mbox{Samantha Copeland\orcid{0000-0002-6946-7165}}}
\affil[2]{Delft University \mbox{of Technology}}
\author[3]{\mbox{Stuart Firestein\orcid{0000-0003-1774-5853}}}
\affil[3]{Columbia University}
\paperaccepted{February 13, 2024}
\paperpublished{March 24, 2024}
\paperpublisheddate{2024-03-24}
\jwebsite{https://journal.trialanderror.org}

\begin{filecontents}{bibliography.bib}
	@inbook{Arfini2023,
    title       = {Serendipity and ignorance studies},
    author      = {Arfini, A. D. D. I. N. ZOTERO\_BIBL {"uncited",],"omitted",],"custom",]} CSL\_BIBLIOGRAPHY and {S}},
    editor      = {Copeland, S. and Ross, W. and Sand, M.},
    url         = {https://doi.org/10.1007/978-3-031-33529-7\_7},
    doi         = {10.1007/978-3-031-33529-7\_7},
    publisher   = {Springer International Publishing},
    date        = {2023},
    pages       = {125--143},
    booktitle     = {Serendipity Science}
}


@article{Arfini2018,
    title       = {The antinomies of serendipity: {H}ow to cognitively frame serendipity for scientific discoveries},
    author      = {Arfini, S. and Bertolotti, T. and Magnani, L.},
    volume      = {39},
    url         = {https://doi.org/10.1007/s11245-018-9571-3},
    doi         = {10.1007/s11245-018-9571-3},
    date        = {2018},
    pages       = {939--948},
    journal     = {Topoi}
}


@article{Barber1958,
    title       = {The case of the floppy-eared rabbits: {A}n instance of serendipity gained and serendipity lost},
    author      = {Barber, B. and Fox, R. C.},
    number      = {2},
    volume      = {64},
    url         = {https://doi.org/10.1086/222420},
    doi         = {10.1086/222420},
    date        = {1958},
    pages       = {128--136},
    journal     = {American Journal of Sociology}
}


@book{Boden2004,
    title       = {The creative mind: {M}yths and mechanisms},
    author      = {Boden, M. A.},
    publisher   = {Routledge},
    date        = {2004}
}


@inproceedings{Bogers2013,
    title       = {Micro-serendipity: {M}eaningful coincidences in everyday life shared on {T}witter},
    author      = {Bogers, T. and Björneborn, L.},
    url         = {https://doi.org/10.9776/13175},
    doi         = {10.9776/13175},
    date        = {2013},
    pages       = {196--208},
    booktitle     = {iConference 2013 Proceedings}
}


@article{Campanario1996,
    title       = {UsingCitation Classics to study the incidence of serendipity in scientific discovery},
    author      = {Campanario, J. M.},
    number      = {1},
    volume      = {37},
    url         = {https://doi.org/10.1007/BF02093482},
    doi         = {10.1007/BF02093482},
    date        = {1996},
    pages       = {3--24},
    journal     = {Scientometrics}
}


@article{Copeland2023,
    title       = {Enacting chance and the space of possibilities},
    author      = {Copeland, S. and Arfini, S. and Ross, W.},
    number      = {3},
    volume      = {1},
    url         = {https://doi.org/10.1177/27538699231191006},
    doi         = {10.1177/27538699231191006},
    date        = {2023},
    pages       = {263--268},
    journal     = {Possibility Studies \& Society}
}


@article{Copeland2018,
    title       = {"Fleming leapt on the unusual like a weasel on a vole”: {C}hallenging the paradigms of discovery in science},
    author      = {Copeland, S. M.},
    number      = {6},
    volume      = {26},
    url         = {https://doi.org/10.1162/posc\_a\_00294},
    doi         = {10.1162/posc\_a\_00294},
    date        = {2018},
    pages       = {694--721},
    journal     = {Perspectives on Science}
}


@article{Darbellay2014,
    title       = {Interdisciplinary research boosted by serendipity},
    author      = {Darbellay, F. and Moody, Z. and Sedooka, A. and Steffen, G.},
    number      = {1},
    volume      = {26},
    url         = {https://doi.org/10.1080/10400419.2014.873653},
    doi         = {10.1080/10400419.2014.873653},
    date        = {2014},
    pages       = {1--10},
    journal     = {Creativity Research Journal}
}


@article{Erdelez2004,
    title       = {Investigation of information encountering in the controlled research environment},
    author      = {Erdelez, S.},
    number      = {6},
    volume      = {40},
    url         = {https://doi.org/10.1016/j.ipm.2004.02.002},
    doi         = {10.1016/j.ipm.2004.02.002},
    date        = {2004},
    pages       = {1013--1025},
    journal     = {Information Processing \& Management}
}


@article{Fine2006,
    title       = {Three principles of {S}erendip: {I}nsight, chance, and discovery in qualitative research},
    author      = {Fine, G. A. and Deegan, J.},
    number      = {4},
    volume      = {9},
    url         = {https://doi.org/10.1080/0951839960090405},
    doi         = {10.1080/0951839960090405},
    date        = {2006},
    pages       = {434--447},
    journal     = {International Journal of Qualitative Studies in Education}
}


@article{Hargrave-Thomas2012,
    title       = {Serendipity in anticancer drug discovery},
    author      = {Hargrave-Thomas, E.},
    number      = {1},
    volume      = {3},
    url         = {https://doi.org/10.5306/wjco.v3.i1.1},
    doi         = {10.5306/wjco.v3.i1.1},
    date        = {2012},
    pages       = {1--6},
    journal     = {World Journal of Clinical Oncology}
}


@article{Henrich2023,
    title       = {What mkes us smart?},
    author      = {Henrich, J. and Muthukrishna, M.},
    volume      = {tops.12656},
    url         = {https://doi.org/10.1111/tops.12656},
    doi         = {10.1111/tops.12656},
    date        = {2023},
    journal     = {Topics in Cognitive Science}
}


@article{Makri2012,
    title       = {Coming across information serendipitously -- Part 1: {A} process model},
    author      = {Makri, S. and Blandford, A.},
    volume      = {68},
    url         = {https://doi.org/10.1108/00220411211256030},
    doi         = {10.1108/00220411211256030},
    date        = {2012},
    pages       = {684--705},
    journal     = {Journal of Documentation}
}


@article{McCay-Peet2015,
    title       = {Examination of relationships among serendipity, the environment, and individual differences},
    author      = {McCay-Peet, L. and Toms, E. G. and Kelloway, E. K.},
    number      = {4},
    volume      = {51},
    url         = {https://doi.org/10.1016/j.ipm.2015.02.004},
    doi         = {10.1016/j.ipm.2015.02.004},
    date        = {2015},
    pages       = {391--412},
    journal     = {Information Processing \& Management}
}


@article{McCulloch2021,
    title       = {Serendipity in doctoral education: {T}he importance of chance and the prepared mind in the PhD},
    author      = {McCulloch, A.},
    number      = {2},
    volume      = {46},
    url         = {https://doi.org/10.1080/0309877X.2021.1905157},
    doi         = {10.1080/0309877X.2021.1905157},
    date        = {2021},
    pages       = {258--271},
    journal     = {Journal of Further and Higher Education}
}


@article{Merton1948,
    title       = {The bearing of empirical research upon the development of social theory},
    author      = {Merton, R.},
    number      = {5},
    volume      = {13},
    date        = {1948},
    pages       = {505--515},
    journal     = {American Sociological Review}
}


@book{Merton2004,
    title       = {The travels and adventures of rerendipity: {A} study in sociological semantics and the sociology of science},
    author      = {Merton, R. and Barber, E.},
    publisher   = {Princeton University press},
    date        = {2004}
}


@article{Pasteur1854,
    title       = {Oeuvres de Pasteur / réunies par Pasteur Vallery-Radot},
    author      = {Pasteur, L.},
    url         = {https://upload.wikimedia.org/wikipedia/commons/6/62/Louis\_Pasteur\_Universit\%C3\%A9\_de\_Lille\_18541857\_dans\_les\_champs\_de\_l\%27observation\_le\_hasard\_ne\_favorise\_que\_les\_esprits\_pr\%C3\%A9par\%C3\%A9s.pdf},
    date        = {1854}
}


@book{Ross2022,
    title       = {The art of serendipity},
    author      = {Ross, W. and Copeland, S.},
    publisher   = {Palgrave McMillan},
    date        = {2022}
}


@article{Ross2022,
    title       = {Accident and agency: {A} mixed methods study contrasting luck and interactivity in problem solving},
    author      = {Ross, W. and Vallée-Tourangeau, F.},
    number      = {4},
    volume      = {28},
    url         = {https://doi.org/10.1080/13546783.2021.1965025},
    doi         = {10.1080/13546783.2021.1965025},
    date        = {2022},
    pages       = {487--528},
    journal     = {Thinking \& Reasoning}
}


@article{Rubin2011,
    title       = {Facets of serendipity in everyday chance encounters: {A} grounded theory approach to blog analysis},
    author      = {Rubin, V. L. and Burkell, J. and Quan-Haase, A.},
    number      = {3},
    volume      = {16},
    url         = {https://doi.org/},
    date        = {2011},
    journal     = {Information Research}
}


@inbook{Thagard2012,
    title       = {Creative combination of representations: {S}cientific discovery and technological invention},
    author      = {Thagard, P.},
    editor      = {Proctor, R. and Capaldi, E.J.},
    publisher   = {Oxford University Press},
    date        = {2012},
    pages       = {389--405},
    booktitle     = {Psychology of science: Implicit and explicit processes}
}


@article{VanAndel1994,
    title       = {Anatomy of the unsought finding},
    author      = {Van Andel, P.},
    volume      = {45},
    date        = {1994},
    pages       = {631--648},
    journal     = {The British Journal for the Philosophy of Science}
}


@online{Walpole1754,
    title       = {Letter from Horace Walpole to Horace Mann},
    author      = {Walpole, H.},
    url         = {https://walpole.library.yale.edu/online-content/digital-resources/horace-walpole-correspondence},
    date        = {1754}
}


@article{Willems2022,
    title       = {Tales of the unexpected: {D}esigning for serendipity in research},
    author      = {Willems, L. and Wade, E. and Herbert, R. and Plume, A.},
    url         = {https://doi.org/10.2139/ssrn.4048549},
    doi         = {10.2139/ssrn.4048549},
    date        = {2022},
    journal     = {SSRN Electronic Journal}
}


@article{Yaqub2018,
    title       = {Serendipity: {T}owards a taxonomy and a theory},
    author      = {Yaqub, O.},
    volume      = {47},
    url         = {https://doi.org/10.1016/j.respol.2017.10.007},
    doi         = {10.1016/j.respol.2017.10.007},
    date        = {2018},
    pages       = {169--179},
    journal     = {Research Policy}
}
\end{filecontents}

\begin{document}
\begin{frontmatter}
  \maketitle
  \begin{abstract}
    \printabstracttext
  \end{abstract}
\end{frontmatter}


	\lettrine{I}{n} the systematic progression of scientific inquiry, serendipity -- happy discovery attributable neither to luck nor skills but rather a combination of the two -- serves as a critical mechanism, fostering incremental advancements through the recombination of existing ideas and observations, often catalyzing novel developments in unanticipated yet fruitful directions. In a recent paper, \textcites[][p. 2]{Henrich2023} write that “new inventions, fresh insights, and novel ideas involve incremental steps that recombine—often through a healthy dose of serendipity—existing ideas, technologies, observations, and concepts”. This is not a new observation. Robert Merton's interest in collecting stories of serendipity began as early as the 1940s, ultimately and arguably founding the current field of research in serendipity with the publication of \emph{The Travels and Adventures of Serendipity }with Elinor Barber \parencites{Merton2004}{Yaqub2018} and influencing foundational research such as Pek Van Andel's (\hspace*{-2pt}\cite*{VanAndel1994}) seminal categorization of collected examples of the phenomenon \parencites{foster2003serendipity}.



	As alluded to above, serendipity refers to the combination of “accident” and “sagacity”: an unexpected and unpredicted event which is noticed by an agent with the right skills to make the most of it. One famous example includes Jocelyn Bell's discovery of pulsars which was made after she noticed an unusual output from a radio telescope \parencites{Arfini2023}. Bell noticed an unpredicted output on the graphical trace and followed it up, eventually discovering the existence of pulsars. Three aspects were needed for the discovery -- the accidental observation, the marking of that observation as something of note and finally someone with the expertise and the networks to make the most of this.



	The rate of serendipitous discovery in science is unclear, although it has been estimated to be high \parencites{Thagard2012}. In the mid-1990s, \textcites{Campanario1996} reported that in 8.3\% of papers from the Citation Classics collection of scientist reports on their key discoveries, they put the discovery down to serendipity. A 2016 survey of over 3000 researchers put the rate at 17\% \parencites{Willems2022}. Other reports have put it as high as 33\% \parencites{Hargrave-Thomas2012}. Such discoveries may be regular occurrences, but they are socially and personally impactful. They play an important role in our understanding of scientific discovery. Nobel Prize speeches such as Alexander Fleming's tout the importance of the accident to discoveries as influential as penicillin. We shout “Eureka” to denote moments like Archimedes' apocryphal insight in his bath about water displacement and the weight of gold in a crown. On the other hand, there is the famous aphorism (commonly attributed to Asimov) that the most exciting phrase in science is not “Eureka!!”, but rather “hmm, that's strange”, suggesting that it is curiosity about an observed anomaly that is the herald of a new discovery.



	Stories about the accidental origins of the microwave, Viagra, X-rays, Velcro, and much of modern medicine abound in collections and are shared time and again in fun editorials. They are also found in the words of scientists themselves, who may offer rational reconstructions in their scientific papers but note the importance of an unexpected observation, an unwitting change in the experimental set-up, and other “accidents” that could have derailed rather than generated a discovery. While most examples herald from the physical sciences or invention, serendipity occurs almost anywhere. Robert Merton began his theoretical exploration of the phenomenon when he found his “serendipity pattern” in an unexpected contribution to sociological theory \parencites{Merton1948}. Walpole's own invention of the term resulted from his finding unexpected confirmation of a historical rumour in a book of heralds \parencites{Walpole1754}. Anthropologists experience it when they happen to be present at a significant cultural event, as a result of relationships formed along the way of investigation \parencites[e.g.,][]{Fine2006}. And many an artist has found inspiration serendipitously while working with materials that respond in unexpected ways, for instance \parencites[see][for examples]{Ross2022}.



	This series is meant not only to add to the repertoire of serendipity stories, but to begin treating these tales as members of a growing archive, in which we attend to the role of chance and the unexpected within our rational pursuits of knowledge. Scientists here will share how accidents and reason intertwined in their practice, and researchers of serendipity will unpack how that happens. Until now, serendipity research has relied very much on narratives shared by scientists and others, from those passed down through history \parencites{VanAndel1994} to written in tweets online \parencites{Bogers2013}{Rubin2011}, as evidence and counterargument for definitions, diagrams and frameworks of serendipity as it happens. Recently, this research has moved away from the narrative and toward empirical research \parencites{Erdelez2004}{Ross2022} as a resource for greater understanding, and with this move the theory has moved from categorization of the phenomena we call serendipity to analysis of its components and the conditions in which it occurs (or does not, for that matter). The narratives in this series, then, will be taken as a testing ground for emergent serendipity theory, illustrating what it can add to our understanding of how discoveries are made as well as offering an opportunity to prove or to question generalizations now being made from historical, familiar cases.



	Notably, serendipity can be an unstable concept, which exists as much in the reporting as the doing, especially if recognizing the event as serendipitous is part of the definition \parencites{Makri2012}{McCay-Peet2015}{Rubin2011}. That is, serendipity stories do suffer from a “publication bias” of their own: we only tell them when they end well, or we only call them serendipity when the accident leads to a discovery. This series offers an opportunity to explore the nuance of success as well as failure -- the reader will see that failures and misdirection can play an important role, sometimes only indirectly, but especially when they at first seem to hinder progress.



	On the other hand, there is a real risk of conceptual bloat if we call every accident a potential origin of serendipitous discovery \parencites{Copeland2023}. Not all the cases examined in this series will be serendipity, and some might open up new debates in serendipity research as well as in our understandings of scientific practice. Any given discovery can be traced back to a moment of surprise or awe, so a line must be drawn, but where we draw the line between chance and purpose is itself an interesting and important question that will be raised by many a serendipity story shared here. We do not anticipate full agreement between the experts but in the space of disagreement we hope to generate understanding.



	Another threshold that can be fuzzy and yet heavily influences our understanding of discovery is that between failure and accident. Theoretically, scientific research expects failure: experiments are a test of a hypothesis, we do them when something is unknown or when we are uncertain. Systems are in place to understand and evaluate failures and while they may turn the narrative of discovery, they do so in an expected or anticipated way. On the other hand, serendipitous discoveries seem to be those which were improbable and for which there is no clear system for exploitation or understanding. This lack of systemic support leads to the heroic genius narrative that often accompanies these tales \parencites{Copeland2018}. One key difference between failure and serendipity may be that failure marks the end of the process of discovery while serendipity marks the start.



	However, while this notion is attractive, it becomes more unstable under scrutiny. The improbable event still must be within the realms of the expected to elicit a reaction of curiosity and enough within the skills of the observer for them to make sense of it. As \textcites[][p. 5]{Arfini2018} write about Fleming's discovery of penicillin:



	\begin{quotation}
        Fleming's “Oh!” reaction was when he managed to frame and understand the antibiotic effect of a mold. He did not enter his laboratory to find a moldy culture singing the chorus of Mamma mia!: that would have sparked another kind of reaction.
    \end{quotation}


	As \textcites{Boden2004} writes, cognitively, serendipity comes from the recognition of a pattern which has come before and the extraction of subtle regularities that can only come from the immersion in a domain. This would fit well the story of Bell above, as she noticed the unexplained trace several times before investigating it. So, there is another “goldilocks zone” to uncover of immersion in and distance from a field. This is reflected in a now-famous comment by Louis Pasteur: “le chance ne favorise que l'ésprits preparés” \parencites{Pasteur1854}; “chance favours the prepared mind”.



	However, the very nature of what it means to be prepared remains under-investigated. While so far stories shared about serendipity have tended to emphasise depictions of the lone genius exploiting an accident without any supporting network, closer looks at how serendipity emerges for even those scientists shows its origins in social networks and success through knowledge exchange. For these reasons, it is at least if not more important to attend to evidence that serendipity arises particularly in groups and within spaces which foster this \parencites{Darbellay2014}{McCulloch2021}. Indeed, an early description of serendipity \parencites{Barber1958} showed it was not the observation of the surprising fact but the surrounding support systems in terms of funding and time which generated the serendipitous discovery -- an important consideration for scientific funding. We predict that the stories collected in this series will tell stories of groups and networks rather than individuals.



	Thus, there is much to debate about serendipity itself; conflicting ideas and understandings of the phenomenon are so far derived from historical examples and more recently empirical work but are only in the early stages of theoretical development. Looking closely at the stories in this series offers an opportunity not only to share and treat as scientifically valuable those parts of science normally left to anecdotes in social media, academic parties, or Nobel Prize speeches. It is also an opportunity to see what emerging theory about serendipity itself can do, when it comes to understanding these stories as practices, producing knowledge in themselves.





\printbibliography



\end{document}