\documentclass[authordate, meta, issue]{jote-new-article}

\jissue{1}
\jvolume{4}
\jpages{5--20}
\paperissued{April 24, 2023}

\articletype{Special Issue - Meta Research}
\specialissue{Reflections on the Unintended Consequences of the Science Reform Movement}

\setcounter{page}{5}


\addbibresource{bibliography.bib}

\begin{filecontents}{bibliography.bib}
	@article{Allum2023,
    title       = {Researchers on research integrity: a survey of {European and American} researchers},
    author      = {Allum, N. and Reid, A. and Bidoglia, M. and Gaskell, G. and Aubert-Bonn, N. and Buljan, I. and Veltri, G.},
    number      = {187},
    volume      = {12},
    url         = {https://doi.org/10.12688/f1000research.128733.1},
    doi         = {10.12688/f1000research.128733.1},
    date        = {2023},
    pages       = {187},
    journal     = {F1000Research}
}


@article{Altenmüller2021,
    title       = {No harm in being self-corrective: Self-criticism and reform intentions increase researchers’ epistemic trustworthiness and credibility in the eyes of the public},
    author      = {Altenmüller, M. S. and Nuding, S. and Gollwitzer, M.},
    number      = {8},
    volume      = {30},
    url         = {https://doi.org/10.1177/09636625211022181},
    doi         = {10.1177/09636625211022181},
    date        = {2021},
    pages       = {962–976},
    journal     = {Public Understanding of Science}
}


@article{Andreoletti2020,
    title       = {Replicability crisis and scientific reforms: Overlooked issues and unmet challenges},
    author      = {Andreoletti, M.},
    number      = {3},
    volume      = {33},
    url         = {https://doi.org/10.1080/02698595.2021.1943292},
    doi         = {10.1080/02698595.2021.1943292},
    date        = {2020},
    pages       = {135–151},
    journal     = {International Studies in the Philosophy of Science}
}


@book{Anonymous2021,
    title       = {It’s 2021... and we are still dealing with misogyny in the name of open science},
    author      = {Anonymous},
    url         = {https://blogs.sussex.ac.uk/psychology/2021/11/25/its-2021-and-we-are-still-dealing-with-misogyny-in-the-name-of-open-science/},
    publisher   = {University of Sussex School of Psychology Blog},
    date        = {2021-11-25}
}



@article{Asendorpf2013,
        doi = {10.1002/per.1919},
        url = {https://doi.org/10.1002/per.1919},
        year = 2013,
        month = {mar},
        publisher = {{SAGE} Publications},
        volume = {27},
        number = {2},
        pages = {108--119},
        author = {Jens B. Asendorpf and Mark Conner and Filip De Fruyt and Jan De Houwer and Jaap J. A. Denissen and Klaus Fiedler and Susann Fiedler and David C. Funder and Reinhold Kliegl and Brian A. Nosek and Marco Perugini and Brent W. Roberts and Manfred Schmitt and Marcel A. G. Van Aken and Hannelore Weber and Jelte M. Wicherts},
        title = {Recommendations for Increasing Replicability in Psychology},
        journal = {European Journal of Personality}
}    

@article{Bak-Coleman2022,
    title       = {Replication does not measure scientific productivity},
    author      = {Bak-Coleman, J. B. and Mann, R. P. and West, J. and Bergstrom, C. T.},
    url         = {https://doi.org/10.31235/osf.io/rkyf7},
    doi         = {10.31235/osf.io/rkyf7},
    date        = {2022-04-28}
}


@article{Barrett2015,
    title       = {Psychology is not in crisis},
    author      = {Barrett, L. F.},
    url         = {https://www3.nd.edu/~ghaeffel/ScienceWorks.pdf},
    date        = {2015-09-01},
    journal     = {The New York Times},
    entrysubtype= {nonacademic}
}


@inbook{Bastian2021,
    title       = {The metascience movement needs to be more self-critical},
    author      = {Bastian, H.},
    url         = {https://absolutelymaybe.plos.org/2021/10/31/the-metascience-movement-needs-to-be-more-self-critical/},
    date        = {2021-10-31},
    booktitle     = {PLOS Blogs: Absolutely Maybe}
}


@article{Bennett2021,
    title       = {Open science from a qualitative, feminist perspective: Epistemological dogmas and a call for critical examination},
    author      = {Bennett, E. A.},
    number      = {4},
    volume      = {45},
    url         = {https://doi.org/10.1177/03616843211036460},
    doi         = {10.1177/03616843211036460},
    date        = {2021},
    pages       = {448–456},
    journal     = {Psychology of Women Quarterly}
}


@article{Bird2020,
    title       = {Understanding the replication crisis as a base rate fallacy},
    author      = {Bird, A.},
    number      = {4},
    volume      = {72},
    url         = {https://doi.org/10.1093/bjps/axy051},
    doi         = {10.1093/bjps/axy051},
    date        = {2020},
    pages       = {965–993},
    journal     = {The British Journal for the Philosophy of Science}
}



@article{Bishop2019,
        doi = {10.1038/d41586-019-01307-2},
        url = {https://doi.org/10.1038/d41586-019-01307-2},
        year = 2019,
        month = {apr},
        publisher = {Springer Science and Business Media {LLC}},
        volume = {568},
        number = {7753},
        pages = {435--435},
        author = {Dorothy V. M. Bishop},
        title = {Rein in the four horsemen of irreproducibility},
        journal = {Nature}
}


@article{Bishop2020,
    title       = {The psychology of experimental psychologists: Overcoming cognitive constraints to improve research:{ The 47th Sir Frederic Bartlett Lecture}},
    author      = {Bishop, Dorothy V. M.},
    number      = {1},
    volume      = {73},
    url         = {https://doi.org/10.1177/1747021819886519},
    doi         = {10.1177/1747021819886519},
    date        = {2020},
    pages       = {1–19},
    journal     = {Quarterly Journal of Experimental Psychology}
}


@article{Boring1919,
    title       = {Mathematical vs. scientific significance},
    author      = {Boring, E. G.},
    number      = {10},
    volume      = {16},
    url         = {https://doi.org/10.1037/h0074554},
    doi         = {10.1037/h0074554},
    date        = {1919},
    pages       = {335–338},
    journal     = {Psychological Bulletin}
}


@article{Brower1949,
    title       = {The problem of quantification in psychological science},
    author      = {Brower, D.},
    number      = {6},
    volume      = {56},
    url         = {https://doi.org/10.1037/h0061802},
    doi         = {10.1037/h0061802},
    date        = {1949},
    pages       = {325–333},
    journal     = {Psychological Review}
}


@article{Buzbas2023,
        doi = {10.1098/rsos.221042},
        url = {https://doi.org/10.1098/rsos.221042},
        year = 2023,
        month = {mar},
        publisher = {The Royal Society},
        volume = {10},
        number = {3},
        author = {Erkan O. Buzbas and Berna Devezer and Bert Baumgaertner},
        title = {The logical structure of experiments lays the foundation for a theory of reproducibility},
        journal = {Royal Society Open Science}
}


@article{Chamberlain2000,
    title       = {Methodolatry and qualitative health research},
    author      = {Chamberlain, K.},
    number      = {3},
    volume      = {5},
    url         = {https://doi.org/10.1177/135910530000500306},
    doi         = {10.1177/135910530000500306},
    date        = {2000},
    pages       = {285–296},
    journal     = {Journal of Health Psychology}
}


@article{Chambers2014,
    title       = {Physics envy: Do ‘hard’ sciences hold the solution to the replication crisis in psychology? },
    journal     = {The Guardian},
    author      = {Chambers, C. D.},
    url         = {http://www.theguardian.com/science/head-quarters/2014/jun/10/physics-envy-do-hard-sciences-hold-the-solution-to-the-replication-crisis-in-psychology},
    date        = {2014-06-10},
    entrysubtype = {nonacademic}
}


@book{Chambers2017,
    title       = {The seven deadly sins of psychology: A manifesto for reforming the culture of scientific practice},
    author      = {Chambers, C. D.},
    publisher   = {Princeton University Press},
    date        = {2017}
}


@inbook{Chambers2018,
    title       = {{Registered Reports} as a vaccine against research bias: Past, present and future},
    author      = {Chambers, C. D.},
    url         = {https://doi.org/10.23668/psycharchives.797},
    doi         = {10.23668/psycharchives.797},
    place       = {Trier, Germany},
    date        = {2018-01-25},
    booktitle     = {Presentation at Registered Reports Workshop}
}


@article{Chambers2022,
    title       = {The past, present and future of {Registered Reports}},
    author      = {Chambers, C. D. and Tzavella, L.},
    volume      = {6},
    url         = {https://doi.org/10.1038/s41562-021-01193-7},
    doi         = {10.1038/s41562-021-01193-7},
    date        = {2022},
    pages       = {29–42},
    journal     = {Nature Human Behaviour}
}


@inbook{Clark2022,
    title       = {Adversarial collaboration: The next science reform},
    author      = {Clark, C. J. and Tetlock, P. E. and Frisby, R. E. and O’Donohue, W. T. and Lilienfeld, S. O.},
    editor      = {Frisby, C.L. and Redding, R.E. and O’Donohue, W.T. and Lilienfeld, S.O.},
    publisher   = {Springer},
    date        = {2022},
    booktitle     = {Political bias in psychology: Nature, scope, and solutions}
}


@inbook{Crețu2019,
    title       = {Perspectival realism},
    author      = {Crețu, A.-M.},
    editor      = {Peters, M.A.},
    publisher   = {Springer},
    date        = {2019},
    booktitle     = {Encyclopedia of educational philosophy and theory}
}


@book{Danziger1990,
    title       = {Constructing the subject: Historical origins of psychological research},
    author      = {Danziger, K.},
    publisher   = {Cambridge University Press},
    date        = {1990}
}


@article{Boeck2018,
    title       = {Perceived crisis and reforms: Issues, explanations, and remedies},
    author      = {De Boeck, P. and Jeon, M.},
    number      = {7},
    volume      = {144},
    url         = {https://doi.org/10.1037/bul0000154},
    doi         = {10.1037/bul0000154},
    date        = {2018},
    pages       = {757–777},
    journal     = {Psychological Bulletin}
}


@article{DelGiudice2021,
    title       = {A traveler’s guide to the multiverse: Promises, pitfalls, and a framework for the evaluation of analytic decisions},
    author      = {Del Giudice, M. and Gangestad, S. W.},
    number      = {1},
    volume      = {4},
    url         = {https://doi.org/10.1177/2515245920954925},
    doi         = {10.1177/2515245920954925},
    date        = {2021},
    journal     = {Advances in Methods and Practices in Psychological Science}
}


@article{Dellsén2018,
    title       = {Scientific progress: Four accounts},
    author      = {Dellsén, F.},
    number      = {11},
    volume      = {13},
    url         = {https://doi.org/10.1111/phc3.12525},
    doi         = {10.1111/phc3.12525},
    date        = {2018},
    pages       = {12525},
    journal     = {Philosophy Compass}
}


@article{Dellsén2020,
    title       = {The epistemic impact of theorizing: Generation bias implies evaluation bias},
    author      = {Dellsén, F.},
    volume      = {177},
    url         = {https://doi.org/10.1007/s11098-019-01387-w},
    doi         = {10.1007/s11098-019-01387-w},
    date        = {2020},
    pages       = {3661–3678},
    journal     = {Philosophical Studies}
}


@article{Derksen2019,
    title       = {Putting {Popper} to work},
    author      = {Derksen, M.},
    number      = {4},
    volume      = {29},
    url         = {https://doi.org/10.1177/0959354319838343},
    doi         = {10.1177/0959354319838343},
    date        = {2019},
    pages       = {449–465},
    journal     = {Theory \& Psychology}
}


@article{Derksen2022,
    title       = {The tone debate: Knowledge, self, and social order},
    author      = {Derksen, M. and Field, S.},
    number      = {2},
    volume      = {26},
    url         = {https://doi.org/10.1177/10892680211015636},
    doi         = {10.1177/10892680211015636},
    date        = {2022},
    pages       = {172–183},
    journal     = {Review of General Psychology}
}


@article{Derksen2022a,
    title       = {Kinds of replication: Examining the meanings of “conceptual replication” and “direct replication”},
    author      = {Derksen, M. and Morawski, J.},
    number      = {5},
    volume      = {17},
    url         = {https://doi.org/10.1177/17456916211041116},
    doi         = {10.1177/17456916211041116},
    date        = {2022},
    pages       = {1490–1505},
    journal     = {Perspectives on Psychological Science}
}


@article{Devezer2019,
    title       = {Scientific discovery in a model-centric framework: Reproducibility, innovation, and epistemic diversity},
    author      = {Devezer, B. and Nardin, L. G. and Baumgaertner, B. and Buzbas, E. O.},
    number      = {5},
    volume      = {14},
    url         = {https://doi.org/10.1371/journal.pone.0216125},
    doi         = {10.1371/journal.pone.0216125},
    date        = {2019},
    journal     = {PloS one}
}


@article{Devezer2021,
    title       = {The case for formal methodology in scientific reform},
    author      = {Devezer, B. and Navarro, D. J. and Vandekerckhove, J. and Ozge Buzbas, E.},
    number      = {3},
    volume      = {8},
    url         = {https://doi.org/10.1098/rsos.200805},
    doi         = {10.1098/rsos.200805},
    date        = {2021},
    journal     = {Royal Society Open Science}
}


@article{Drummond2019,
    title       = {Is the drive for reproducible science having a detrimental effect on what is published?},
    author      = {Drummond, C.},
    number      = {1},
    volume      = {32},
    url         = {https://doi.org/10.1002/leap. 1224},
    doi         = {10.1002/leap. 1224},
    date        = {2019},
    pages       = {63–69},
    journal     = {Learned Publishing}
}


@article{Errington2021a,
    title       = {Reproducibility in cancer biology: Challenges for assessing replicability in preclinical cancer biology},
    author      = {Errington, T. M. and Denis, A. and Perfito, N. and Iorns, E. and Nosek, B. A.},
    volume      = {10, Article e67995},
    url         = {https://doi.org/10.7554/eLife.67995},
    doi         = {10.7554/eLife.67995},
    date        = {2021},
    journal     = {Elife}
}


@article{Errington2021b,
    title       = {Investigating the replicability of preclinical cancer biology},
    author      = {Errington, T. M. and Mathur, M. and Soderberg, C. K. and Denis, A. and Perfito, N. and Iorns, E. and Nosek, B. A.},
    volume      = {10, Article e71601},
    url         = {https://doi.org/10.7554/eLife.71601},
    doi         = {10.7554/eLife.71601},
    date        = {2021},
    journal     = {Elife}
}


@article{Fabrigar2020,
    title       = {A validity-based framework for understanding replication in psychology},
    author      = {Fabrigar, L. R. and Wegener, D. T. and Petty, R. E.},
    number      = {4},
    volume      = {24},
    url         = {https://doi.org/10.1177/1088868320931366},
    doi         = {10.1177/1088868320931366},
    date        = {2020},
    pages       = {316–344},
    journal     = {Personality and Social Psychology Review}
}


@article{Fanelli2018,
    title       = {Opinion: Is science really facing a reproducibility crisis, and do we need it to?},
    author      = {Fanelli, D.},
    number      = {11},
    volume      = {115},
    url         = {https://doi.org/10.1073/pnas.1708272114},
    doi         = {10.1073/pnas.1708272114},
    date        = {2018},
    pages       = {2628–2631},
    journal     = {Proceedings of the National Academy of Sciences}
}


@article{Feest2019,
    title       = {Why replication is overrated},
    author      = {Feest, U.},
    number      = {5},
    volume      = {86},
    url         = {https://doi.org/10.1086/705451},
    doi         = {10.1086/705451},
    date        = {2019},
    pages       = {895–905},
    journal     = {Philosophy of Science}
}


@article{Fiedler2018,
    title       = {The creative cycle and the growth of psychological science},
    author      = {Fiedler, K.},
    number      = {4},
    volume      = {13},
    url         = {https://doi.org/10.1177/1745691617745651},
    doi         = {10.1177/1745691617745651},
    date        = {2018},
    pages       = {433–438},
    journal     = {Perspectives on Psychological Science}
}


@article{Fiedler2016,
    title       = {Questionable research practices revisited},
    author      = {Fiedler, K. and Schwarz, N.},
    number      = {1},
    volume      = {7},
    url         = {https://doi.org/10.1177/1948550615612150},
    doi         = {10.1177/1948550615612150},
    date        = {2016},
    pages       = {45–52},
    journal     = {Social Psychological and Personality Science}
}


@dissertation{Field2022,
    title       = {Charting the constellation of science reform},
    author      = {Field, S. M.},
    url         = {https://doi.org/10.31219/osf.io/udfw4},
    doi         = {10.31219/osf.io/udfw4},
    date        = {2022-07-13},
    journal     = {PsyArXiv},
    publisher   = {PsyArXiv}
}


@article{Field2021,
    title       = {Experimenter as automaton; experimenter as human: Exploring the position of the researcher in scientific research},
    author      = {Field, S. M. and Derksen, M.},
    volume      = {11, Article 11},
    url         = {https://doi.org/10.1007/s13194-020-00324-7},
    doi         = {10.1007/s13194-020-00324-7},
    date        = {2021},
    journal     = {European Journal for Philosophy of Science}
}


@book{Firestein2012,
    title       = {Ignorance: How it drives science},
    author      = {Firestein, S.},
    publisher   = {Oxford University Press},
    date        = {2012},

}


@article{Firestein2016,
    title       = {Why failure to replicate findings can actually be good for science},
    author      = {Firestein, S.},
    url         = {https://www.latimes.com/opinion/op-ed/la-oe-0214-firestein-science-replication-failure-20160214-story.html},
    date        = {2016-02-14},
    journal     = {LA Times},
    entrysubtype= {nonacademic}
}


@article{Fiske2016,
    title       = {A call to change science’s culture of shaming},
    author      = {Fiske, S. T.},
    volume      = {29},
    url         = {https://www.psychologicalscience.org/observer/a-call-to-change-sciences-culture-of-shaming},
    date        = {2016-10-31},
    journal     = {APS Observer},
    entrysubtype= {nonacademic}
}


@article{Flis2019,
    title       = {Psychologists psychologizing scientific psychology: An epistemological reading of the replication crisis},
    author      = {Flis, I.},
    number      = {2},
    volume      = {29},
    url         = {https://doi.org/10.1177/0959354319835322},
    doi         = {10.1177/0959354319835322},
    date        = {2019},
    pages       = {158–181},
    journal     = {Theory \& Psychology}
}


@article{Flis2022,
    title       = {The function of literature in psychological science},
    author      = {Flis, I.},
    number      = {2},
    volume      = {26},
    url         = {https://doi.org/10.1177/10892680211066466},
    doi         = {10.1177/10892680211066466},
    date        = {2022},
    pages       = {146–156},
    journal     = {Review of General Psychology}
}


@article{Freiling2021,
    title       = {The science of open (communication) science: Toward an evidence-driven understanding of quality criteria in communication research},
    author      = {Freiling, I. and Krause, N. M. and Scheufele, D. A. and Chen, K.},
    number      = {5},
    volume      = {71},
    url         = {https://doi.org/10.1093/joc/jqab032},
    doi         = {10.1093/joc/jqab032},
    date        = {2021},
    pages       = {686–714},
    journal     = {Journal of Communication}
}


@inbook{Gao2014,
    title       = {Methodologism/methodological imperative},
    author      = {Gao, Z.},
    editor      = {Teo, T.},
    url         = {https://doi.org/10.1007/978-1-4614-5583-7_614},
    doi         = {10.1007/978-1-4614-5583-7_614},
    publisher   = {Springer},
    date        = {2014},
    booktitle     = {Encyclopedia of critical psychology}
}


@article{Gervais2021,
    title       = {Practical methodological reform needs good theory},
    author      = {Gervais, W. M.},
    number      = {4},
    volume      = {16},
    url         = {https://doi.org/10.1177/1745691620977471},
    doi         = {10.1177/1745691620977471},
    date        = {2021},
    pages       = {827–843},
    journal     = {Perspectives on Psychological Science}
}


@book{Giere2006,
    title       = {Scientific perspectivism},
    author      = {Giere, R. N.},
    publisher   = {Chicago Press},
    date        = {2006}
}


@article{Greenfield2017,
    title       = {Cultural change over time: Why replicability should not be the gold standard in psychological science},
    author      = {Greenfield, P. M.},
    number      = {5},
    volume      = {12},
    url         = {https://doi.org/10.1177/1745691617707314},
    doi         = {10.1177/1745691617707314},
    date        = {2017},
    pages       = {762–771},
    journal     = {Perspectives on Psychological Science}
}


@book{Grossmann2021,
    title       = {How social science got better: Overcoming bias with more evidence, diversity, and self-reflection},
    author      = {Grossmann, M.},
    publisher   = {Oxford University Press},
    date        = {2021}
}


@article{Guttinger2020,
    title       = {The limits of replicability},
    author      = {Guttinger, S.},
    number      = {2},
    volume      = {10},
    url         = {https://doi.org/10.1007/s13194-019-0269-1},
    doi         = {10.1007/s13194-019-0269-1},
    date        = {2020},
    pages       = {1–17},
    journal     = {European Journal for Philosophy of Science}
}


@article{Haig2022,
    title       = {Understanding replication in a way that is true to science},
    author      = {Haig, B. D.},
    number      = {2},
    volume      = {26},
    url         = {https://doi.org/10.1177/10892680211046514},
    doi         = {10.1177/10892680211046514},
    date        = {2022},
    pages       = {224–240},
    journal     = {Review of General Psychology}
}


@article{Hamlin2017,
    title       = {Is psychology moving in the right direction? {A}n analysis of the evidentiary value movement},
    author      = {Hamlin, J. K.},
    number      = {4},
    volume      = {12},
    url         = {https://doi.org/10.1177/1745691616689062},
    doi         = {10.1177/1745691616689062},
    date        = {2017},
    pages       = {690–693},
    journal     = {Perspectives on Psychological Science}
}


@article{Hardwicke2023,
    title       = {Reducing bias, increasing transparency, and calibrating confidence with preregistration},
    author      = {Hardwicke, T. E. and Wagenmakers, E.},
    volume      = {7},
    url         = {https://doi.org/10.1038/s41562-022-01497-2},
    doi         = {10.1038/s41562-022-01497-2},
    date        = {2023},
    pages       = {15–26},
    journal     = {Nature Human Behaviour}
}


@article{Hartgerink2016,
    title       = {Research practices and assessment of research misconduct},
    author      = {Hartgerink, C. H.J. and Wicherts, J. M.},
    number      = {0},
    volume      = {0},
    url         = {https://doi.org/10.14293/S2199-1006.1.SOR-SOCSCI.ARYSBI.v1},
    doi         = {10.14293/S2199-1006.1.SOR-SOCSCI.ARYSBI.v1},
    date        = {2016},
    pages       = {1–10},
    journal     = {ScienceOpen Research}
}


@article{Hoekstra2021,
    title       = {Aspiring to greater intellectual humility in science},
    author      = {Hoekstra, R. and Vazire, S.},
    number      = {12},
    volume      = {5},
    url         = {https://doi.org/10.1038/s41562-021-01203-8},
    doi         = {10.1038/s41562-021-01203-8},
    date        = {2021},
    pages       = {1602–1607},
    journal     = {Nature Human Behaviour}
}


@article{Holcombe2021,
    title       = {Ad hominem rhetoric in scientific psychology},
    author      = {Holcombe, A. O.},
    number      = {2},
    volume      = {113},
    url         = {https://doi.org/10.1111/bjop. 12541},
    doi         = {10.1111/bjop. 12541},
    date        = {2021},
    pages       = {434–454},
    journal     = {British Journal of Psychology}
}


@article{Hostler2022,
    title       = {Open research reforms and the capitalist university’s priorities and practices: Areas of opposition and alignment},
    author      = {Hostler, T.},
    url         = {https://doi.org/10.31235/osf.io/r4qgc},
    doi         = {10.31235/osf.io/r4qgc},
    date        = {2022},
    journal     = {SocArXiv}
}


@article{Ioannidis2014,
    title       = {Publication and other reporting biases in cognitive sciences: Detection, prevalence, and prevention},
    author      = {Ioannidis, J. P. and Munafo, M. R. and Fusar-Poli, P. and Nosek, B. A. and David, S. P.},
    number      = {5},
    volume      = {18},
    url         = {https://doi.org/10.1016/j.tics.2014.02.010},
    doi         = {10.1016/j.tics.2014.02.010},
    date        = {2014},
    pages       = {235–241},
    journal     = {Trends in Cognitive Sciences}
}


@article{Iso-Ahola2020,
    title       = {Replication and the establishment of scientific truth},
    author      = {Iso-Ahola, S. E.},
    volume      = {11, Article 2183},
    url         = {https://doi.org/10.3389/fpsyg.2020.02183},
    doi         = {10.3389/fpsyg.2020.02183},
    date        = {2020},
    journal     = {Frontiers in Psychology}
}


@article{Jamieson2023,
    title       = {Reflexivity in quantitative research: A rationale and beginner’s guide},
    author      = {Jamieson, M. K. and Pownall, M. and Govaart, G. H.},
    volume      = {e12735},
    url         = {https://doi.org/10.1111/spc3.12735},
    doi         = {10.1111/spc3.12735},
    date        = {2023},
    journal     = {Social and Personality Psychology Compass, Article}
}


@article{John2012,
    title       = {Measuring the prevalence of questionable research practices with incentives for truth telling},
    author      = {John, L. K. and Loewenstein, G. and Prelec, D.},
    number      = {5},
    volume      = {23},
    url         = {https://doi.org/10.1177/0956797611430953},
    doi         = {10.1177/0956797611430953},
    date        = {2012},
    pages       = {524–532},
    journal     = {Psychological Science}
}


@article{Kessler2021,
    title       = {Open for whom? {T}he need to define open science for science education},
    author      = {Kessler, A. and Likely, R. and Rosenberg, J. M.},
    number      = {10},
    volume      = {58},
    url         = {https://doi.org/10.1002/tea.21730},
    doi         = {10.1002/tea.21730},
    date        = {2021},
    pages       = {1590–1595},
    journal     = {Journal of Research in Science Teaching}
}


@article{Leonelli2018,
    title       = {Rethinking reproducibility as a criterion for research quality. Including a symposium on {Mary Morgan}: Curiosity, imagination, and surprise},
    author      = {Leonelli, S.},
    volume      = {36B},
    url         = {https://doi.org/10.1108/S0743-41542018000036B009},
    doi         = {10.1108/S0743-41542018000036B009},
    publisher   = {Emerald Publishing},
    date        = {2018},
    pages       = {129–146},
    journal     = {Research in the History of Economic Thought and Methodology}
}


@article{Leonelli2022,
    title       = {Open science and epistemic diversity: Friends or foes?},
    author      = {Leonelli, S.},
    number      = {5},
    volume      = {89},
    url         = {https://doi.org/10.1017/psa.2022.45},
    doi         = {10.1017/psa.2022.45},
    date        = {2022},
    pages       = {991–1001},
    journal     = {Philosophy of Science}
}


@article{Leung2011,
    title       = {Presenting post hoc hypotheses as a priori: Ethical and theoretical issues},
    author      = {Leung, K.},
    number      = {3},
    volume      = {7},
    url         = {https://doi.org/10.1111/j.1740-8784.2011.00222.x},
    doi         = {10.1111/j.1740-8784.2011.00222.x},
    date        = {2011},
    pages       = {471–479},
    journal     = {Management and Organization Review}
}


@article{Levin2017,
    title       = {How does one “open” science? {Q}uestions of value in biological research},
    author      = {Levin, N. and Leonelli, S.},
    number      = {2},
    volume      = {42},
    url         = {https://doi.org/10.1177/0162243916672071},
    doi         = {10.1177/0162243916672071},
    date        = {2017},
    pages       = {280–305},
    journal     = {Science, Technology, \& Human Values}
}


@article{Lewandowsky2020,
    title       = {Low replicability can support robust and efficient science},
    author      = {Lewandowsky, S. and Oberauer, K.},
    volume      = {11, Article 358},
    url         = {https://doi.org/10.1038/s41467-019-14203-0},
    doi         = {10.1038/s41467-019-14203-0},
    date        = {2020},
    journal     = {Nature Communications}
}


@book{Longino1990,
    title       = {Science as social knowledge: Values and objectivity in scientific inquiry},
    author      = {Longino, H. E.},
    publisher   = {Princeton University Press},
    date        = {1990}
}


@article{Malich2022,
    title       = {Metascience is not enough – A plea for psychological humanities in the wake of the replication crisis},
    author      = {Malich, L. and Rehmann-Sutter, C.},
    number      = {2},
    volume      = {26},
    url         = {https://doi.org/10.1177/10892680221083876},
    doi         = {10.1177/10892680221083876},
    date        = {2022},
    pages       = {261–273},
    journal     = {Review of General Psychology}
}


@book{Massimi2022,
    title       = {Perspectival realism},
    author      = {Massimi, M.},
    publisher   = {Oxford University Press},
    date        = {2022}
}


@article{Maxwell2015,
    title       = {Is psychology suffering from a replication crisis? {W}hat does “failure to replicate” really mean?},
    author      = {Maxwell, S. E. and Lau, M. Y. and Howard, G. S.},
    number      = {6},
    volume      = {70},
    url         = {https://doi.org/10.1037/a0039400},
    doi         = {10.1037/a0039400},
    date        = {2015},
    pages       = {487–498},
    journal     = {American Psychologist}
}


@article{McDermott2022,
    title       = {Breaking free: How preregistration hurts scholars and science},
    author      = {McDermott, R.},
    number      = {1},
    volume      = {41},
    url         = {https://doi.org/10.1017/pls.2022.4},
    doi         = {10.1017/pls.2022.4},
    date        = {2022},
    pages       = {55–59},
    journal     = {Politics and the Life Sciences}
}


@article{Merton1987,
    title       = {Three fragments from a sociologist’s notebooks: Establishing the phenomenon, specified ignorance, and strategic research materials},
    author      = {Merton, R. K.},
    number      = {1},
    volume      = {13},
    url         = {https://doi.org/10.1146/annurev.so.13.080187.000245},
    doi         = {10.1146/annurev.so.13.080187.000245},
    date        = {1987},
    pages       = {1–29},
    journal     = {Annual Review of Sociology}
}


@article{Moody2022,
    title       = {Reproducibility in the social sciences},
    author      = {Moody, J. W. and Keister, L. A. and Ramos, M. C.},
    volume      = {48},
    url         = {https://doi.org/10.1146/annurev-soc-090221-035954},
    doi         = {10.1146/annurev-soc-090221-035954},
    date        = {2022},
    pages       = {65–85},
    journal     = {Annual Review of Sociology}
}


@article{Moran2022,
    title       = {I know it’s bad, but I have been pressured into it: Questionable research practices among psychology students in {Canada}},
    author      = {Moran, C. and Richard, A. and Wilson, K. and Twomey, R. and Coroiu, A.},
    number      = {1},
    volume      = {64},
    url         = {https://doi.org/10.1037/cap0000326},
    doi         = {10.1037/cap0000326},
    date        = {2022},
    pages       = {12–24},
    journal     = {Canadian Psychology}
}


@article{Morawski2019,
    title       = {The replication crisis: How might philosophy and theory of psychology be of use?},
    author      = {Morawski, J.},
    number      = {4},
    volume      = {39},
    url         = {https://doi.org/10.1037/teo0000129},
    doi         = {10.1037/teo0000129},
    date        = {2019},
    pages       = {218–238},
    journal     = {Journal of Theoretical and Philosophical Psychology}
}


@article{Morawski2022,
    title       = {How to true psychology’s objects},
    author      = {Morawski, J.},
    number      = {2},
    volume      = {26},
    url         = {https://doi.org/10.1177/10892680211046518},
    doi         = {10.1177/10892680211046518},
    date        = {2022},
    pages       = {157–171},
    journal     = {Review of General Psychology}
}


@article{Morey2019,
    title       = {You must tug that thread: Why treating preregistration as a gold standard might incentivize poor behavior},
    publisher   = {Psychonomic Society},
    author      = {Morey, R.},
    url         = {https://featuredcontent.psychonomic.org/you-must-tug-that-thread-why-treating-preregistration-as-a-gold-standard-might-incentivize-poor-behavior/},
    date        = {2019}
}


@article{Munafò2017,
        doi = {10.1038/s41562-016-0021},
        url = {https://doi.org/10.1038/s41562-016-0021},
        year = 2017,
        month = {jan},
        publisher = {Springer Science and Business Media {LLC}},
        volume = {1},
        number = {1},
        author = {Marcus R. Munaf{\`{o}} and Brian A. Nosek and Dorothy V. M. Bishop and Katherine S. Button and Christopher D. Chambers and Nathalie Percie du Sert and Uri Simonsohn and Eric-Jan Wagenmakers and Jennifer J. Ware and John P. A. Ioannidis},
        title = {A manifesto for reproducible science},
        journal = {Nature Human Behaviour}
}


@article{Nelson2018,
    title       = {Psychology’s renaissance},
    author      = {Nelson, L. D. and Simmons, J. and Simonsohn, U.},
    volume      = {69},
    url         = {https://doi.org/10.1146/annurev-psych-122216-011836},
    doi         = {10.1146/annurev-psych-122216-011836},
    date        = {2018},
    pages       = {511–534},
    journal     = {Annual Review of Psychology}
}


@article{Norton2015,
    title       = {Replicability of experiment},
    author      = {Norton, J. D.},
    number      = {2},
    volume      = {30},
    url         = {https://doi.org/10.1387/theoria.12691},
    doi         = {10.1387/theoria.12691},
    date        = {2015},
    pages       = {229–248},
    journal     = {Theoria. Revista de Teoría, Historia y Fundamentos de la Ciencia}
}


@article{Nosek2019,
        doi = {10.1016/j.tics.2019.07.009},
        url = {https://doi.org/10.1016/j.tics.2019.07.009},
        year = 2019,
        month = {oct},
        publisher = {Elsevier {BV}},
        volume = {23},
        number = {10},
        pages = {815--818},
        author = {B. A. Nosek and Emorie D. Beck and Lorne Campbell and Jessica K. Flake and Tom E. Hardwicke and David T. Mellor and Anna E. van 't Veer and Simine Vazire},
        title = {Preregistration Is Hard, And Worthwhile},
        journal = {Trends in Cognitive Sciences}
}


@article{Nosek2018,
    title       = {The preregistration revolution},
    author      = {Nosek, B. A. and Ebersole, C. R. and DeHaven, A. C. and Mellor, D. T.},
    number      = {11},
    volume      = {115},
    url         = {https://doi.org/10.1073/pnas.1708274114},
    doi         = {10.1073/pnas.1708274114},
    date        = {2018},
    pages       = {2600–2606},
    journal     = {Proceedings of the National Academy of Sciences}
}


@article{Nosek2022,
    title       = {Replicability, robustness, and reproducibility in psychological science},
    author      = {Nosek, B. A. and Hardwicke, T. E. and Moshontz, H. and Allard, A. and Corker, K. S. and Dreber, A. and Vazire, S.},
    volume      = {73},
    url         = {https://doi.org/10.1146/annurev-psych-020821-114157},
    doi         = {10.1146/annurev-psych-020821-114157},
    date        = {2022},
    pages       = {719–748},
    journal     = {Annual Review of Psychology}
}


@article{Nosek2014,
    title       = {Registered reports},
    author      = {Nosek, B. A. and Lakens, D.},
    number      = {3},
    volume      = {45},
    url         = {https://doi.org/10.1027/1864-9335/a000192},
    doi         = {10.1027/1864-9335/a000192},
    date        = {2014},
    pages       = {137–141},
    journal     = {Social Psychology}
}


@article{Nosek2012,
    title       = {Scientific utopia: II. Restructuring incentives and practices to promote truth over publishability},
    author      = {Nosek, B. A. and Spies, J. R. and Motyl, M.},
    number      = {6},
    volume      = {7},
    url         = {https://doi.org/10.1177/1745691612459058},
    doi         = {10.1177/1745691612459058},
    date        = {2012},
    pages       = {615–631},
    journal     = {Perspectives on Psychological Science}
}


@article{Oberauer2019,
    title       = {Preregistration of a forking path – What does it add to the garden of evidence?},
    publisher   = {Psychonomic Society},
    author      = {Oberauer, K.},
    url         = {https://featuredcontent.psychonomic.org/preregistration-of-a-forking-path-what-does-it-add-to-the-garden-of-evidence/},
    date        = {2019-01-15}
}


@article{Oberauer2019a,
    title       = {Addressing the theory crisis in psychology},
    author      = {Oberauer, K. and Lewandowsky, S.},
    number      = {5},
    volume      = {26},
    url         = {https://doi.org/10.3758/s13423-019-01645-2},
    doi         = {10.3758/s13423-019-01645-2},
    date        = {2019},
    pages       = {1596–1618},
    journal     = {Psychonomic Bulletin \& Review},
    entrysubtype = {article}
}


@article{Collaboration2015,
    title       = {Estimating the reproducibility of psychological science},
    author      = {{Open Science Collaboration}},
    number      = {6251},
    volume      = {349},
    url         = {https://doi.org/10.1126/science.aac4716},
    doi         = {10.1126/science.aac4716},
    date        = {2015},
    journal     = {Science},
    article     = {aac4716}
}


@article{Penders2022,
    title       = {Process and bureaucracy: Scientific reform as civilisation},
    author      = {Penders, B.},
    number      = {4},
    volume      = {42},
    url         = {https://doi.org/10.1177/02704676221126388},
    doi         = {10.1177/02704676221126388},
    date        = {2022},
    pages       = {107–116},
    journal     = {Bulletin of Science, Technology \& Society}
}


@article{Peterson2020,
    title       = {Metascience as a scientific social movement},
    author      = {Peterson, D. and Panofsky, A.},
    url         = {https://osf.io/preprints/socarxiv/4dsqa/},
    date        = {2020-08-04}
}


@article{Peterson2021,
    title       = {Arguments against efficiency in science},
    author      = {Peterson, D. and Panofsky, A.},
    number      = {3},
    volume      = {60},
    url         = {https://doi.org/10.1177/05390184211021383},
    doi         = {10.1177/05390184211021383},
    date        = {2021},
    pages       = {350–355},
    journal     = {Social Science Information}
}


@article{Pham2021,
    title       = {Preregistration is neither sufficient nor necessary for good science},
    author      = {Pham, M. T. and Oh, T. T.},
    number      = {1},
    volume      = {31},
    url         = {https://doi.org/10.1002/jcpy.1209},
    doi         = {10.1002/jcpy.1209},
    date        = {2021},
    pages       = {163–176},
    journal     = {Journal of Consumer Psychology}
}


@article{Pownall2022,
    title       = {Is replication possible for qualitative research?},
    author      = {Pownall, M.},
    url         = {https://doi.org/10.31234/osf.io/dwxeg},
    doi         = {10.31234/osf.io/dwxeg},
    date        = {2022-06-14},
    journal     = {PsyArXiv}
}


@inbook{Pownall2021,
    title       = {Embedding open and reproducible science into teaching: A bank of lesson plans and resources},
    author      = {Madeleine Pownall and Flavio Azevedo and Alaa Aldoh and Mahmoud Elsherif and Martin Vasilev and Charlotte R. Pennington and Olly Robertson and Myrthe Vel Tromp and Meng Liu and Matthew C. Makel and Natasha Tonge and David Moreau and Ruth Horry and John Shaw and Loukia Tzavella and Ronan McGarrigle and Catherine Talbot and Sam Parsons.},
    url         = {https://doi.org/10.1037/stl0000307},
    doi         = {10.1037/stl0000307},
    date        = {2021},
    booktitle     = {Scholarship of Teaching and Learning in Psychology}
}


@article{Pownall2022a,
    title       = {Slow science in scholarly critique},
    author      = {Pownall, M. and Hoerst, C.},
    volume      = {35},
    url         = {https://thepsychologist.bps.org.uk/volume-35/february-2022/slow-science-scholarly-critique},
    date        = {2022},
    pages       = {2},
    journal     = {The Psychologist}
}


@article{Prosser2022,
    title       = {When open data closes the door: Problematising a one size fits all approach to open data in journal submission guidelines},
    author      = {Annayah M. B. Prosser and Richard J. T. Hamshaw and Johanna Meyer and Ralph Bagnall and Leda Blackwood and Monique Huysamen and Abbie Jordan and Konstantina Vasileiou and Zoe Walter},
    url         = {https://doi.org/10.1111/bjso.12576},
    doi         = {10.1111/bjso.12576},
    date        = {2022},
    journal     = {British Journal of Social Psychology}
}


@article{Proulx2021,
    title       = {Beyond statistical ritual: Theory in psychological science},
    author      = {Proulx, T. and Morey, R. D.},
    number      = {4},
    volume      = {16},
    url         = {https://doi.org/10.1177/17456916211017098},
    doi         = {10.1177/17456916211017098},
    date        = {2021},
    pages       = {671–681},
    journal     = {Perspectives on Psychological Science}
}


@inbook{Reiss2020,
    title       = {Scientific objectivity},
    author      = {Reiss, J. and Sprenger, J.},
    url         = {https://plato.stanford.edu/entries/scientific-objectivity/},
    date        = {2020},
    booktitle     = {The Stanford Encyclopedia of Philosophy}
}


@article{Rosnow1983,
    title       = {Von Osten's horse, Hamlet's question, and the mechanistic view of causality: Implications for a post-crisis social psychology},
    author      = {Rosnow, R. L.},
    number      = {3},
    volume      = {4},
    url         = {http://www.jstor.org/stable/43852983},
    date        = {1983},
    pages       = {319–337},
    journal     = {The Journal of Mind and Behavior}
}


@article{Rubin2017a,
    title       = {An evaluation of four solutions to the forking paths problem: Adjusted alpha, preregistration, sensitivity analyses, and abandoning the {Neyman-Pearson} approach},
    author      = {Rubin, M.},
    number      = {4},
    volume      = {21},
    url         = {https://doi.org/10.1037/gpr0000135},
    doi         = {10.1037/gpr0000135},
    date        = {2017},
    pages       = {321–329},
    journal     = {Review of General Psychology}
}


@article{Rubin2017b,
    title       = {When does {HARK}ing hurt? {I}dentifying when different types of undisclosed post hoc hypothesizing harm scientific progress},
    author      = {Rubin, M.},
    number      = {4},
    volume      = {21},
    url         = {https://doi.org/10.1037/gpr0000128},
    doi         = {10.1037/gpr0000128},
    date        = {2017},
    pages       = {308–320},
    journal     = {Review of General Psychology}
}


@article{Rubin2020,
    title       = {Does preregistration improve the credibility of research findings?},
    author      = {Rubin, M.},
    number      = {4},
    volume      = {16},
    url         = {https://doi.org/10.20982/tqmp. 16.4.p376},
    doi         = {10.20982/tqmp. 16.4.p376},
    date        = {2020},
    pages       = {376–390},
    journal     = {The Quantitative Methods for Psychology}
}


@article{Rubin2021a,
    title       = {What type of {Type I} error? {C}ontrasting the {Neyman-Pearson and Fisherian} approaches in the context of exact and direct replications},
    author      = {Rubin, M.},
    volume      = {198},
    url         = {https://doi.org/10.1007/s11229-019-02433-0},
    doi         = {10.1007/s11229-019-02433-0},
    date        = {2021},
    pages       = {5809–5834},
    journal     = {Synthese}
}


@article{Rubin2021b,
    title       = {When to adjust alpha during multiple testing: A consideration of disjunction, conjunction, and individual testing},
    author      = {Rubin, M.},
    volume      = {199},
    url         = {https://doi.org/10.1007/s11229-021-03276-4},
    doi         = {10.1007/s11229-021-03276-4},
    date        = {2021},
    pages       = {10969–11000},
    journal     = {Synthese}
}


@article{Rubin2022,
    title       = {The costs of {HARK}ing},
    author      = {Rubin, M.},
    number      = {2},
    volume      = {73},
    url         = {https://doi.org/10.1093/bjps/axz050},
    doi         = {10.1093/bjps/axz050},
    date        = {2022},
    pages       = {535–560},
    journal     = {British Journal for the Philosophy of Science}
}



@article{Rubin2022a,
        doi = {10.1080/09515089.2022.2113771},
        url = {https://doi.org/10.1080/09515089.2022.2113771},
        year = 2022,
        month = {aug},
        publisher = {Informa {UK} Limited},
        pages = {1--29},
        author = {Mark Rubin and Chris Donkin},
        title = {Exploratory hypothesis tests can be more compelling than confirmatory hypothesis tests},
        journal = {Philosophical Psychology}
}


@article{Sacco2019,
    title       = {Grounds for ambiguity: {J}ustifiable bases for engaging in questionable research practices},
    author      = {Sacco, D. F. and Brown, M. and Bruton, S. V.},
    number      = {5},
    volume      = {25},
    url         = {https://doi.org/10.1007/s11948-018-0065-x},
    doi         = {10.1007/s11948-018-0065-x},
    date        = {2019},
    pages       = {1321–1337},
    journal     = {Science and Engineering Ethics}
}


@article{Scheel2021,
    title       = {An excess of positive results: {C}omparing the standard psychology literature with {Registered Reports}},
    author      = {Scheel, A. M. and Schijen, M. R. and Lakens, D.},
    number      = {2},
    volume      = {4},
    url         = {https://doi.org/10.1177/25152459211007467},
    doi         = {10.1177/25152459211007467},
    date        = {2021},
    pages       = {1–12},
    journal     = {Advances in Methods and Practices in Psychological Science}
}


@article{Schimmack2020,
    title       = {A meta-psychological perspective on the decade of replication failures in social psychology},
    author      = {Schimmack, U.},
    number      = {4},
    volume      = {61},
    url         = {https://doi.org/10.1037/cap0000246},
    doi         = {10.1037/cap0000246},
    date        = {2020},
    pages       = {364–376},
    journal     = {Canadian Psychology}
}


@article{Shrout2018,
    title       = {Psychology, science, and knowledge construction: Broadening perspectives from the replication crisis},
    author      = {Shrout, P. E. and Rodgers, J. L.},
    volume      = {69},
    url         = {https://doi.org/10.1146/annurev-psych-122216-011845},
    doi         = {10.1146/annurev-psych-122216-011845},
    date        = {2018},
    pages       = {487–510},
    journal     = {Annual Review of Psychology}
}


@article{Simmons2011,
    title       = {False-positive psychology: Undisclosed flexibility in data collection and analysis allows presenting anything as significant},
    author      = {Simmons, J. P. and Nelson, L. D. and Simonsohn, U.},
    number      = {11},
    volume      = {22},
    url         = {https://doi.org/10.1177/0956797611417632},
    doi         = {10.1177/0956797611417632},
    date        = {2011},
    pages       = {1359–1366},
    journal     = {Psychological Science}
}


@article{Simmons2021,
    title       = {Pre‐registration: Why and how},
    author      = {Simmons, J. P. and Nelson, L. D. and Simonsohn, U.},
    number      = {1},
    volume      = {31},
    url         = {https://doi.org/10.1002/jcpy.1208},
    doi         = {10.1002/jcpy.1208},
    date        = {2021},
    pages       = {151–162},
    journal     = {Journal of Consumer Psychology}
}


@article{Smithson1996,
    title       = {Science, ignorance and human values},
    author      = {Smithson, M.},
    number      = {1},
    volume      = {2},
    url         = {https://doi.org/10.1177/097168589600200107},
    doi         = {10.1177/097168589600200107},
    date        = {1996},
    pages       = {67–81},
    journal     = {Journal of Human Values}
}


@misc{bib118,
    author      = {{Society for the Improvement of Psychological Science}},
    title       = {Mission Statement},
    url         = {https://improvingpsych.org/mission/},
    date        = {2022}
}


@inbook{Spellman2018,
    title       = {Open science},
    author      = {Spellman, B. A. and Gilbert, E. A. and Corker, K. S.},
    editor      = {Wixted, J.T. and Wagenmakers, E.-J.},
    volume      = {5},
    edition     = {4},
    publisher   = {Wiley},
    date        = {2018},
    pages       = {729–775},
    booktitle     = {Stevens’ handbook of experimental psychology and cognitive neuroscience: Volume 5 {Methodology}},
    doi         = {10.1002/9781119170174.epcn519},
    url         = {https://doi.org/10.1002/9781119170174.epcn519}
}


@article{Stanley2014,
    title       = {Expectations for replications: Are yours realistic?},
    author      = {Stanley, D. J. and Spence, J. R.},
    number      = {3},
    volume      = {9},
    url         = {https://doi.org/10.1177/1745691614528518},
    doi         = {10.1177/1745691614528518},
    date        = {2014},
    pages       = {305–318},
    journal     = {Perspectives on Psychological Science}
}


@article{Stanley2018,
    title       = {What meta-analyses reveal about the replicability of psychological research},
    author      = {Stanley, T. D. and Carter, E. C. and Doucouliagos, H.},
    number      = {12},
    volume      = {144},
    url         = {https://doi.org/10.1037/bul0000169},
    doi         = {10.1037/bul0000169},
    date        = {2018},
    pages       = {1325–1346},
    journal     = {Psychological Bulletin}
}


@article{Stroebe2014,
    title       = {The alleged crisis and the illusion of exact replication},
    author      = {Stroebe, W. and Strack, F.},
    number      = {1},
    volume      = {9},
    url         = {https://doi.org/10.1177/1745691613514450},
    doi         = {10.1177/1745691613514450},
    date        = {2014},
    pages       = {59–71},
    journal     = {Perspectives on Psychological Science}
}


@article{Strong1991,
    title       = {Theory-driven science and naïve empiricism in counseling psychology},
    author      = {Strong, S. R.},
    number      = {2},
    volume      = {38},
    url         = {https://doi.org/10.1037/0022-0167.38.2.204},
    doi         = {10.1037/0022-0167.38.2.204},
    date        = {1991},
    pages       = {204–210},
    journal     = {Journal of Counseling Psychology}
}


@article{Szollosi2021,
    title       = {Arrested theory development: The misguided distinction between exploratory and confirmatory research},
    author      = {Szollosi, A. and Donkin, C.},
    number      = {4},
    volume      = {16},
    url         = {https://doi.org/10.1177/1745691620966796},
    doi         = {10.1177/1745691620966796},
    date        = {2021},
    pages       = {717–724},
    journal     = {Perspectives on Psychological Science}
}


@article{Szollosi2020,
    title       = {Is preregistration worthwhile?},
    author      = {Szollosi, A. and Kellen, D. and Navarro, D. J. and Shiffrin, R. and van Rooij, I. and Van Zandt, T. and Donkin, C.},
    number      = {2},
    volume      = {24},
    url         = {https://doi.org/10.1016/j.tics.2019.11.009},
    doi         = {10.1016/j.tics.2019.11.009},
    date        = {2020},
    pages       = {94–95},
    journal     = {Trends in Cognitive Science}
}


@article{Ulrich2020,
    title       = {Meta-research: Questionable research practices may have little effect on replicability},
    author      = {Ulrich, R. and Miller, J.},
    volume      = {9, Article e58237},
    url         = {https://doi.org/10.7554/eLife.58237},
    doi         = {10.7554/eLife.58237},
    date        = {2020},
    journal     = {Elife}
}


@article{UygunTunç2022,
    title       = {Is open science neoliberal? {P}erspectives on Psychological Science},
    author      = {Uygun Tunç, D. and Tunç, M. N. and Eper, Z. B.},
    url         = {https://doi.org/10.1177/17456916221114835},
    doi         = {10.1177/17456916221114835},
    date        = {2022}
}


@article{Vancouver2018,
    title       = {In defense of {HARK}ing},
    author      = {Vancouver, J. N.},
    number      = {1},
    volume      = {11},
    url         = {https://doi.org/10.1017/iop. 2017.89},
    doi         = {10.1017/iop. 2017.89},
    date        = {2018},
    pages       = {73–80},
    journal     = {Industrial and Organizational Psychology}
}


@book{Dijk2021,
    title       = {How to tackle confirmation bias?},
    author      = {van Dijk, T.},
    url         = {https://www.delta.tudelft.nl/article/how-tackle-confirmation-bias},
    publisher   = {Journalistic Platform TU Delft},
    place       = {Delta},
    date        = {2021-06-22}
}


@article{vanRooij2021,
    title       = {Theory before the test: How to build high-verisimilitude explanatory theories in psychological science},
    author      = {van Rooij, I. and Baggio, G.},
    number      = {4},
    volume      = {16},
    url         = {https://doi.org/10.1177/1745691620970604},
    doi         = {10.1177/1745691620970604},
    date        = {2021},
    pages       = {682–697},
    journal     = {Perspectives on Psychological Science}
}


@article{Vazire2018,
    title       = {Implications of the credibility revolution for productivity, creativity, and progress},
    author      = {Vazire, S.},
    number      = {4},
    volume      = {13},
    url         = {https://doi.org/10.1177/1745691617751884},
    doi         = {10.1177/1745691617751884},
    date        = {2018},
    pages       = {411–417},
    journal     = {Perspectives on Psychological Science}
}


@article{Vazire2022,
    title       = {Credibility beyond replicability: Improving the four validities in psychological science},
    author      = {Vazire, S. and Schiavone, S. R. and Bottesini, J. G.},
    number      = {2},
    volume      = {31},
    url         = {https://doi.org/10.1177/09637214211067779},
    doi         = {10.1177/09637214211067779},
    date        = {2022},
    pages       = {162–168},
    journal     = {Current Directions in Psychological Science}
}


@article{Wagenmakers2012,
    title       = {A year of horrors},
    author      = {Wagenmakers, E. J.},
    volume      = {27},
    date        = {2012},
    pages       = {12–13},
    journal     = {De Psychonoom}
}


@article{Wagenmakers2012a,
    title       = {An agenda for purely confirmatory research},
    author      = {Wagenmakers, E. J. and Wetzels, R. and Borsboom, D. and van der Maas, H. L. and Kievit, R. A.},
    number      = {6},
    volume      = {7},
    url         = {https://doi.org/10.1177/1745691612463078},
    doi         = {10.1177/1745691612463078},
    date        = {2012},
    pages       = {632–638},
    journal     = {Perspectives on Psychological Science}
}


@article{Walkup2021,
    title       = {Replication and reform: Vagaries of a social movement},
    author      = {Walkup, J.},
    number      = {2},
    volume      = {41},
    url         = {https://doi.org/10.1037/teo0000171},
    doi         = {10.1037/teo0000171},
    date        = {2021},
    pages       = {131–133},
    journal     = {Journal of Theoretical and Philosophical Psychology}
}


@article{Wentzel2021,
    title       = {Open science reforms: Strengths, challenges, and future directions},
    author      = {Wentzel, K. R.},
    number      = {2},
    volume      = {56},
    url         = {https://doi.org/10.1080/00461520.2021.1901709},
    doi         = {10.1080/00461520.2021.1901709},
    date        = {2021},
    pages       = {161–173},
    journal     = {Educational Psychologist}
}


@article{Whitaker2020,
    title       = {{\#bropenscience is broken science}},
    author      = {Whitaker, K. and Guest, O.},
    volume      = {33},
    url         = {https://thepsychologist.bps.org.uk/volume-33/november-2020/bropenscience-broken-science},
    date        = {2020},
    pages       = {34–37},
    journal     = {The Psychologist}
}


@article{Wiggins2019,
    title       = {The replication crisis in psychology: An overview for theoretical and philosophical psychology},
    author      = {Wiggins, B. J. and Christopherson, C. D.},
    number      = {4},
    volume      = {39},
    url         = {https://doi.org/10.1037/teo0000137},
    doi         = {10.1037/teo0000137},
    date        = {2019},
    pages       = {202–217},
    journal     = {Journal of Theoretical and Philosophical Psychology}
}


@article{Wood2019,
    title       = {No crisis but no time for complacency},
    author      = {Wood, W. and Wilson, T. D.},
    number      = {7},
    volume      = {32},
    url         = {https://www.psychologicalscience.org/observer/no-crisis-but-no-time-for-complacency},
    date        = {2019-08-22},
    journal     = {APS Observer}
}


@article{Zwaan2018,
    title       = {Making replication mainstream},
    author      = {Zwaan, R. A. and Etz, A. and Lucas, R. E. and Donnellan, M. B.},
    volume      = {41, Article E120},
    url         = {https://doi.org/10.1017/S0140525X17001972},
    doi         = {10.1017/S0140525X17001972},
    date        = {2018},
    journal     = {Behavioral and Brain Sciences}
}
\end{filecontents}


\abstracttext{Metascientists have studied \emph{questionable research practices} in science. The present article considers the parallel concept of \emph{questionable metascience practices} (QMPs). A QMP is a research practice, assumption, or perspective that has been questioned by several commentators as being potentially problematic for metascience and/or the science reform movement. The present article reviews ten QMPs that relate to criticism, replication, bias, generalization, and the characterization of science. Specifically, the following QMPs are considered: (1) rejecting or ignoring self-criticism; (2) a fast ‘n’ bropen scientific criticism style; (3) overplaying the role of replication in science; (4) assuming a replication rate is “too low” without specifying an “acceptable” rate; (5) an unacknowledged metabias towards explaining the replication crisis in terms of researcher bias; (6) assuming that researcher bias can be reduced; (7) devaluing exploratory results as being more “tentative” than confirmatory results; (8) presuming that questionable research practices are problematic research practices; (9) focusing on knowledge accumulation; and (10) focusing on specific scientific methods. It is stressed that only \emph{some} metascientists engage in \emph{some} QMPs \emph{some} of the time, and that these QMPs may not \emph{always} be problematic. Research is required to estimate the prevalence and impact of QMPs. In the meantime, QMPs should be viewed as invitations to ask questions about how we go about doing better metascience.}

\jotetitle{Questionable Metascience Practices}
\keywordsabstract{metascience, open science, questionable research practices, replication crisis, \mbox{science reform}}
\runningauthor{Rubin}
\corremail{\href{mark-rubin@outlook.com}{mark-rubin@outlook.com}}
\corraddress{Durham University}
\jname{Journal of Trial \& Error}
\paperdoi{10.36850/mr4}
\paperreceived{September 8, 2022}
\author[1]{Mark Rubin\orcid{0000-0002-6483-8561}}
\authorone{Mark Rubin}
\affil[1]{Durham University}
\paperaccepted{December 13, 2022}
\jwebsite{https://journal.trialanderror.org}
\jyear{2023}
\paperpublished{April 24, 2023}
\paperpublisheddate{2023-04-24}

\begin{document}

\begin{frontmatter}
  \maketitle
  \begin{abstract}
    \printabstracttext
  \end{abstract}
\end{frontmatter}



\lettrine{I}{n} \citeyear{Simmons2011}, Simmons et al. demonstrated that researchers can present “anything as significant” (p. 1359) by conducting numerous analyses (e.g., using different outcome variables, sample sizes, and/or covariates) and then selectively reporting only those analyses that yield significant results. A year later, \citeauthor{John2012}'s \mbox{(\hspace*{-.21em}\citeyear{John2012})} published the results of a survey which purported to show that \emph{questionable research practices }(QRPs)\emph{, }such as HARKing and \emph{p}-hacking, are prevalent among psychologists. A few years later, an attempt to replicate 100 psychology studies found that only 39\% of effects were rated as replicable \parencite{Collaboration2015}.



In light of this and other work, some metascientists have concluded that QRPs play a significant role in increasing the publication of “false positive” results and, therefore, lowering replication rates \parencites[e.g.,][]{Bishop2019}{Bishop2020}{Munafò2017}{Nosek2012}{Collaboration2015}{Schimmack2020}{Spellman2018}. Partly in response, science reformers have advocated new “open science” research practices that are intended to reveal and reduce QRPs \parencites[e.g., preregistered research plans, publicly accessible research data and materials][]{Munafò2017}.



In the present article, I consider questionable research practices in the field of metascience. A \emph{questionable metascience practice }(QMP) is a research practice, assumption, or perspective that has been questioned by several commentators as being potentially problematic for metascience and/or the science reform movement. I outline ten QMPs that are grouped into the five broad categories of (a) criticism, (b) replication, (c) bias, (d) generalization, and (e) science characterization.



Please note that I have not provided an exhaustive list of QMPs \parencites[for some additional QMPs, please see][p. 2]{Devezer2021}. In addition, unlike \citeauthor{John2012}'s \mbox{(\hspace*{-.21em}\citeyear{John2012})} study of QRPs, I have not attempted to estimate the prevalence of the QMPs that I consider. It is possible that only a \emph{few} metascientists have engaged in the QMPs, and that they have engaged in only a \emph{few} QMPs a \emph{few} times. Nonetheless, under some circumstances, a few low frequency QMPs may be quite influential and problematic, especially when they are undertaken by prominent metascientists who are regarded as leaders and role models in the field. Hence, it is worthwhile considering QMPs even if they have a low prevalence.



Finally, in my view, QMPs are not always problematic. They are merely “questionable” in the sense that they warrant questioning before a conclusion is reached about whether they are problematic in any given situation. Hence, my aim is not to cast aspersions on the field of metascience but, instead, to encourage a deeper consideration of its more questionable research practices, assumptions, and perspectives.\footnote{Researcher positionality statement: I identify as a White, Western, heterosexual, middle-class, cisgender man. My primary field of research is social psychology. However, I have recently published several articles in the field of metascience. Here, I have adopted a relatively nuanced and contextualist approach to issues such as HARKing, \emph{p}-hacking, and multiple testing \parencites[e.g.,][]{Rubin2017a}{Rubin2017b}{Rubin2021b}{Rubin2022}. I have also adopted a relatively critical approach to some science reforms, such as (a) preregistration \parencites[e.g.,][]{Rubin2020}{Rubin2022} and (b) greater adherence to Neyman-Pearson hypothesis testing \parencites[e.g.,][]{Rubin2021a}. My current philosophy of science is closest to perspectival realism \parencites{Crețu2019}{Giere2006}{Massimi2022}. For more information about my work, please see \url{https://sites.google.com/site/markrubinsocialpsychresearch/}}



\section{Criticism-Related QMPs}



\subsection{Rejecting or Ignoring Self-Criticism}



As several commentators have noted, some metascientists react particularly negatively and defensively towards criticisms of their proposed science reforms \parencites{Bastian2021}{Gervais2021}[p. 5]{Malich2022}[p. 132]{Walkup2021}. For example, as \textcite{Flis2022} explained, there was a rather extreme negative reaction on social media to an article by \textcite{Szollosi2020} that criticized the open science practice of preregistration. Flis suggested that this highly negative reaction may have represented a defensive response that was learned during metascientists’ interactions with so-called “status-quoers” who questioned the reality of the replication crisis and opposed the need for science reform. In other words, some first-generation metascientists and reformers may have adopted a particularly negative reaction to self-criticism because they perceived it to be a challenge to their raison d’être.



Instead of rejecting self-criticism, some metascientists may simply ignore it, especially in the more authoritative space of the published literature. For example, as of February 2023, 228 articles have cited a pro-preregistration article by \textcite{Nosek2019} that was published around the same time and in the same journal as \citeauthor{Szollosi2020}'s \mbox{(\hspace*{-.21em}\citeyear{Szollosi2020})} critical article. However, only 17\% of these 228 articles (\emph{k }= 39) have also cited Szollosi et al. (To identify these 39 articles, I clicked on “cited by” in Google Scholar for the Nosek et al. article and then selected “search within citing articles” and searched for “Is preregistration worthwhile?”). This low co-citation rate may reflect a citation bias against an article that is critical of a prominent science reform \parencites[for another example of potential citation bias, please see][p. 6]{Flis2022}. This type of citation bias creates an illusion of consensus in the literature, and it may obstruct the motive for theory improvement by giving the impression that current theories are adequate and undisputed \parencites[see also][]{Bishop2020}[p. 1604]{Hoekstra2021}. Hence, “failing to cite publications that contradict your beliefs” is regarded as a QRP \parencites[p. 8]{Allum2023}. To prevent this QRP from becoming a QMP, metascientists should encourage self-criticism, cite their critics’ work, and respond in a thoughtful manner \parencites{Altenmüller2021}[p. 828]{Gervais2021}[p. 235]{Haig2022}[p. 1604]{Hoekstra2021}. To be clear, metascientists do not always need to concede to their critics’ arguments. However, they do need to engage with those arguments publicly, formally, and carefully \parencites[see also][]{Longino1990}.



\subsection{Fast ‘n’ Bropen Criticism}



Concerns have also been raised about the style and tone of some metascientists’ interactions with scientists, especially on social media \parencites[e.g.,][]{Fiske2016}[p. 692]{Hamlin2017}{Pownall2022a}{Whitaker2020}. For example, \textcite{Whitaker2020} coined the term \emph{bropenscience} to refer to a dismissive, mocking, school-yard style of scientific criticism that some metascientists sometimes use on social media \parencites[e.g.,][]{Anonymous2021}[see also][]{Derksen2022}[pp. 529-530]{Pownall2021}. Similarly, \textcite{Pownall2022} noted that, in contrast to the appeal for more thoughtful and “slower” science, there is a “growing culture of fast, hostile, and superficial critiques of research” on social media.\footnote{Some critics of the science reform movement also have problematic communication styles at times \parencites[for some examples, see][]{Holcombe2021}. Nonetheless, two wrongs don’t make a right!}



Although a \emph{fast ‘n’ bropen} criticism style may be used rarely and by few metascientists, it can be problematic if it is used by relatively prominent metascientists who are regarded as being representative of the field. In particular, it may (a) distract from and/or deter legitimate criticism, (b) cause scientists to feel personally attacked and/or excluded \parencites[e.g.,][]{Derksen2022}[p. 692]{Hamlin2017}{Pownall2021}, (c) damage the reputation of metascience, and/or (d) reduce the uptake of beneficial science reforms \parencite{Gervais2021}. Metascientists should undertake thoughtful, “critical evaluation with civility and mutual respect” \parencite{bib118}.



\section{Replication-Related QMPs}



\subsection{Overplaying Replication}



Some metascientists assume that direct replications are a method for assessing the “truth” of a claim or effect. For example, \textcite[p. 617]{Nosek2012} stated that “replication is a means of increasing the confidence in the truth value of a claim”; \textcite[p. 520]{Nelson2018} stated that, “to a scientist, a true effect is one that replicates under specifiable conditions”; and \textcite[p. 153]{Simmons2021} stated that “many published findings do not replicate under specifiable conditions and so are, by the standards of science, untrue” \parencites[for further examples, see][pp. 6-8]{Devezer2021}. Some metascientists also regard replication as an essential\emph{ }and\emph{ }defining aspect of science. For example, the \textcite[p. 1]{Collaboration2015} described reproducibility as “a defining feature of science,” and \textcite[p. 13]{Zwaan2018} explained that replication is “an essential component of science…a foundational principle of the scientific method” \parencites[see also][p. 108]{Asendorpf2013}[p. 48]{Chambers2017}[p. 618]{Nosek2012}[for further examples, see][p. 64]{Drummond2019}[p. 226]{Haig2022}[p. 487]{Maxwell2015}. In response, critics have argued that these sorts of statements overplay the role of replication in science \parencites{Boeck2018}{Devezer2019}{Devezer2021}{Feest2019}{Greenfield2017}{Guttinger2020}{Haig2022}{Iso-Ahola2020}{Leonelli2018}{Norton2015}.



Replication does not indicate whether research claims or findings are true. Replicable results may be “false” due to model misspecification, reliable but invalid measures, or overly liberal evidence thresholds, and “true” results may be nonreplicable due to model misspecification, unreliable methods, or irreversible changes in the population over time \parencites{Bak-Coleman2022}{Buzbas2023}{Boeck2018}{Devezer2019}{Devezer2021}{Errington2021b}{Guttinger2020}{Iso-Ahola2020}{Norton2015}[p. 739]{Nosek2022}{Rubin2021a}{Stanley2014}. Furthermore, replication is not an essential component of science. Scientists often use other methods to demonstrate the reliability of their results, such as robustness analyses \parencites{Haig2022}{Leonelli2018}. Alternatively, they may provide a repeat demonstration of the existence of a phenomenon within the same study using a different set of variables that are nonetheless representative of the theoretical constructs that were used in the original demonstration.



Certainly, replication is important in some areas of science. However, it is a QMP to overplay replication as an “essential” aspect of science that indexes the “truth” of findings \parencites[p. 10]{Devezer2021}.



\subsection{Unspecified Replication Rate Targets}



Some metascientists claim that replication rates need to be improved. For example, the \textcite[p. 7]{Collaboration2015} concluded that “there is room to improve reproducibility in psychology,” and  \textcite[p. 1]{Munafò2017} explained that “data from many fields suggests reproducibility is lower than is desirable.” However, it is unclear how replication rates can be judged to be “low” and in need of improvement in the absence of clear targets for “acceptable” replication rates. Logically, this reasoning represents an incomplete comparison.



In their recent review,\textcite{Nosek2022} found that 64\% of 307 replications reported statistically significant evidence in the same direction as the original studies. Is this replication rate “too low” or is it “acceptable?” Nosek et al. were unsure, asking: “what degree of replicability should be expected?” (p. 730) and “what is the optimal replicability rate at different stages of research maturity?” (p. 738). They suggested that these questions should be addressed in future metascience research \parencite[see also][p. 7]{Collaboration2015}. However, the deferral of this question implies that metascientists are trying to solve a problem that they are not yet sure exists. After all, future research may reveal that current replication rates are “acceptable” \parencites{Bird2020}[p. 692]{Freiling2021}[p. 8]{Guttinger2020}{Lewandowsky2020}. Alternatively, the meaningfulness of quantifying replication rates may be called into question \parencites{Buzbas2023}{Rubin2021a}.



In the absence of clear targets for “acceptable” replication rates, it is not surprising that several commentators have questioned whether current replication rates are at “crisis��� levels \parencites[e.g.,][]{Barrett2015}{Bird2020}{Buzbas2023}{Fanelli2018}{Firestein2016}{Freiling2021}{Haig2022}{Maxwell2015}{Morawski2019}{Shrout2018}{Stroebe2014}{Wood2019}. Certainly, claiming that a replication rate is “too low” without specifying an “acceptable” replication rate represents a QMP.



\section{Bias-Related QMPs}



\subsection{Metabias}



As several commentators have observed, contemporary metascientists tend to be concerned with how bias and motivated reasoning influence scientists’ methods, analyses, and interpretations \parencites{Field2021}{Flis2019}{Morawski2019}{Morawski2022}[p. 7]{Peterson2020}[for examples, see][]{Bishop2020}[chapter 1]{Chambers2017}{Chambers2022}{Hardwicke2023}{Ioannidis2014}{Munafò2017}{Nosek2012}[p. 153]{Simmons2021}. Indeed, \textcite{Morawski2022} has suggested that metascientists may be biased towards explaining the replication crisis in terms of researcher bias because they are overrepresented by psychologists \parencites{Moody2022}[see also][]{Flis2019}{Malich2022}, who tend to be familiar with cognitive and motivational biases (i.e., a type of availability heuristic bias). Consistent with Morawski’s interpretation, it is interesting to note that psychologists’ metabias may also explain their emphasis on researcher bias during the 1960s-1970s crisis of confidence in social psychology \parencites[p. 600]{Peterson2021}{Rosnow1983}. In this previous crisis, psychologists were concerned about researchers biasing the behavior of their participants (e.g., experimenter expectancy effects). In the current replication crisis, they are more concerned about researchers biasing their methods and analyses.



To be consistent with their concerns about researcher bias, metascientists should acknowledge their own \emph{metabias} towards explanations of the replication crisis that refer to researcher bias. There are multiple mutually compatible explanations for failed replications that do not refer to researcher bias, including data errors, fraud, a base rate fallacy, low power, unreliable measurement, poor validity, hidden moderators, and heterogenous effects \parencites[e.g.,][]{Bird2020}{Boeck2018}{Fabrigar2020}{Maxwell2015}{Rubin2021a}{Stanley2014}. Researcher bias and associated QRPs represent only one potential explanation, yet they have been given a disproportionate amount of attention in explanations of, and solutions to, the replication crisis \parencites[e.g.,][]{Hardwicke2023}{Munafò2017}[p. 372]{Schimmack2020}. Focusing on researcher bias at the expense of other viable explanations represents a form of causal reductionism \parencites[p. 17]{Devezer2019}, and an acknowledgement of metabias may help to produce a more balanced and comprehensive multicausal account of the replication crisis.




\newpage
\subsection{The Bias Reduction Assumption}



Some metascientists believe that preregistration and registered reports reduce researcher bias. For example, \textcite[p. 15]{Hardwicke2023} explained that “preregistration…reduces the risk of bias by encouraging outcome-independent decision-making”; \textcite[p. 166]{Vazire2022} explained that “the aim of the Registered Report format is to reduce bias by eliminating many of the avenues for undisclosed flexibility in research”; and \textcite{Chambers2018} described “Registered Reports as a vaccine against research bias” \parencites[see also][p. 32]{Chambers2022}[p. 2]{Scheel2021}[for commentary, see][]{Field2021}. There are three problems with this claim.



First, researcher bias influences not only the post hoc selection of hypotheses, data, analyses, and results (i.e., \emph{selective reporting}), but also the a priori selection of hypotheses, methods, analyses, evidence thresholds, and interpretations \parencites[i.e., \emph{selective questioning};][]{Rubin2022a}, and considering selective reporting without also considering selective questioning may lead to a biased evaluation of researcher bias. For example, preregistering the number of times that a researcher will toss a coin may help to identify and reduce any selective reporting of their results (e.g., only reporting when the coin lands heads and not when it lands tails). However, the reduction of this selective reporting will not reduce researcher bias if the researcher’s preregistered decision rule is “heads I win, tails you lose!” As \textcite{Clark2022} put it, “the dice have often been loaded before pre-registration” \parencites[p. 13, see also][]{Dellsén2020}{Jamieson2023}. Consequently, it is a QMP to assume that a preregistered study is less biased than a non-preregistered study, because selective questioning in the preregistered study may be more problematic than selective reporting in the non-preregistered study \parencites[for similar concerns, see][p. 16]{Devezer2021}[p. 698]{Freiling2021}{Jamieson2023}{McDermott2022}{Oberauer2019}[p. 167]{Pham2021}{Rubin2022a}[p. 95]{Szollosi2020}[p. 212]{Wiggins2019}.



Second, it might be argued that preregistration reduces selective reporting when all other variables are held constant, including variables associated with selective questioning. However, even if, ceteris paribus, preregistration reduces selective reporting, it may also \emph{increase} other types of researcher bias, such as (a) the \emph{researcher commitment bias} (sticking with a planned research approach, even when it is inappropriate), (b) the \emph{researcher prophecy bias} (misattributing a researcher’s lucky, atheoretical prophecy to a theory’s predictive power), and (c) a bias towards committing data fraud \parencites[for a discussion, please see][]{Rubin2022a}. Again, it is a QMP to consider bias reduction in terms of selective reporting per se and ignore other forms of researcher bias.



Finally, and more generally, the metascientific concept of “bias reduction” assumes that researchers can get closer to an “unbiased” evaluation, which smacks of \emph{naïve} \emph{objectivism,} \emph{naïve empiricism, naïve realism, }and\emph{ value-free science} \parencites{Field2021}[p. 228]{Morawski2019}{Reiss2020}{Strong1991}{Dijk2021}{Wiggins2019}. According to these philosophical positions, scientists can observe an immutable reality directly and in an unbiased and objective manner. However, contrary to these positions, research is always undertaken from one perspective or another, so it is always “biased” from one perspective or another, and what are seen as decreases in bias from one perspective may be regarded as increases in bias from another. Consequently, a more tenable position is that open science practices help to \emph{reveal different} \emph{perspectives} rather than to \emph{reduce} \emph{bias} \parencites{Field2021}{Grossmann2021}{Jamieson2023}{Pownall2022}. For example, a robustness or multiverse analysis allows readers to understand how different analytical approaches produce or “enact” different results \parencites{DelGiudice2021}{Morey2019}{Rubin2020}[for a discussion of the “enactment” perspective, see][]{Derksen2022a}. In addition, researcher positionality statements can reveal researchers’ perspectives rather than reduce their biases \parencites{Jamieson2023}.



\section{Sweeping Generalization QMPs}



\subsection{Devaluing Exploratory Hypothesis Tests}



Some metascientists devalue unplanned exploratory tests of post hoc hypotheses relative to preregistered confirmatory tests of a priori hypotheses, even when the exploratory tests are correctly reported as being exploratory. For example, relative to the results of confirmatory hypothesis tests, the results of exploratory tests are supposed to have a “higher risk of bias” \parencites[p. 19]{Hardwicke2023} and entail greater “uncertainty” \parencites[p. 2601]{Nosek2018}, which makes their associated conclusions more “tentative” \parencites[p. 19]{Errington2021a}[p. 238]{Ioannidis2014}[p. 519]{Nelson2018}[p. 138]{Nosek2014}[p. 154]{Simmons2021}. Consequently, “confirmatory analyses…have much greater evidential impact than exploratory analyses” \parencites[p. 13]{Wagenmakers2012}, and research conclusions should be “appropriately weighted in favour of the confirmatory outcomes” \parencites[p. 36]{Chambers2022}. There are two problems with this perspective.



First, critics have argued that the distinction between exploratory and confirmatory hypothesis tests is unclear and irrelevant, both from a statistical perspective \parencites{Devezer2021}{Rubin2020}{Rubin2021b} and from a philosophical standpoint \parencites{Rubin2020}{Rubin2022}{Rubin2022a}{Szollosi2021}. In particular, it has been shown that the “double use” of the same data to (a) generate hypotheses and then (b) test those hypotheses is not necessarily problematic \parencites{Devezer2021}, and that any “circular reasoning” involved in this process can be identified by checking the \emph{contents} of the reasoning without needing to know the \emph{timing} of the reasoning \parencites{Rubin2022a}.



Second, even if we accept the validity of the confirmatory-exploratory distinction and agree that, \emph{all other things being equal,} exploratory results tend to be more tentative than confirmatory results, it would be a fallacy of the general rule to conclude that \emph{all} exploratory results are more tentative than \emph{all} confirmatory results. For example, an exploratory result may be evaluated as being \emph{less} tentative than a confirmatory result when it is based on higher quality theory, methods, and analyses than the confirmatory result and when it is accompanied by greater transparency vis-à-vis robustness analyses and open data and materials \parencites{Devezer2021}{Morey2019}{Rubin2020}{Szollosi2020}. Consequently, it would be a QMP to argue that “exploratory studies cannot be presented as strong evidence in favor of a particular claim” \parencites[p. 635]{Wagenmakers2012a}, because \emph{high quality} exploratory studies can provide stronger evidence than \emph{low quality} confirmatory studies \parencites[see also][p. 314]{Rubin2017b}.



\subsection{Presuming QRPs are Problematic}



Another sweeping generalization QMP is to presume that \emph{questionable} research practices are always \emph{problematic} research practices. For example, \textcite[p. 1]{Hartgerink2016} defined QRPs as “practices that are detrimental to the research process…[and that] harm the research process”; \textcite{Chambers2014} described QRPs as “soft fraud”; and \textcite[p. 372]{Schimmack2020} proposed that “the most obvious solution [to the replication crisis] is to ban the use of questionable research practices and to treat them like other types of unethical behaviours.” There are two problems with this position.



First, QRPs can be perfectly acceptable research practices \parencites{Fiedler2016}[Table 6]{Moran2022}[p. 551]{Rubin2022}{Sacco2019}. For example, the QRP of “failing to report all of a study’s dependent measures” \parencites[p. 525]{John2012} may not indicate \emph{p}-hacking if (a) there are good reasons to exclude the measures from the research report and (b) the excluded measures are irrelevant to the final research conclusions \parencites[p. 46]{Fiedler2016}[p. 531]{John2012}{Rubin2017b}{Rubin2020}. As their name implies, QRPs need to be “questioned” by other researchers and interpretated in specific research situations before they can be judged to be potentially problematic.



Second, even potentially problematic research practices such as HARKing and \emph{p}-hacking may not always be problematic for research credibility and replicability \parencites[e.g.,][]{Bak-Coleman2022}{Devezer2019}{Fanelli2018}{Leung2011}{Rubin2017a}{Rubin2017b}{Rubin2020}{Rubin2022}{Stanley2018}{Ulrich2020}{Vancouver2018}. Hence, a more tenable position is to assume that only \emph{some} QRPs are \emph{potentially} problematic in specific research situations, and only \emph{some} potentially problematic research practices are \emph{actually} problematic under \emph{some} conditions.



\section{Science Characterization QMPs}



\subsection{Focusing on Knowledge Accumulation}



Some metascientists assume that the goal of science is to accumulate knowledge \parencites[e.g.,][p. 1]{Errington2021b}{Munafò2017}{Nosek2012}{Vazire2018}. For example, \textcite[p. 617]{Nosek2012} explained that “the primary objective of science as a discipline is to accumulate knowledge about nature,” and \textcite[p. 416]{Vazire2018} explained that “the common goal among all scientists is to accumulate knowledge.” Commentators have noted that, from this perspective, some metascientists view low replication rates as indicating an “inefficient” accumulation of knowledge \parencites{Morawski2022}{Peterson2021}[for examples, see][]{Errington2021b}{Munafò2017}{Nosek2012}{Vazire2018}[for discussions, see][]{Hostler2022}{UygunTunç2022}. The proposed open science reforms are supposed to improve the efficiency of knowledge accumulation \parencites[e.g.,][p. 37]{Chambers2022}[p. 626]{Nosek2012}. For example, \textcite[p. 626]{Nosek2012} concluded that “scientific practices can be improved to enhance the efficiency of knowledge building.”



However, there are two reasons that knowledge accumulation may not be regarded as the primary objective of science. First, different philosophies of science emphasize different goals. For example, besides knowledge accumulation, \textcite{Dellsén2018} described three alternative goals of science: truth-seeking, problem-solving, and understanding. Second, any philosophy of science that posits knowledge as a goal should also acknowledge the complementary role of ignorance: “What does this unexpected effect mean?” and “why did we find a null result in this study?” These sorts of known unknowns are essential for scientific progress because they motivate the generation of hypotheses for future studies. Hence, according to this “knowledge-\emph{and}-ignorance” perspective, scientific progress is achieved through not only knowledge accumulation, but also \emph{specified ignorance} \parencites{Firestein2012}{Merton1987}[][p. 7]{Collaboration2015}[p. 5826]{Rubin2021a}{Smithson1996}.



Importantly, knowledge accumulation and specified ignorance have opposite associations with replicability. Successful replications represent scientific progress by confirming current hypotheses. However, failed replications also represent scientific progress by motivating the generation of new hypotheses that explain why the replications failed \parencites[e.g., by positing boundary conditions; for an example, see][]{Firestein2012}. Hence, although low replication rates may indicate poor knowledge accumulation, they may also represent scientific progress vis-à-vis greater specified ignorance.



In summary, definitions of scientific progress depend on the types of goals to be achieved \parencites[p. 236]{Haig2022}. Metascientists who assume that knowledge accumulation is central to scientific progress should also acknowledge that (a) other philosophies of science regard other objectives as being more important, and (b) specified ignorance is equally as important as knowledge accumulation.



\begin{table*}
  \begin{fullwidth}
    \caption{Questionable Metascience Practices}
    \label{tab:Table1}
    \begin{tabularx}{\linewidth}{@{} l >{\RaggedRight\arraybackslash}p{12em} >{\RaggedRight\arraybackslash}X >{\RaggedRight\arraybackslash}X@{}}
      \toprule
         & Name                                                                                                                                                                                                                                    & Definition                                                                                                                                                                                           & Recommended Practice \\

      \midrule
      1  & Rejecting or ignoring self-criticism                                                                                                                                                                                                    & Rejecting or ignoring criticisms of metascience and/or science reform
         & Encourage self-criticism, cite critics’ work, and respond in a thoughtful manner
      \\

      2  & Fast ‘n’ bropen criticism                                                                                                                                                                                                               & A quick, superficial, dismissive, and/or mocking style of scientific criticism
         & Undertake careful “critical evaluation with civility and mutual respect” \parencite{bib118}
      \\

      3  & Overplaying replication                                                                                                                                                                                                                 & Assuming that replication is essential to science, and that it indexes “the truth”
         & Qualify and contextualize claims about the centrality and role of replication in science
      \\

      4  & Unspecified replication rate targets                                                                                                                                                                                                    & Assuming that a replication rate is “too low” without specifying an “acceptable” rate
         & Elaborate on the meaning of “low” when discussing “low replication rates”
      \\

      5  & Metabias                                                                                                                                                                                                                                & A bias towards explaining the replication crisis in terms of researcher bias
         & Undertake a more balanced and comprehensive assessment of explanations for the replication crisis
      \\

      6  & The bias reduction assumption                                                                                                                                                                                                           & Focusing on selective reporting as the primary form of researcher bias and assuming that it can be reduced without increasing other forms of bias
         & Consider other forms of researcher bias (e.g., selective questioning, researcher commitment bias) and reveal different research perspectives (e.g., through robustness analyses and researcher positionality statements)
      \\

      7  & Devaluing exploratory hypothesis tests                                                                                                                                                                                                  & Devaluing an exploratory result as being more “tentative” than a confirmatory result without considering other relevant issues (e.g., quality of associated theory, methods, analyses, transparency)
         & Acknowledge that some exploratory results can be \emph{less} tentative than some confirmatory results
      \\

      8  & Presuming QRPs are problematic                                                                                                                                                                                                          & Presuming that questionable research practices are always problematic research practices
         & Acknowledge that only \emph{some} QRPs are \emph{potentially} problematic in specific research situations, and only \emph{some} potentially problematic research practices are \emph{actually} problematic under \emph{some} conditions
      \\

      9  & Focusing on knowledge accumulation                                                                                                                                                                                                      & Conceiving knowledge accumulation as the primary objective of science without considering (a) the role of specified ignorance or (b) different objectives in other philosophies of science
         & Acknowledge that (a) knowledge accumulation and specified ignorance go hand-in-hand and (b) different philosophies of science define scientific progress differently
      \\

      10 & Homogenizing science                                                                                                                                                                                                                    & Focusing on specific approaches as “the scientific method”
         & Diversify membership in the metascience community and embrace scientific diversity and pluralism
      \\
      \bottomrule
    \end{tabularx}
  \end{fullwidth}
\end{table*}




\subsection{Homogenizing Science}



As several commentators have noted, some metascientists appear to assume that there is a single scientific method rather than a collection of diverse methods \parencites[for commentators, see][]{Drummond2019}{Malich2022}[p. 21]{Peterson2020}[see also][p. 2]{Guttinger2020}. \textcite[pp. 4-6]{Malich2022} argued that this “homogenizing view” is apparent every time a metascientist refers to “the scientific method” in the singular and without qualification \parencites[e.g.,][p. 7]{Munafò2017}[p. 618]{Nosek2012}[p. 13]{Zwaan2018}[for further examples, see][p. 64]{Drummond2019}.



In addition, and at the risk of homogenizing metascience \parencites{Field2022}, some (not all) metascientists focus their concerns on particular aspects of “the scientific method” \parencite{Flis2019}. In particular, the contemporary metascientific view of science tends to focus on:

\begin{enumerate}


  \item a priori predictions \parencites[e.g.,][p. 36]{Chambers2022}[p. 154]{Simmons2021};



  \item quantitative methods \parencites{Bennett2021}[p. 691]{Hamlin2017}[p. 530]{Pownall2021};


  \item rigorous statistical analyses \parencites[for a review, see][]{Moody2022};


  \item replicable effects \parencites[e.g.,][p. 617]{Nosek2012}[p. 153]{Simmons2021};



  \item unbiased interpretations \parencites[e.g.,][]{Hardwicke2023}[p. 166]{Vazire2022}; and



  \item a Popperian philosophy of science \parencites{Flis2019}[p. 74]{Grossmann2021}{Morawski2019}{Morawski2022}[for examples, see][]{Derksen2019}.


\end{enumerate}

However, from a critical perspective, these foci may be associated with:

\begin{enumerate}


  \item \emph{predictivism}: the view that a priori predictions are superior to post hoc inferences \parencites[p. 1605]{Oberauer2019a}{Rubin2017b}{Rubin2022};



  \item \emph{methodolatory/methodologism}: the prioritizing of methodological rigor over other research concerns, such as theory \parencites{Chamberlain2000}[p. 5]{Danziger1990}{Gao2014};



  \item \emph{statisticism/mathematistry}: an overemphasis on statistics as both a problem and a solution in science \parencites{Boring1919}{Brower1949}{Fiedler2018}{Proulx2021};


  \item \emph{naïve empiricism} \parencites{Strong1991}: the view that science progresses through the accumulation of replicable effects \parencites{Flis2022}{Proulx2021}{vanRooij2021};


  \item \emph{naïve objectivism}: the view that it is possible for scientists to adopt unbiased and objective perspectives \parencites{Field2021}{Penders2022}{Wiggins2019}; and



  \item a fairly narrow and outdated philosophy of science \parencites{Derksen2019}[p. 170]{Flis2019}[p. 74]{Grossmann2021}[p. 226, p. 233]{Morawski2019}.


\end{enumerate}

Furthermore, several commentators have noted that these metascientific foci may have the unintended consequence of alienating scientists whose work does not fit with this particular view of science \parencites{Bennett2021}{Kessler2021}{Levin2017}{Malich2022}[p. 58]{McDermott2022}{Penders2022}[p. 530]{Pownall2021}{Prosser2022}[p. 170]{Wentzel2021}. To address this problem, and to facilitate the recognition of their own biases, metascientists should continue to diversify their membership and embrace scientific diversity and pluralism \parencites{Andreoletti2020}{Flis2022}{Gervais2021}{Grossmann2021}{Leonelli2022}{Pownall2022}.



Table 1 summarizes the 10 QMPs that I have discussed and includes recommended practices in relation to each one.




\section{Conclusion}



Paralleling \citeauthor{John2012}'s \mbox{(\hspace*{-.21em}\citeyear{John2012})} concept of questionable research practices, the present article considered a nonexhaustive list of 10 questionable metascience practices\emph{.} Readers may disagree about the importance of specific QMPs. However, in my view, it remains useful for metascientists to consider the basic concept of QMPs and to reflect on the ways in which they (a) handle criticism, (b) conceptualize replication, (c) consider researcher bias, (d) avoid sweeping generalizations, and (e) acknowledge the diversity and pluralism of science.



In discussing QMPs, we should be careful not to homogenize metascience \parencites{Field2022} or to presume that QMPs are necessarily problematic. It is likely that only \emph{some} metascientists engage in \emph{some} QMPs \emph{some} of the time and that QMPs are only problematic in \emph{some} situations. Future metascientific research may wish to assess the prevalence and impact of various QMPs in order to obtain a clearer understanding of these issues. In the meantime, QMPs should be regarded as invitations to reflect on metascientific practices, assumptions, and perspectives and to ask “questions” about how we go about doing better metascience.



\nocite{*}

\printbibliography






\end{document}