\documentclass[authordate, empirical]{jote-new-article}

\usepackage{caption}

\usepackage{tabularx}

\usepackage{graphicx}

\usepackage{hyperref}

\usepackage[backend=biber,style=apa]{biblatex}

\addbibresource{bibliography.bib}

\jotetitle{Ragworms Anecdotes – Shared by Jim van Belzen}
\keywordsabstract{ragworms, Hediste diversicolor, saltmarsh, seedling establishment, waves, ecological interactions, serendipity}
\runningauthor{van Belzen}
\jname{Journal of Trial \& Error}
\jyear{2024}
\paperdoi{10.36850/r6}
\paperreceived{August 2, 2024}
\author[1]{\mbox{Jim van Belzen\orcid{0000-0003-2099-1545}}}
\affil[1]{Wageningen Marine Research (WMR) & Department of Estuarine and Delta Systems (EDS), Royal Netherlands Institute for Sea Research (NIOZ)}
\corremail{\href{mailto:jim.van.belzen@nioz.nl}{jim.van.belzen@nioz.nl}}
\corraddress{Wageningen Marine Research (WMR) & Department of Estuarine and Delta Systems (EDS), Royal Netherlands Institute for Sea Research (NIOZ)}
\runningauthor{van Belzen}
\paperaccepted{August 2, 2024}
\paperpublished{August 16, 2024}
\paperpublisheddate{2024-08-16}
\jwebsite{https://journal.trialanderror.org}



\begin{document}
\begin{frontmatter}
  \maketitle
  \begin{abstract}
    \printabstracttext
  \end{abstract}
\end{frontmatter}


	\section{Ragworms anecdotes -- Shared by Jim van Belzen}



	When appointed as PhD-candidate to work on coastal ecosystems as nature-based coastal protection I started working on the processes that could limit saltmarsh ecosystems from establishing on tidal flats. Saltmarshes seaward of the sea defence attenuate waves and offer flood protection in case of a dike breach and therefore these vegetated wetlands improve coastal resilience (Temmerman et al., 2013; van den Hoven et al., 2023). Previous research within our department (Spatial Ecology Department, Centre for Estuarine and Marine Ecology, the Netherlands Institute for Ecology (NIOO-CEME), which is referred to today as Department of Estuarine and Delta Systems, The Netherlands Institute for Sea Research (NIOZ-EDS)) pointed to wave action as a key factor hindering the establishment of pioneer saltmarsh vegetation on bare tidal flats seaward of the saltmarsh. However, the method used by my predecessors was problematic.



	The results were likely influenced by the planting method of seedling grasses into the sediments of the tidal flat, as the young plants were grown on sand in the lab. After four weeks, the roots were rinsed out and then placed in a hole on-site which was created by pushing a finger in the mud. The roots, important for the anchoring of the pioneering plants in the mud, were certainly disrupted by this procedure. We therefore planned to change the procedure. The seedlings would grow in the mud collected at the place where we would also replant the grown grasses. In essence; with a small clod to not disturb the roots.



	After setting up the pots with mud collected from the transplantation site into an automated flooding-and-drainage system outside, mimicking the tidal inundation, the potted, just-germinated seedlings suddenly disappeared. After replanting the seedlings into the pots, the most disappeared again the next day. We were incredibly surprised as my experienced research assistants and I didn't anticipate this sudden disappearance. We hypothesized about various possible explanations. For example, flushing due to the mimicking of the tidal inundation or birds eating them. Yet, these hypotheses could not be confirmed and seemed rather unlikely due to various reasons.



	The third time we replanted the few seedlings that were left, our assistant stuck around to observe what would happen with the seedlings. Staying at the tanks they watched the pots for about an hour observing what was happening with the seedlings. Suddenly a ragworm (\emph{Hediste diversicolor}) came out of the sediment, grabbed the stem of a seedling, and pulled it into the sediment. These mud-dwelling marine polychaete worms, here a few centimeters long and about two millimeters thick, possess segmented bodies adorned with bristle-like structures called chaetae, used for both locomotion and defense (Scaps, 2002). They play a crucial role in marine ecosystems as scavengers and predators. Their observation put us on a track of various observations and research revealing the key role ragworms can play in understanding marsh establishment.



	We were now set on a track to investigate the role of the ragworm in the establishment of marsh vegetation. Since we could not rule out the effects of waves on vegetation establishment, we set up a full factorial field experiment to study the effects of waves and ragworms. This helped us determine which factor is more important for vegetation establishment on a wave swept tidal flat. Therefore, we dampened the influence of waves using sandbag barriers and excluded ragworms from the plot by defaunation of the tidal flats. We used this field experiment supplemented with experimentation in laboratory and wave flume. Via this, we found that ragworms can limit seedling establishment by both eating seedlings and making the sediment more erodible to waves due to their bioturbating activities of the sediment causing earlier uprooting. Thus, ragworms turned out to play a more prominent role than expected simply based on this surprise observation. They were even interacting with waves and boosting wave impacts. This work is currently under revision for the Journal of Ecology.



	But we also found that ragworms seem to garden food. The follow-up anecdote:



	As a result of the surprising find Zhenchang Zhu and I started doing various follow-up experiments with ragworms, seeds and seedlings. In one such experiment, Zhu put mimic seeds made from plastic nurdles into the sediment. Another surprising observation followed. It seemed that the ragworms were selective and removed them from the sediment. At the same time, they grabbed real seeds and moved them into the burrow. Yet, after some days these seeds were retrieved from the pots and not eaten. This was strange as the plastic nurdles were not retained in the sediment. After some further experimentation it turned out that ragworms cannot eat the seeds due to a hard seed husk. Yet, once the seed sprouts the food becomes available and is even more nutritious than unhusked seeds. The ragworms grow faster on eating the sprouts compared to unhusked seeds. This work (Zhu et al., 2016) suggested that ragworms had developed a way to rear nutritious food.







	References:



	Scaps, P. (2002). A review of the biology, ecology and potential use of the common ragworm \emph{Hediste diversicolor }(O.F. Müller) (Annelida: Polychaeta). \emph{Hydrobiologia}, \emph{470}, 203-218. http://doi.org/10.1023/A:1015681605656



	Temmerman, S., Meire, P., Bouma, T. J., Herman, P. M., Ysebaert, T., \& De Vriend, H. J. (2013). Ecosystem-based coastal defence in the face of global change. \emph{Nature}, \emph{504}(2013), 79-83. http://doi.org/10.1038/nature12859



	van den Hoven, K., van Belzen, J., Kleinhans, M. G., Schot, D. M., Merry, J., van Loon-Steensma, J. M., \& Bouma, T. J. (2023). How natural foreshores offer flood protection during dike breaches: An explorative flume study. \emph{Estuarine, Coastal and Shelf Science}, \emph{294}, Article 108560. https://doi.org/10.1016/j.ecss.2023.108560



	Zhu, Z., van Belzen, J., Hong, T., Kunihiro, T., Ysebaert, T., Herman, P. M., \& Bouma, T. J. (2016). Sprouting as a gardening strategy to obtain superior supplementary food: Evidence from a seed-caching marine worm. \emph{Ecology}, \emph{97}(12), 3278-3284. http://doi.org/10.1002/ecy.1613


\end{document}