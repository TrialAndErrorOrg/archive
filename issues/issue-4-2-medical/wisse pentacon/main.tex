\documentclass[authordate, empirical]{jote-new-article}

\usepackage{caption}

\usepackage{tabularx}

\usepackage{graphicx}

\usepackage{hyperref}

\usepackage[backend=biber,style=apa]{biblatex}

\usepackage{float}

\addbibresource{bibliography.bib}

\jotetitle{Partial Endothelial Trepanation versus Deep Anterior Lamellar Keratoplasty in keratoconus patients: Results of the PENTACON trial}
\keywordsabstract{keratoconus, lamellar surgery, surgical safety, clinical trial, DALK}
\abstracttext{The purpose of this research was to report on the surgical safety and outcomes of two distinct techniques of anterior lamellar corneal surgery: the Big Bubble Deep Anterior Lamellar Keratoplasty (DALK) versus Busin’s Partial ENdothelial Trepanation (PET) in addition to anterior lamellar keratoplasty in keratoconus patients. In short, the PENTACON trial. In this multicenter trial, patients were randomized to receive either a DALK or PET procedure. Primary outcome was the occurrence of a intra-operative complication necessitating conversion to a full-thickness corneal graft. Secondary outcomes were uncorrected and best corrected visual acuity (UCVA/BCVA), manifest refraction, corneal astigmatism, and adverse events at 1 year follow-up. Fourteen eyes of 14 patients were enrolled in this study. At baseline, mean logMAR UCVA and BCVA were 1.59 (\emph{SD} = 0.35) and 0.89 (\emph{SD} = 0.69) respectively. Mean thinnest pachymetry was 322 (\emph{SD} = 66µm), with a mean Kmax of 76.7 (\emph{SD} = 14.1D). In five of 13 surgeries a full-thickness conversion occurred (DALK:PET 3:2, (\emph{p} =  .592). Overall, logMAR UCVA and BCVA increased to 0.52 (\emph{SD} = 0.20, (\emph{p} = .003) and 0.26  (\emph{SD} = 0.36, (\emph{p} = .03) respectively at 1 year follow-up. Mean refractive astigmatism was 3.8 (\emph{SD} = 2.2D). No significant differences were observed between both treatment groups for any of the secondary outcomes parameters. No conclusions can be drawn on the primary outcome based on this underpowered clinical trial. However, the PET technique was not as safe as expected. A low trial inclusion rate and lack of scientific equipoise prompted trial termination. We regard it our ethical obligation to report these results.}
\runningauthor{Wisse et al.}
\jname{Journal of Trial \& Error}
\jyear{2025}
\paperdoi{10.36850/0550-4e9c}
\paperreceived{June 25, 2024}
\author[1]{\mbox{Robert P.L. Wisse\orcid{0000-0002-2844-9868}}}
\affil[1]{University Medical Center, Utrecht, the Netherlands}
\corremail{\href{mailto:Rplwisse@gmail.com}{Rplwisse@gmail.com}}
\corraddress{University Medical Center Utrecht}
\runningauthor{Wisse et al.}
\author[2]{\mbox{Cathrien A. Eggink}}
\affil[2]{University Medical Center St. Radboud, Nijmegen, the Netherlands}
\author[3]{\mbox{Bart T.H. van Dooren\orcid{0000-0001-8802-2770}}}
\affil[3]{Amphia Ziekenhuis, Breda, the Netherlands}
\author[1]{\mbox{Allegonda van der Lelij\orcid{0000-0002-2237-6995}}}
\paperaccepted{March 18, 2025}
\paperpublished{May 25, 2025}
\paperpublisheddate{2025-05-25}
\jwebsite{https://journal.trialanderror.org}

\specialissue{Scientific Failure and Uncertainty in the Health Domain}
\articletype{Special Issue - Empirical}

\begin{document}
\begin{frontmatter}
  \maketitle
  \begin{abstract}
    \printabstracttext
  \end{abstract}
\end{frontmatter}

	



	\begin{takeHomeMessage}
		The PENTACON trial, which compared Big Bubble Deep Anterior Lamellar Keratoplasty (DALK) techniques in advanced keratoconus patients, highlighted major challenges in clinical research. These included identifying feasible patient selection criteria, managing a prolonged study, and coordinating a multicenter design. Although the trial was underpowered and ultimately terminated, it underscores the ethical imperative to report all findings. These techniques have a steep learning curve that is frequently underreported.
	\end{takeHomeMessage}

	\section{Introduction}



	Keratoconus is a progressive corneal condition characterized by irregular refractive properties that reduce visual acuity. Keratoconus usually arises in adolescence, is bilateral, and has an estimated incidence of 1:2,000 (Kennedy et al., 1986). Treatment is aimed at improving vision, principally using (rigid) gas permeable contact lenses. With progression of the disease, non-correctable refractive abnormalities and/or corneal scars arise. For these advanced stages of keratoconus, and in contact lens intolerance, a corneal transplant is a viable treatment modality.



	The first corneal transplant for keratoconus was conducted in 1936 by Ramon Castroviejo in New York's Columbia Presbyterian Medical Centre. Ever since, corneal grafting has been subject to many technical developments. With the advent of refractive surgery in the 1990s (Buratto et al., 1992), equipment was developed to split a cornea in horizontal lamellae. This made partial thickness grafting possible, tailoring grafts according to the nature and location of corneal pathology. For keratoconus, only the affected anterior part of the cornea needs to be transplanted: the Deep Anterior Lamellar Keratoplasty (DALK). This technique circumvents transplanting the host endothelium, leading to a lower rate of graft rejection (Ang et al., 2008; Fontana et al., 2007). The main drawback of DALK is the risk of accidental corneal perforation during surgery, as the fragile Descemet membrane can easily rupture, potentially requiring conversion to a full-thickness graft. To prevent inadvertent perforation, several techniques are described to split the stroma from the posterior lying Descemet membrane and corneal endothelium, using either fluid (Amayem \& Anwar, 2000), viscoelastic devices (Melles et al., 2000), or air (Anwar \& Teichmann, 2002). Failure and perforation are described in 20\% of cases, though, leading to poor surgical predictability (Cheng et al., 2011). The DALK techniques require a long learning curve, and the reported perforation rates might be an underestimate (Kasbekar et al., 2014).



	To circumvent this problem, a technique was developed in which, in addition to a mechanized anterior lamellar keratoplasty, a Partial Endothelial Trepanation (PET) is performed. This technique was first performed by Prof. Massimo Busin, in the Villa Serena Hospital, Forli, Italy (Busin et al., 2012).\textsuperscript{ }The endothelium and Descemet membrane are paracentrally and circularly loosened, but a certain proportion is left intact. This ‘island' is able to mold to the healthy donor curvature. By doing this, the surgeon can retain a safer graft thickness margin leading to a lowered number of preoperative perforations. The introduction of PET is believed to make corneal grafting safer and more predictable.



	Here, we study the outcomes of this new technique in a randomized clinical trial, with the DALK technique as comparator technique: the Partial Endothelial Trepanation in addition to anterior lamellar keratoplasty in keratoconus patients, or the PENTACON, trial. Our primary goal was to assess the surgical safety of both techniques. Secondly, we assessed secondary treatment outcomes in terms of visual acuity, manifest refraction, corneal astigmatism, and endothelial cell density at 1 year post-treatment.



	\section{Materials and methods}



	\subsection{Study design}



	This multicenter randomized clinical trial was conducted from March 2011 until June 2015. Study participation was granted by the University Medical Center Utrecht, University Medical Center Nijmegen St. Radboud, Rotterdam Eye Hospital, Amphia Ziekenhuis Breda, and Westfriesgasthuis Hoorn. The conduction of this study was approved by the Ethics Review Board of all participating centers and was performed in accordance with local laws, the European guidelines of Good Clinical Practice, and the tenets of the Declaration of Helsinki. The study was registered at ISCRTN (no° ISRCTN39068025) and clinicaltrials.gov (no° 30756.041.10).



	Patients eligible for study participation were randomized using a permutated block size and were stratified for the presence of atopic diseases. The web-based randomization tool was hosted by our institutions biostatistical department (UMCU Julius Center).



	\subsection{Patient selection}



	Inclusion criteria included: age ≥ 18 years, clinical and topographic evidence of keratoconus (KISA\% index >100\%), and reduced best corrected visual acuity from corneal scarring or contact lens intolerance (Rabinowitz \& Rasheed, 1999; Tang et al., 2005).



	Exclusion criteria included: prior corneal or refractive surgery, corneal thickness <300 µm, corneal steepness preventing suction ring placement, corneal endothelial disease on specular microscopy, or any other significant ocular pathology that could reduce visual acuity beyond keratoconus itself.



	\subsection{Primary and secondary outcomes}



	The event of a surgical complication necessitating conversion to a full-thickness corneal graft was considered as primary outcome parameter. Hereto, all surgical and post-operative adverse events and protocol deviations were recorded in study specific case report forms.



	Secondary study objectives focused on the effectiveness of both techniques at 6 months and 1 year follow-up: uncorrected and best spectacle corrected visual acuity (UCVA/BCVA), manifest refraction, corneal astigmatism, contact lens use (soft/rigid/scleral) or spectacle use, graft rejection and failure rate, corneal endothelial function, and correlation of outcomes with atopic constitution. Graft rejection was assessed by slit lamp examination (Folks, 2005). Graft failure is related to endothelial cell dysfunction and graded concordantly as corneal endothelial disease. Atopical constitution is defined by the presence of allergic conjunctivitis at time of screening or confirmation of atopy (e.g. allergy, asthma, eczema, laboratory testing with elevated IgE levels) by patient history. All patients were routinely screened for total IgE serum levels.



	\subsection{Clinical protocol and used equipment}



	Examinations were scheduled at baseline, at 6, and at 12 months follow up. The ophthalmic examination consisted of a brief history, use of (ocular) medication, use of visual aids (spectacles/contact lenses), and the occurrence of adverse events. UCVA and BCVA were assessed using an EDTRS visual acuity chart. Manifest refraction was taken by an optometrist or ophthalmic assistant. Slitlamp examination focused on the presence of corneal pathology, corneal clarity, and suture related complications. Hereto, dedicated case report forms were employed. A dilated fundus exam assessed the incidence of cataract, glaucoma, or macular disease.



	Corneal topography and pachymetry were acquired using the Oculus Pentacam HR Type 70900 (Oculus Optikgeräte GmbH, Wetzlar, Germany). Endothelial cell counts were acquired with the Topcon Sp-3000p, Topcon Corporation, Tokyo, Japan. Intraocular pressure was measured using the Topcon CT-80. If unattainable, a Goldmann applanation tonometer was used.



	\subsection{Surgical technique and donor preparation}



	All donor corneas were supplied by the European Cornea Bank, Beverwijk, conform EEBA medical standards (European Eye Bank Association, 2008).



	Patients were randomly divided into two groups, which received different treatments. Group A received a Partial Endothelial Trepanation (PET) in addition to anterior lamellar keratoplasty Part of the described technique is published by Busin (Busin et al., 2012).\textsuperscript{ }We applied the technique following these instructions: The donor cornea is mounted on an artificial anterior chamber (ALTK, Moria S.A., Antony, France) with the epithelium up, and an anterior corneal lamella is cut with a 350μm microkeratome head and a hand-driven microkeratome (CBm, Moria S.A., Antony, France). Then the anterior corneal lamella of the recipient is prepared by applying the suction ring to the eye of the patient, and the intraocular pressure is increased to >65 mmHg. Balanced salt solution (ALCON, Fort Worth, Texas, USA) is instilled on the corneal surface and the same hand-driven microkeratome is advanced in the tract until the anterior lamella is completely severed from the underlying recipient stroma. At least 100μm residual tissue should be left in place. Thereafter a partial trepanation with a 6.5 mm disposable hand trephine is made into the remaining stroma. In this grove of the remaining tissue, including Descemet's membrane and endothelium, a cut is completed manually and oblique with a Thornton knife over 180-270°. This small ‘island' will stay in place. The diameter of the exposed stromal bed is measured with a caliper, and the diameter of the donor graft is chosen accordingly. Finally, the lamellar graft is sutured in place under tension of 16 interrupted 10-0 nylon sutures. After removal of the speculum, the eye is patched.



	For Group B we applied the conventional Deep Anterior Lamellar Keratoplasty (DALK) type Big Bubble technique according to Anwar and Teichmann (2002).

	\begin{table}[H]
		\begin{fullwidth}
			\caption{Baseline characteristics of both treatment groups}
			\begin{tabularx}{\linewidth}{@{} l l l l l l l l l l l l l l l @{}}
				\toprule
	
			 	  &   & PET & DALK & \emph{p*} \\
	
			 	\multicolumn{2}{l}{Gender (\% male)} & 86\% & 33\% & 0.06 \\
	
			 	\multicolumn{2}{l}{Age} & 36.4 ±10.8 & 40.5 ±14.2 & 0.56 \\
	
			 	\multicolumn{2}{l}{Atopy} & 57\% & 67\% & 0.97 \\
	
			 	\multicolumn{2}{l}{UCVA (logMAR)} & 1.76 ±0.21 & 1.36 ± 0.39 & 0.14 \\
	
			 	\multicolumn{2}{l}{BCVA (logMAR)} & 1.12 ±0.79 & 0.74 ±0.65 & 0.42 \\
	
			 	\multicolumn{2}{l}{Manifest refraction} &   &   &   \\
	
			 	Sphere (D) & -9.25 ±6.33 & -4.17 ±5.41 & 0.21 \\
	
			 	Cylinder (D) & -2.88 ±2.22 & -3.71 ±1.96 & 0.55 \\
	
			 	\multicolumn{2}{l}{Kflat (D)} & 58.65 ±6.25 & 58.45 ±6.69 & 0.96 \\
	
			 	\multicolumn{2}{l}{Ksteep (D)} & 66.45 ±10.12 & 63.23 ±7.00 & 0.54 \\
	
			 	\multicolumn{2}{l}{Kmax (D)} & 78.05 ±17.43 & 75.37 ±11.24 & 0.76 \\
	
			 	\multicolumn{2}{l}{Corneal astigmatism (D)} & 3.32 ±1.99 & 4.77 ±3.08 & 0.36 \\
	
				\bottomrule
				\multicolumn{5}{l}{\parbox[t]{0.9\textwidth}{PET: Partial Endothelial Trepanation. DALK: Deep Anterior Lamellar Keratoplasty. UCVA: uncorrected visual acuity. BCVA: best corrected visual acuity. D: diopter. *independent students t-test}}
			\end{tabularx}	
		\end{fullwidth}
	\end{table}


	\subsection{Statistical analysis and power analysis}



	Baseline measurements between the treatment groups were compared using an independent samples t-test. Fischer's exact test (two-tailed) was used to determine the relation between treatment and risk of conversion to a perforating keratoplasty. Decimal visual acuity was converted to the logarithm of the minimal angle of resolution (logMAR). Normality and homoscedasticity of the residuals were tested visually and in a Q-Q plot and scatterplot, respectively. A \emph{p}-value < .05 was considered statistically significant. Data are recorded as mean ± standard deviation. All tests were performed in SPSS version 22.0 for Windows.



	With an expected perforation risk reduction of 85\% (current DALK ratio = 20\%, 3\% perforation reported by Busin et al. (2012)), incorporating a sequential power calculation with a two-sided alpha 0.05 and beta 0.80, approximately 30 patients needed to be included in each treatment arm\textsuperscript{ }(Chow et al., 2003; D'Agostino et al., 1988).



	\section{Results}

	\subsection{Clinical characteristics}



	A total of 14 eyes from 14 patients were enrolled in this trial. Two external centers participated (Radboud UMC \emph{n }= 2, Amphia Ziekenhuis Breda \emph{n} = 1).One patient postponed his surgery after randomization and was excluded from analysis. Six DALK procedures and 7 PET procedures were therefore included. Two randomized cases (one DALK, one PET) developed a corneal hydrops while on the waiting list for surgery, and following the intention-to-treat analysis both were included. Mean age was 38.3 years (\emph{SD =} 12.1) and 61.5\% of the patients was male. At baseline, mean logMAR UCVA and BCVA were 1.59 (\emph{SD }=\emph{ }0.35) and 0.89 (\emph{SD }=\emph{ }0.69) respectively. Mean refractive astigmatism was 3.4 (\emph{SD }=\emph{ }2.0D), mean IOP 10 (\emph{SD }=\emph{ }2.1mmHg), and mean thinnest pachymetry 322 (\emph{SD }=\emph{ }66µm). Endothelial cell counts were only attainable in two cases (2110 and 2558 cells/mm\textsuperscript{2}). Topographic indices on average were a K\textsubscript{max} of 76.7 (\emph{SD }=\emph{ }14.1D), K\textsubscript{flat} 58.6 (\emph{SD }=\emph{ }6.17D), K\textsubscript{steep} 64.8 (\emph{SD }=\emph{ }8.5D), and an astigmatism of 4.0 (\emph{SD }=\emph{ }2.6D). Baseline characteristics did not differ significantly between both groups (see table 1).
	

	\subsection{Donor characteristics}



	All patients received their donors from the European Cornea Bank, Beverwijk, the Netherlands. Mean donor age was 61.3 years (\emph{SD }=\emph{ }9.8) and the average time between death and enucleation was 15½ hours. On average the difference between the age of the donor and patient was 21.4 years (\emph{SD }=\emph{ }12.2), range 4-42. Mean ECD was 2692 (\emph{SD }=\emph{ }170 cells/mm\textsuperscript{2}), range 2400-3000. Donor characteristics did not differ significantly between both groups (data not shown).

	
	
	\begin{originalPurpose}
		The PENTACON trial was designed to evaluate the surgical safety and clinical outcomes of two advanced corneal transplantation techniques—Big Bubble Deep Anterior Lamellar Keratoplasty (DALK) and Busin's Partial Endothelial Trepanation (PET) in patients with keratoconus. Keratoconus is a progressive corneal disease characterized by thinning and protrusion, leading to significant visual impairment. Traditional treatment approaches, including full-thickness corneal transplantation, carry a risk of graft rejection and complications, and the DALK technique is know for its steep learning curve and relatively high rate of intra- and post-operative complications. PET was conceptualized as a technically less demanding technique, potentially yielding comparable outcomes.
	
		Our primary goal was therefore to assess whether the PET technique could offer a safer alternative to the DALK procedure. Secondary objectives included evaluating the visual acuity outcomes, refractive stability, corneal astigmatism, and adverse events 1 year post-operatively in both treatment arms. By comparing these parameters, we aimed to determine if the PET technique could provide comparable or superior visual rehabilitation while enhancing surgical safety and predictability.

		In essence, the PENTACON trial was an endeavor to innovate and refine corneal transplantation techniques to improve patient outcomes, reduce surgical complications, and ultimately enhance the quality of life for individuals affected by keratoconus. Through rigorous multicenter collaboration and randomized controlled trial methodology, we aimed to contribute substantial evidence to guide clinical practice in corneal surgery.
	\end{originalPurpose}


	\subsection{Primary outcome}



	The primary study outcome was defined as the incidence of surgical adverse events necessitating conversion to a penetrating keratoplasty. Adverse events occurred in ten of thirteen surgeries. Five of thirteen surgeries were converted to perforating keratoplasties (DALK:PET 3:2, \emph{p} = 0.592, Fisher's exact test), including both cases with a previous corneal hydrops. Only two surgeries (both PET) had no complications. Notable protocol deviations included five full perforations, two microperforations (both DALK), three poor microkeratome cuts (all PET), three post-operative rebubblings (DALK:PET 1:2), and two pre-Descemetic preparations (both DALK).



	\subsection{Secondary outcomes}



	Secondary outcomes were assessed at 6 months and 12 months post-operatively. Overall, at 6 months mean logMAR UCVA and BCVA increased to 0.93 (\emph{SD }=\emph{ }0.27, \emph{p }= .02) and 0.48 (\emph{SD }=\emph{ }0.27, \emph{p} = .15) respectively. At 12 months this further increased to 0.52 (\emph{SD }=\emph{ }0.20, \emph{p }= .003) and 0.26 (\emph{SD }=\emph{ }0.36, \emph{p }= .03). The following parameters are only reported at the 12 months assessment since topographic data and manifest refraction were often not attainable at the 6 months' time point. Due to the low number of cases only overall outcomes were reported; a valid comparison between both techniques was not feasible. All corneas were clear (one or two out of six) at the final follow-up visit, and all sutures were removed. Two PET cases had some Descemet folds, however (Snellen BCVA 0.45 \& 0.55). One case (DALK) was suspected of an epithelial rejection and treated subsequently (Snellen BCVA 0.7). Two-thirds of the patients used scleral contact lenses after their surgery. Endothelial cell densities were too often unattainable or not recorded; only three viable measurements were recorded, data not shown. Mean refractive astigmatism was 3.8 (\emph{SD }=\emph{ }2.2D), with one case (DALK) of 8D astigmatism. On average the topographical indices were a K\textsubscript{max} of 53.0 (\emph{SD }=\emph{ }2.8D), K\textsubscript{flat} 42.4 (\emph{SD }= 5.1), K\textsubscript{steep }46.5 (\emph{SD }= 3.5D), and an astigmatism of 4.9 (\emph{SD }= 3.4D). No significant differences were observed between both treatment groups for any of the secondary outcomes parameters. No long term sequelae like suture related complications, graft failure, cataract, glaucoma, or ocular hypertension were noted during trial follow-up.



	\section{Discussion}



	In general, no solid conclusions can be drawn with regards to the primary outcome based on this underpowered clinical trial. Whether the partial trepanation technique proposed by Busin is superior to the regular DALK technique in terms of surgical safety is still open for debate. On average, UCVA and BCVA improved significantly after 12 months, and visual acuity improved in all eyes. Post-operative mean refractive and topographic astigmatism were in line with other studies (Cheng et al., 2011; Söğütlü Sari et al., 2012). No long term sequelae from corneal surgery were recorded, although two-thirds of the patients used (scleral) contact lenses at the 12 months follow-up. Some findings of this study however deserve to be discussed.



	Firstly, this trial was heavily underpowered. Consistent with Tolstoy's (1877/1997, p. 1) reference to (un)happy families, numerous disruptive events arose during the study. The most significant issues were:

	\begin{enumerate}


		\item challenges integrating the trial into routine clinical practice,



		\item
		the introduction of corneal crosslinking mid-study (Godefrooij et al., 2016), and



		\item a narrow surgical indication (mild cases do well with contact lenses, and severe cases often have post-hydrops scarring that makes them unsuitable).


	\end{enumerate}

	During the 4 years that the trial was open for participation only 14 eyes were included, and we considered it unrealistic that the pre-defined power of 60 inclusions could eventually be met. Finally, we lost scientific equipoise: We could no longer maintain the belief that both treatments were equally advisable for our patients. We considered the PET a safer alternative to the Big Bubble DALK, although this premise was not held since adverse events occurred in both groups alike. A learning curve effect might have interfered this finding. The combination of a low trial inclusion rate and serious doubts on study safety prompted the termination of this trial in June 2015. We regard it our ethical obligation towards the participants to report these trial results nonetheless (Edwards et al., 1997).



	Both treatments arms were confronted with a remarkably high rate of adverse events (AE), and these can be viewed from different perspectives. From a trial perspective, conversion to a perforating surgery was the most relevant AE. From an ethical/juridical perspective the AEs that require a re-operation, i.e., the detached Descemet membranes, could be considered the most severe. From a patient perspective, however, the AEs that negatively influence optimal visual acuity can be regarded the most burdensome, in other words, the Descemet folds that impair visual acuity on the long term. Apart from the intrinsic difficulties and long learning curve associated with lamellar surgery\textsuperscript{ }(Kasbekar et al., 2014), the degree of keratoconus in this study was very severe, with an average K\textsubscript{max} of 76.6D and an average pachymetry of 322µm. If these two mean values are considered a compound index of the staging of keratoconus severity, interesting comparisons can be made with other surgical studies (Anwar \& Teichmann, 2002; Busin et al., 2012; Ghanem et al., 2015), should the baseline characteristics be adequately reported. The increased availability and the clinical experience with scleral contact lenses in the Netherlands can be considered a contributing factor for this difference: With adequately fitted scleral lenses virtually all clear keratoconus corneas can achieve a good visual acuity (Visser et al., 2016).



	Another consideration is that the treatment protocol did not formally exclude scarred corneas. In clinical practice, however, lamellar surgery in these cases pertains an even higher risk of Descemet perforation/rupture, and lamellar surgery after a sustained hydrops is rarely successfully completed (Wisse et al., 2014). During the course of the trial, cases with a sustained corneal hydrops were not considered suitable for trial participation, mainly because performing a successful Descemet baring DALK becomes increasingly technically demanding. Secondly, these cases were considered unsuitable because the scarred residual stroma in a PET procedure might preclude optimal visual recovery. Apart from abovementioned alterations, the surgical and clinical protocol remained virtually unchanged. This could be considered a strength of this study, in the light of the difficult equilibrium between trial obligations and surgical innovation. Researchers have debated that the timeframe of a well-conducted trial spans many years (Beks et al., 2017); years in which the investigated technique can be adjusted and improved. What, then, is the value of a trial if it provides evidence based medicine for yesterday's procedures? The latter is of particular relevance in corneal surgery. Busin himself published an improved technique for keratoconus surgery which renders the previously reported PET technique obsolete (Busin et al., 2016).



	In conclusion, a significant increase of uncorrected and corrected visual acuity was recorded for the group as a whole 1 year after corneal transplantation surgery for keratoconus. The added value of the PET over the DALK technique in terms of surgical safety cannot be deducted from these data, nor could we assess differences in the secondary outcomes (e.g., visual acuity, endothelial cell loss). However, in either treatment arm the incidence of intra-operative adverse events was higher than expected.



	\section{Disclosure}
	None of the authors have any conflict of interest to disclose.

	\section{Funding}
	This research was funded by an unrestricted grant from the Dr. F.P. Fisscher Stichting Utrecht, The Netherlands, facilitated by Stichting Vrienden van het UMC Utrecht.



	\section{References }



	Amayem, A. F., \& Anwar, M. (2000). Fluid lamellar keratoplasty in keratoconus. \emph{Ophthalmology}, \emph{107}(1), 76-79. \url{http://doi.org/10.1016/S0161-6420(99)00002-0}



	Ang, L., Boruchoff, S., \& Azar, D. T. (2009). Penetrating keratoplasty. In D. M. Albert, J. W. Miller, D. T. Azar, \& B. A. Blodi (Eds.), \emph{Albert \& Jakobiec's} \emph{Principles and practice of ophthalmology }(3\textsuperscript{rd} ed., pp. 813-827). Elsevier.



	Anwar, M., \& Teichmann, K. D. (2002). Big-bubble technique to bare Descemet's membrane in anterior lamellar keratoplasty. \emph{Journal of Cataract \& Refractive Surgery, 28}(3), 398-403. \url{http://doi.org/10.1016/S0886-3350(01)01181-6}



	Beks, R. B., Houwert, R. M., \& Groenwold, R. H. H. (2017). Meerwaarde van observationeel onderzoek in chirurgie. \emph{Nederlands Tijdschrift voor Geneeskunde, 161}, Article D1493.



	Buratto, L., Ferrari, M., \& Rama, P. (1992). Excimer laser intrastromal keratomileusis. \emph{American Journal of Ophthalmology, 113}(3), 291-295. \url{http://doi.org/10.1016/s0002-9394(14)71581-8}



	Busin, M., Scorcia, V., Leon, P., \& Nahum, Y. (2016). Outcomes of air injection within 2 mm inside a deep trephination for deep anterior lamellar keratoplasty in eyes with keratoconus. \emph{American Journal of Ophthalmology, 164}, 6-13. \url{http://doi.org/10.1016/j.ajo.2015.12.033}



	Busin, M., Scorcia, V., Zambianchi, L., \& Ponzin, D. (2012). Outcomes from a modified microkeratome-assisted lamellar keratoplasty for keratoconus. \emph{Archives of Ophthalmology}, \emph{130}(6), 776-782. \url{http://doi.org/10.1001/archophthalmol.2011.1546}



	Cheng, Y. Y. Y., Visser, N., Schouten, J. S., Wijdh, R.-J., Pels, E., van Cleynenbreugel, H., Eggink, C. A., Zaal, M. J. W., Rijneveld, W. J., \& Nuijts, R. M. M. A. (2011). Endothelial cell loss and visual outcome of deep anterior lamellar keratoplasty versus penetrating keratoplasty: A randomized multicenter clinical trial. \emph{Ophthalmology, 118}(2), 302-309. \url{http://doi.org/10.1016/j.ophtha.2010.06.005}



	Chow, S-C., Wang, H., \& Shao, J. (Eds.). (2003). \emph{Sample Size Calculations in Clinical Research}. Chapman \& Hall/CRC.



	D'Agostino, R. B., Chase, W., Belanger, A. (1988). The appropriateness of some common procedures for testing the equality of two independent binomial populations. \emph{The American Statistician, 42}(3), 198-202. \url{http://doi.org/10.1080/00031305.1988.10475563}



	Edwards, S. J. L., Lilford, R. J., Braumholtz, D., \& Jackson, J. (1997). Why “underpowered” trials are not necessarily unethical. \emph{The Lancet, 350}(9080), 804-807. \url{http://doi.org/10.1016/S0140-6736(97)02290-3}



	European Eye Bank Association. (2008). Agreement on minimal standards (AMS). \url{https://www.eeba.eu/files/pdf/EEBA%20Minimum%20Medical%20Standards%20Revision%205%20Final.pdf}



	Folks, G. (2005). Diagnoses and management of corneal allograft rejection. In J. Krahmer, M. Mannis, \& E. Holland (Eds.), \emph{CORNEA }(2\textsuperscript{nd} ed., pp. 1284-1314). Mosby.



	Fontana, L., Parente, G., \& Tassinari, G. (2007). Clinical outcomes after deep anterior lamellar keratoplasty using the big-bubble technique in patients with keratoconus. \emph{American Journal of Ophthalmology, 143}(1), 117-124. \url{http://doi.org/10.1016/j.ajo.2006.09.025}



	Ghanem, R. C., Bogoni, A., \& Ghanem, V. C. (2015). Pachymetry-guided intrastromal air injection (“pachy-bubble”) for deep anterior lamellar keratoplasty: Results of the first 110 cases. \emph{Cornea}, \emph{34}(6), 625-631. \url{http://doi.org/10.1097/ICO.0000000000000413}



	Godefrooij, D. A., Gans, R., Imhof, S. M., \& Wisse, R. P. L. (2016). Nationwide reduction in the number of corneal transplantations for keratoconus following the implementation of crosslinking. \emph{Acta Ophthalmologica, 94}(7), 675-678. \url{http://doi.org/10.1111/aos.13095}



	Kasbekar, S. A., Jones, M. N. A., Ahmad, S., Larkin, D. F. P., \& Kaye, S. B. (2013). Corneal transplant surgery for keratoconus and the effect of surgeon experience on deep anterior lamellar keratoplasty outcomes. \emph{American Journal of Ophthalmology, 158}(6), 1239-1246. \url{http://doi.org/10.1016/j.ajo.2014.08.029}



	Kennedy, R. H., Bourne, W. M., Dyer, J. A. (1986). A 48-year clinical and epidemiologic study of keratoconus. \emph{American Journal of Ophtalmology, 101}(3), 267-273. \url{http://doi.org/10.1016/0002-9394(86)90817-2}



	Melles, G. R., Remeijer, L., Geerards, A. J., Beekhuis, W., \& Houdijn, M. D. (2000). A quick surgical technique for deep, anterior lamellar keratoplasty using visco-dissection. \emph{Cornea}, \emph{19}(4), 427-432. \url{http://doi.org/10.1097/00003226-200007000-00004}



	Rabinowitz, Y. S., \& Rasheed, K. (1999). KISA\% index: A quantitative videokeratography algorithm embodying minimal topographic criteria for diagnosing keratoconus. \emph{Journal of Cataract \& Refractive Surgery, 25}(10), 1327-1335. \url{http://doi.org/10.1016/S0886-3350(99)00195-9}



	Söğütlü Sari, E., Kubaloğlu, A., Ünal, M., Piñero Llorens, D., Koytak, A., Ofluoglu, A. N., \& Özertürk, Y. (2012). Penetrating keratoplasty versus deep anterior lamellar keratoplasty: Comparison of optical and visual quality outcomes. \emph{British Journal of Ophthalmology, 96}(8), 1063-1067. \url{http://doi.org/10.1136/bjophthalmol-2011-301349}



	Tang, M., Shekhar, R., Miranda, D., \& Huang, D. (2005). Characteristics of keratoconus and pellucid marginal degeneration in mean curvature maps. \emph{American Journal of Ophthalmology, 140}(6), 993-1001. \url{http://doi.org/10.1016/j.ajo.2005.06.026}



	Tolstoy, L. (1997). \emph{Anna Karenina} (L. Maude \& A. Maude, Trans.). Wordsworth Editions Limited. (Original work published in 1877)



	Visser, E. S., Wisse, R. P. L., Soeters, N., Imhof, S. M., \& van der Lelij, A. (2016). Objective and subjective evaluation of the performance of medical contact lenses fitted using a contact lens selection algorithm. \emph{Contact Lens \& Anterior Eye, 39}(4), 298-306. \url{http://doi.org/10.1016/j.clae.2016.02.006}



	Wisse, R. P. L., van den Hoven, C. M. L., \& van der Lelij, A. (2014). Does lamellar surgery for keratoconus experience the popularity it deserves? \emph{Acta Ophthalmologica, 92}(5), 473-477. \url{http://doi.org/10.1111/aos.12281}


\end{document}