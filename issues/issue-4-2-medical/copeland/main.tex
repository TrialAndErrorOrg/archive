\documentclass[authordate, empirical]{jote-new-article}

\usepackage{caption}

\usepackage{tabularx}

\usepackage{graphicx}

\usepackage{hyperref}

\usepackage[backend=biber,style=apa]{biblatex}

\addbibresource{bibliography.bib}

\jotetitle{Ragworm Anecdote Commentary}
\keywordsabstract{serendipity, ragworms, scientific practice, discovery, novelty}
\runningauthor{Copeland}
\jname{Journal of Trial \& Error}
\jyear{2024}
\paperdoi{10.36850/r6}
\paperreceived{November 12, 2023}
\author[1]{\mbox{Samantha Copeland\orcid{0000-0002-6946-7165}}}
\affil[1]{Delft University of Technology}
\author[2]{\mbox{Stuart Firestein\orcid{0000-0003-1774-5853}}}
\affil[2]{Columbia University}
\corremail{\href{mailto:s.m.copeland@tudelft.nl}{s.m.copeland@tudelft.nl}}
\corraddress{Delft University of Technology}
\runningauthor{Copeland}
\paperaccepted{June 27, 2024}
\paperpublished{September 12, 2024}
\paperpublisheddate{2024-09-12}
\jwebsite{https://journal.trialanderror.org}

\begin{filecontents}{bibliography.bib}
	@article{bib0
}
\end{filecontents}

\begin{document}
\begin{frontmatter}
  \maketitle
  \begin{abstract}
    \printabstracttext
  \end{abstract}
\end{frontmatter}


	\section{Commentary: Ragworms Anecdotes, By Samantha Copeland \& Stuart Firestein}



	This story exemplifies a typical understanding of serendipity, as ‘looking for one thing and finding something else'. The author and his team were looking for a way to mimic nature in their experiment to produce the results they wanted and learned something unexpected and completely new about that natural environment; they looked for results to increase their knowledge of one species, and learned something remarkable about a different one they hadn't even considered. As a typical example, however, it raises a common question: How novel does the discovery have to be, to earn the title of serendipitous? That is, if we label all new directions in science that begin with an unexpected observation that is followed up, it seems that all of science is essentially accident-driven. Does this allow too much into that category, or does this say something interesting about scientific practice itself?



	Note that the story captures nicely the processual and non-linear fashion in which serendipity in science tends to work. When something doesn't work the way scientists will hope, the next step is to try something different—it may seem ad hoc, but this is the way experiments are created in novel or innovative situations. When designing a new experiment, there is rarely a perfect model to follow, but rather trial and error are a natural part of the process. If one doesn't know what the error is, or rather why the experiment hasn't worked, this process is observational: One must wait for what happens to see if the changes made have worked. Thus, a gap is opened where new and unexpected observations are likely to take place. That is, the mind of the experimenter experimenting with experiments is open to new observations, while at the same time still deep in the search for a particular thing.

	\begin{companion}
		van Belzen (2024) \vskip.5\smallwidth
		\emph{Ragworms Anecdotes - Shared by Jim van Belzen}\vskip.5\smallwidth
		\href{https://doi.org/10.36850/a1}{DOI: 10.36850/a1}
		\vskip-2\baselineskip
	\end{companion}

	Some question many cases of serendipity, suggesting that if there is too much already ‘in the mind' or ‘in the search' guiding the finder toward her finding, then it cannot count as ‘true' serendipity. Real surprise is a necessary element—we could ask, were these researchers as surprised as they say? Did they know nothing about ragworms? Were the questions they answered far enough out of reach of their imagination to be truly by chance? Such questions arise in relation to creativity as well, when we consider the threshold between reform and radicalization, or innovation and improvement; like the increasing number of stalks in a growing haystack, there is no specific number at which the change from pile to heap can be said to occur. Should it matter that these unexpected discoveries occur in the same field, among known factors, and to those educated in the relevant fields to know what the observations might mean?



	What does matter, then, or rather what is the difference that \emph{makes} a difference in our categorizations of these kinds of discovery? First of all, noticing is key. In this case, the noticing was in itself intentional, as a researcher sat purposely beside the experiment to witness what was happening. They were already paying attention, which clouds the element of surprise: Even though the ragworms themselves were entirely unexpected, something unexpected was in fact just what they were looking for. This mix of purpose and surprise is not unusual—often an unexpected observation triggers its study, and this moves science from the realm of serendipity into the realm of intentional discovery.



	In 2010, psychopharmacologist Alan Baumeister and colleagues called for standardization of the use of the word serendipity when discussing pharmacological research (Baumeister et al., 2010). For them, this movement from surprise observation to study marked the end of the serendipity process; further discoveries made down the line are not also serendipitous, but rather the results of methodology. Philosopher of science Paul Thagard (2002) likewise treats serendipity as a trigger to discovery, but discoveries themselves are the product of search and method. However, this paints a rather linear picture of science as moving only from observation to confirmation and thereby discovery. In contrast, stories such as this one about the ragworms show that during and post-discovery, research can yet branch out further on the back of new, also surprising, observations, in an iterative process. This brings us to the question: At what point does a discovery process begin? While the original observation of the ragworms was unexpected, it might still be captured in a traditional picture of science, but the follow-up anecdote suggests that another new direction will be taken as a result of the investigation of the original surprising observation, with the researchers moving from surprise to study and then encountering surprise again. Not only did the original surprising observation provide a resolution to their mystery, it also opened up a new avenue for study where another surprising observation could be made.

\newpage

	\section{Reply by Jim van Belzen}



	I do think that the failure experience described, and our response by course correcting our research, is indeed exemplary of common scientific practice. New knowledge and insights are emerging from the iterative practice of the scientific method, which is often a linear chain of logical follow-up experiments and investigations. However, sometimes it changes course or even branches, due to an unexpected observation or failure of setting up the experiment successfully. It could be failure to validate a scientific hypothesis, to see what the aim of the experiment was anyway, or failure as the result of human errors. The overall structure arising seems to resemble a sort of evolutionarily branching process. Failure and errors are a necessary ingredient of the 'search engine' to explore natural regularities.

	% \begin{invitation}
	% 	Want to continue the discussion? Comment on PubPeer!\vskip.5\smallwidth
	% 	[LINK]\vskip-2\baselineskip
	% \end{invitation}

	In reaction to the questions raised, some clarifications on our state-of-mind before we obtained the unexpected results:



	First, we did not have a vague idea, but we were very (maybe too much) focused on the physical effects of waves (i.e., hydrodynamic forcing) and at that point didn't consider biological factors to be a relevant factor. However, due to our ecological background we were aware of ecological studies in which biological factors had a top-down effect, e.g. snails and crabs in the tidal marshes of the USA. Yet in those marshes, tidal influences were expected to be much less relevant. The Westerschelde has a relatively large tidal range and is fairly wave-exposed, and previous work done in our research department on this location suggested that waves and consequent sediment dynamics were the dominant factors, thereby priming our mindset to be focused on physical rather than biological explanations for establishment bottlenecks.



	Second, at the time we set up the initial experiment we didn't know anything about the potential role of ragworms. But as the result of our experience, I did some further literature research and found a couple of earlier studies in which birds (Esselink et al., 1997; Mitchell \& Perrow, 1998) and ragworms (Hughes \& Paramor, 2004) were identified as the culprits of seedling establishment failure that supported our observations to be more generally applicable. Thus, although we considered birds as a potential culprit for the disappearance of seedlings from the pots in the mesocosms outside, the role of ragworms was still not on our minds. Once we identified ragworms as the cause of the seedling disappearance, this newfound literature helped us to go into more depth setting up experiments to develop mechanistic understanding into this interaction only after the fact.



	Lastly, the whole experience did help me in later research to be more open-minded about other possibilities than the ones we aimed to understand. More than once this open-minded approach did help to consider unusual suspects and unlikely processes to be considered anyway. For example, it helped us discover that oysters can be found much higher in the intertidal, i.e., in the lower part of tidal marshes, than usually considered possible (Fivash et al., 2021).


	






	\section{References}


	Baumeister, A. A., Hawkins, M. F., López-Muñoz, F. (2010). Toward standardized usage of the word serendipity in the historiography of psychopharmacology. \emph{Journal of the History of the Neurosciences}, \emph{19}(3), 253-270. http://doi.org/10.1080/09647040903188205


	Esselink, P., Helder, G. J., Aerts, B. A., \& Gerdes, K. (1997). The impact of grubbing by Greylag Geese (\emph{Anser anser}) on the vegetation dynamics of a tidal marsh. \emph{Aquatic Botany}, \emph{55}(4), 261-279. https://doi.org/10.1016/S0304-3770(96)01076-5



	Fivash, G. S., Stüben, D., Bachmann, M., Walles, B., van Belzen, J., Didderen, K., Temmink, R. J., Lengkeek, W., van der Heide, T., \& Bouma, T. J. (2021). Can we enhance ecosystem-based coastal defense by connecting oysters to marsh edges? Analyzing the limits of oyster reef establishment. \emph{Ecological Engineering}, \emph{165}, Article 106221. https://doi.org/10.1016/j.ecoleng.2021.106221



	Hughes, R. G., \& Paramor, O. A. L. (2004). On the loss of saltmarshes in south-east England and methods for their restoration. \emph{Journal of Applied Ecology}, \emph{41}(3), 440-448. http://doi.org/10.1111/j.0021-8901.2004.00915.x



	Mitchell, S. F., \& Perrow, M. R. (1998). Interactions between grazing birds and macrophytes. In E. Jeppesen, M. Søndergaard, M. Søndergaard, \& K. Christoffersen (Eds.), \emph{The structuring role of submerged macrophytes in lakes} (pp. 175-196). Springer New York. http://doi.org/10.1007/978-1-4612-0695-8\_9



	Scaps, P. (2002). A review of the biology, ecology and potential use of the common ragworm \emph{Hediste diversicolor }(O.F. Müller) (Annelida: Polychaeta). \emph{Hydrobiologia}, \emph{470}, 203-218. https://doi.org/10.1023/A:1015681605656


	Thagard, P. (2002). Curing cancer? Patrick Lee's path to the reovirus treatment. \emph{International Studies in the Philosophy of Science}, \emph{16}(1), 79-93. https://doi.org/10.1080/02698590120118846













\end{document}