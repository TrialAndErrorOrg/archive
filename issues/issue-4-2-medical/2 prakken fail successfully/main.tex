\documentclass[authordate, editorial, issue]{jote-new-article}

\usepackage{caption}

\usepackage{tabularx}

\usepackage{graphicx}

\usepackage{hyperref}

\usepackage[backend=biber,style=apa]{biblatex}

\addbibresource{bibliography.bib}

\jotetitle{How to fail successfully}
\keywordsabstract{translational medicine, immune tolerance, education innovation, translational scientist}
\runningauthor{Prakken}
\jname{Journal of Trial \& Error}
\jyear{2025}
\paperdoi{10.36850/b7f0-4fb1}
\paperreceived{October 16, 2024}
\author[1]{\mbox{Berent Prakken\orcid{0000-0001-8488-4816}}}
\affil[1]{UMC Utrecht, Utrecht, the Netherlands}
\corremail{\href{mailto:bprakken@umcutrecht.nl}{bprakken@umcutrecht.nl}}
\corraddress{UMC Utrecht}
\runningauthor{Prakken}
\paperaccepted{March 3, 2025}
\paperpublished{April 1, 2025}
\paperpublisheddate{2025-04-01}
\jwebsite{https://journal.trialanderror.org}


\setcounter{page}{7}
\jissue{2}
\jvolume{4}
\jpages{7--16}
\specialissue{Scientific Failure and Uncertainty in the Health Domain}
\articletype{Special Issue - Editorial}

\begin{document}
\begin{frontmatter}
  \maketitle
  \begin{abstract}
    \printabstracttext
  \end{abstract}
\end{frontmatter}






	\begin{quote}“Boys we were - but good boys. If I may say so myself. We're much smarter now, so smart it's pathetic.” (Nescio, 2012)\end{quote}


	\begin{tikzpicture}[remember picture, overlay]
		\node[align=left, anchor=north west] 
		at ([xshift=4.7cm, yshift=-0.3cm]current page.north west) 
		{
			\begin{minipage}{15cm}
			\raggedright
			\textbf{Correction notice} \\
			Incorrect Special Issue Labeling (Article erroneously excluded): This article was previously not labeled as part of a special issue due to an error. This has now been corrected.\vspace{2pt}
			{\color{joteorange}\rule{\linewidth}{1pt}}
			\end{minipage}
		};
	\end{tikzpicture}




	\lettrine{T}{his} quote from the Dutch writer Nescio (1933/2012, p.35) perfectly captured how I felt that morning in Paris in 2005. Salvo Albani and I were sitting in a coffee bar — the kind that actually sells coffee. Outside, the rain was falling steadily, while inside, it was warm and humid. The coffee tasted bitter.







	I had arrived in Paris the day before, traveling by train from Utrecht, in high spirits. Salvo, who had flown in from San Diego, was equally upbeat. He was on the verge of signing an agreement with a venture capital firm that evening, a deal that would secure funding for his biotech spin-off. This funding was crucial — it would finance the next, pivotal clinical trials that we hoped would definitively prove the effectiveness of peptide immunotherapy for arthritis. Less than 24 hours later, everything had changed. It was so close — so terribly close. And then, it was out of reach again.



	\section{First encounter}



	Salvo and I had first met eleven years earlier, in 1994, in Pavia. The occasion was the European Pediatric Rheumatology Congress, where I presented two abstracts to an international audience for the first time. I had meticulously prepared for the meeting, anticipating the kinds of questions the sharpest minds in the audience might pose. I even prepared two additional slides as a reserve to address the most challenging questions I could imagine. Twice, a tall, well-dressed and good-looking young Italian man with a distinctive accent approached the microphone and asked exactly the two questions I had prepared for. It was Salvo. After my second presentation, he approached me, introduced himself, and said: “Since you had slides ready to answer my questions, everyone in the audience now thinks we're secretly working together. So why don't we start working together now?” That marked the beginning of a close collaboration and a friendship that has lasted for nearly 30 years. During these years we did not reach our target (a brilliant breakthrough treatment for autoimmunity), but while failing to do so we accidentally achieved something with perhaps more long-term impact: the establishment of an institute for translational scientists — Eureka!







	On the surface, we seemed as different as our birthplaces: Amsterdam (mine) and Siracusa (Salvo's). Salvo was more outspoken and bolder; I tended to be quieter and more cautious. Yet, we soon discovered we had far more in common than what set us apart. We were both pediatric immunologists and deeply passionate about translational science, that is, science with true impact. After our first meeting, our professional relationship quickly deepened, ultimately even evolving into a close friendship. We shared our struggles with journal submissions, horrific reviewers, and the challenges of managing a growing lab. For me, the most important thing was probably the reassurance of knowing that someone else, albeit on the other side of the world, was thinking along the same lines.



	Looking back, I can see clearly that the success of our personal partnership was built on three key elements: mutual respect, trust, and a shared passion to make a difference.







	\section{The journey starts}



	At the time of our first meeting I was doing my PhD at the lab of professor Willem van Eden, at the time world-famous for his discovery of a link between heat shock proteins and arthritis (van Eden et al., 1988). Salvo already had his own research group at the University of California San Diego and had recently made a pivotal discovery on the role of molecular mimicry in arthritis that also involved a peptide derived from a heat shock protein (Albani et al., 1995). Following our first meeting in Pavia, things developed rapidly. The more we talked the more we realized in how many ways we were alike. After finishing my PhD in Utrecht I moved my family to San Diego and joined Salvo's group for the next two years. By the time we met in Paris in 2005, Salvo was working at the world-renowned Burnham Institute for Molecular Medicine in San Diego, while I had returned to the Netherlands and built my own lab at the University Medical Center Utrecht. Our two labs collaborated closely: We shared data, patient samples, antibodies, and even original ideas. The researchers in our two groups also felt the connection, which was further improved with the exchange of personnel.







	We were relatively young -- in our early forties, by now typical early-mid career professionals. The same career phase during which today, unfortunately, many young scientists often decide to leave the life sciences field and seek a career elsewhere. Salvo and I never felt the urge to leave science. Looking back, I see two reasons for our resilience. Firstly we were in this together and, secondly, we saw the large unmet need in the patients we were treating every day. Quitting was simply not an option.







	\subsection{The unmet need}



	Salvo and I were both pediatric immunologists, treating children with Juvenile Idiopathic Arthritis (JIA) in our clinics. In 1994, when Salvo and I met, the only effective treatment for JIA was Methotrexate (MTX), a drug developed decades earlier primarily for other conditions, such as malignancies (Calasan \& Wulffraat, 2014).



	MTX was and in many ways still is a mysterious drug. It was initially proposed for the treatment of autoimmune diseases because of its immunosuppressive effects. However, based on the dosage used, it was highly unlikely that it had any significant immunosuppressive impact. Indeed, over the years, no clear immunosuppressive effect has been confirmed, either \emph{in vitro} or \emph{in vivo}. While MTX was proven effective, it had clear limitations. The most prominent was the severe nausea it induced in the majority of patients (Calasan et al., 2013). This nausea could become so intense that children would feel sick merely at the thought of taking MTX, upon hearing the word "MTX," or even when seeing the same yellow color that the pills have. Another major issue was that, although MTX suppressed the symptoms, in many cases it did not provide a lasting solution: a disease- and medication-free remission. In other words, it did not cure the disease.







	\subsection{Time travel}



	Until now, treatments for autoimmune diseases have primarily focused on suppressing symptoms. Over the years, we have made tremendous strides in symptom management, often with life-saving outcomes. Consider the discovery and use of insulin for Type 1 Diabetes. This was certainly a groundbreaking treatment (Lewis \& Brubaker, 2021), but today, almost 100 years after Sir Frederick Banting's discovery of insulin at the University of Toronto, diabetes patients still mostly rely on insulin to manage their symptoms. We felt the current approach was taking us in the wrong direction. What we envisioned was something radically different: a disease- and medication-free remission—a true cure. In essence, we wanted to travel back in time to the period before the patient became ill. At the time, this was science fiction, and sadly, it largely remains so today.







	\subsection{Under fire}



	Our vision was to reprogram the gatekeepers of inflammation—CD4+ T cells—into T cells capable of modulating inflammation. It would make sense that such cells existed simply based on the principle: What goes up, must come down. And we believed, based on data mostly obtained in animal models, that this should be possible in a so-called antigen specific manner (Prakken et al., 2002). By using antigen-specific immune modulation we thought we could bypass systemic side effects and specifically control the immune response (Albani et al., 2011; Albani \& Prakken, 2006), an approach Salvo likened to a dimmer. However, not everyone agreed with us. And that was a problem.







	Our critics, mostly anonymous reviewers with probably far more seniority than we had at that time, were quick to point out why we all got it wrong. First, we began our work on T cells with immune regulatory capacity while just a few years earlier, another type of “suppressor” T cells (CD8+ T cells) had fallen out of favor in the scientific community after a brief period of popularity. Although we were targeting a different subset of T cells (CD4+ T cells) under a completely different hypothesis and framework, the mere association with "suppressor" T cells was enough to reject our work. At that time, having the term "suppressor T cells" in the title of a paper almost guaranteed rejection. But this was only the beginning of the challenges we faced.







	Second, we were attempting to modulate T cells using mucosal tolerance induction via nasal application of peptides. The problem was that about a decade earlier, other researchers had introduced oral tolerance induction for autoimmune diseases, creating a sensation in the field (Trentham et al., 1993; Weiner et al., 1994). This led to a surge of excitement and numerous papers on mucosal tolerance, particularly in autoimmune diseases. This enthusiasm culminated prematurely in clinical trials using oral myelin basic protein for MS and oral collagen for Rheumatoid Arthritis, both of which unfortunately failed. Despite significant differences between this and our approach, the failure of these trials led to widespread dismissal of the field by the scientific community.







	Third, we applied peptides derived from heat shock proteins (HSPs). HSPs are evolutionarily conserved proteins that are also immunodominant. Since HSPs are upregulated during cell stress, they were considered potential candidates for inducing antigenic mimicry, possibly leading to autoimmunity. However, attempts to use recombinant proteins for immune modulation had shown that some of the remarkable positive effects observed in experimental models were due, at least in part, to contamination with bacterial compounds. Although we used peptides instead of proteins and thus avoided this issue, the reputation of HSPs was already tarnished—they also had fallen out of fashion.







	So, we were using — and even combining — three different methods, each of which was unpopular, to say the least. Some of our mentors advised us to pursue a less contentious direction, but we were too convinced of our approach to follow their advice. Perhaps it was naïveté or even arrogance, but we believed we had strong intellectual arguments for choosing this path. Our reviewers and competitors, however, were far less convinced of the soundness of our choices, and they were not shy about making their opinions clear. One review I received, which I've meticulously saved over the years, ended with the line: \emph{"}Altogether, this can be regarded as an interesting yet completely futile experimental exercise\emph{.}"\emph{ }







	Why didn't we do the smart thing and abandon our risky approach? In a way, instead of discouraging us, these negative reviews almost had the opposite effect: Since we saw little intellectual merit in the often-harsh critiques, our confidence only grew. And there was another factor: We were not alone. In hindsight, I realize that we were practicing team science before it became a recognized and fashionable concept — which, fortunately, it is becoming today.



	\subsection{More poor timing}



	If you think our timing was poor, that wasn't the half of it. We embarked on this work at a time when MTX was still the only proven drug for severe cases of autoimmune diseases. We were unaware that a new class of drugs was about to revolutionize the field of rheumatology: biological agents that block cytokine pathways. Beginning with anti-TNF-alpha therapy, these drugs had a profound impact, demonstrated first in Rheumatoid Arthritis, then in Juvenile Idiopathic Arthritis, and in many other autoimmune diseases. Their efficacy in suppressing symptoms was impressive. However, in our view, this did not diminish the need for the specific tolerance-inducing treatments we had in mind. These cytokine and pathway-blocking therapies were non-specific and therefore carried the potential for side effects. Although short-term side effects seemed mild, early treatments for example revealed the re-emergence of latent tuberculosis. Moreover, these therapies did not cure the disease: Symptoms almost invariably returned once treatment was stopped. This essentially meant that patients would require lifelong treatment, thereby increasing the risks of long-term side effects. Blocking a cytokine pathway for a short period may be relatively safe, but doing so for years, or even decades, could significantly increase risks, such as vulnerability to microbial agents or disruption of intrinsic immune homeostasis. We were convinced that there remained a critical need to develop alternative therapies to achieve our ultimate goal of "time travel”: restoring the natural immune homeostasis that existed before the onset of autoimmune disease.







	\subsection{The lab in which the sun never sets}



	The playing field had changed, and the bar was now set even higher than before. But instead of being discouraged, we continued on the same track working nonstop to explore new ways to induce immune tolerance. Our two labs collaborated closely: We shared data, patient samples, antibodies, and even a sense of camaraderie. The two groups were very complementary and despite the cultural and organizational differences, we created a strong team with unifying traditions and habits and a thriving intrinsic motivation -- now arising not only from just us two but also (and most importantly) from other lab group members. We referred to our partnership as "the lab in which the sun never sets", because with the time difference, someone was always conducting an experiment somewhere. Both lab entrances were adorned with a name plate reading “Iacopo Institute for Translational Medicine”; named for Salvo's giant friendly dog who would always accompany us on our walks in San Diego. At that time translational medicine was not yet a buzz word, which in subsequent years it has become.



	We identified suitable peptide epitopes and explored different ways to administer those peptides in various experimental models, including combinations with stem cell therapy and blockade of the TNF-alpha pathway (Delemarre et al., 2011; Delemarre et al., 2014; Kamphuis et al., 2005; Kamphuis et al., 2006; Zonneveld-Huijssoon et al., 2012). Looking back, I—naively—placed far too much trust in the significance of animal models. I was captivated by the simplicity and endless possibilities of these models. It took time even for me, a clinician at heart, to fully appreciate the vast gap between model systems and real-life clinical settings in actual patients. Looking back this may be one of my biggest failures.







	My lab also studied specific immune responses in patients during different stages of disease, looking for patterns that may give us clues about the natural immune regulatory response. These studies benefited from the excellent clinical organization in Utrecht set up by Nico Wulffraat and colleagues, which allowed us to do precise testing in well-defined cohorts of patients. To enhance the immune studies in patients we also developed new techniques, such as the multiplex immune assay for use in human samples. For this we worked together with Vicki Seyfert as a core facility of the Immune Tolerance Network of NIH (de Jager et al., 2005). Salvo on the other hand did more basic immunology studies and took the most crucial step towards application in patients. He pulled off something truly amazing: a successful Phase I/II clinical trial with dnaJP1, the peptide developed in his lab (Koffeman et al., 2009; Prakken et al., 2004). At the time, his study design was quite unique; not just looking for side effects but also using extensive immunological testing of patient samples for indications of successful tolerance induction \emph{in vitro} in patients using surrogate parameters in peripheral blood samples. It turned out that this therapy was not only safe, but there also seemed to be a stunning correlation between the observed immune tolerance induction in patient cells and clinical outcomes (Prakken et al., 2004). The resulting publications led to renewed attention to T-cell specific immune therapy. Salvo also did something else I did not even consider: He launched a company to secure the funding and organization needed for a large multicenter placebo-controlled study in patients with RA. I admired (and still admire) Salvo for his boldness and his persistence in making this happen. I never seriously considered this option — I probably am too much of an academic, and certainly too cautious to take this step.



	Besides, I cherished my academic freedom and loved working with the technicians, PhD and Master's students, and postdocs in my growing lab. I now realize that I had this luxury of focusing on academic lab work because of Salvo's incessant search for funding and strong connections with potential investors.







	\subsection{Back to Paris}



	Salvo was in Paris to do exactly that, talk with investors. I joined him as I usually would when he was close to Utrecht. It allowed us to talk freely and coordinate our plans. We often thought along similar lines and moved quickly, with the risk that we ended up doing exactly the same thing. Thus, it was important we talked regularly. In these years long before COVID, Zoom was unavailable and Skype barely operational, so apart from weekly calls on regular phone lines, we used every opportunity to meet. Over the years we would meet in Rome, Chicago, Utrecht, Tucson, London, Lyon, New York, Berlin, Singapore, Genoa and many more places. This time, as Salvo was in Paris, I took the train to meet him and this was why we were now sitting across from each other in this dull cafeteria the morning after his meeting with prospective investors. The next studies, which would provide final proof of his approach, could not be achieved as the investors unexpectedly pulled out, at least for now. We were now sitting, disappointed, in this dreary bar in gray rainy Paris.



	\subsection{A Eureka moment}



	Our meetings would often not only be about immunology and all our shared lab projects, but also venture into other subjects --our families, relationships (mine being significantly more monotonous than Salvo's), books, and life in general. These areas would, weirdly enough, never include politics. Primarily because we did not like the opportunistic and simplistic nature of politics and especially politicians. But we also realized that arguing about political issues would not get us anywhere; an advice by the way that I would like to share with anyone.



	I do not recall who brought up the topic -- as there often was strong synchronicity in our thinking, one of us probably started to vocalize what the other was thinking. But at some moment we found ourselves talking about what a difference it could have made if we knew ten years ago what we knew now. And from there it was a small step towards our Eureka moment: Why not let other, younger colleagues benefit from our past (and present) failures and thus prevent them from making the same mistake?. The atmosphere changed from gloom to excitement. It felt as if a new window was opened, blowing in fresh air. And above all, it opened a totally new field in which we frankly did not have much knowledge: How can translational scientists learn from our failures? How do people learn anyway? Often crazy ideas tend to stay at that very first stage — just an idea. Not for us, when we would collaborate. When an idea struck, giving up simply became unthinkable. From that moment on we kept at it, brainstorming, searching for people with similar mindsets and talking to them about our ideas. Most of the people we talked to considered our plans crazy and unrealistic but a small group of people was very enthusiastic and encouraged us to go on.



	The plan gradually evolved into an Institute for Translational Medicine focusing on training young translational scientists and building a network for translational research. Part of the plan was to make Siracusa (Sicily), the birth town of Salvo, the homebase of the new institute because of its many possibilities, and the benefit of a beautiful historical city with an impressive cultural and scientific past. And thus, precisely a year later, on a warm Summer day in 2006, we traveled to Siracusa to set up the first legal entity that would be the start of the institute. After signing the paperwork in a hot notary's office full of books, binders, and papers, we went for a swim in the Mediterranean Sea. I will never forget the feeling as we dived into the water — it felt as if we were going to change the world.







	\subsection{From Eureka moment to Eureka Institute}



	The problem was that we barely had more than a broad outline of an idea. Or maybe that was our strength: We did not foresee all the problems that could arise if we followed up on this idea. Being ignorant about the potential risks and difficulties helped us to simply go ahead and move, step by step, towards our goal.



	That Summer I spent 2 months in San Diego for a short sabbatical with Salvo and we continuously kept thinking and talking about it — sometimes we even forgot to talk about immunology. We slowly but steadily worked on a plan that with every meeting became more and more concrete.



	There was no way back anymore. A year later I found myself together with Salvo again in the same hot notary's office in Siracusa, to officially launch the institute that we now decided to call Eureka. And again one year later, in 2008, we were ready for a huge next step: We had secured some minor grant funding. It was just enough to invite a small group of opinion leaders in the field of translational medicine, who we hoped were likeminded, to join us for an inaugural meeting of this new institute in Siracusa. I was pretty nervous about this first meeting. What if they just did not get it? What if they thought that there was no problem at all? What if they understood the problem but did not like our solution?



	The meeting was held in a small room, almost without daylight, in the basement of a hotel in Siracusa. The group included Janet and David Hafler from Yale, Norm Rosenblum from the University of Toronto, Lucca Guidotti (then UCSD, now San Raffaela), Juan Carlos Lopez (Nature Medicine) and others; all experts with very different backgrounds but all committed to Translational Medicine. My worries were unnecessary: It was one of the most inspiring meetings I ever had. The team quickly found a joint mission: The institute should inspire, mentor, and educate translational researchers worldwide, bring them together in a network, and help to create impactful research. From there on the developments went quickly. Already at the same meeting a plan was drafted for an international certificate course to be held in Siracusa one year later.







	\subsection{Into reality }



	The first Eureka course was held in Siracusa in 2009 with young translational scientists from all over the world participating. The course followed a unique blend of knowledge of the translational medicine pathway offered by experts in the field and educational innovation. The latter was mainly provided by Janet Hafler, a prominent medical educator from Yale. Thanks to her input, the course structure was rock-solid and different from what both participants and faculty were used to: It was truly learner-centered and set in an environment that encouraged interaction, discussion, and self-reflection. In that setting the participants met with peers who came from different countries and institutions with different cultures, but who were struggling with similar issues. Just listening to each other's experiences and realizing that they were not alone impacted the participants. The presence of renowned international faculty members with abundant experience and knowledge on every aspect of the translational medicine pathway was the other factor. In a conventional model, faculty staff would be teaching the participants by giving impressive lectures about their own achievements. Their role at Eureka was different: They were in the course to listen to the participants, support, and advise them. And in their presentations, they were asked not to just simply speak about their successes but also about their own insecurities and yes, about their failures.



	\subsection{Life and career changing?}



	Already at this first try it became clear that we had started something special. Though still in its first raw version, and thus with inconsistencies and shortcomings, the course turned out to have a major impact on both the participants and the faculty. Something magical happened — already during the very first edition of the course participants were saying that the course was life and career changing. But what does “life and career changing” mean? Now, 15 years later I still do not completely understand this. Clearly there is something that Salvo many years later called “the two souls of Eureka”. One soul being the access to expertise and knowledge of the full translational pathway -- you cannot play a game if you do not know the rules of the game. And how better to learn these rules than from true experts in the field, the ones who had done it, who had gotten their hands dirty and understood it not just from theory but from practice? The other soul could be the deliberate exposure to complementary or 21\textsuperscript{st} century skills such as critical thinking, problem solving, communication, and creativity. And then, maybe a third soul is the hidden curriculum: a safe environment in which the participants can freely interact with the faculty in such a way that the classical division based on seniority and hierarchy becomes irrelevant and even non-existing. In peer mentoring sessions they learn from each other's problems which often bear remarkable similarities to their own issues and worries. It strengthens the feeling that you are not alone in the so-called Valley of Death between bench and bedside. Last and not least it helped to bring people back to the reason why they once started to do research, namely to search for discoveries that really can make a difference for patients. With the entrance of role models such as Pat Furlong -- parent and advocate for patients with muscular dystrophy -- in one of the first courses, patients and societal impact literally came to stand in the middle of Eureka.



	Young translational scientists grow up in a hypercompetitive system that forces them to focus on grants and papers to be successful, while they lose their strong intrinsic motivation. At least this is how it feels for them. Paradoxically over the years we could see that when they let go of this extrinsic motivation and focus on the “why” of their research, classical success in the form of papers and grants followed also.



	It was a lucky coincidence that the growth of Eureka coincided with the international movement towards Open Science and the development of a new system for recognition and reward of scientists (Benedictus et al., 2016). For once our timing was perfect.







	\subsection{From movement to Institute}



	In the following years, Eureka unfolded with an almost logarithmic speed. However, at the start, Eureka was met with some reluctance and even skepticism. Arguments against it varied from ”we already have courses on translational medicine” to ”why invest so much in such a small number of participants”, and “we see no real problem here”. But Salvo and I were used to far worse criticism, so this did not slow us down. Besides, we were not alone anymore: We had clear support from the other pioneers with whom we started Eureka in that basement in Siracusa back in 2008. Gradually it became clear that the success of the first course was not a lucky shot. The long-term impact on the participants became more evident (Weggemans et al., 2018). Consequently, we progressively gained more support culminating in institutional support from strong academic medical partners, starting with UMC Utrecht and Duke NUS. Next, other partners stepped in to support Eureka's mission to improve translational medicine: patient organizations, government, granting agencies, and more university medical centers. Thus, steadily, Eureka changed from a bottom-up initiative from a group of motivated individual scientists into a vibrant international network of universities, translational scientists, and patients, and closely linked to society. Remarkably we kept hearing the phrase “life and career changing” in many different settings and courses. So often that we started to perform educational research to better understand this apparent impact. Margot Weggemans, one of the PhD students studying translational scientists and the impact of Eureka, showed in a follow up study that the certificate course had a profound and lasting effect on the participants. She found that more than 85\% of participants reported that Eureka changed the way they performed research. Most remarkably, this change persisted over time (Weggemans et al., 2018).



	\section{Epilogue}



	 This journey had an unexpected impact for me personally. It opened a whole new field of (educational) research— a type of research that turned out to be just as rich and complex as the biomedical research I was trained in. Janet Hafler, Olle ten Cate and Marieke van der Schaaf introduced me to the deeper layers of educational research, and I was mesmerized by what they told me. I started to read more and more about it, followed a post-master's course, and devoured books and articles on pedagogy psychology and the philosophy of education. Then, one day in 2017, to my own surprise I heard myself saying wholeheartedly yes when I was asked to apply for the position of vice dean for education at my home institution.







	\subsection{Failure}



	The first basic idea for setting up Eureka came from the simple idea that others would need to learn from our failures. Though the program and learning objectives of Eureka ultimately became much more elaborate and more extensive, the idea of sharing of and learning from failures stays deep in the DNA of Eureka. No quote better signifies the spirit of Eureka than the famous quote of Samuel Beckett: “Ever tried, ever failed? No matter. Try again. Fail again. Fail better” (Samuel Beckett quoted in Marshall, 2017). This may seem remarkable because in the highly competitive environment of life sciences, making failures can easily translate into being a failure. However, it is indeed crucial to admit to failures, to be unafraid to make them, and to learn from them and try again.







	\subsection{Lessons Learned}



	It was impossible to foresee the chain of events we set in motion that morning in Paris in 2005. With hindsight, it is always easier to discern patterns that you did not see in the midst of the storm—or that may not have even been there. What we did aligned with the spirit of the time and resonated with then-unseen trends that eventually led to Open Science.



	A few factors helped us along the way:





	\begin{enumerate}


		\item Conviction. We were so convinced — and so blinded by our idea — that we were simply too stubborn to give up.



		\item
		Link to Practice. We were trained as clinicians and translational scientists, and we had experienced the Valley of Death ourselves.



		\item Team. We sought out others to join us, and since that first meeting in 2008, we were no longer alone.



		\item Fun. We genuinely enjoyed the journey.



		\item Luck. No further explanation needed.


	\end{enumerate}





	In closing, I realize now that we had started climbing what David Brooks calls the “second mountain” (Brooks, 2019). It was no longer about us: It was about creating impact for others. Looking back, this is what gave us the resilience to keep going.























	P.S. I asked Salvo to proofread this article. He agreed with everything I wrote above but noted that I missed one important message: “The best is yet to come.”



	\section{References}



	Albani, S., Keystone, E. C., Nelson, J. L., Ollier, W. E., La Cava, A., Montemayor, A. C., Weber, D. A., Montecucco, C., Martini, A., \& Carson, D. A. (1995). Positive selection in autoimmunity: Abnormal immune responses to a bacterial dnaJ antigenic determinant in patients with early rheumatoid arthritis. \emph{Nature Medicine},\emph{ 1}(5), 448-452. \url{https://doi.org/10.1038/nm0595-448}



	Albani, S., Koffeman, E. C., \& Prakken, B. (2011). Induction of immune tolerance in the treatment of rheumatoid arthritis. \emph{Nature Reviews Rheumatology},\emph{ 7}(5), 272-281. \url{https://doi.org/10.1038/nrrheum.2011.36}



	Albani, S., \& Prakken, B. (2006). T cell epitope-specific immune therapy for rheumatic diseases. \emph{Arthritis \& Rheumatism},\emph{ 54}(1), 19-25. \url{https://doi.org/10.1002/art.21520}



	Benedictus, R., Miedema, F., \& Ferguson, M. W. (2016). Fewer numbers, better science. \emph{Nature},\emph{ 538}(7626), 453-455. \url{https://doi.org/10.1038/538453a}



	Brooks, D. (2019). \emph{The second mountain: The quest for a moral life}. Random House Publishing Group.



	Calasan, M. B., van den Bosch, O. F., Creemers, M. C., Custers, M., Heurkens, A. H., van Woerkom, J. M., \& Wulffraat, N. M. (2013). Prevalence of methotrexate intolerance in rheumatoid arthritis and psoriatic arthritis. \emph{Arthritis Research \& Therapy},\emph{ 15}(6), Article R217. \url{https://doi.org/10.1186/ar4413}



	Calasan, M. B., \& Wulffraat, N. M. (2014). Methotrexate in juvenile idiopathic arthritis: Towards tailor-made treatment. \emph{Expert Review of Clinical Immunology},\emph{ 10}(7), 843-854. \url{https://doi.org/10.1586/1744666X.2014.916617}



	de Jager, W., Prakken, B. J., Bijlsma, J. W., Kuis, W., \& Rijkers, G. T. (2005). Improved multiplex immunoassay performance in human plasma and synovial fluid following removal of interfering heterophilic antibodies. \emph{Journal of Immunology Methods},\emph{ 300}(1-2), 124-135. \url{https://doi.org/10.1016/j.jim.2005.03.009}



	Delemarre, E., Roord, S., Wulffraat, N., van Wijk, F., \& Prakken, B. (2011). Restoration of the immune balance by autologous bone marrow transplantation in juvenile idiopathic arthritis. \emph{Current Stem Cell Research \& Therapy},\emph{ 6}(1), 3-9. \url{https://doi.org/10.2174/157488811794480726}



	Delemarre, E. M., Roord, S. T., van den Broek, T., Zonneveld-Huijssoon, E., de Jager, W., Rozemuller, H., Martens, A. C., Broere, F., Wulffraat, N. M., Glant, T. T., Prakken, B. J., \& van Wijk, F. (2014). Brief report: Autologous stem cell transplantation restores immune tolerance in experimental arthritis by renewal and modulation of the Teff cell compartment. \emph{Arthritis \& Rheumatology},\emph{ 66}(2), 350-356. \url{https://doi.org/10.1002/art.38261}



	Kamphuis, S., Hrafnkelsdottir, K., Klein, M. R., de Jager, W., Haverkamp, M. H., van Bilsen, J. H., Albani, S., Kuis, W., Wauben, M. H., \& Prakken, B. J. (2006). Novel self-epitopes derived from aggrecan, fibrillin, and matrix metalloproteinase-3 drive distinct autoreactive T-cell responses in juvenile idiopathic arthritis and in health. \emph{Arthritis Research \& Therapy},\emph{ 8}, Article R178. \url{https://doi.org/10.1186/ar2088}



	Kamphuis, S., Kuis, W., de Jager, W., Teklenburg, G., Massa, M., Gordon, G., Boerhof, M., Rijkers, G. T., Uiterwaal, C. S., Otten, H. G., Sette, A., Albani, S., \& Prakken, B. J. (2005). Tolerogenic immune responses to novel T-cell epitopes from heat-shock protein 60 in juvenile idiopathic arthritis. \emph{The} \emph{Lancet},\emph{ 366}(9479), 50-56. \url{https://doi.org/10.1016/S0140-6736(05)66827-4}



	Koffeman, E. C., Genovese, M., Amox, D., Keogh, E., Santana, E., Matteson, E. L., Kavanaugh, A., Molitor, J. A., Schiff, M. H., Posever, J. O., Bathon, J. M., Kivitz, A. J., Samodal, R., Belardi, F., Dennehey, C., van den Broek, T., van Wijk, F., Zhang, X., Zieseniss, P., … Albani, S. (2009). Epitope-specific immunotherapy of rheumatoid arthritis: Clinical responsiveness occurs with immune deviation and relies on the expression of a cluster of molecules associated with T cell tolerance in a double-blind, placebo-controlled, pilot phase II trial. \emph{Arthritis \& Rheumatism},\emph{ 60}(11), 3207-3216. \url{https://doi.org/10.1002/art.24916}



	Lewis, G. F., \& Brubaker, P. L. (2021). The discovery of insulin revisited: Lessons for the modern era. \emph{The} \emph{Journal of Clinical Investigation},\emph{ 131}(1), Article e142239. \url{https://doi.org/10.1172/JCI142239}



	Nescio (2012). \emph{Amsterdam Stories}. (D. Searls, Trans.) New York Review Books. (Original work published 1933)



	Marshall, C. (2017, December). Samuel Beckett's mantra: Try again, fail again, fail better. \emph{Goethe Institute USA, Los Angeles.} \url{https://www.goethe.de/ins/us/en/sta/los/bib/feh/21891928.html}



	Prakken, B. J., Roord, S., van Kooten, P. J., Wagenaar, J. P., van Eden, W., Albani, S., \& Wauben, M. H. (2002). Inhibition of adjuvant-induced arthritis by interleukin-10-driven regulatory cells induced via nasal administration of a peptide analog of an arthritis-related heat-shock protein 60 T cell epitope. \emph{Arthritis \& Rheumatism},\emph{ 46}(7), 1937-1946. \url{https://doi.org/10.1002/art.10366}



	Prakken, B. J., Samodal, R., Le, T. D., Giannoni, F., Yung, G. P., Scavulli, J., Amox, D., Roord, S., de Kleer, I., Bonnin, D., Lanza, P., Berry, C., Massa, M., Billetta, R., \& Albani, S. (2004). Epitope-specific immunotherapy induces immune deviation of proinflammatory T cells in rheumatoid arthritis. \emph{Proceedings of the National Academy of Sciences of the United States of America},\emph{ 101}(12), 4228-4233. \url{https://doi.org/10.1073/pnas.0400061101}



	Trentham, D. E., Dynesius-Trentham, R. A., Orav, E. J., Combitchi, D., Lorenzo, C., Sewell, K. L., Hafler, D. A., \& Weiner, H. L. (1993). Effects of oral administration of type II collagen on rheumatoid arthritis. \emph{Science},\emph{ 261}(5129), 1727-1730. \url{https://doi.org/10.1126/science.8378772}



	van Eden, W., Thole, J. E., van der Zee, R., Noordzij, A., van Embden, J. D., Hensen, E. J., \& Cohen, I. R. (1988). Cloning of the mycobacterial epitope recognized by T lymphocytes in adjuvant arthritis. \emph{Nature},\emph{ 331}(6152), 171-173. \url{https://doi.org/10.1038/331171a0}



	Weggemans, M. M., van der Schaaf, M., Kluijtmans, M., Hafler, J. P., Rosenblum, N. D., \& Prakken, B. J. (2018). Preventing translational scientists from extinction: The long-term impact of a personalized training program in translational medicine on the careers of translational scientists. \emph{Frontiers in Medicine},\emph{ 5}, Article 298. \url{https://doi.org/10.3389/fmed.2018.00298}



	Weiner, H. L., Friedman, A., Miller, A., Khoury, S. J., al-Sabbagh, A., Santos, L., Sayegh, M., Nussenblatt, R. B., Trentham, D. E., \& Hafler, D. A. (1994). Oral tolerance: Immunologic mechanisms and treatment of animal and human organ-specific autoimmune diseases by oral administration of autoantigens. \emph{Annual Review of Immunology},\emph{ 12}, 809-837. \url{https://doi.org/10.1146/annurev.iy.12.040194.004113}



	Zonneveld-Huijssoon, E., van Wijk, F., Roord, S., Delemarre, E., Meerding, J., de Jager, W., Klein, M., Raz, E., Albani, S., Kuis, W., Boes, M., \& Prakken, B. J. (2012). TLR9 agonist CpG enhances protective nasal HSP60 peptide vaccine efficacy in experimental autoimmune arthritis. \emph{Annals of the Rheumatic Diseases},\emph{ 71}(10), 1706-1715. \url{https://doi.org/10.1136/annrheumdis-2011-201131}






\end{document}