\documentclass[authordate, empirical]{jote-new-article}

\usepackage{caption}

\usepackage{tabularx}

\usepackage{graphicx}

\usepackage{hyperref}

\usepackage[backend=biber,style=apa]{biblatex}

\addbibresource{bibliography.bib}

\jotetitle{The Music Must Play On: \mbox{The Music Therapy} Sessions That Should Not Have Stopped}
\keywordsabstract{music therapy, clinical vignette, residual schizophrenia, therapeutic mistake}
\abstracttext{Seventy-six-year-old Rose was referred to me for music therapy with a diagnosis of residual schizophrenia. Rose was very passive and only wanted to listen to French chansons. After two years, I ended the therapy out of a belief that our music therapy sessions were not meaningful for her. About a year later, I took on more work hours and Rose surprisingly returned to the therapy room and requested to listen to music. I now realize that I was wrong to believe that the absence of active participation indicated a lack of meaning. A receptive state is not a passive state, and a relationship with a client can also be formed by listening to music together.}
\runningauthor{Dassa}
\jname{Journal of Trial \& Error}
\jyear{2023}
\paperdoi{10.36850/e19}
\paperreceived{May 22, 2023}
\author[1]{\mbox{Ayelet Dassa\orcid{0000-0001-7641-3918}}}
\affil[1]{Bar-Ilan University}
\corremail{\href{mailto:ayelet.dassa@biu.ac.il}{ayelet.dassa@biu.ac.il}}
\corraddress{Bar-Ilan University}
\runningauthor{Dassa}
\paperreceived{September 7, 2023}
\paperaccepted{September 14, 2023}
\paperpublished{November 12, 2023}
\jwebsite{https://journal.trialanderror.org}

\articletype{Mistakes in Music Therapy - Clinical Vignette}

\begin{filecontents}{bibliography.bib}
	@article{Geretsegger2017,
    title       = {Music therapy for people with schizophrenia and schizophrenia-like disorders},
    author      = {Geretsegger, M. and Mössler, K. A. and Bieleninik, Ł. and Chen, X. J. and Heldal, T. O. and C, Gold},
    volume      = {5},
    url         = {https://doi.org/10.1002/14651858.CD004025.pub4},
    doi         = {10.1002/14651858.CD004025.pub4},
    place       = {Art},
    date        = {2017},
    journal     = {Cochrane Database of Systematic Reviews}
}


@article{Goldstone2020,
    title       = {Unmet medical needs and other challenges in the treatment of patients with schizophrenia. The},
    author      = {Goldstone, L. W.},
    number      = {3},
    volume      = {26},
    url         = {https://doi.org/10.37765/ajmc.2020.43011},
    doi         = {10.37765/ajmc.2020.43011},
    date        = {2020},
    pages       = {48--54},
    journal     = {American Journal of Managed}
}


@article{ofHealth2023,
    title       = {Supported housing in the community for people with mental illness who are eligible for a rehabilitation basket},
    author      = {of Health, Ministry},
    url         = {https://www.gov.il/en/service/rehabilitation-housing},
    date        = {2023}
}


@article{Pedersen2021,
    title       = {Music therapy vs. music listening for negative symptoms in schizophrenia: {R}andomized, controlled, assessor-and patient-blinded trial},
    author      = {Pedersen, I. N. and Bonde, L. O. and Hannibal, N. J. and Nielsen, J. and Aagaard, J. and Gold, C. and Rye Bertelsen, L. and Jensen, S. B. and Nielsen, R. E.},
    number      = {738810},
    volume      = {12},
    url         = {https://doi.org/10.3389/fpsyt.2021.738810},
    doi         = {10.3389/fpsyt.2021.738810},
    date        = {2021},
    journal     = {Frontiers in Psychiatry}
}


@article{Tseng2016,
    title       = {Significant treatment effect of adjunct music therapy to standard treatment on the positive, negative, and mood symptoms of schizophrenic patients: a meta-analysis. {BMC} psychiatry, 16, 16},
    author      = {Tseng, P. T. and Chen, Y. W. and Lin, P. Y. and Tu, K. Y. and Wang, H. Y. and Cheng, Y. S. and Chang, Y. C. and Chang, C. H. and Chung, W. and Wu, C. K.},
    url         = {https://doi.org/10.1186/s12888-016-0718-8},
    doi         = {10.1186/s12888-016-0718-8},
    date        = {2016}
}


@article{Veerman2017,
    title       = {Treatment for negative symptoms in schizophrenia},
    author      = {Veerman, S. R. T. and Schulte, P. F. J. and de Haan, L.},
    number      = {13},
    volume      = {77},
    url         = {https://doi.org/10.1007/s40265-017-0789-y},
    doi         = {10.1007/s40265-017-0789-y},
    date        = {2017},
    pages       = {1423--1459},
    journal     = {A comprehensive}
}
\end{filecontents}

\begin{document}
\begin{frontmatter}
  \maketitle
  \begin{abstract}
    \printabstracttext
  \end{abstract}
\end{frontmatter}

\lettrine{R}{ose's} case takes me back about a decade, to my work as a music therapist at a hostel for older adults with mental illness (OAMI). In the following clinical vignette, I will share the thoughts I had after writing and reflecting on this case. A renewed contemplation has enabled me to delve deeper into the case and understand that the mistake I made then was a misperception of the significance of that therapy.

I started working at the hostel after several years of working with people with dementia, so I had become well-acquainted with working with older adults, but less so with OAMI. Most of the residents at the hostel were over 60 years old and had spent a large portion of their lives hospitalized at psychiatric institutions. At the time, there was a change in the approach to the mentally ill, which led to a trend of integrating them in the community. The rehabilitation field began to gain momentum in Israel when the Rehabilitation Law was passed in 2000. The aim of the Law was to work on rehabilitating and integrating people with mental illness in the community, to enable them to achieve the highest degree of functional independence and quality of life as possible, while maintaining their dignity in the spirit of the Basic Law of Human Dignity and Liberty. As part of the Rehabilitation Law, people with mental illness were eligible to live at a hostel—a rehabilitation setting for people whose ability to function independently was compromised by mental illness and who thus required intensive assistance with activities of daily living (Ministry of Health, 2023).



The hostel at which I worked housed around 40 individuals. Their physical needs were met in a sheltered setting that was adapted to their abilities. The residents participated in various recreational activities at the hostel and were free to go out alone, though most tended to stay at the hostel throughout the day. I worked at the hostel twice a week and joined a small team that was very experienced in treating OAMI. The team included a social worker, psychiatrist, geriatrician, nurse, occupational supervisor, and counselors, as well as the hostel manager—a social worker who also supervised me during that period. The hostel was a simple place that did not have a lot of resources, but it did have a therapeutic soul and a strong love for the residents. The staff was very dedicated to each and every resident and went above and beyond to enable them to have a good quality of life. Rose was referred to me for music therapy after a weekly staff meeting. She did not participate in group activities and the staff thought that it was worth trying individual therapy to give her a space where she could feel safe enough to develop a relationship.



I met Rose after she had been living at the hostel for several years. She was then 76 years old, had been diagnosed with residual schizophrenia, and had spent over 30 years at a psychiatric hospital before moving to the hostel. Negative symptoms among patients with schizophrenia lead to social and cognitive decline and are associated with a general decline in quality of life (Pedersen et al., 2021). Over time, patients with schizophrenia become prone to flat affect, impaired motivation, reduction in spontaneous speech, reduced feeling of pleasure in the activities of daily living, social withdrawal, and difficulty being consistent with activities (Goldstone, 2020). In addition to negative primary symptoms that are directly associated with the disease, there are also secondary negative symptoms that lead to its exacerbation; these are partly caused by increased consumption of high doses of antipsychotic drugs over the years and living in a psychiatric institution for a prolonged period of time (Veerman et al., 2017).



Rose was a heavy smoker, like most of the residents at the hostel. She was skinny and quiet, tended to sit alone, holding her pack of cigarettes, and did not participate in any social activities. Even in the yard, she tended to sit separately from the others and almost never conversed with the residents or staff members. The few words she spoke with other residents were usually gruff outbursts such as “leave me alone” or “get out of here” when they asked her for a cigarette or money, and when staff suggested she join a social activity, Rose generally replied “I don't have the energy” or “not coming.” She rarely initiated a conversation or sat with another resident. Occasionally I saw her sitting in the occupation room, doing embroidery. Because of her tendency to isolate, I feared that Rose would resist partaking in music therapy, but I hoped that because it was not a group activity but rather an individual session, she would agree to try. And indeed, Rose accepted my invitation to go to the music therapy room.



The music therapy room was on the top floor of the building. The offices of the manager and the social worker were at the entrance to the floor, and in order to reach the room it was necessary to choose the corridor that led to the roof. There, tucked away from it all, was a room that we dedicated to music therapy. The room had rugs, an old sofa and armchairs, a record player with records (an item that was anchored in the culture that the population was familiar with), a CD stereo system (the popular audio system at the time), an old keyboard that was donated, a guitar, and several percussion instruments that were bought with a small budget we managed to raise. In the first decade of the 21\textsuperscript{st} Century, it looked like we had created a 1970's style room. The room was designed as such due to the few items we had, but it was inviting and pleasant. And indeed, the residents at the hostel loved the room. Rose would come for therapy once per week on a regular basis, every Tuesday at 5:00 p.m. Right away, in the first sessions, she asked to hear the music she loved, French chansons, and mentioned a few singers' names. I prepared discs of Edith Piaf, Charles Aznavour, Jacques Brel, and other popular singers from that period. Rose would sit in an armchair and look at me, and after replying “fine” to my question of how she was doing, she tended to say “play music”.



A meta-analysis of studies in the field found that music therapy has a significant potential to address negative symptoms such as social withdrawal, impaired motivation and interest, and a marked cognitive improvement mainly among chronically ill people (Geretsegger et al., 2017; Tseng et al., 2016; Veerman et al., 2017). Therefore, my primary aim in therapy was to forge a relationship with Rose and to give her a sense of a safe space in the room. I also looked for ways to expand her expressional range and to create interaction through additional channels other than conversation, which, as mentioned, was very sparse and limited. For this, I used a variety of tools I had at my disposal as a music therapist. I tried to interest her in playing the percussion instruments together, I encouraged her to sing along to the songs that played in the background, and I tried to spark a conversation about the content of the songs we heard. However, Rose avoided my suggestions to play a percussion instrument, she did not sing along, and when I tried to address the content of songs she barely conversed and gave perfunctory replies such as, “it's a love song”. I printed out the lyrics of the songs and asked her to help me translate the words from French to Hebrew. She was barely agreeable, and only translated a word here and there.



Despite not complying with my therapeutic interventions, Rose continued to come to therapy each and every week, without fail, and each time to listen once again to chansons from the collection of singers she selected during the first session. She did not express any interest in expanding the repertoire. When a song ended, she said “Put another one on”. When I attempted to elicit a response to the songs, she mainly replied “it's nice”.



In the sessions with Rose, I felt that I was having a hard time overcoming the barrier of disinterest and lack of cooperation with the interventions I proposed. As time passed, I began to feel a sense of emptiness and boredom in the sessions with her. Although I had experience with sessions of a repetitive nature in my work with people with dementia, there was a significant difference here. Whilst in therapy with people with dementia the songs repeated themselves and the client tended to sing the same song over and over again, there was typically an active dynamic to the session: the client would sing along with me (even if they were only capable of singing a small portion of the song), make eye contact, respond as much as they could, and interact with me. With Rose, I found myself listening to the same songs repeatedly, but without any singing along or eye contact, and with the pervasive feeling that there was no interaction between us.



In supervision meetings at that time, I tried to understand what meaning this session held for Rose. I shared my feelings that my presence in the room was meaningless and that my role was only to press PLAY. To a certain degree I gave up trying to elicit an active response and encourage Rose to engage, and I simply sat with her and listened to music. And, believe it or not, almost two years of sessions with Rose passed in this manner.



After two years, I requested to work one day less at the hostel due to academic commitments. Consequently, I reorganized my schedule and checked which treatments I could conclude. During a discussion with the professional staff, Rose's name came up first. We didn't see a reason to continue with the therapy: we couldn't point to any change in her behavior outside of the music therapy room and I didn't think that the sessions were meaningful to her. I offered to prepare discs for Rose so she could listen to the songs she loved in her room. After all, that's all we did during our sessions together, so I did not see any need to continue. I may have even felt a sense of relief that the opportunity to end the therapy had presented itself. When I told Rose that I would no longer be working on Tuesdays and therefore we would stop meeting, her only reply was “fine, so be it.” In our final sessions I tried to summarize our shared journey, I addressed the songs we listened to, and I offered to prepare her discs so she could continue to listen to the music she loved. Rose took the discs, but in her characteristic style, she did not say much. Our final session ended in a similar manner to all of our other sessions.



Throughout the next year, I continued to work at the hostel one day a week, and I would greet Rose whenever I saw her. She would greet me back. When I asked whether she listened to the discs, she would simply say “no,” and nothing more. These chance encounters reinforced my feeling that Rose did not ascribe any special importance to the music, and I felt that I had made the right decision to conclude the music therapy sessions with her.



One year later, I took on more hours at the hostel and resumed my work on Tuesdays. Once again, I sat with the manager of the hostel and the professional staff to decide which residents were eligible for music therapy. I gathered information from the social worker and the recreational supervisor, and I sat down to reorganize my schedule. On the first Tuesday I returned to the hostel, while sitting in the music therapy room with my papers and lists, Rose entered the room at 5:00 p.m. I was very surprised she had come; a year had passed since our last therapy session, and it was clear that we had concluded the therapy back then. Despite this, Rose came in and sat down. I asked her what made her come today and she replied, “You weren't here, so I didn't come.” I felt somewhat embarrassed, at that moment I thought that she had not understood me, had not realized that we had finished the therapy. Therapy had already ended as far as I was concerned, even though I had not completely managed to understand what had transpired during it, and suddenly she was back and settling into the room. I tried to explain that we had actually concluded the therapy and we were not supposed to continue, and Rose replied, “You're here, I'm alive, play music”.



In the supervision meeting with the hostel manager that week, we discussed Rose. We discussed whether resuming therapy with her was the right thing from a schedule aspect, and whether there were no other residents who were in greater need of therapy at that stage. However, I realized that there was something very powerful and clear in those few words that Rose vocalized and in her physical presence back in the armchair in the music therapy room. Rose, a seemingly passive woman, got up and did something so different to how I had perceived her. It was clear to me that Rose demanded her place back, the sessions, the music. I realized that we were resuming our sessions every Tuesday at 5:00 p.m. In retrospect, I contemplate that period, and through my own reflection and writing I can re-process and gain a more in-depth understanding of what happened there and the therapeutic mistake I made. From my perspective then as a therapist, I did not attempt to understand Rose's perspective, but rather acted out of my own beliefs and perceptions, and therefore I missed important things in understanding our relationship.



Approaches in music therapy make a distinction between active doing, a state where the client is invited to play or sing and participate in musical activities such as improvisation and songwriting, and a receptive state, where the client listens to recorded music or music played by the therapist, which was selected by the client or the therapist (Geresegger et al., 2017). I now realize that I made a mistake when I viewed the therapy from the perspective of the music therapist, who expects the client's active participation. I believed that if there was no active participation in the music therapy room - if Rose did not sing, did not play an instrument, did not converse, and just sat and listened to music - there was no real importance to the therapeutic session and it could be replaced with preparing discs for her so she could listen to them in her free time. In hindsight, I understand how narrow and superficial this perspective was, and how significant it is to listen to the client's preferred music together during the session, even when the client is in a receptive state. I also understand that a receptive state is not a passive state at all, and through the shared experience of listening to music together, the connection is forged. This was Rose's way of enabling a relationship with me, and this was apparently what was the right thing for her at the time. Moreover, the degree of cooperation and perseverance in music therapy among patients with schizophrenia who have a history of numerous hospitalizations is lower than among patients with fewer previous hospitalizations (Tseng et al., 2016); despite this, Rose, who had previously been hospitalized for a number of years, came to our sessions consistently. Now I also understand that her perseverance could be perceived as success of the therapy in its own right—something to which I did not ascribe any importance at the time.



Since that day, we met every Tuesday at 5 p.m. until Rose passed away several months later. When Rose died, I thought to myself how important it was that I had not rejected her when she showed up for our sessions the second time around. Rose understood something that I had not, and thanks to her intuitive understanding, I was able to rectify the clinical mistake I made when I tried to end the therapy, in the belief that it was not meaningful.















\section{References}



Geretsegger, M, Mössler, K. A., Bieleninik, Ł, Chen, X. J., Heldal, T. O., and Gold C. (2017). Music therapy for people with schizophrenia and schizophrenia-like disorders.\emph{ Cochrane Database of Systematic Reviews}, 5. Art. No.: CD004025. \href{https://doi.org/10.1002/14651858.CD004025.pub4}{https://doi.org/10.1002/14651858.CD004025.pub4}



Goldstone, L. W. (2020). Unmet medical needs and other challenges in the treatment of patients with schizophrenia. \emph{The American Journal of Managed Care}, \emph{26}(3), 48-54. \href{https://doi.org/10.37765/ajmc.2020.43011}{https://doi.org/10.37765/ajmc.2020.43011}



Ministry of Health (2023). \emph{Supported housing in the community for people with mental illness who are eligible for a rehabilitation basket. }\href{https://www.gov.il/en/service/rehabilitation-housing}{https://www.gov.il/en/service/rehabilitation-housing}



Pedersen, I. N., Bonde, L. O., Hannibal, N. J., Nielsen, J., Aagaard, J., Gold, C., Rye Bertelsen, L., Jensen, S. B., and Nielsen, R. E. (2021). Music therapy vs. music listening for negative symptoms in schizophrenia: Randomized, controlled, assessor-and patient-blinded trial. \emph{Frontiers in Psychiatry 12}:738810. \href{https://doi.org/10.3389/fpsyt.2021.738810}{https://doi.org/10.3389/fpsyt.2021.738810}



Tseng, P. T., Chen, Y. W., Lin, P. Y., Tu, K. Y., Wang, H. Y., Cheng, Y. S., Chang, Y. C., Chang, C. H., Chung, W., \& Wu, C. K. (2016). Significant treatment effect of adjunct music therapy to standard treatment on the positive, negative, and mood symptoms of schizophrenic patients: a meta-analysis. \emph{BMC psychiatry}, \emph{16}, 16. \href{https://doi.org/10.1186/s12888-016-0718-8}{https://doi.org/10.1186/s12888-016-0718-8}



Veerman, S. R. T., Schulte, P. F. J., \& de Haan, L. (2017). Treatment for negative symptoms in schizophrenia: A comprehensive Review. \emph{Drugs}, \emph{77}(13), 1423--1459. \href{https://doi.org/10.1007/s40265-017-0789-y}{https://doi.org/10.1007/s40265-017-0789-y}


\end{document}