% Document class options:
% =======================
%
% serif: Sets the body font to be serif.
%
% twocolumn: Sets the body text in two - column layout.
%
% Fill in the type of article here.
% empirical
  % reflection
  % meta
  % rga
  % editorial
  %
% Using other bibliography styles:
% =======================
% authordate = APA
  % number =[1]
  %
    \documentclass[twocolumn, issue, reflection, authordate]{jote-new-article}


  %%% Add the bibliography, make sure it's in the same directory
\addbibresource{glaveanu.bib}

%%% Add additional packages here if required.Usually not needed, except when doing things with figures and tables, god help you then

  % This package is for generating Lorem Ipsum, usage: \lipsum[X] where X is the Xth paragraph of lorem psum.OR use[1 - 5] to generate the first five, etc.

\usepackage{gensymb}
\usepackage{graphicx}
\usepackage[export]{adjustbox}% http://ctan.org/pkg/adjustbox


%%% TODO: Make this a 1 - 5 option scale to reduce the chance of mistyping

  % Enter the title, in Title Case Please
    % Try to keep it under 3 lines
\jotetitle{Cognition stays wild: A commentary on Ross~and~Vallée-Tourangeau’s \textit{Rewilding~Cognition}}

% List abbreviations here, if any.Please note that it is preferred that abbreviations be defined at the first instance they appear in the text, rather than creating an abbreviations list.
%\abbrevs{
%  ABC, a black cat; DEF, doesn't ever fret; GHI, goes home immediately.}

    % Include full author names and degrees, when required by the journal.
% Use the \authfn to add symbols for additional footnotes and present addresses, if any.Usually start with 1 for notes about author contributions; then continuing with 2 etc if any author has a different present address.
% Format: \author[1, 2]{FirstName LastName} \author[2]{...}
  \author[1,2]{Vlad P. Glăveanu}
  \author[3,4]{Alex Gillespie}

    % Fill it in again for the PDF metadata.Lame workaround but it works
      % Format: \authorone{...} \authortwo{...}
  \authorone{Vlad P. Glăveanu}
  \authortwo{Alex Gillespie}

    % List the contribution effort here, they will be listed at the end of the page
      % List the acknowledgments.If there is no companion piece, this is listed below the author info
        %\acknowledgments{Author Two would like to thank Author One for doing all the work while they could slack off.}
% List possible conflict of interest.Will default to saying no conflict exists.
%\interests{Author One was paid for by Big Failed Experiment}
% List funding
    %\funding{}
% Include full affiliation details for all authors
\affil[1]{Webster University Geneva}
\affil[2]{University of Bergen}
\affil[3]{London School of Economics \& Political Science}
\affil[4]{Oslo New University College}


    % List the correspondence email of the main correspondent
  \corremail{\href{glaveanu@webster.ch} {glaveanu@webster.ch}}

% Optionally list the present address of one of the authors
    %\presentadd[\authfn{2}]{Department, Institution, City, State or Province, Postal Code, Country}

% Fill in the DOI of the paper

    % Always starts with '10.36850/' and is suffixed with one of the following plus a number
      % e  : empirical
        % r  : reflection
          % mr : meta - research
            % rga: rejected grant application
              % ed : editorial
  \paperdoi{10.36850/r4}

% Include the name of the author that should appear in the running header
  \runningauthor{Glăveanu \& Gillespie}

% The name of the Journal
  \jname{Journal of Trial and Error}

% The year that the article is published
  \jyear{2021}
\jissue{1}
\jvolume{2}
% The Volume Number
    %\jvolume{Fall}

% The website that's listed in the bottom right
  \jwebsite{
    https://www.jtrialerror.com}

%%% Only \paperpublished is necessary, any combination of the other two is possible

      % When the paper was received
        % Format: 1 January, 1900
    \paperreceived{11 October, 2021}
% When the paper was accepted
      % Format: 1 January, 1900
    \paperaccepted{17 November, 2021}
% When the paper will be published
      % Format: 1 January, 1900
    \paperpublished{30 November, 2021}
    \paperissued{2 March, 2022}
% When the paper is published but in YYYY - MM - DD format, for the crossmark button
    \paperpublisheddate{2021-11-30}

% The pages of the article, comment out if rolling article
      %\jpages{1 - 12}
% Link to the logo, might be redundant
 %   \jlogo{media/jote_logo_full.png}

% Fill something here if this is a rolling / online first article, will make ROLLING ARTICLE show up on the first page


% Sets the paragraph skip to be zero, this should be in the CLS
    

%%% Companion Piece

      % Reflection and Empirical articles have each other as companion pieces.Add the DOI, Title, and Abstract of the respective Companion piece here

        %%% Abstract

        % These two set the height and width of the abstract.There's no solution to do this automatically at the moment so fiddle with these a bit. height-width should be 5mm, 
% Fill in the keywords that will appear in the abstract, max 7
    \keywordsabstract{cognition, problem-solving, interactivity, materiality,
sociality, experiments}

%%%%%%%%%%%%%%%%%%%%%%%%%%%%%%%%%%%%%%%%%%%%%%%%%%%
% Document Starts
      %%%%%%%%%%%%%%%%%%%%%%%%%%%%%%%%%%%%%%%%%%%%%%%%%%%

    \begin{document}
%%% This starts the frontmatter, which includes everything that's on the front page execpt the text of the article
    \begin{frontmatter}
    \maketitle
      % Type your abstract between these things.Max 250 words.Be sure to include the \noindent, looks bad otherwise
    \begin{abstract}
    A ``failed'' experiment (\pdftooltip{{\color{tip}Ross \& Vallée-Tourangeau,}}{Ross, W.,\& Vallée-Tourangeau, F. (2021). Rewilding Cognition: Complex Dynamics in Open Experimental Systems. https://doi.org/10.36850/e4}~\citeyear{Ross2021})
tried to reveal the role played by materiality in solving an insight
problem that made reference to embodied action, leading to valuable
insights about the nature of cognition and the experimental method. In
this commentary, we argue that this study reveals various forms of
interactivity and brings new evidence against the idea that ``pure''
cognition can be isolated from either materiality or sociality. The
question becomes, then, not whether the use of objects helps or hinders
problem solving, but how objects, bodies, and other people participate
in it, even in controlled lab settings, and to what effect. Reflections
are offered on why and how cognition stays wild (i.e., embodied,
dialogical, and surprising) and what this means for experimental work.
    \end{abstract}
    \end{frontmatter}




\begin{companion}
Ross and Vallée-Tourangeau (2021),\vskip.5\smallwidth
\textit{Rewilding Cognition: Complex Dynamics in Open Experimental Systems.}\vskip.3\smallwidth
\newline \href{https://doi.org/10.36850/e4}{https://doi.org10.36850/e4}
\end{companion}

\lettrine{R}\pdftooltip{{\color{tip}oss and Vallée-Tourangeau,}}{Ross, W.,\& Vallée-Tourangeau, F. (2021). Rewilding Cognition: Complex Dynamics in Open Experimental Systems. https://doi.org/10.36850/e4} (\citeyear{Ross2021}) distinguish first-order (hands-on and
interactive) and second-order (cognitive and abstract) problem solving
and set out to experimentally demonstrate the neglected superiority of
first-order problem solving for certain creative tasks. To this end,
they use the socks problem: ``If you have a drawer with brown socks and
black socks mixed in a ratio of 4:5, how many would you have to pull out
in order to guarantee a pair?'' They expected, with good reasons, that
participants who interacted with socks would be better at resolving the
problem compared to participants who only imagined interacting with
socks. However, their results revealed an inconclusively small
difference between the conditions in terms of the number of participants
who solved the problem and the time taken to do so.

The finding of this ostensibly failed experiment is insightful and
richly developed by Ross and Vallée-Tourangeau. They note, with the
power of hindsight, that they had been overly focused on the
interactivity with the socks while failing to consider the multiple
other dimensions of interactivity within the research design:
interactivity with the instructions, internal dialogical interactivity,
interactivity with the researcher, and interactivity with any available
resources. Indeed, even the interactivity of participants who had socks
was uncontrollable, with some upending the sock bag and getting
distracted by counting the socks. They conclude that interactivity
permeated both conditions, and this realization has broad implications
for experimental research that tries to control interactivity.
Interactivity, it seems, is so fundamental to humans that it cannot be
experimentally isolated.

\section*{Pervasive Interactivity}

To further illustrate the point, there is yet another layer of
interactivity that was not considered by Ross and Vallée-Tourangeau, but
which is evident in their data, namely the larger dialogical context of
what it means for participants to be in a ``psychology experiment.'' Few
contemporary participants are engaging in their first experiment, and
few have not heard about experiments with confederates, deceptions, and
creative twists. Thus, another dimension of interactivity is that
participants know that the experiment is a game and that the
experimenter may have hidden motives. They expect a trick, and this
makes the seemingly simple instructions seem puzzling. We see this in
their data. Participants 41, five minutes into the task asks: ``so I
have to pull out a pair?'' (5:21). Participant 26, nearly two minutes
into the task states: ``I'm so confused by the question'' (1:45). Yet,
the instructions given to participants were ostensibly straightforward.
We suggest that what these participants are confused about is what they
are ``meant'' to do (i.e., what is the experimental context around the
text of the instruction). Participants are thus in a mental dialogue
with the genre of the experiment. They are unsure what the experimenters
were looking for and whether even the question itself was a trick. In
short, the experimental situation has layers of interactivity that go
far beyond the situation of waking up in the morning and trying to find
a pair of matching socks in a sock drawer.

This pervasive interactivity within the experiment brings us to the
``rewilding'' metaphor used in the article. Their choice to introduce
materiality (intentionally) and sociability (unintentionally) into the
experimental design is the equivalent of opening the door to the
``wilder'', harder to measure and control, elements of cognition.
Reminiscent of older notions of ``cognition in the wild'' (\pdftooltip{{\color{tip}Hutchins}}{Hutchins, E. (1995). Cognition in the wild. MIT press.},~\citeyear{Hutchins1995}), where sanitized laboratories are replaced by messy, interactive
settings of action and interaction, the use of ``wild'' here and
elsewhere raises a legitimate question: did we ever manage to
domesticate cognition in the first place? Did we achieve experimental
designs that can, at will, cut off the mind from body, others,
institutions, and culture (to name but a few ``unruly'' elements), in
order to study ``pure'' thought (or what Vallée-Tourangeau and March in
2020 called second-order problem solving)? Decisively, no. We can
certainly create methodologies that foreground abstract thought and
place bodily movement, object manipulation, and dialogues with others
into the background. Still, the influence of the latter can never be
discounted. The fact that participants don't visibly use their hands to
tinker with objects doesn't mean that they are any less embodied or that
material forms of engagement have no part to play in their ongoing
mental operations. Similarly, being left alone to solve a task doesn't
make that particular moment or context asocial. Inner forms of
dialogicality are ever-present---the other is always there, just like
bodies are. The question is, then, not to try and compare ``wild'' and
``tame'' forms of cognition, ``muddied'' and ``pure'', distributed and
internal, in terms of the participants' performance in an experiment. A
more interesting question is the one Ross and Vallée-Tourangeau raise:
how exactly do materiality and sociability participate in cognition
(here, problem-solving) in ways that either accelerate or hinder
performance?

Sociality, we suggest, contributes to creativity by expanding
possibility through introducing alternative perspectives (\pdftooltip{{\color{tip} Glăveanu}}{Glăveanu, V. (2020).The possible: A sociocultural theory. Oxford University Press},~\citeyear{Glaveanu2020}; \pdftooltip{{\color{tip}Zittoun \& Gillespie}}{Zittoun, T.,\& Gillespie, A. (2015). Imagination in human and cultural development. Routledge.},~\citeyear{Zittoun2015}). While creativity tasks are often based on
divergent thinking, insight problems call for a better balance between
divergent and convergent thinking. They also depend on
social-psychological processes like perspective-taking (\pdftooltip{{\color{tip}Glăveanu}}{Glăveanu, V., Gillespie, A.,\& Karwowski, M. (2019). Are people workingtogether inclined towards practicality? a process analysis of cre-ative ideation in individuals and dyads.Psychology of Aesthetics,Creativity, and the Arts,13(4), 388–401. https://doi.org/10.1037/aca0000171},~\citeyear{Glaveanu2015})
that connect ``creators'' to their audiences. In the context of the
study, the most direct audience was the experimenter and as such, it is
not surprising that participants tried to propose, clarify, or specify
their perspectives directly to her. This reaching out to the other not
only serves the purpose of reaching the most suitable outcome, but it is
also an engine for creative ideation. Creative insight is not the
personal or internal moment most imagine it to be. It is, in fact, the
result of past and present dialogical experiences in which the
perspectives of the self are placed in dialogue with those of others
(and the more diverse these others are, the more likely it is for
participants to reach more creative solutions; \pdftooltip{{\color{tip}Gassmann}}{Gassmann, O. (2001). Multicultural teams: Increasing creativity and innovation by diversity. Creativity and Innovation Management, 10(2), 88–95.},~\citeyear{Gassmann2001}). The
experiment reported by Ross and Vallée-Tourangeau was not designed to
allow for creative outcomes but to end with a numeric answer, the number
of pulls needed to reach a pair. It is interesting to imagine if allowing the participants to have a more playful and social approach to the problem (e.g., being in dialogue with other participants or the experimenter) might itself lead to more creatively productive aims.

The pervasive interactivity across both experimental conditions does not
mean that Ross and Vallée-Tourangeau's initial guiding research question
was mistaken. There is a lot of evidence for humans having two types of
cognition, as evident in the many ``dual-process'' models of cognition.
It is evident in \pdftooltip{{\color{tip}Vygotsky \& Luria's}}{Vygotsky, L. S.,\& Luria, A. (1994). Tool and Symbol in Child Development. In R. Veer\& J. Valsiner (Eds.), The Vygotsky Reader (pp. 99–174).} (\citeyear{Vygotsky1994}) distinction between the
mental functions humans share with other primates and the symbolically
mediated mental functions that seem more peculiar to humans. And, more
recently, it is evident in the distinction between thinking fast and
slow (\pdftooltip{{\color{tip}Kahneman,}}{Kahneman, D. (2011). Thinking, Fast and Slow. Macmillan.}~\citeyear{Kahneman2011}). These two modes of thought seem pervasive in
human cognition, not just creative problem solving. However, this does
not mean they can be experimentally separated with ease. A mind capable
of both first- and second-order problem solving is likely to leverage
both within any real problem solving episode (why limit oneself to only
one cognitive process?) The creative mind is characterized by movement
within thought and between styles of thought (\pdftooltip{{\color{tip}Gillespie \& Zittoun}}{Gillespie, A.,\& Zittoun, T. (2013). Meaning making in motion: Bodies and minds moving through institutional and semiotic structures. Culture\& Psychology, 19(4), 518–532.},~\citeyear{Gillespie2013}). In the first-order condition, participants may close their eyes
to conceptualize the problem abstractly. In the second-order condition,
participants may simulate the socks using pen and paper or their mind's
eye. In short, in cognition, as it naturally occurs in the wild, humans
likely oscillate between these two modes and perhaps even participate in
both simultaneously---and it is this overlap of cognitive modes that
holds the key to creative cognition.

Interestingly, this idea, that creative cognition stems from moving
between concrete and abstract modes of thought, is evidenced by Ross and
Vallée-Tourangeau's own methodology. They begin with a quantitative
analysis that is abstracted, with conditions differentiated by numbers.
However, in the qualitative analysis, they engage with the concrete
particulars---just like participants who had access to the bag of socks,
they opened up the experiment to have a look inside and see what was
actually going on. This leads them to observe interactivity in the
so-called non-interactive condition. Participants talked to the
experimenter, asked questions about the question, and pitched answers to
see the response. These concrete observations were then integrated back
into a more abstract understanding of rewilding cognition not just in
their own experiment, but in experiments more generally. Thus, Ross and
Vallée-Tourangeau arguably, engaged in both first- and second-order
problem solving in their own creative analysis.

One of the innovative aspects of the article, which Ross and
Vallée-Tourangeau only make passing reference to, is the combination of
qualitative analysis within an experimental design. It enables them to
be both concrete about what went on in the experiment (analyzing videos)
and also to be more abstract (assessing statistical differences). In the
nomenclature of mixed methods research, they used a peculiar variant of
an explanatory sequential design (\pdftooltip{{\color{tip}Creswell \& Creswell,}}{Creswell, J. W.,\& Creswell, J. D. (2018). Research Design: Qualitative, Quantitative, and Mixed Methods Approaches. Sage.}~\citeyear{Creswell2018}). These
designs start with a surprising quantitative finding and then use
qualitative research to generate plausible explanations. Ross and
Vallée-Tourangeau's variant of this design is peculiar because the
quantitative and qualitative parts of the analysis pertain to the same
events (participant behavior in the experiment). Normally, mixed methods
sequential designs use separate data sources (e.g., a survey followed by
an interview). Combining qualitative methods within the quantitative
experiment provides two different lenses on the same behaviors—thus
enabling each to elucidate the other. This qualitative-quantitative
analysis of the behaviors within the experiment enabled them to zoom out
to identify a surprising lack of difference between conditions and then
zoom in to describe what participants were actually doing.

\section*{Conclusion}

What the present studies and several others (e.g., \pdftooltip{{\color{tip} Glăveanu et al.}}{Glăveanu, V., Gillespie, A.,\& Karwowski, M. (2019). Are people workingtogether inclined towards practicality? a process analysis of cre-ative ideation in individuals and dyads.Psychology of Aesthetics,Creativity, and the Arts,13(4), 388–401. https://doi.org/10.1037/aca0000171},~\citeyear{Glaveanu2019}; \pdftooltip{{\color{tip}Hawlina}}{Hawlina, H., Gillespie, A.,\& Zittoun, T. (2019). Difficult Differences: A Socio-Cultural Analysis of How Diversity Can Enable and Inhibit Creativity. 53(2), 133–144. https://doi.org/10.1002/jocb.182},~\citeyear{Hawlina2019}) show us is that rethinking experiments in
psychology, and other disciplines, remains an urgent task. The video
recording of participants in experimental studies or conducting
post-experiment interviews or focus groups should be the rule in this
kind of research, not the exception. The possibility of collecting
qualitative data within research designs that are meant to be
quantitative and focused on measurement should be seen as a valuable
opportunity by experimentalists. Detailed data about the actions,
interactions, and beliefs of one's participants are not meant to be
analyzed mainly when the statistical analysis failed to offer the
desired finding---it should become a source of deeper reflection about
the phenomenon under study and about the participants being studied in
most cases. Of course, there are pragmatic reasons for not overloading
researchers with collecting and especially analyzing datasets that, on
the surface, seem highly distinct. But, as Ross and Vallée-Tourangeau's
research shows, these are not even necessarily different datasets---they
can be part of a single account of the course of action prompted by the
experiment, the origin of both quantification and of qualitative forms
of interpretation. Adding cameras or making room for interviews within a
traditional experiment is not meant to disturb standardization or reduce
control. On the contrary, these additions can offer precious insights
into how things happened and what actually happened, above and beyond
the narrow measurement of changes in a few pre-selected dependent
variables (see also \pdftooltip{{\color{tip} Glăveanu~et~al.}}{Glăveanu, V., Gillespie, A.,\& Karwowski, M. (2019). Are people workingtogether inclined towards practicality? a process analysis of cre-ative ideation in individuals and dyads.Psychology of Aesthetics,Creativity, and the Arts,13(4), 388–401. https://doi.org/10.1037/aca0000171},~\citeyear{Glaveanu2019}).

\pdftooltip{{\color{tip}Moscovici}}{Moscovici, S. (1991). Experiment and Experience: An Intermediate Step from Sherif to Asch. 21(3), 253–268. https://doi.org/10.1111/j.1468-5914.1991.tb00197.x} (\citeyear{Moscovici1991}) lamented the separation between experiments and
experience when experiments in psychology necessarily entail
manipulations of participants' experience. Too rarely have experimenters
``opened the black box'' of the experiment to examine what is actually
going on (\pdftooltip{{\color{tip}Corti et al.,}}{Corti, K., Reddy, G., Choi, E.,\& Gillespie, A. (2015). The Researcher as Experimental Subject: Using Self-Experimentation to Access Experiences, Understand Social Phenomena, and Stimulate Reflexivity. 49(2), 288–308. https://doi.org/10.1007/s12124-015-9294-6}~\citeyear{Corti2015}; \pdftooltip{{\color{tip}Psaltis}}{Psaltis, C.,\& Duveen, G. (2007). Conservation and Conversation Types: Forms of Recognition and Cognitive Development. 25(1), 79–102. https://doi.org/10.1348/026151005x91415},~\citeyear{Psaltis2007}). Ross and
Vallée-Tourangeau's use of qualitative methods within the experiment is
an exciting illustration of the insights that can be obtained by
studying what goes on in experiments, by moving between the abstract
statistical type of analysis and the more concrete, grounded qualitative
type of analysis. Ironically, their delving into the concrete
particulars of their own experiment to examine what happened
demonstrates the point that they failed to establish experimentally,
namely, that first-order concrete thinking can pump insight and problem
solving. It is also a vivid reminder that lab-based cognition doesn't
become ``wild'' whenever the door is opened for the physical
manipulation of objects or the possibility to be in dialogue with the
experimenter. It stays as wildly embodied, dialogical, and surprising as
it ever was.


\nocite{*}
\printbibliography

\end{document}