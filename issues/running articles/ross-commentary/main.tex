\documentclass[authordate, commentary]{jote-new-article}

\usepackage{caption}

\usepackage{tabularx}

\usepackage{graphicx}

\usepackage{hyperref}

\usepackage[backend=biber,style=apa]{biblatex}

\addbibresource{bibliography.bib}

\jotetitle{Commentary: Stumbling upon Indirect Self-Enhancement in Free Will Beliefs}
\keywordsabstract{serendipity, failure, indirect self-enhancement beliefs, free will}
\runningauthor{Ross et al.}
\jname{Journal of Trial \& Error}
\jyear{2024}
\paperdoi{10.36850/8c95-4881}
\paperreceived{October 10, 2023}
\author[1]{\mbox{Wendy Ross\orcid{0000-0002-0461-7660}}}
\affil[1]{London Metropolitan University}
\corremail{\href{mailto:wendyamross@gmail.com}{wendyamross@gmail.com}}
\corraddress{London Metropolitan University}
\runningauthor{Ross, Grüning, \& Firestein}
\author[2]{\mbox{Stuart Firestein\orcid{0000-0003-1774-5853}}}
\affil[2]{Columbia University}
\paperaccepted{July 8, 2024}
\paperpublished{September 8, 2024}
\paperpublisheddate{2024-09-08}
\jwebsite{https://journal.trialanderror.org}


\begin{document}
\begin{frontmatter}
  \maketitle
  \begin{abstract}
    \printabstracttext
  \end{abstract}
\end{frontmatter}



	The paper describes a story, the pattern of which may be familiar to many scientists. A clear and thought-out hypothesis which is grounded in the empirical literature -- in this case that the better-than-average effect which has been robustly demonstrated across several domains would replicate in the domain of free will -- falls at the first empirical hurdle. Rather than give up, the researchers take this unplanned anomaly in the scientific process and use it as a springboard to understand better not only the better-than-average effect but also our ways of understanding free will. The final outcome is far richer than the initial project.



	This story would neatly fit the process model of serendipity laid out by Makri \& Blandford (2012). In this model, the following events needs to take place:



	\subsection{Make a new connection involving a mix of unexpected circumstances and insight}



	The data are unexpected and the realization that the attribution of free will is something more binary than at first assumed is described as a “revelation”. The connection is made between these unexpected data and the categorization of human universal and human traits. In this case, external support came from outside of the system to disrupt a linear knowledge trajectory that posits science as a neat process of idea, discovery of supporting empirical evidence, and assumed understanding and recast it as something messier (Ross, 2023).



	\subsection{Project the potential value of the outcome}



	The null results did not deter the team in this instance. Rather than “file drawering” the data, they were determined to investigate them further. Note in this case the extremely social nature of the value projection and subsequent sensemaking of the accidence. This social nature is often overlooked in stories of serendipitous discovery but is clear across most reports (Copeland, 2019; McCulloch, 2021). 

	\begin{companion}
		David Grüning (2024) \vskip.5\smallwidth
		\emph{Stumbling upon Indirect Self-Enhancement in Free Will Beliefs}\vskip.5\smallwidth
		\href{https://doi.org/10.36850/ef7f-4632}{DOI: 10.36850/ef7f-4632}
		\vskip-2\baselineskip
	\end{companion}


	\subsection{Exploit the connection}



    This exploitation is where the work comes into play. In the folk imagination, serendipity is too often seen as the flash-bulb moment of inspiration: An accident occurs and the individual genius immediately seizes on it (Copeland, 2018). While an idea can be sparked in this way, that idea needs to be enacted and worked through to become a valuable, serendipitous discovery. This is described by the authors in this paper as a “practical enthusiasm” and it may be that this level of application is key to exploiting the role of chance, because most of the time the active pursuit of failure is cognitively costly (Ormerod, 2023).



	\subsection{Uncover valuable unanticipated outcomes}



	The outcomes were valuable and were unanticipated at the start of the project. Nevertheless, they have led to valuable additions in the understanding of the boundary conditions of the better-than-average effect, as well as a series of experiments and a research paper.



	\subsection{Reflect on the value of the outcome and consider as serendipity}



	This last aspect has been clearly met by the submission of this paper. The authors have thought carefully about the experience and the fortuitous nature of the initial unexpected data.



	Despite this, this story does not stand out as a clear case of serendipitous discovery for me. Instead, it is rather a good example of how the \emph{sense of} serendipity - that is the propensity for people to label events as serendipity and to value serendipity - sometimes does not overlap with the events as they unfold from a more objective standpoint. However, understanding why this is not a clear case of serendipitous discovery is a useful way to highlight one key aspect of what it means to be serendipitous, which is often missing from current theorizing.

 

    Serendipity is to some extent ubiquitous (Copeland et al., 2023). The nature of being in a world in a state of environmental flux means that there is a constant encountering of the unexpected and the navigation and sensemaking of those unexpected moments is part of being a functional human. From the perspective of scientific research, experimental research is designed to provide checks on the theory. Were the “accident” which is required for an event to be called serendipitous simply something of this level, then serendipity would cease to become a meaningful category (Ross, 2022). 



	Rather, to constitute a moment of serendipity, the accident should be something which is not only unexpected but impossible within existing frameworks (Boden, 2004). In conceiving of the serendipitous accident this way, I draw from work on surprise (notably Loewenstein, 2019) to argue that serendipitous accidents are not merely anomalous but defy easy explanation and require an updating of our existing model to understand them and to use them to make accurate predictions. In other words, we carry a model of anticipated possibilities even if some are unlikely (Byrne, 2023), but the possibilities introduced by serendipitous accidents should be beyond the bounds of the imaginable. 



	At first glance, this is the case with the story outlined above. The unexpected datum was the failure of the experiment which then precipitated a chain of events which, combined with the skill of the researchers, led to an outcome which was fortuitous at a personal and an epistemic level. However, the failure of an experiment is not an event which is outside the projected possibility space when the experiment is designed. In addition, I suggest that a serendipitous accident should carry with it epistemic value beyond the negative. This particular failure did not bring with it additional knowledge beyond the poor fit of the original hypothesis for the data that were collected. That knowledge was generated through the non-documented but common scientific method of discussion and trial and error until a better model was proposed.

	
	Failure in scientific research is undervalued, particularly by grant awarding bodies and other academic institutions (Barwich, 2019; Firestein, 2012; Willems et al., 2022). However, it is an essential part of the scientific story. Productive failure as illustrated by this story requires far greater investigation than it is currently afforded, so that the fortuitous outcome illustrated here can be repeated. In this case failure is not only valuable retrospectively, i.e., learning from mistakes, but actually leads to discovery of something you did not know you did not know.  Note, however, that understanding and learning from failure requires time and the existence of social networks that can bring in new ideas and allow the incubation of old ones. With the increasing pressures of the publish or perish culture (Frith, 2020), the target article demonstrates quite clearly the value of failure and its importance to our scientific process.



	\section{Author's Response}



	I highly appreciate the detailed categorization and localization of the provided story within the broader perspective of serendipity and what it means to make a truly serendipitous discovery. The purpose of my response is twofold. On the one hand, I agree with and want to briefly comment on the first half of the author's commentary, expanding on Makri \& Blandford's (2012) model. At the same time, in the second half, I want to lightly challenge a central aspect of the commentary, in the hope that this discussion will culminate in a valuable addition to understanding the nature of serendipity in psychology and the social sciences more broadly. My suggestions and arguments are drawn from scientific thinking and research experience in the psychological and related social sciences.



	For the former goal, I would like to expand on the author's address of Makri \& Blandford (2012) structure of serendipitous discovery via a pragmatic Open Science (OS) perspective:



	\subsection{A combination of unexpected circumstances and resulting insight}



	I want to emphasize the combination of unexpected findings and resulting insight as a substantially underdeveloped facet of scientific discovery. In the wake of the OS movement, researchers often misunderstand efforts to preregister their hypotheses before testing them as an argument that the unexpected has no place in discovery. On the contrary, I argue that this argument has lost sight of the general and central idea of OS, namely transparency. Unexpectedness is inevitable in any scientific process. In the spirit of OS, transparency also means openly acknowledging unexpected findings and rigorously testing their implications in prepared and registered follow-up investigations.



	\subsection{Social nature of making sense of the unexpectedness}



	Unexpected discovery must be met with open discourse in the scientific community. As such, an unexpected discovery (if methodological rigor can be guaranteed) should not be a case for the file drawer, but should be recognized as its exact opposite, namely a potential paradigm shift. This potential must be discussed with due rigor, for which social, i.e. community-involved, discourse is indispensable.



	\subsection{Finding momentum to change}



	Reworking existing paradigms and established ways of thinking about the topic at hand is effortful. A social, shared approach to reworking removes some of the weight of individual responsibility. For this to be possible, unexpected results (arguably even more than expected results) should be subject to a high degree of transparency and detail about their circumstances (e.g., study design, statistical tools, and subject of analysis).



	\subsection{Fostering the value of the unexpected}



	In sum, the unexpected can shift paradigms by challenging established ways of thinking. Unexpected discovery seems to be undervalued, or at least underformalized, even though it is an inherent part of the scientific process. Unexpected results, especially, require a social approach. For this to happen, and for it to be translated into action, full transparency of the circumstances of unexpected results is the highest imperative.

	\begin{invitation}
		Want to continue the discussion? Comment on PubPeer!\vskip.5\smallwidth
		Click \href{}{\underline{here}} to visit this article.\vskip-2\baselineskip
	\end{invitation}

	\subsection{Recognizing serendipity}



	Unexpected results afford more reflection in at least two ways. First, recognizing serendipity means conducting research in a responsible manner, that can be subject to rigorous testing. Second, once an irregularity is detected, further action in the form of scientific investigation is required. On a smaller scale, an individual bears responsibility to acknowledge irregularities and actively investigate their source (e.g., methodological, statistical, or theoretical errors). On a larger scale, the entire scientific community is called upon to create spaces for individual researchers and research groups to express their responsibility.



	As a second goal of the present response, I would like to reflect briefly on the suggestion that the unexpected discovery in the present story lacks a central characteristic of a serendipitous discovery. The author of the commentary argues that, for a discovery to be truly serendipitous, it must not only be unexpected, but also seemingly impossible. However, throughout the commentary, this criterion of impossibility is referred to in three different forms. Specifically, in addition to impossibility, the authors refer to the imaginability of a result and its contradiction with existing models. However, that a result is (1) impossible, (2) unimaginable, or (3) requires updating an existing model is not conceptually congruent. The first criterion, I argue, is difficult to reconcile with the way most theoretical frameworks in the social sciences work. Even when they are substantially based on mathematical groundwork, theories in the social sciences have loose ties of deduction. That is, they often lack the theoretical interconnectivity of their existing counterparts in the natural sciences. On a practical level, the retrospective labeling of a discovery as impossible is particularly challenging given hindsight bias (see, e.g., Fischhoff, 1975; Hoffrage et al., 2003). In particular, the fluency of an alternative explanation, once presented, promotes the perception that the discovered alternative state of the world was never unlikely in the first place. At the extreme of this argument, the only truly impossible outcomes are those of practical failure (e.g., in the design of a study or statistical analysis of data). The idea that serendipitous discovery needs to be beyond imaginable is, in my view, more appropriate for social science theories. At the very least, certain outcomes of a model may be so far out of reach that they are not considered given current theory. Lastly, that an outcome requires an update of existing beliefs seems to be the least stringent criterion, and may be too loose as a characteristic of serendipity.



	Reflecting on what serendipity means in the psychological and, more broadly, the social sciences is essential to fostering the conditions for revolutionary discoveries. In this respect, I wholeheartedly agree with commentator's final plea to encourage failure in science. At the very least, the recognition of unexpected results due to methodological and other practical failures informs future efforts. At best, and I could not possibly end the response on a better note than to cite the commentary authors: the unexpected "leads to the discovery of something you didn't know you didn't know.”

  





	\section{References}







	Barwich, A.-S. (2019). The value of failure in science: The story of grandmother cells in neuroscience. \emph{Frontiers in Neuroscience}, \emph{13}, Article 1121. \url{https://doi.org/10.3389/fnins.2019.01121}



	Boden, M. A. (2004). \emph{The creative mind: Myths and mechanisms} (2nd ed). Routledge.



	Byrne, R. M. J. (2023). How people think about possibilities. \emph{Possibility Studies \& Society}, \emph{1}(1--2), 29--36. \url{https://doi.org/10.1177/27538699321}.



	Copeland, S. M. (2018). “Fleming leapt on the unusual like a weasel on a vole”: Challenging the paradigms of discovery in science. \emph{Perspectives on Science}, \emph{26}(6), 694--721. \url{https://doi.org/10.1162/posc\_a\_00294}



	Copeland, S. M. (2019). On serendipity in science: Discovery at the intersection of chance and wisdom. \emph{Synthese}, \emph{196}, 2385--2406. \url{https://doi.org/10.1007/s11229-017-1544-3}



	Firestein, S. (2012). \emph{Ignorance: How it drives science}. Oxford University Press.



	Fischhoff, B. (2003). Hindsight ≠ foresight: The effect of outcome knowledge on judgment under uncertainty. \emph{BMJ Quality \& Safety}, \emph{12}(4), 304-311. \url{http://doi.org/10.1136/qhc.12.4.304}
	
	Frith, U. (2020). Fast lane to slow science. \emph{Trends in Cognitive Sciences}, \emph{24}(1), 1--2. \url{https://\allowbreak doi.org/10.1016/j.tics.2019.10.007}



	Hoffrage, U., \& Pohl, R. (2003). Research on hindsight bias: A rich past, a productive present, and a challenging future. \emph{Memory}, \emph{11}(4-5), 329-335. \url{https://doi.org/10.1080/09658210344000080}



	Loewenstein, J. (2019). Surprise, recipes for surprise, and social influence. \emph{Topics in Cognitive Science}, \emph{11}(1), 178--193. \url{https://doi.org/10.1111/tops.12312}



	Makri, S., \& Blandford, A. (2012). Coming across information serendipitously -- Part 1: A process model. \emph{Journal of Documentation}, \emph{68}(5), 684--705. \url{https://doi.org/10.1108/00220411211256030}



	McCulloch, A. (2021). Serendipity in doctoral education: The importance of chance and the prepared mind in the PhD. \emph{Journal of Further and Higher Education}, \emph{46}(2), 258--271. \url{https://doi.org/10.1080/0309877X.2021.1905157}



	Ormerod, T. C. (2023). Possible, yes, but useful? Why the search for possibilities is limited but can be enhanced by expertise. \emph{Possibility Studies \& Society}, \emph{1}(1--2).



	Ross, W. (2022). Heteroscalar serendipity and the importance of accidents. In W. Ross \& S. Copeland (Eds.), \emph{The Art of Serendipity} (pp. 75--99). Palgrave MacMillan.



	Ross, W. (2023). The possibilities of disruption: Serendipity, accidents and impasse driven search. \emph{Possibility Studies \& Society}, \emph{1}(4), 489--501. \url{https://doi.org/10.1177/27538699231173625}



	Willems, L., Wade, E., Herbert, R., \& Plume, A. (2022). Tales of the unexpected: Designing for serendipity in research. \emph{International Center for the Study of Research}, \emph{9}. \url{https://doi.org/10.2139/ssrn.4048549}









\end{document}