\documentclass[authordate, commentary]{jote-new-article}

\usepackage{caption}

\usepackage{tabularx}

\usepackage{graphicx}

\usepackage{hyperref}

\usepackage[backend=biber,style=apa]{biblatex}

\addbibresource{bibliography.bib}

\jotetitle{Commentary: The Popcorn Effect}
\keywordsabstract{serendipity, analogy, scientific practice, models}
\runningauthor{Ross}
\jname{Journal of Trial \& Error}
\jyear{2025}
\paperdoi{10.36850/8ca0-4ae4}
\paperreceived{September 21, 2024}

\author[1]{\mbox{Wendy Ross\orcid{0000-0002-0461-7660}}}
\affil[1]{London Metropolitan, London, the United Kingdom}
\corremail{\href{mailto:wendyamross@gmail.com}{wendyamross@gmail.com}}
\corraddress{London Metropolitan}
\runningauthor{Ross}
\paperaccepted{November 28, 2024}
\paperpublished{February 7, 2025}
\paperpublisheddate{2025-02-07}
\jwebsite{https://journal.trialanderror.org}



\begin{document}
\begin{frontmatter}
  \maketitle
  \begin{abstract}
    \printabstracttext
  \end{abstract}
\end{frontmatter}


	\section{Commentary: The Popcorn Effect, by Wendy Ross}


	This anecdote reports a discovery in the real time implementation of thermic biomass gasification made after an undergraduate student noticed an unanticipated consequence of taking a shortcut. It is a good example of the necessary combination of accident and sagacity in a serendipitous event. The story hinges on two aspects central to serendipity. First, the accident. In this scenario, this was an accident caused by the shortcut of not fully drying wooden pellets before feeding them into the machine. This shortcut stemmed from the all-too-familiar feeling of time pressure, in this case for an undergraduate student with a non-negotiable project end point. This led to an unanticipated consequence which first caused a feeling of failure but then, by imagination and careful thought, yielded an unanticipated discovery that changed both theoretical and practical understanding. This anecdote highlights the importance of \emph{both} the unanticipated accident \emph{and} the intelligence to understand the implications of the initial failure in generating novel insights from failure.



	The discovery also demonstrates the importance of teamwork in understanding how serendipity unfolds. As the author notes, the challenge to authorized models was a brave move from an undergraduate student, but it also required those at higher levels of the project to listen and not dismiss the observations. Alongside this openness, they needed to have and be willing to use the resources to investigate them. This anecdote further illustrates Copeland's (2019) argument that these serendipitous moments require additional structures to be realized than simply the moment of noticing by an isolated individual.

	\begin{companion}
		Maximilian Roßmann (2025) \vskip.5\smallwidth
		\emph{The Popcorn Effect}\vskip.5\smallwidth
		\href{https://doi.org/10.36850/e1c8-4c62}{DOI: 10.36850/e1c8-4c62}
		\vskip-2\baselineskip
	\end{companion}

	There are two key things that I would like to highlight in this anecdote to help us to understand the way that serendipity unfolds in scientific practice. The first is one highlighted by the author: the distance between scientific models presented on the page and science as it is practiced in an imperfect laboratory and subject to many different pressures. These pressures are often marginalized in the story of scientific discovery. The story told here illustrates the way that serendipity often occurs when the messiness of the real world extends into and is incorporated into the world of the theoretical. The second key aspect that this story allows us to explore is again highlighted by the author, the importance of analogical thinking in making sense of these unanticipated observations and the role of the imagination in making the connection between seemingly unrelated topics.



	In the case described here, what jumps out from the start is that the author describes the practice of science as far messier than it looks on paper and also far more \emph{human}. The distinct phases of gasification mapped out in the process models used to understand those phases were not reflective of the unfolding of the process in the laboratory even prior to the serendipitous discovery. Rather, the author tells us that the processes were not as distinct as they were presented in these models, and that there was an overlap across the different phases. The author even demonstrates the importance of additional senses, such as smell, to understand these. Added to this was the very real process of muddling that unfolds in a scientific laboratory when the materials required for experimentation do not fully work and the people who are operating them are under additional forms of pressure themselves. Indeed, the start of this anecdote describes a situation of failure and frustration with the equipment and the skipping of the traditional order because of ongoing time pressures.



	Many serendipity stories arise from what has been called “the principle of limited sloppiness” (Delbruck, cited in Grinnell, 2009). Perhaps most famously, the discovery of penicillin was sparked by a form of mold that grew over a Christmas break in an untidy laboratory, but there are many other stories that hang on this sort of accident. For example, Goodyear finally solved his problem of the stickiness of rubber at high temperatures through accidentally dropping sulfur and rubber together on a stove in his home and discovering vulcanization (Yaqub, 2018). However, it is important to appreciate that for Delbruck, the sloppiness is \emph{limited} because it is not indiscriminate. The principle of limited sloppiness simply serves to acknowledge that the conceptual understanding of a situation is always unclear and limited rather than an unintended messiness. Delbruck makes the case that experimental designs sometimes test more than the intended question and that the role of the experimenter is to notice these unintended consequences. In this way, the messiness differs from that described in the anecdote presented here. The case in this anecdote was not that of explicit experimentation with messily fuzzy edges. However, the author's act of noticing the unintended consequence of the shortcut demonstrated the same skill of noticing and understanding the theoretical implications of the unintended outcome outlined by Delbruck.



	It remains to be seen, as an increasing amount of research practice is automated, how the parameters of limited sloppiness can be maintained and whether in the nature intended by Delbruck or the more human messiness exemplified in this anecdote. For example, while in my own domain of psychological science there has been much concern about the effect of “bots” in online data collection, there has been less discussion of the gradual elimination of unintended data collection that is part of these forms of tightly controlled experimentation. Anecdotes such as this and others in this series remind us that those who are conducting experiments in messy real world have important understanding which should be respected. If we do not pay attention to the messiness on the margins of discovery, we may lose something along with the efficiency gains of increased automation and abstraction of research. Similarly, and as discussed by Grüning (Ross et al, 2024), it remains to be seen if the current focus on Open Science (OS) in research will also lead to the erasure of this principle of limited sloppiness. Currently, rightly or wrongly, the move towards OS is being interpreted as a reduction of this sloppiness rather than transparency about the messiness of science. The next iteration of this movement must be to actively engage with this messiness.



	Of course, we cannot explicitly build sloppiness into our theoretical models. There is no space for sloppiness in the models of a scientific process because models act to clarify rather than describe. As the author of this anecdote writes, the role of models is to simplify the real. Leaving space for serendipity is simply leaving space to update these abstracted models in light of important information from the real world that they aim to represent. Serendipity is the process of understanding the ways that theoretical aspects can be updated when something unexpected happens. The messy everyday of complexity cannot be simplified (Smaldino, 2017).



	I propose that serendipity then arises from the moment when the non-modelled world meets and changes the theoretical models. For this same reason Yaqub (2018) uses the famous quote by Pasteur to show us how accidental discovery comes from the observation of something which is theoretically unanticipated. The relationship with theory is threaded throughout stories of serendipitous discovery in the sciences. Serendipity involves the surprise that comes from a change in understanding (Simonton, 2022). I have previously described the disruptive nature of this shift in theoretical understanding, which can sometimes lead to it being rejected by the individual or the sociocultural surroundings (Ross, 2023). Serendipity highlights the messiness of the scientific process but also the paucity of our understanding of the stability or otherwise of our models' parameters.



	The second aspect of this story that can shed light on the process of innovation is the role of analogy in serendipity. Analogical transfer is, broadly, the ability to map findings from one domain to another. It means that the learning or understanding that occurs in one domain can be used in another. For several theorists this has long been considered to be one of the key aspects of creativity and intelligence (Sternberg, 1977). Researchers in this area will present participants with a problem (the source) and then, after a break, present them with a problem similar in structure (the target) to see whether they are able to recognize the links between the two. However, within the laboratory, transfer from one area to another is hard to elicit and hints or even explicit instructions are often needed (Ormerod \& MacGregor, 2017; Sala \& Gobet, 2017).



	The story here hangs on an analogy. In this instance, it is an example of what would be called spontaneous analogical transfer (Ormerod, 2023). Spontaneous because the author of the anecdote was not prompted by anyone else to make the link -- instead the link between the way that the pellets puffed up and popcorn was made without prompting. This link then led to a clearer sense of the potential of the wooden pellets and, in his words, the “\emph{unconstrained analogy }… can explain the relationship between accidental parameter variation and measurement data sufficiently to plan follow-up experiments.” The author of this anecdote explains clearly the role of imagination in creating the link between the real and the abstracted. This illustrates why the notion of far transfer is closely linked to intelligence and creativity.



	However, to say that transfer occurred spontaneously undermines the role of the accident in generating this analogy. The accident revealed something about the structure of the wooden pellets that happened beyond the thoughts of the individual. Three things were necessary -- knowledge of the source domain and of the target domain, the accident born from a short cut which unintentionally drew a link between the two, and the imagination of the observer to trace this link. This highlights the key aspects of serendipity -- knowledge and attunement to a domain, imaginative skill to project beyond the two domains, and an accident to reveal an underlying connection. This also highlights an unresearched and unconsidered aspect of analogy: The analogy functions as a method of communication and a way of scaffolding sense making. Recognizing and drawing the analogy was not the end of the process but rather the beginning. Much as argued in Ross and Arfini (2023), serendipity and associated psychological phenomena such as insight or analogy should be seen as the start of the process of discovery rather than its end point.



	\section{Reply by Maximilian Roßmann}



	I would like to thank Dr. Wendy Ross for her generous and insightful commentary on my anecdote. Her feedback has greatly assisted me in refining my philosophical reflection.



	I must say that I find this format of comment and response truly invigorating. For the first time, this revision process feels like an inspiring discussion with a fellow scholar who ignited a genuine interest in her references and discourse, rather than simply placing the burden of citing them upon me. However, I do recognize that I need more time to engage thoroughly with this fascinating discourse on serendipity.

	\begin{invitation}
		Want to continue the discussion? Comment on PubPeer!\vskip.5\smallwidth
		Click \href{}{\underline{here}} to visit this article.\vskip-2\baselineskip
	\end{invitation}


	Still, I would like to briefly elaborate on the topic of analogical thinking in the make-believe discourse. Salis and Frigg (2020) also discuss analogies to sensual experiences and mental images, such as puffing popcorn or a crackling bonfire. Imagining puffing wood pellets inside the reactor, then, transcends analogy by not only “playing back on screen what has been recorded previously” but creatively putting pieces together (p. 9). They, however, further emphasize that most thinking is “propositional” and thereby question the relevance of mental imagery to make-believe and develop a hypothesis. Still, I find it convincing that unconstrained analogies and diverse sensual experiences from different and detached knowledge domains should be seen “as the start of the discovery” (Ross, 2024, p. X). [ATTENTION, page nr?] In contrast to only imagining propositions, they significantly increase the variance of possible imaginings as required in creative discovery. Constraining the initial analogy to the propositional imagination that evaporating water from an incomplete drying phase makes particles puff up like popcorn and react faster, seems like the selection step in an evolution process. It leaves behind the potential relevance of colors, sounds, touches, and smells associated with memories of bonfire and popcorn to clearly prescribe what is to be imagined when justifying follow-up experiments. While such a sterile imagining fits well with the stereotype of rational, reasoning scientists, I suspect that it underestimates the relevance of the appealing analogies and “additional structures” (Ross, 2024, p. X) [ATTENTION, page nr?] to rephrase and emphasize crucial relationships in interdisciplinary research collaborations and science communication with different audiences.


	I would be curious to discuss this fascinating perspective on serendipitous discoveries further on another occasion and include the question of how novelty comes into the world. These questions also provide interesting new perspectives to my empirical interest in multimodal representations. Thank you once again for your thoughtful insights!



	\section{References:}



	Copeland, S. M. (2019). On serendipity in science: Discovery at the intersection of chance and wisdom. \emph{Synthese}, \emph{196}, 2385--2406. \url{https://doi.org/10.1007/s11229-017-1544-3}



	Grinnell, F. (2009). Discovery in the lab: Plato's paradox and Delbruck's principle of limited sloppiness. \emph{The FASEB Journal}, \emph{23}(1), 7--9. \url{https://doi.org/10.1096/fj.09-0102ufm}



	Ormerod, T. C. (2023). Analogy and the transfer of creative insights. In L. J. Ball \& F. Vallée-Tourangeau (Eds.), \emph{Routledge handbook of creative cognition }(1st ed., pp. 94-108). Routledge. \url{https://doi.org/10.4324/9781003009351}



	Ormerod, T. C., \& MacGregor, J. N. (2017). Enabling spontaneous analogy through heuristic change. \emph{Cognitive Psychology}, \emph{99}, 1--16. \url{https://doi.org/10/gcs256}



	Ross, W. (2023). The possibilities of disruption: Serendipity, accidents and impasse driven search. \emph{Possibility Studies \& Society}, \emph{1}(4), 489-501. \url{https://doi.org/10.1177/27538699231173625}

	Ross, W. (2024). A commentary on the popcorn effect. \emph{Journal of Trial and Error}. [ATTENTION, full source?]

	Ross, W., \& Arfini, S. (2023). Serendipity and creative cognition. In L. J. Ball \& F. Vallée-Tourangeau (Eds.), \emph{Routledge handbook of creative cognition }(1st ed., pp. 46-64). Routledge. \url{https://doi.org/10.4324/9781003009351}



	Ross, W., Grüning, D. \& Firestein, S. (2024). Commentary: Stumbling upon indirect self-enhancement in free will beliefs. \emph{Journal of Trial \& Error}. \url{https://doi.org/10.36850/8c95-4881}




	Salis, F., \& Frigg, R. (2020). Capturing the scientific imagination. In A. Levy \& P. Godfrey-Smith (Eds.), \emph{The Scientific Imagination} (pp. 17--50). Oxford University Press. \url{https://doi.org/10.1093/oso/9780190212308.003.0002}



	Sala, G., \& Gobet, F. (2017). Does far transfer exist? Negative evidence from chess, music, and working memory training. \emph{Current Directions in Psychological Science}, \emph{26}(6), 515--520. \url{https://doi.org/10.1177/0963721417712760}



	Simonton, D. K. (2022). Serendipity and creativity in the arts and sciences: A combinatorial analysis. In W. Ross \& S. Copeland (Eds.), \emph{The art of serendipity }(1st ed., pp. 290-320). Palgrave MacMillan.



	Smaldino, P. E. (2017). Models are stupid, and we need more of them. In R. R. Vallacher, S. J. Read, \& A. Nowak (Eds.), \emph{Computational Social Psychology} (1st ed., pp. 311--331). Routledge. \url{https://doi.org/10.4324/9781315173726-14}



	Sternberg, R. J. (1977). Component processes in analogical reasoning. \emph{Psychological Review}, \emph{84}(4), 353--378. \url{https://doi.org/10.1037/0033-295X.84.4.353}



	Yaqub, O. (2018). Serendipity: Towards a taxonomy and a theory. \emph{Research Policy}, \emph{47}(1), 169--179. \url{https://doi.org/10.1016/j.respol.2017.10.007}






\end{document}