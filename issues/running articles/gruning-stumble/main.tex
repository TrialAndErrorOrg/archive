\documentclass[authordate, anecdote]{jote-new-article}

\usepackage{caption}

\usepackage{tabularx}

\usepackage{graphicx}

\usepackage{hyperref}

\usepackage[backend=biber,style=apa]{biblatex}

\addbibresource{bibliography.bib}

\jotetitle{Stumbling upon Indirect Self-Enhancement in Free Will Beliefs}
\keywordsabstract{better-than-average effect, free will belief, self-enhancement, lay people, serendipity}
\runningauthor{Grüning}
\jname{Journal of Trial \& Error}
\jyear{2024}
\paperdoi{10.36850/ef7f-4632}
\paperreceived{August 9, 2023}
\author[1, 2]{\mbox{David Grüning\orcid{0000-0002-9274-5477}}}
\affil[1]{Psychology Department, Heidelberg University}
\affil[2]{Department of Survey Design \& Methodology, GESIS - Leibniz-Institute for the Social Sciences}
\corremail{\href{mailto:david.gruening@psychologie.uni-heidelberg.de}{david.gruening@psychologie.uni-heidelberg.de}}
\corraddress{David Grüning}
\runningauthor{Grüning}
\paperaccepted{July 8, 2024}
\paperpublished{November 25, 2024}
\paperpublisheddate{2024-10-25}
\jwebsite{https://journal.trialanderror.org}

\begin{document}
\begin{frontmatter}
  \maketitle
  \begin{abstract}
    \printabstracttext
  \end{abstract}
\end{frontmatter}







	The Better-Than-Average Effect (BTAE, Zell et al., 2020) describes the psychological phenomenon that individuals overestimate their positive characteristics in comparison to others. The BTAE is a special case of a broader phenomenon of self-enhancement effects, which is a robust finding of psychology. Among other contexts, it can be replicated in practical skill assessments (e.g., driving, Roy \& Liersch, 2013), individuals' evaluation of achievements (Trope, 1986), and strategic thinking (e.g., Grüning \& Krueger, 2021).



	At the beginning of my PhD, I focused on exploring self-enhancement in different settings of strategic games. Soon, however, due to the substantial influence of one of my mentors, who is central to my scientific curiosity to this day, I was captivated by lay people's (dis)beliefs in free will and determinism. I was especially interested in how individuals integrated these fundamental beliefs about the world into their general perspective on themselves. The very first question that naturally came to mind was concerned with how self-enhancement would manifest in people's free will (dis)beliefs (for another idea on free will beliefs sparked at this time, see Grüning \& Krueger, 2023). Because self-enhancement is such a pervasive psychological phenomenon, we assumed it to be essential also for individuals' self-assigned capacity to have a free will. Specifically, the then-existent reasoning was that having a free will was the same as intelligence, certain topical knowledge, and physical abilities, which can be perceived as an individual capability that one could possess in different degrees. This would be the basis for the common BTAE, namely, that an individual perceives their will to be freer and their future to be less determined than others.



	Demonstrating the BTAE in a field of such fundamentality and relevance to many not just psychological but philosophical debates seemed an exciting endeavor. We were highly motivated to prepare and run a first experiment to test this self-enhancement effect in free will beliefs in a student sample. We finished collecting the data in May of 2022 and were surprised when we found no acknowledgeable differences indicated by participants for their own free will compared to others' wills. Contrary to the abundance of literature on the robustness of self-enhancement under diverse circumstances (e.g., Grüning \& Krueger, 2021; 2024; Roy \& Liersch, 2013; Sedikides et al., 2003; Sinha \& Krueger, 1998; Trope, 1986), this need for enhancement did not manifest for free will.



	In a first reaction to the result, we concluded that the null results could have been due to sample characteristics (e.g., the demography or size). We were not discouraged by the results but rather fascinated by this apparent anomaly of self-enhancement that we had gotten wrong. We found ourselves coming back to the unexpected results repeatedly in future discussions on free will and soon dedicated additional studies to the null results. A revelation in the ensuing discussions had triggered this practical enthusiasm: Our interpretation of the null results was that we had made a mistake in assuming that free will is perceived in a similar way to other personal characteristics such as intelligence or physical ability. Our starting assumption was that as most people believe their will is free (Nadelhoffer et al., 2014) and because this belief is morally charged (Mele, 2018), people feel their own will is freer than the will of the average person.



	Yet, there was a reasonable alternative. A differential like the BTAE can manifest more easily where individual differences are thought to exist. By contrast, a belief regarding a human universal, such as free will or determinism, constrains interindividual variance by definition and thus limits the room left for self-enhancement. In other words, folk conceptions of human nature may crowd out perceptions of self-other differences. As an illustration, when applied to the classical syllogism by Socrates, if all humans are mortal, Socrates cannot be more mortal than others. Consistent with this view, Feldman (2017, p. 1) defines free will belief as “the general belief that human behavior is free from internal and external constraints across situations for both self and others.” In summary, in our first approach we overlooked the potential difference individuals perceive between human universals (e.g., free will) and variable human traits (e.g., extraversion).



	Further, we reflected on other potential opportunities for individuals to self-enhance without giving up on their human universal of an (non-)existent free will. If respondents do not self-enhance directly with regard to perceptions of free will, they may do so indirectly. This prediction was grounded in research documenting the influence of the self-concept on social perception (Alicke et al., 2005; DiDonato et al., 2011). Individuals tend to evaluate others favorably insofar as these others share their preferences, traits, or beliefs. This matching effect indirectly contributes to self-enhancement by validating the person's own beliefs. In other words, by thinking highly of an individual who shares one's views, a perceiver can embrace an indirect way of construing self-related beliefs as desirable (Clement \& Krueger, 1998). Such indirect self-enhancement can safeguard the consistency of a personal worldview while simultaneously elevating the self. The belief in the existence of free will is reinforced by valuing others who think likewise without denying free will as a basic human property. Since then, the revised hypothesis that free will beliefs are largely incompatible with direct but compatible with indirect paths of self-enhancement has been tested via two samples and the resulting paper is in preparation (Grüning \& Krueger, in preparation), soon to be submitted to Collabra.



	We learned that serendipitous results can be the beginning of a stimulating discovery or new theorizing that may be even more profound than the starting hypotheses. For this, however, we need to be open for what at first seems to be failure and understand the more profound meaning of science as a cumulative endeavor of trial, error, reflection, and improvement.



	\section{References}



	Abele, A. E., Hauke, N., Peters, K., Louvet, E., Szymkow, A., \& Duan, Y. (2016). Facets of the fundamental content dimensions: Agency with competence and assertiveness—Communion with warmth and morality. \emph{Frontiers in Psychology}, \emph{7}, Article 1810. \url{https://doi.org/10.3389/fpsyg.2016.01810}



	Alicke, M. D., \& Govorun, O. (2005). The Better-Than-Average Effect. In M. D. Alicke, D. A. Dunning, \& J. I. Krueger (Eds.), \emph{The Self in Social Judgment} (1\textsuperscript{st} ed., pp. 85--106). Psychology Press. \url{http://doi.org/10.4324/9780203943250}



	Clement, R. W., \& Krueger, J. (1998). Liking persons versus liking groups: A dual-process hypothesis. \emph{European Journal of Social Psychology},\emph{ 28}(3), 457-469. \url{https://doi.org/10.1002/(SICI)1099-0992(199805/06)28:3\%3C457::AID-EJSP880\%3E3.0.CO;2-T}



	DiDonato, T. E., Ullrich, J., \& Krueger, J. I. (2011). Social perception as induction and inference: An integrative model of intergroup differentiation, ingroup favoritism, and differential accuracy. \emph{Journal of Personality and Social Psychology}, \emph{100}(1), 66--83. \url{http://doi.org/10.1037/a0021051}



	Feldman, G. (2017). Making sense of agency: Belief in free will as a unique and important construct. \emph{Social and Personality Psychology Compass}, \emph{11}(1), Article e12293. \url{https://doi.org/10.1111/spc3.12293}



	Grüning, D. J., \& Krueger, J. (2023). Indeterminism belief protects against uncertainty: First empirical findings. \emph{PsyArXiv}. \url{https://doi.org/10.31234/osf.io/hz947}



	Grüning, D. J., \& Krueger, J. I. (2021). Strategic thinking: A random walk into the rabbit hole. \emph{Collabra: Psychology}, \emph{7}(1), Article 24921. \url{https://doi.org/10.1525/collabra.24921}



	Grüning, D. J., \& Krueger, J. I. (2024). Strategic thinking in the shadow of self-enhancement: Benefits and costs. \emph{British Journal of Social Psychology}. \url{https://doi.org/10.1111/bjso.12747}



	Grüning, D. J., \& Krueger, J. I. (in preparation). You are free like me, but incompetent and immoral: Self-enhancement in free will.



	Mele, A. R. (2018). Free will, moral responsibility, and scientific epiphenomenalism. \emph{Frontiers in Psychology, 9,} Article 2536. \url{https://doi.org/10.3389/fpsyg.2018.02536}



	Nadelhoffer, T., Shepard, J., Nahmias, E., Sripada, C., \& Ross, L. T. (2014). The free will inventory: Measuring beliefs about agency and responsibility. \emph{Consciousness and Cognition}, \emph{25}, 27-41. \url{https://doi.org/10.1016/j.concog.2014.01.006}



	Roy, M. M., \& Liersch, M. J. (2013). I am a better driver than you think: Examining self-enhancement for driving ability. \emph{Journal of Applied Social Psychology}, \emph{43}(8), 1648-1659. \url{https://doi.org/10.1111/jasp.12117}



	Sinha, R. R., \& Krueger, J. (1998). Idiographic self-evaluation and bias. \emph{Journal of Research in Personality}, \emph{32}(2), 131-155. \url{https://doi.org/10.1006/jrpe.1997.2211}



	Sedikides, C., Gaertner, L., \& Toguchi, Y. (2003). Pancultural self-enhancement. \emph{Journal of Personality and Social Psychology, 84}(1), 60--79. \url{https://doi.org/10.1037/0022-3514.84.1.60}



	Trope, Y. (1986). Self-enhancement and self-assessment in achievement behavior. In R. M. Sorrentino \& E. T. Higgins (Eds.), \emph{Handbook of motivation and cognition: Foundations of social behavior} (pp. 350--378). Guilford Press.



	Zell, E., Strickhouser, J. E., Sedikides, C., \& Alicke, M. D. (2020). The better-than-average effect in comparative self-evaluation: A comprehensive review and meta-analysis. \emph{Psychological Bulletin}, \emph{146}(2), 118--149. \url{https://doi.org/10.1037/bul0000218}


\end{document}