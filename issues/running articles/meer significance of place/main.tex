\documentclass[authordate, reflection]{jote-new-article}

\usepackage{caption}

\usepackage{tabularx}

\usepackage{graphicx}

\usepackage{hyperref}

\usepackage[backend=biber,style=apa]{biblatex}
\addbibresource{bibliography.bib}

\jotetitle{On the Significance of Place: Vaccination Refusal as a Situated Phenomenon}
\keywordsabstract{vaccination hesitancy, vaccination refusal, conceptual replication, public health, place, knowledge deficit}
\abstracttext{Expert endorsement seems a promising tool in countering vaccine hesitancy. Yet findings from an experiment in the United Kingdom, published in this journal, found that repeated expert‐backed “debunking” messages had little effect on vaccination intentions or behaviors. At the same time, a similar study in Italy had earlier observed a slight increase in the intent to vaccinate—despite actual uptake remaining unchanged. In this article, I reflect on the differences between these studies and argue for a situated public health approach: one that opens up on diversity and responds to local trust dynamics, cultural nuances, and community values in shaping vaccine responses. Some publics may be reassured by scientific endorsement. Others could remain cautious, influenced by deeply held beliefs about risk and norms and values prioritized within their communities. I propose to interpret the null result published in the \emph{Journal of Trial and Error} as pointing in the direction of public health communication strategies that move beyond a one-size-fits-all model, adapting to the unique social landscapes in which individuals live together and the places where their views are formed and expressed.}
\runningauthor{van der Meer}
\jname{Journal of Trial \& Error}
\jyear{2025}
\paperdoi{10.36850/dfab-4bcf}
\paperreceived{January 23, 2025}
\author[1]{\mbox{Martijn van der Meer\orcid{0000-0002-5557-0225}}}
\affil[1]{Kavli Center for Ethics, Science, and the Public, University of California, Berkeley, United States}
\affil[2]{Department of Public Health, Erasmus MC, Rotterdam, the Netherlands}
\affil[3]{Department of History, Erasmus University, Rotterdam, the Netherlands}
\corremail{\href{mailto:vandermeer@eshcc.eur.nl}{vandermeer@eshcc.eur.nl}}
\corraddress{Erasmus University, Rotterdam, the Netherlands}
\runningauthor{van der Meer}
\paperaccepted{March 12, 2025}
\paperpublished{April 1, 2025}
\paperpublisheddate{2025-04-01}
\jwebsite{https://journal.trialanderror.org}

\begin{document}
\begin{frontmatter}
  \maketitle
  \begin{abstract}
    \printabstracttext
  \end{abstract}
\end{frontmatter}






	\section{Introduction}



	How do we ensure that an entire country gets vaccinated? By "we," I mean scientists and policymakers committed to safeguarding population health. Since nation-states began prioritizing the health of their populations, vaccination has become a deeply political issue (e.g., Porter, 2005; Vargha \& Wilkins, 2023). A single jab operates on two levels: It protects the individual from future disease while also subjecting them to potential side effects. At the societal level, widespread vaccination can lead to “herd immunity,” shielding the population once enough individuals are immunized. Achieving this state requires healthy individuals to accept the small personal risks of vaccination, even when the likelihood of encountering the pathogen or developing side-effects is negligible.







	Yet—or perhaps, so: Not everyone is willing to participate in immunization programs. Some doubt the efficacy of vaccines, even when they are clinically approved and provided by governments. Others prioritize the risks for themselves to outweigh the benefits for the population (Hobson-West, 2003). Whether these concerns are valid or not, refusal presents a challenge for public health authorities: Non-participation undermines the collective goal of population immunity. This is especially problematic in contexts where respect for individual autonomy and bodily integrity prevents governments from mandating vaccination. If vaccination remains a matter of personal choice, scientists and policymakers must grapple with a pressing question: How can we change beliefs, intentions, and, ultimately, behavior to ensure both individual protection and collective health?







	Historically speaking, this question is not new. But it has gained new urgency as vaccination rates in national child immunization programs—particularly in the global North—continue to decline. In many countries where vaccines are available for free or at low cost, a growing number of parents have refused to vaccinate their children against diseases such as measles, whooping cough, rubella, and tetanus (e.g., Flatt et al., 2024; Seither, 2024; van Lier et al., 2024). In response, some ethicists have started tying access to shared spaces such as kindergartens to vaccination status, echoing measures introduced during the COVID-19 pandemic (Pierik \& Verweij, 2024). At the height of that public health disaster, multiple national governments employed digital tools to restrict access to public spaces. In doing so, they temporarily transformed their countries into social laboratories, testing how individual behavior could be influenced to achieve collective goals by providing them with better or more information. Social scientists such as Folco Panizza and his colleagues (2023) took up this challenge and came up with an experiment.







	As part of a Horizon 2020 project on “Policy, Expertise, and Trust in Action,” Panizza and his colleagues set out to investigate whether hesitant individuals could be persuaded when confronted with the consensus of scientists and medical experts (European Commission, 2024). The researchers focused not on the \emph{message} but on the \emph{messenger}. Would explicit expert endorsement change minds? In 2021, the team conducted an experiment targeting vaccine-hesitant individuals and exposed them to repeated “debunking” messages over 3 months. The research ran simultaneously in two countries: Italy and the United Kingdom. Based on the order of listed co-authors, it appears that Piero Ronzani led the Italian arm of the study, while Folco Panizza oversaw the UK component.





\vspace{\baselineskip}

	\noindent It was Piero who got lucky.


	\vspace{\baselineskip}




	In June 2022, \emph{Vaccine}—a Q1 journal in public health and immunology—published the initial findings under the title “Countering vaccine hesitancy through medical expert endorsement.” The results from Italy showed that “scientist[s] and medical experts are not simply a generally trustworthy category but also a well[-]suited messenger in contrasting disinformation during vaccination campaigns” (Ronzani et al., 2022, p. 4635). Over a year later, the same research group, now led by Panizza, published a second paper, this time with a starkly different title: “Medical expert endorsement fails to reduce vaccine hesitancy in U.K. residents” (Panizza et al., 2023). A nearly identical experimental design had yielded completely different outcomes. In the United Kingdom, expert endorsement appeared to have no significant impact on vaccination intentions. This divergence raises important questions: Why did the intervention succeed in Italy but fail in the UK? Is the difference simply a matter of replication, or does it reveal something about the contexts in which these studies took place? And if so, then what?







	Picking up on these questions, my intention with this reflection article is to contribute to theory building on vaccine hesitancy. I argue that Ronzani and Panizza's work supports the hypothesis that the effectiveness of expert endorsement hinges on localized trust dynamics and the alignment of public health engagement with community priorities. To develop this argument, I first compare the findings from Italy and the UK studies, situating the UK study as a conceptual replication. While direct replication reproduces an experiment as closely as possible to verify its findings, conceptual replication tests the same hypothesis with modifications in design, population, or conditions to assess its generalizability (e.g., Crandall \& Sherman, 2016; Schmidt 2009). Beyond this comparison, I draw on recent work in the medical humanities to argue that vaccination campaigns should expand from a singular reliance on the authority of scientific claims toward a more contextualized form of engagement. Finally, I briefly sketch the broader implications of these findings for public health strategies aimed at countering vaccine hesitancy.







	\section{Expert endorsement does not work universally}



	To test the hypothesis that hesitant citizens might change their minds about vaccination when addressed by medical and scientific experts, the Horizon 2020 research team designed an online experiment. They recruited 2,277 participants in Italy and 2,247 participants in the UK via the Prolific platform (Palan \& Schitter, 2018). Following a randomized controlled trial design, participants were divided into experimental and control groups. Over a 3-month period, the experimental groups received debunking messages explicitly endorsed by experts. For instance, to counter fears about the rapid development of COVID-19 vaccines, participants were reassured that accelerated timelines stemmed from minimizing bureaucratic delays rather than compromising scientific rigor. In both studies, data were collected over seven waves between April and June 2021, spaced 10 days apart. To measure the impact of expert endorsement, the researchers assessed three outcomes: whether participants got vaccinated during the study, their intention to do so, and any changes in beliefs about vaccines' protective capabilities.







	The two experiments also differed—not just in terms of an Italian versus a UK study population, but also in subtle aspects of study design. First, the Italian study tailored its debunking messages based on participants' responses in a pre-trial exploratory phase, whereas in the UK, the same materials used for Italian participants were simply re-used without local customization (see Appendix 3 in: Panizza et al., 2023). Second, the control groups were structured differently: In Italy, control group participants received debunking messages endorsed by “generic respondents” (e.g., “the majority of survey participants”), whereas in the UK, the control group received no debunking messages at all. This distinction allowed the Italian study to measure expert endorsement against a baseline of general messaging, while the UK experiment tested the presence or absence of expert-endorsed messaging alone. Third, as the research team themselves note, both experiments were “quasi-experimental,” reflecting different real-world contexts. The UK rollout of COVID-19 vaccines was already well underway by the time its study began—meaning most citizens had already received a vaccine—whereas the Italian vaccination campaign had only just began (Panizza et al., 2023).







	These differences in design, population, and context make it clear that the two studies are not direct replications. Nevertheless, they both tested the same hypothesis—whether expert-endorsed messaging can reduce vaccine hesitancy—across distinct contexts (Crandall \& Sherman, 2016; Schmidt 2009). As such, the UK experiment can still be considered a “conceptual replication” of the Italian study, even given the three most important differences in study design. There are three key reasons why. First, while the Italian study customized its messages based on an exploratory phase, the UK study did not, which may have enhanced effectiveness in Italy but did not alter the goal to assess if expert endorsement causes a shift in vaccine \emph{behavior}. Second, the Italian control group received messages attributed to generic respondents, whereas the UK control group received no debunking messages at all—yet both experiments measured whether expert endorsement affected participants' attitudes \emph{relative to a baseline}. Third, the timing of the UK study may have limited its impact on vaccination intent, as most participants had already received the COVID-19 vaccine by then. However, as the researchers also note, this so-called “ceiling effect” does not explain the lack of change in vaccine-related \emph{beliefs}. If expert-endorsed messaging were universally effective, one would expect it to influence beliefs about vaccine effectiveness, even in a population with high vaccination rates—and that was not what the researchers observed.







	Despite testing the same intervention, the two studies produced contrasting results. The Italian study showed a significant effect of expert endorsement on the intention to get vaccinated and on the belief that vaccination protects others. However, there was no significant effect on the belief that vaccination protects oneself, and neither expert endorsement nor tailored debunking led to higher actual vaccination uptake. In contrast, the UK study found that intentions to vaccinate increased in both the experimental and control groups, with no significant difference between them. Moreover, the intervention had no meaningful effect on vaccination behavior or beliefs regarding the protective capabilities of the COVID-19 vaccines—whether for oneself or the community. Only in Italy did expert-endorsed debunking, compared to generic messaging, yield a modest uptick in the intent to vaccinate (+1.6\%) and potentially bolster beliefs about protecting others.







	Hence, an interpretation of Panizza's paper as a conceptual replication challenges the notion that communication explicitly backed by scientists or medical experts is a universal solution for increasing vaccination uptake. Here, “universal” refers to an approach that would be effective under all conditions without exception. While the Italian findings indicate that expert-endorsed debunking can be beneficial, the UK results demonstrate that this outcome is not guaranteed. Rather than interpreting the UK data as a refutation of expert-endorsed debunking, the study underscores the need to investigate \emph{when} and \emph{how} such messages work, rather than assuming their global effectiveness. In relation to each other and interpreted against the background of recent work in medical anthropology, the findings from Ronzani, Panizza and their colleagues problematize two connected assumptions that underly information campaigns: first, that hesitancy stems primarily from misunderstanding scientific facts, and second, that it is a universal phenomenon unaffected by local context. The following sections sketches why and how these assumptions could be reconsidered.



	\section{On the messengers: Perhaps there is not necessarily a knowledge deficit}



	For the research team, vaccination hesitancy served as a case study to analyze the role of expert endorsement in persuasion. They drew inspiration from theoretical frameworks such as the “elaboration likelihood model,” which posits that the “credibility of the source, including perceived expertise, acts as a persuasive factor” (Panizza et al., 2023; Petty \& Cacioppo, 1986). Vaccination hesitancy presents an interesting test case for this theory: It is often attributed to “misinformation,” such as the “misplaced fear that [the vaccine] could cause autism in children (a case of false information)” (Panizza et al., 2023). This misinformation, according to the authors, could be “debunked” by “directly providing evidence from scientific studies demonstrating no correlation between vaccination and autism in children” (Panizza et al. 2023).







	The main query for the authors was centered on accomplishing this most effectively. Their answer: by using a persuasive source. They reasoned that because experts and medical scientists are generally highly trusted, these groups would be the most effective in addressing the misunderstanding of science among vaccine-hesitant individuals. This position frames vaccine hesitancy as a conflict between science and ignorance—the former portrayed as unproblematic and the latter as flawed and in need of correction (Goldenberg, 2016). Hence, Panizza and his colleagues framed vaccine hesitancy in alignment with the “knowledge deficit model,” which assumes that expert knowledge alone provides a sufficient foundation for resolving major public policy issues (Wynne, 1991). From this perspective, beliefs that contradict expert-endorsed directions should be corrected through education, science communication, or—indeed—“debunking.”







	However, historians, philosophers, and sociologists of science have long critiqued the assumptions underlying a knowledge deficit model (e.g., Hilgartner, 1990; Lewenstein, 1992; Miller, 2001; Wynne 1991; 1992). Publics whose beliefs or behaviors contradict scientific consensus are not necessarily rejecting scientific knowledge outright, these scholars point out. They are not “anti-science,” but they mistrust science as an organization. The problem could therefore better be framed in terms of mistrust instead of ignorance. How does this apply to vaccine hesitancy? Philosopher Maya Goldenberg (2016) argues that “some of the previously secure relations of trust between science and the public that gave consensus statements their epistemic weight in the eyes of the lay public no longer hold” (p. 564). Based on ethnographic research, Goldenberg suggests that vaccine-hesitant parents do not unequivocally reject the scientific consensus on vaccines. Rather, they mistrust the priorities underpinning policy recommendations made by scientific experts. Regarding childhood immunization, Goldenberg notes that many parents she interviewed incorporated “established knowledge that immune responses do vary” and sought to address gaps in understanding causal or preceding events. These parents viewed the “presence of rare but serious adverse events as a safety priority rather than, as health officials see it, a reasonable risk” (Goldenberg, 2016, p. 566; p. 564). This reframes vaccine hesitancy not as the product of misinformation but as a reflection of diverging priorities: the safety of their children versus the safety of the population (Hobson-West, 2003; Hobson-West, 2007).







	Goldenberg's argument cannot simply be applied wholesale to vaccine hesitancy during the COVID-19 pandemic, but it invites a reinterpretation of the results of Panizza's experiment. Perhaps the findings reflect a lack of trust in the priorities of science rather than in consensus-based scientific knowledge itself. In the Italian study, participants were significantly more convinced that the vaccine protected others, yet there was no observable increase in vaccination uptake following the expert-endorsed messaging. This suggests that trust in scientific knowledge does not necessarily translate into trust in policy recommendations. That observation prompts an important question: Do the Italian and UK studies indicate that vaccine hesitancy stems from mistrust in the priorities of science rather than ignorance of its consensus? It would have been valuable to explore whether participants believed the vaccine was effective in preventing both disease occurrence and its transmission but still prioritized concerns about potential side effects in their individual cases. If this were true, it would suggest that people can simultaneously trust scientific knowledge while mistrusting public policies endorsed by scientists and medical experts.







	It is unfair to characterize Panizza's experiment as purely one-directional. The researchers did account for participants' concerns when tailoring certain debunking messages. However, they also assumed that scientists held primary expertise on what mattered most when deciding to get vaccinated against COVID-19. This may be a mistake, as scholars in the medical humanities have suggested. Rather than focusing solely on which beliefs about vaccines underlie hesitation, these scholars argue for a more expansive approach to public engagement. Goldenberg, for instance, advocates moving toward a “dialogical” and “communicative” form of interaction that involves hesitant individuals in co-defining priorities (Goldenberg, 2016). Rather than presuming non-participation stems from ignorance to be corrected, dialogue could begin by recognizing local concerns and the communities whose health hesitant individuals prioritize. This form of public engagement fits recent calls for “community-based participatory research” in which stakeholders are involved equitably, and collaboratively articulate which research topics are of importance to specific communities with the aim of combining knowledge and action for social change to improve community health (Minkler \& Chang, 2019; Wallerstein \& Duran, 2010). From this perspective—and as Panizza and his colleagues themselves acknowledge—the specific context in which hesitant individuals live truly matters. The contrasting findings in Italy and the UK thus prompt a new research question: Does a single messaging strategy work across all contexts? We already know the answer, and it brings into sharp focus a second key assumption of the research group: That vaccination hesitancy is the same phenomenon everywhere.







	\section{On the receivers: Perhaps vaccination hesitancy is not a global phenomenon}



	On the surface, it may seem plausible to view non-participants in vaccination campaigns as a single group: They all decline a readily available vaccine endorsed by governments and medical experts. Since vaccines and the diseases they address are global, one might assume vaccine hesitancy is similarly universal. From this perspective, it makes sense that Panizza and his colleagues expected their successful intervention in Italy to produce comparable outcomes in the United Kingdom. Yet in practice, it did not. In seeking an explanation, the researchers primarily scrutinized their quasi-experimental design and noted that factors such as the timing of vaccination campaigns and government communication efforts likely influenced the divergent results. Even more noteworthy is their suggestion that their intervention “might simply not be effective in certain populations.” As they also remarked in their Italian study, they stress that in contrast to the UK, “the cultural characteristics of Italy make it peculiar in more than one way, and we may observe a certain degree of heterogeneity between populations” (Panizza et al. 2023). Hence, their findings imply that vaccination hesitancy is not uniformly distributed worldwide but rather shaped by distinct local contexts.







	What does it mean, then, to consider vaccination hesitancy as a context-dependent phenomenon? Scholars in the medical humanities have long noted that hesitancy does not necessarily reflect direct opposition to government mandates or scientific consensus. Instead, it often arises from social interaction rather than purely rational deliberation. Anthropologist Elisa Sobo, for example, frames non-participation as a form of solidarity within specific communities, calling it “refusal” rather than “resistance” to capture this nuance. Whereas “resistance” implies direct confrontation with institutional power, “refusal” operates as a generative act of solidarity—reaffirming a community's social fabric. In other words, communities that refuse vaccination may still acknowledge science but opt to uphold communal priorities or shared identity, rather than actively resisting external authority (Sobo, 2016).







	This perspective invites for moving beyond portraying “vaccine refusers” as a monolithic group defined solely by shared (mis)beliefs about vaccines. If refusal represents an act of solidarity, the more pressing question becomes: Which community is the refuser reaffirming? Anthropologists such as Mia Hammerlin emphasize that non-participation often hinges on the “place” in which it occurs—a localized setting of proximal beliefs, customs, and daily interactions. In this sense, “local context” speaks to the historical, institutional, and cultural specificities that shape a community's way of living together, while “proximity” denotes not only physical closeness but also social interconnectedness and mutual dependence (Hammerlin, 2022). Rather than focusing solely on the reasons cited for not vaccinating, we might look more closely at \emph{where}—and why there—groups of people opt out together. This helps situate vaccine refusal in its local context rather than in its abstract similarities across various publics, as Lawrence and her colleagues suggest (Lawrence et al., 2014).







	Although focusing on the local “place” of vaccine refusal fosters a descriptive and actor-focused mode of analysis, it does not rule out the possibility of overarching mechanisms. Certain factors not yet included in existing taxonomies of vaccine uptake could explain why the same intervention succeeds in one setting but fails in another (MacDonald et al., 2022; Thomson et al., 2016). If such variables were identified in exploratory historical or ethnographic studies and then investigated across multiple locations, they might reveal one or more universal drivers of vaccine hesitancy—drivers that manifest differently depending on the “proximities” binding refusing communities. From this perspective, a one-size-fits-all framework could still work at a broad theoretical level, as long as it allows for the specific contextual factors that shape its practical effectiveness. This sensitivity facilitates a pragmatic and “situated” public health approach in response to vaccination refusal—one rooted in dialogue over priorities and an empirically traceable understanding of community values.







	\section{Conclusion: Situated public health}



	This reflection articles proposes to interpret the conflicting outcomes in Italy and the United Kingdom as highlighting that the “knowledge deficit model” alone cannot explain why some communities remain hesitant regarding vaccination. If providing accurate, expert-endorsed information were universally sufficient, the researchers would very likely not find such a divergence in responses. Instead, their results underscore a “contextualist” perspective advocated by historians, sociologists, and philosophers of science: Neither “science” nor “the public” is a uniform entity. Both arise within specific historical and spatial environments, through interactions that co-produce expert knowledge and local understanding (Hilgartner,1990; Jasanoff, 2004; Lewenstein, 1992; Miller, 2001; Wynne 1991; 1992). In this light, as Maya Goldenberg argues, a \emph{dialogical} approach to vaccine refusal becomes essential: Health organizations must recognize and genuinely engage with the priorities and concerns of those who refuse vaccination, rather than simply viewing them as uninformed (Goldenberg, 2016). Other medical humanities scholars, such as Heidi Lawrence and her colleagues, further emphasize that this dialogue ought to begin at the \emph{local} level, with qualitative insight into populations' particular values and identities (Lawrence et al., 2014). This approach echoes community-based participatory research (CBPR), where researchers and local stakeholders collaborate at every stage—from defining the problem to designing interventions (Wallerstein \& Duran, 2010). Such a contextualized approach to vaccine hesitancy could help align the diverse needs of individuals, communities, and nations in a way that top-down messaging alone cannot achieve.







	Practically speaking, this is difficult. A situated response to vaccination refusal admittedly demands more work than rolling out expert-endorsed “debunking” materials. Yet, if the scientific reasoning used to dispel vaccine doubts has failed to eliminate skepticism over the past two hundred years, “they are not going to start working now,” as Lawrence and co-authors argue (Lawrence et al., 2014, p. 127). The findings from Panizza, Ronzani, and colleagues' research slightly refines this view: Although expert-backed messaging did not boost actual uptake in Italy, it significantly increased people's \emph{intent} to vaccinate. Fixing the knowledge deficit may in fact be effective—sometimes, and in specific places. Yet, before using science-based misinformation campaigns as a panacea for vaccine refusal, it may be useful to first ask three questions: \emph{Where} is hesitancy localized, \emph{what} do the people in that setting have in common, and \emph{how} could scientific reasoning align with their beliefs and practices?







	Answers to these questions may clarify vaccine refusal but also pose a serious normative challenge for public health advocates: What if locally situated non-participation endangers overall population health? Measures like restricting daycare access, as public health ethicists such as Pierik and Verweij (2024) propose under John Stuart Mill's no-harm principle, might be warranted when voluntary uptake fails. However, these interventions risk undermining the communal ties that refusal often reinforces. Consequently, health officials must ensure that local solidarities are not dismissed, while still prioritizing broader public health goals. A contextualized understanding of refusal could help them navigate this tension when co-designing interventions to maximize participation in vaccine campaigns. Perhaps community leaders, rather than distant experts, are the most convincing messengers; trusted clerics could very well be more persuasive than health officials. Maybe religious doubts should be met with faith-based reassurance. And if trust in governments is low, civil society organizations might have better luck administering vaccines. These are mere suggestions in need of empirical verification.







	In the end, I would say that the main oversight in Panizza's approach was assuming that the arguments driving hesitancy in Italy, identified through a pre-trial explorative study, would directly apply to the UK. I wonder whether a more context-sensitive approach—attuned to the diversity of cultural, social, and historical specifics, or at least informed by responses from UK participants—might have produced different results. Such adjustments could have made their paper as a conceptual replication even more compelling. Nevertheless, if we accept the possibility that vaccine hesitancy is rooted in specific local realities, then grouping participants by country borders is likely not the most meaningful analytic strategy. The real takeaway from these studies is that effective public health interventions should probably do more than correct a supposed misunderstanding of science or amplify expert authority. They should also resonate with the decisions hesitant individuals make—or don't make—within the places they live. Ultimately, Panizza and his colleagues' null result is not a “failed experiment”. It is a gesture in the direction of situated public health interventions.







	\section{Acknowledgements}



	I have greatly benefited from interacting with Elena Conis and the fellows from the Kavli Center for Ethics, Science, and the Public at the University of California at Berkeley. I would also like to thank David Jones and Allan Brandt from Harvard History of Science; Hans van Vliet from the Dutch Institute for Public Health and the Environment; Noortje Jacobs, Timo Bolt, and Laura Hartman from the Erasmus Medical Center in Rotterdam; and Ralf Futselaar from the Erasmus University in Rotterdam for discussing my interpretation of vaccine refusal at various occasions. The three challenging reviews from the \emph{Journal of Trial and Error} have been invaluable to sharpen and nuance the final text. Yet above all, I would like to thank Folco Panizza and his colleagues for ensuring that their findings did not end up in a file drawer.







	\section{References}



	Crandall, C. S., \& Sherman, J. W. (2016). On the scientific superiority of conceptual replications for scientific progress. \emph{Journal of Experimental Social Psychology}, \emph{66}, 93--99. \url{https://doi.org/10.1016/j.jesp.2015.10.002}



	European Commission. (2024). \emph{Policy, expertise, and trust in action | PERITIA Project | Results | H2020. }CORDIS | European Commission. \url{https://cordis.europa.eu/project/id/870883/results}



	Flatt, A., Vivancos, R., French, N., Quinn, S., Ashton, M., Decraene, V., Hungerford, D., \& Taylor-Robinson, D. (2024). Inequalities in uptake of childhood vaccination in England, 2019-23: Longitudinal study. \emph{British Medical Journal,} \emph{387}, Article e079550. \url{https://doi.org/10.1136/bmj-2024-079550}



	Goldenberg, M. J. (2016). Public misunderstanding of science? Reframing the problem of vaccine hesitancy. \emph{Perspectives on Science}, \emph{24}(5), 552--581. \url{https://doi.org/10.1162/POSC\_a\_00223}



	Hammerlin, M.-M. (2022). This is home: Vaccination hesitancy and the meaning of place. \emph{Ethnologia Scandinavica, 52}, 202-220.



	Hilgartner, S. (1990). The dominant view of popularization: Conceptual problems, political uses. \emph{Social Studies of Science}, \emph{20}(3), 519--539. \url{http://doi.org/10.1177/030631290020003006}



	Hobson-West, P. (2003). Understanding vaccination resistance: Moving beyond risk. \emph{Health, Risk \& Society,} \emph{5}(3), 273--283. \url{https://doi.org/10.1080/13698570310001606978}



	Hobson-West, P. (2007). ‘Trusting blindly can be the biggest risk of all': Organised resistance to childhood vaccination in the UK. \emph{Sociology of Health \& Illness,} \emph{29}(2), 198--215. \url{https://doi.org/10.1111/j.1467-9566.2007.00544.x}



	Jasanoff, S. (Ed.). (2004). \emph{States of knowledge: The co-production of science and social order}. International Library of Sociology. Routledge.



	Lawrence, H. Y., Hausman, B. L., \& Dannenberg, C. J. (2014). Reframing medicine's publics: The local as a public of vaccine refusal. \emph{Journal of Medical Humanities}, \emph{35}(2), 111--129. \url{https://doi.org/10.1007/s10912-014-9278-4}



	Lewenstein, B. (Ed.) (1992). \emph{When science meets the public: Proceedings of a workshop organized by the American Association for the Advancement of Science, Committee on Public Understanding of Science and Technology, February 17, 1991, Washington, DC. }American Association for the Advancement of Science. \url{https://ecommons.cornell.edu/bitstream/handle/1813/70154/Lewenstein.1992.When\%20Science\%20Meets\%20Public.pdf?sequence=3}



	Miller, S. (2001). Public understanding of science at the crossroads. \emph{Public Understanding of Science,} \emph{10}(1), 115--120. \url{https://doi.org/10.3109/a036859}



	Minkler, M., \& Chang, C. (2019). Community-Based Participatory Research: A promising approach for studying and addressing immigrant health.” In Marc B. Schenker, Xóchitl Castañeda, \& Alfonso Rodriguez-Lainz (Eds.), \emph{Migration and Health }(pp. 361--376). University of California Press. \url{https://doi.org/10.1525/9780520958494-020.}



	Palan, S., \& Schitter, C. (2018). Prolific.ac—A subject pool for online experiments. \emph{Journal of Behavioral and Experimental Finance}, \emph{17}, 22--27. \url{https://doi.org/10.1016/j.jbef.2017.12.004}



	Panizza, F., Ronzani, P., Martini, C., Savadori, L., \& Motterlini, M. (2023). Medical expert endorsement fails to reduce vaccine hesitancy in U.K. residents. \emph{Journal of Trial and Error}. \url{https://doi.org/10.36850/e15}



	Petty, R. E., \& Cacioppo, J. T. (1986). The elaboration likelihood model of persuasion. In Richard E. Petty \& John T. Cacioppo (Eds.), \emph{Communication and persuasion: Central and peripheral routes to attitude change }(pp. 1--24). Springer. \url{https://doi.org/10.1007/978-1-4612-4964-1\_1}



	Pierik, R. \& Verweij, M. (2024). \emph{Inducing immunity? Justifying immunization policies in times of vaccine hesitancy}. MIT Press. \url{https://books.google.com/books?hl=nl\&lr=\&id=nCfHEAAAQBAJ\&oi=fnd\&pg=PR5\&dq=inducing+immunity+verweij\&ots=NYjzF6JNyW\&sig=c9BeX659N8NPL5OSOeU8bWStbnw}



	Porter, D. (2005). \emph{Health, civilization and the state: A history of public health from ancient to modern times}. Routledge.



	Ronzani, P., Panizza, F., Martini, C., Savadori, L., \& Motterlini, M. (2022). Countering vaccine hesitancy through medical expert endorsement. \emph{Vaccine,} \emph{40}(32), 4635--4643. \url{https://doi.org/10.1016/j.vaccine.2022.06.031}



	Schmidt, S. (2009). Shall we really do it again? The powerful concept of replication is neglected in the social sciences. \emph{Review of General Psychology,} \emph{13}(2), 90--100. \url{https://doi.org/10.1037/a0015108}



	Seither, R. (2024). Coverage with selected vaccines and exemption rates among children in kindergarten — United States, 2023--24 School Year. \emph{Morbidity and Mortality Weekly Report},\emph{ 73}(41), 925-932. \url{https://doi.org/10.15585/mmwr.mm7341a3}



	Sobo, E. J. (2016). Theorizing (vaccine) refusal: Through the looking glass. \emph{Cultural Anthropology,} \emph{31}(3), 342--350. \url{https://doi.org/10.14506/ca31.3.04}



	van Lier, E. A., Hament, J.-M., Knijff, M., Westra, M., \& Giesbers, H. (2024). Vaccinatiegraad Rijksvaccinatieprogramma Nederland, verslagjaar 2024. Rijksinstituut voor Volksgezondheid en Milieu (RIVM). \url{https://doi.org/10.21945/RIVM-2024-0044}



	Vargha, D., \& Wilkins, I. (2023). Vaccination and pandemics. \emph{Isis,} \emph{114}(S1), S50--S70. \url{https://doi.org/10.1086/726980}



	Wallerstein, N., \& Duran, B. (2010). Community-Based Participatory Research contributions to intervention research: The intersection of science and practice to improve health equity. \emph{American Journal of Public Health,} \emph{100}(S1), S40--46. \url{https://doi.org/10.2105/AJPH.2009.184036}



	Wynne, B. (1991). Knowledges in context. \emph{Science, Technology, \& Human Values,} \emph{16}(1), 111--121. \url{https://doi.org/10.1177/016224399101600108}



	Wynne, B. (1992). Misunderstood misunderstanding: Social identities and public uptake of science. \emph{Public Understanding of Science,} \emph{1}(3), 281--304. \url{https://doi.org/10.1088/0963-6625/1/3/004}






\end{document}