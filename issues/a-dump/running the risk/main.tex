\documentclass[authordate, anecdote]{jote-new-article}

\usepackage{caption}

\usepackage{tabularx}
\PassOptionsToPackage{absolute,overlay}{textpos}
\RequirePackage{xcolor}
\usepackage{graphicx}

\usepackage{hyperref}

\usepackage[backend=biber,style=apa]{biblatex}

\addbibresource{bibliography.bib}

\jotetitle{Running the risk: On serendipity in comparative research across cultures}
\keywordsabstract{serendipity, comparative research, risk governance}
\runningauthor{Beumer}
\jname{Journal of Trial \& Error}
\jyear{2025}
\paperdoi{10.36850/4811-44cc}
\paperreceived{October 13, 2024}
\author[1]{\mbox{Koen Beumer\orcid{0000-0003-4533-3581}}}
\affil[1]{Copernicus Institute of Sustainable Development, Utrecht University, Utrecht, the Netherlands}
\corremail{\href{mailto:k.beumer@uu.nl}{k.beumer@uu.nl}}
\corraddress{Utrecht University}
\runningauthor{Beumer}
\paperaccepted{November 4, 2024}
\paperpublished{March 17, 2025}
\paperpublisheddate{2025-03-17}
\jwebsite{https://journal.trialanderror.org}

\begin{document}
\begin{frontmatter}
  \maketitle
  \begin{abstract}
    \printabstracttext
  \end{abstract}
\end{frontmatter}


	\begin{tikzpicture}[remember picture, overlay]
		\node[align=left, anchor=north west] 
		at ([xshift=4.7cm, yshift=-0.3cm]current page.north west) 
		{
			\begin{minipage}{15cm}
			\raggedright
			\textbf{Correction notice} \\
			Incorrect Article Type Labeling: This article was previously mislabeled as an empirical article. This has now been corrected.\vspace{2pt} % Optional: adds a little space between text and line
			{\color{joteorange}\rule{\linewidth}{1pt}}
			\end{minipage}
		};
	\end{tikzpicture}
	The governance of risk is a highly technical endeavor. If we want to know whether a product like a chemical substance poses a risk to human health and should be regulated, we have to conduct laboratory tests under highly controlled conditions. We may for example expose a number of mice to the chemical substance and see if they are harmed. The variables of such tests -- the number of mice, the dosage, exposure time, and organs tested -- are precisely defined and stipulated in international standards, which result from hundreds of rigorous scientific studies yielding a scientific consensus.



	For my PhD project (Beumer, 2016), I wanted to investigate how different countries in the so-called ‘global South' governed such risks. I particularly wanted to study this for nanotechnologies -- technologies that make use of the unique properties that some materials express at the nano-scale. At the time, many stakeholders were excited about nanotechnology; various countries in the global South, including India, Kenya, and South Africa, were pursuing nanotechnologies, hoping that they could help to fight poverty and bring development. But there were also concerns that nanotechnologies could pose new risks to human health. The consensus was that these risks needed adequate governance.



	Our project was a collaboration between researchers in India, Kenya, and the Netherlands, and we had understood that the governance of risks was a prominent concern in each country. For my subproject I would compare the measures that different Southern countries were taking.



	About a year into my PhD, I first traveled to Kenya for fieldwork. To my surprise, I found that hardly any measures were taken that could possibly count as risk governance. While many people were \emph{talking }about risks, not much else was going on. There were hardly any scientific studies being conducted, there were no regulations, no policy documents, no committees, no conferences, no guidelines, no standards.



	The conventional response to the absence of governance measures in so-called developing countries was to argue that they were lagging behind. That Kenya was not governing risks only because they did not have the capacity to do so, and that they needed Western support to build that capacity. Kenya, so this story implies, was not governing nanotechnology risks \emph{yet}, but sooner or later they would catch up.



	We wondered, however, if a supposed lack of capacity could adequately explain what was going on. We encountered several examples of materials that undisputedly posed risks to human health in Southern countries that clearly did have the capacity to govern those risks, yet these risks were hardly governed. In India, for example, there is hardly any stringent regulation for asbestos. The Indian supreme court refused to ban asbestos as recently as 2011 and the Indian asbestos industry is growing over 10\% a year. Moreover, a supposed lack of regulatory capacity could not explain why there was nevertheless so much talk about risk, despite the clear lack of activity to mitigate it.



	Using a relational theory of risks (Boholm \& Corvellec, 2011), we realized there were different forces at play that made people talk about risks, regardless of whether they were planning to govern these risks, and regardless of whether they even thought governance was desirable. This approach helped us to see that ideas about risk governance reflect a value judgment about the importance of human health vis-à-vis other values that may be at stake when bringing technologies to the market, such as economic progress or employment. In Europe and the United States, where risk governance approaches originated, there is a strong consensus that health risks are legitimate reasons for regulating technologies, even if this comes at the cost of other values. Health is thus valued above anything else. In different settings , such as in Kenya and India, however, the balance between health and other values may tilt in another direction. In some contexts, health risks may be deemed more acceptable in the face of dire societal problems that new technologies may help to solve.



	We found that countries in the global South adopt such ‘risk talk' despite this difference in relative valuation of health because risk governance had been incorporated in global trade regulations. Keeping up the appearance that risks were governed was a requirement to sell one's products in Western markets. This sheds a very different light on evaluations of risk governance approaches: Rather than taking the Western model for granted and asking to what extent Southern countries have successfully adopted risk governance models, our finding raises the question of how risk governance frameworks impose Western norms and values upon countries in the global South, and how Southern countries can be empowered to govern technologies in a way that aligns to their locally specific values and needs.



	This is not the result we were looking for. But it was a valuable result nonetheless. It provided us with a lens for looking at risk governance in Kenya and India that helped us to understand such governance processes in a new way.






	\section{References}



	Beumer, K. (2016). \emph{Nanotechnology and development: Styles of governance in India, South Africa, and Kenya}. [Doctoral Thesis, Maastricht University] Maastricht University Press. Pure.



	Beumer, K. (2018). Travelling risks: How did nanotechnology become a risk in India and South Africa? \emph{Journal of Risk Research, 21}(11), 1362-138. \url{https://doi.org/10.1080/13669877.2017.1304978}



	Boholm, Å., \& Corvellec, H. (2011). A relational theory of risk. \emph{Journal of Risk Research, 14}(2), 175--190. \url{https://doi.org/10.1080/13669877.2010.515313}






\end{document}