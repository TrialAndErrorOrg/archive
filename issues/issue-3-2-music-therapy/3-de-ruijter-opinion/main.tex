\documentclass[authordate, empirical, issue]{jote-new-article}


\jotetitle{Embracing the “Wrong” in Classical Music Interpretation: About Finding Balance Between Tradition and Creativity in Classical Music Performance}
\keywordsabstract{classical music tradition, creative interpretation and experimentation, nonconformist Music Learning Space, autonomy and authenticity in music performance}
\abstracttext{This article discusses the relationship between tradition and personal interpretation in classical music, mainly focusing on the experience of conservatoire music students. I argue that the traditional approach to classical music in education leaves too little room for creativity and finding one’s voice, resulting in a lack of personal development, expression and experimentation. To address this issue, I propose artistic experimentation and teaching strategies that emphasize the importance of creating nonconformist music learning spaces with accountability guidelines for musical learning, encouraging experimentation, risk-taking, and self-expression. In addition, I aim to demonstrate how classically trained musicians can find a balance between tradition and personal interpretation in performance. By fostering an environment that values both tradition and creativity, musicians can explore new ways of performing classical works outside standard interpretive forms. This practice-led research has demonstrated new possibilities for musicians’ performances, as well as opening new paths for familiar music to have a lasting and meaningful impact. This suggests alternatives, possibilities, and opportunities within the discipline. The article concludes that the search for a balance between tradition and creativity is ongoing.}
\runningauthor{de Ruijter}
\jname{Journal of Trial \& Error}
\jyear{2023}
\paperdoi{10.36850/h3yn-bd82}
\jvolume{3}
\jpages{15--21}
\jissue{2}
\paperreceived{May 31, 2023}
\author[1]{\mbox{Sophie de Ruijter\orcid{0009-0002-5289-4301}}}
\affil[1]{Fontys Academy of the Arts}
\corremail{\href{mailto:sophie@deruijtermusic.nl}{sophie@deruijtermusic.nl}}
\corraddress{Fontys Academy of the Arts}
\runningauthor{de Ruijter}
\paperaccepted{September 24, 2023}
\paperpublished{December 22, 2023}
\paperpublisheddate{2023-12-22}
\paperissued{December 22, 2023}
\paperissued{December 22, 2023}
\jwebsite{https://journal.trialanderror.org}
\articletype{Entangled Strings - Opinion}


\specialissue{Untangling Strings -- Further Explorations of Mistakes in Music Therapy}

\begin{document}
\begin{frontmatter}
  \maketitle
  \begin{abstract}
    \printabstracttext
  \end{abstract}
\end{frontmatter}

\setcounter{page}{15}



\lettrine{I}{} was seventeen years old when I started my journey towards a professional career in classical music. As an aspiring flute player, I found myself navigating through an intense four years at the conservatory. During those years, the pressure to conform to my teachers' ideas and interpretations left me feeling insecure about my own musical ideas and creative process, and many of the questions that arose regarding creativity were left unanswered. Why did I not feel able to ascribe the same emotion to that Mozart concerto that my teachers did, and why could I not choose another tempo in the second movement? I supported my interpretative ideas with a lot of historical materials and was given just one opportunity to talk about these during my Bachelor of Music. Unfortunately, my ideas were considered “wrong” and no space was given to me to explore the reasons why. I was told that before discussing my own interpretation and vision on classical works, I needed to be more technically skilled.



Unconventional performances are often perceived as erroneous by musicians who adhere strictly to established norms. Giving alternative performances of classical works clashes with the ideology and its gatekeepers in Western Classical Music (WCM), which is often limited by the need to conform to traditional norms and practices and leaves little space for personal creativity and experimentation. I had the opportunity to conduct practice-led research and to investigate questions of personal interpretation and finding one's voice in contemporary WCM (de Ruijter, 2021; The Performer's Library, n.d.). To my surprise, instead of being met with criticism or dismissal, I was met with curiosity and support from my colleagues. I found that students today are dealing with the same questions I had had as a student over a decade earlier.



The purposes of this article are to create awareness of and discuss the tension between tradition and creativity in classical music performance, as well as to propose the use of artistic experiments as a means of searching for a balance between the two. It is not about rejecting or dismissing the teachings of mentors; rather, it is about honoring their guidance while also exploring creative instincts and developing the skills and techniques necessary to communicate one's own unique musical ideas effectively. My practice-led research and this article do not solve a “problem.” They show alternatives, possibilities, and opportunities within the discipline.























\section{Embracing the “Wrong” in Classical Music Interpretation}

\emph{About finding balance between tradition and creativity in classical music performance.}







\begin{quote}
  Anson, I couldn't agree more, especially the Prokofiev is a shame, trashy playing like this shouldn't be allowed, artists should first and foremost respect the scores! I really wonder what great teachers like Dmitri Bashkirov or Menahem Pressler would say if they heard her play, they probably would throw up their hands in despair\emph{...}(Predota, 2021).
\end{quote}



This is how Kathia Buniatishvili, one of the world's most famous contemporary pianists, was criticized for one of her performances. She is known for her eccentric appearance as well as her free interpretations of works of classical music. As a classically trained musician today, one is still supposed to be faithful to the musical work (as arguably embodied in the notated score) and its composer, as well as to the norms, values and traditions of the genre. This leaves young musicians struggling with the feeling that they have little space to experiment outside of contemporary norms, or to investigate authenticity, subjectivity and creative impulses by performing classical works from their own perspective. Unconventional performances, like the one described above,\footnote{ This performance of the Prokofiev Piano Sonata No. 7 (III. Precipitato) by Kathia Buniatishvili is considered unconventional due, among other things, to a faster tempo of this third movement and her way of communicating with her audience (Prokofiev, 1942/2018). } are generally viewed as erroneous by those who adhere strictly to the established norms. A musician searching for their own voice often battles against established traditions and derogatory attitudes within the culture. However, bucking contemporary norms and embracing experimentation can lead young musicians to experience creativity, thus opening up new possibilities for their performance and development.



In \emph{The Imaginary Museum of Musical Works }(2007, pp.14-43, 205-242), philosopher Lydia Goehr criticizes the prevailing analytical approach to music. She argues that before 1800, music was seen as an activity driven by creativity. Thanks to the rise of the bourgeoisie and the invention of sound recording, music turned into an object, with compositions labelled as works that themselves became objects of activity. Goehr calls this \emph{the work concept}. Classical works are so strong in identity that they form a fixed whole, the Western canon. This canon ensures recognition, but also imposes a very tight frame of reference in the minds of Western classically trained musicians. One generally judges and listens to musical works by means of this frame of reference, consciously or unconsciously (de Munck, 2019). This has resulted in the formation of a strict ideology called “Werktreue”—being true to the work (Goehr, 2007).



By following the composer's intentions as faithfully as possible, the performer, in the service of the composer, should guarantee the identity and “objectivity” of the classical work. Musicologist Richard Taruskin suggests that when music is played in as “neutral” a fashion as possible, musicians hope to maintain the authenticity of a work, meaning it has no discernible input of creativity or originality from the performer (Taruskin, 1982). That is not to say that there is no room for expression in the performance of a classical work, but expression in this case represents something uninvolved and is therefore not part of the identity and creativity of the performer (Cook, 2014, p.222; Hendricks et al., 2014). Embracing such a philosophy means there is inevitably little room for creativity and authenticity from performers. Stravinsky, one of the most well-known composers of the 20\textsuperscript{th} century, argued that musicians should reproduce the composer's work with respect for historical facts instead of serving their own career interests. The performer is expected to have a strictly objective approach known as “execution”, which refers to following an explicit will without adding anything beyond its specific commands (Stravinsky, 1970).



The unequal relationship between performing musicians and authoritative composers and scores has several consequences within the WCM-ideology, including within conservatoire training. First of all, a musician's critical and reflective ability will often remain weak in an environment where they do not learn to make independent choices in their own practice. For example, performers are afraid to deviate from contemporary performance norms, and instead conform to agreed-upon expectations of how particular composers' styles should sound (Stam, 2019, pp. 46-47). It is important to note that classical music education is generally a one-on-one affair rather than a group activity. Students therefore see their teacher as the indicator of what is correct or incorrect, and make few choices of their own, thereby largely anticipating the judgment of others (Hill, 2018, pp. 163-166). This conformity is reinforced through examination testing, re-emphasizing these norms and values. The assessments provide a one-sided perspective of the program from which teachers teach and tend to reinforce both the work concept and conventional tastes (Hill, 2018, pp. 163-166). These systems stimulate technical skills and knowledge of classical music and also guarantee the successful implementation of performance standards, but leave students little room to choose their own path and discover other interpretations. The pressure to conform to contemporary performance practice or to take a creative stance regarding a score is an act of non-conformity within the ideology, which is often socially undesirable within training and considered to be a misconception by gatekeepers\footnote{The performance of WCM is closely regulated by various individuals and entities involved in the profession. These gatekeepers work together to uphold established performance standards and norms within WCM ideology. This ideology aims to limit diversity and encourages conformist execution based on established norms. Its purpose is to regulate the boundaries of acceptability within the industry and promote performances that align with predetermined standards (Leech-Wilkinson, 2020). } like teachers, concert programmers and examiners. Hill (2018, pp. 4-30; Leech-Wilkinson, 2020) argues that making space for diverse expressions of a classical work helps musicians to question value judgments and encourages them to explore positions of their own. Musicians in this context can experience more freedom to experiment and dare to take creative risks when performing. These are abilities that, in my view, allow the performer to “rediscover” a classical work and make it meaningful in potentially infinite ways. This is done not as a reaction against WCM tradition, but in addition to it. Musicologist Christopher Small (2011, pp. 26-34) suggests that the processes of composing, practicing and rehearsing, performing, and listening are not separate activities but interconnected by what he calls \emph{musicking} (Small, 2011, p. 26): “To music is to take part, in any capacity, in a musical performance, whether by performing, by listening, by rehearsing or practicing, by providing material for performance (what is called composing), or by dancing.” All individuals possess a theory of musicking, whether they are conscious of it or not. This theory encompasses their understanding of what musicking entails, what it does not entail, and its significance in their lives. When this theory remains unexamined and unacknowledged, it not only influences and confines people's musical endeavors, but also exposes them to manipulation by individuals with ulterior motives such as power, social standing, or financial gain.



In \emph{De }\emph{Vlucht}\emph{ van de }\emph{Nachtegaal}\emph{ }(The Nightingale's Flight; 2019), philosopher Marlies de Munck argues in favor of personal choice. This could be as faithful as possible to the notated score or completely autonomous from it. This applies not only to professional musicians, but also to those training at the conservatory and in the master-apprentice relationships between teachers and students. Musicologist Daniel Leech-Wilkinson (2020) stresses: “An ethical music school should not be gatekeeping to suit the gatekeepers: it should be putting pressure on them to rise to the challenge of thinking and responding as imaginatively as its students.” In contemporary performance practice, musicians experience pressure to comply perfectly with the composer's perceived intentions, as well to win the approval of other gatekeepers. Students strive for perfect execution and are afraid to receive negative feedback, such as that given to Kathia Buniatishvili. Pedagogue-philosopher Gert Biesta (2017, p. 50) argues that when the teacher shows an interest in the freedom of a pupil, the teacher must control their own desire to control. The pupil is not an object, but rather a subject that enters the world in a “grown-up way”. What the student does with the information obtained is up to them. Education should never be about control. After all, creating and playing music is a social practice (Regelski, 2016). Education must create a “pedagogical space” in which possibilities are tested that can be valuable for both the student and society (Masschelein \& Simons, 2019, pp. 349-366). But what is needed for this sort of experimentation within pedagogical spaces?



Hill (2018, pp. 4-30) proposes six important conditions for creation: generativity (the ability to create), agency, interaction, nonconformity (the freedom to differ), recycling (the new use of existing ideas), and flow. In “Creating Safe Spaces for Music Learning,” Hendricks et al. (2014) propose a practical approach to creating safe and positive learning environments for music students. They emphasize the importance of effective teaching strategies, attitudes and behaviors that encourage trust and respect, experimentation, risk-taking, and self-expression. On the other hand, Boost Rom (1998) suggested two decades ago that education should not prioritize safety and comfort. In order to be ready for the world beyond the classroom, students must receive constructive criticism and face challenges that help them refine their own point of view. Many organizations claim to provide “safe and brave” spaces, but according to Elise Ahenkorah (2020), this promise is not practical. To promote inclusivity and understanding of diverse experiences in real-time, it is better to adopt accountable space guidelines. “Accountability means being responsible for yourself, your intentions, words, and actions. It means entering a space with good intentions, but understanding that aligning your intent with action is the true test of commitment” (Ahenkorah, 2020).



\section{Method }



\subsection{Adventure is out there.}



\noindent\emph{Navigating the Path to Creativity in Classical Music through Artistic Experiments in Practice-Led Research}
\\


\noindent The pressure to conform to contemporary performance practice and the approval of gatekeepers can limit the creativity and individuality of musicians. But what if we look at performances as paths to creativity? By conducting various artistic experiments, that focus on stimulating creativity and establishing a dialogue, I explore how the classical music student might be able to find a balance between tradition and personal interpretation in the performance of WCM. The experiments in MUSIC LAB, experienced individually and in groups, are partly based on methods from \emph{Challenging Performance} by Leech-Wilkinson (2020)\footnote{\emph{Challenging Performance} is an eBook which critically examines the various ways in which performers are restricted from expressing creativity or inventiveness in their interpretations of classical scores. It aims to inspire performers to explore a wider range of interpretations that bring musical meaning to classical scores. This is illustrated through numerous (audio) examples of performances and tips on how to perform scores in various ways.} and on methods from the 18\textsuperscript{th} century. For instance, the experiment Phrase, based on a theatre exercise, asks musicians rephrase the same melody as often as possible with different characteristics, dynamics or articulation (see Nick Hern Books, 2015). Another example is Quodlibet, which is a musical composition or performance that combines several different melodies. Applying aspects of one part of a score to another (and vice versa) was common practice in music education during the Baroque period. It is a type of “musical mash-up” that often includes popular songs or tunes that are familiar to the audience. The term “quodlibet” derives from the Latin phrase \emph{quod}\emph{ }\emph{libet}, meaning “whatever you please.” These kinds of experiments challenge the musician to deal creatively with an existing work and to immerse themselves in the various possibilities for interpretation and expression, even and especially where they fall outside the norms and values of WCM. The performer's sense of liberty lies in their choice to deviate from established conventions, concepts and interpretations found in preexisting scores, recordings, or transcriptions. This entails embracing a greater degree of interpretive freedom, being conscious of this freedom, and assuming a creative role within a musical composition at their own discretion, allowing themselves to take risks and explore. While there is no inherent reason to avoid them, the experiments do not suggest incorporating other musical genres in classical performance like jazz or improvisation. This is because there is a risk of intersection and being confined to a safe category that no longer challenges mainstream practices. Any ready-made category limits what can be achieved and no longer pushes boundaries. Therefore, it is sensible to avoid work that can be easily labeled as an “arrangement” for now, or at least to resist that label strongly (Leech-Wilkinson, 2020).



In addition to the creative aspect, MUSIC LAB also offers space for dialogue and reflection. In order to ensure a safe environment for experimentation, dialogue and reflection, it is important to establish accountable space guidelines for all participants, including educators. Musicians need to be encouraged to see challenges as a part of the learning process, allowing them to experiment, reflect, and grow from their experiences. By emphasizing the importance of respectful communication, we ensure that each participant has an opportunity to speak without unnecessary pressure. By actively listening to others, one can better understand different perspectives and engage in meaningful dialogue. Based on open questions, which are called “nutcrackers”, students reflect on their own and others' actions after each experiment. These questions can, among other things, address the way the musician performs with respect to sound or technique, or the individual choices musicians make in the classical piece while experimenting. In addition, self-reflection plays an important role in the accountable spaces. Musicians are encouraged to engage in introspection and reflect on their own actions, words, and biases, allowing them to gain a deeper understanding of their own perspectives and how they may impact others in a learning mindset.



Embracing the mindset of pushing boundaries is crucial not just for individual artists, but also for collective performance styles and habits. When we collectively strive for innovation, we foster an environment that promotes experimentation and growth. This enables the performing arts to adapt and stay current in a constantly evolving world.











\section{Analysis}



Through MUSIC LAB, interviews, and observations, I explored the thoughts and actions of (student) musicians on training and how they make independent choices when taking a first step in searching for personal creativity in the performance of classical works. The project provides new insights about approaching classical music through experimental methods.







\subsection{“But I don't think playing according to the norms and traditions is really making art”\protect\footnotemark}
\footnotetext{Student, personal communication, June 07, 2022.}




\subsubsection{Student insights on experimental classical music approaches}



Experiments provide multiple perspectives and more tools to approach a piece, but the outcome can also be frustrating. There is the difficulty of letting go of what they already know in order to enter an experiment freely and openly and with a nonconformist attitude. Students highlight the importance of having a creative idea and musical opinion, it empowers them and can also lead to progress also with regard to technical skills. Students reflect on, and become more aware of, the WCM culture and system and how this can make them unsure of themselves and hinder creativity.

\subsection{“It would be nice to be able to talk to them as if they were coaches and not as if they are like… holy beings who judge you at the end, right?”\protect\footnotemark}

\footnotetext{Student, personal communication, June 07, 2022.}

\subsubsection{Creating Nonconformist Music Learning space for Experimentation: A desire among students in main subject lessons}



My research indicates that musicians perceive the pedagogical climate, didactic action, and the accompanying critical and reflective capacity of a teacher in different ways. During an individual main subject lesson there is primarily a clear and strict master-apprentice attitude with the teacher, and no relationship to the teacher as supervisor. “To start the discussion with a teacher doesn't help. It is impossible. They are really convinced about a performance practice, so there's only one way” (Student, personal communication, June 07, 2022). There is little room for starting a dialogue or discussion about different performance practices. Teachers tend to have a set frame of reference, so students must possess a lot of knowledge before being considered credible. “I mean, that could be partly the student, but it could also be that you're not taught to explore your opinion. You are just taught to follow instructions” (Student, personal communication, June 07, 2022). This allows little space for independence in making choices and creativity among students. Students, however, expect a teacher's support in making musical choices, and a greater focus on their achieving a ‘be your own teacher' perspective. They express a desire for a wide nonconformist learning space to experiment without feeling judged or pressured. They feel limited in their ability to experiment during a main subject lesson and wish for more space to try things out. Students also believe that an accountable safe space should be a trial-and-error environment where they can freely express their opinions and ideas. In addition, they desire more creativity and sincerity in their playing.



\section{Conclusion}



The tension between tradition and creativity in classical music is ongoing, but musicians should not need to choose between them. Experimentation and artistic exploration can lead to new possibilities and interpretations of classical works, and can help students develop critical and reflective skills. This may also provide a deeper understanding of how to analyze one's own artistic work through the lens of self-guided learning. However, creating and embracing safe, accountable and nonconformist music learning spaces for experimentation is crucial, and teachers should focus on supporting students in making independent choices and expressing their own musical opinions. By navigating the path to creativity in classical music through artistic experiments, musicians can explore a balance between tradition and personal interpretation, create meaningful authentic performances, and contribute to the evolution of WCM.















































\section{References}


\hspace*{\parindent} Ahenkorah, E. (2020, September 21). Safe and Brave Spaces Don't Work (and What You Can Do Instead). \emph{Medium}. \href{https://medium.com/@elise.k.ahen/safe-and-brave-spaces-dont-work-and-what-you-can-do-instead-f265aa339aff}{https://medium.com/@elise.k.ahen/safe-and-brave-spaces-dont-work-and-what-you-can-do-instead-f265aa339aff}



Biesta, G. (2017). \emph{Door kunst onderwezen willen worden: Kunsteducatie 'na' Joseph Beuys}. Artez Press.



Boost Rom, R. (1998). Safe Spaces: Reflections on a Educational Metaphor. \emph{Journal of Curriculum Studies, 30}(4), 397-4084. \url{https://doi.org/10.1080/002202798183549}



Cook, N. (2014). \emph{Beyond the Score: Music as Performance}. Oxford University Press. \url{https://doi.org/10.1093/acprof:oso/9780199357406.001.0001}



de Munck, M. (2019). \emph{De vlucht van de nachtegaal: Een filosofisch pleidooi voor de muzikant.} Letterwerk.



de Ruijter, S. (2021, June 19). \emph{The Performers' Library }[video]. Vimeo. \url{https://vimeo.com/560564200}



Goehr, L. (2007). \emph{The Imaginary Museum of Musical Works.} Oxford University Press. \url{https://doi.org/10.1093/0198235410.001.0001}



Hendricks, K. S., Smith, T. D., \& Stanuch, J. (2014). Creating Safe Spaces for Music Learning. \emph{Music Educators Journal, 101}(1), 35-40. \url{https://doi.org/10.1177/0027432114540337}



Hill, J. (2018). \emph{Becoming Creative: Insights from Musicians in a Diverse World.} Oxford University Press. url{https://doi.org/10.1093/oso/9780199365173.001.0001}



Leech-Wilkinson, D. (2020). \emph{Challenging Performance: Classical Music Performance Norms and How to Escape Them} (V2.18 ed.). \url{https://challengingperformance.com}



Masschelein, J., \& Simons, M. (2019). Het failliet van onderwijs op maat: naar pedagogische werkplekken. \emph{Pedagogiek, 39}(3), pp. 349 - 366. \url{https://doi.org/10.5117/PED2019.3.006.MASS}



Nick Hern Books (November 9, 2015). \emph{Actions: The 60-Second Challenge }[video]. YouTube. \url{https://www.youtube.com/watch?v=Ohbh3bo5HDA}



Predota, G. (2021, January 10). \emph{Khatia Buniatishvili: “Beyond the Eccentricity of Planet Pogorelich”}. Retrieved October 2023, from Interlude: \href{https://interlude.hk/khatia-buniatishvili-beyond-the-eccentricity-of-planet-pogorelich/}{https://interlude.hk/khatia-buniatishvili-beyond-the-eccentricity-of-planet-pogorelich/}



Prokofiev, S. S. (2018). Piano Sonata No 7, Precipitato [Song recorded by Khatia Buniatishvili]. (Original work published in 1942)



Regelski, T. (2016). Music, Music Education and Institutional Ideology: A Praxial Philosophy of Musical Sociality. \emph{Action, Criticism, and Theory for Music Education, 15}(2), pp. 10-45.



Small, C. (2011). \emph{Musicking, the meanings of performing and listening.} Wesleyan University Press.



Stam, E. W. (2019). \emph{In Search of a Lost Language: Performing in Early-Recorded Style in Viola and String Quartet Repetoires }[Doctoral dissertation, Leiden University]. Leiden Repository. \url{https://hdl.handle.net/1887/79999}



Stravinsky, I. (1970). \emph{Poetics of Music in the Form of Six Lessons.} Harvard University Press.



Taruskin, R. (1982). On Letting the Music Speak for Itself: Some Reflections on Musicology and Performance. \emph{The Journal of Musicology, 1}(3), pp. 338-349. \url{https://doi.org/10.2307/763881}



\emph{The Performers' Library }(n.d.). Retrieved October 11, 2023, from \url{https://theperformerslibrary.weebly.com}






\end{document}