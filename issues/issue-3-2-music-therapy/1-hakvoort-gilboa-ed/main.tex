\documentclass[authordate, empirical, issue]{jote-new-article}

\usepackage{caption}

\usepackage{tabularx}

\usepackage{graphicx}

\usepackage{hyperref}

\usepackage[backend=biber,style=apa]{biblatex}

\addbibresource{bibliography.bib}


\jotetitle{Untangling Strings}
\keywordsabstract{mistakes, music therapy, Music Therapy Education}
\runningauthor{Hakvoort et al.}
\jname{Journal of Trial \& Error}
\jyear{2023}
\abstracttext{\vspace*{-\baselineskip}}
\paperreceived{October 20, 2023}
\author[1]{\mbox{Laurien Hakvoort}}
\affil[1]{ArtEZ University of the Arts}
\corremail{\href{mailto:LaurienHakvoort@gmail.com}{LaurienHakvoort@gmail.com}}
\corraddress{ArtEZ University of the Arts}
\runningauthor{Hakvoort \& Gilboa}
\author[2]{\mbox{Avi Gilboa}}
\affil[2]{Bar Ilan University}
\paperaccepted{November 18, 2023}
\jwebsite{https://journal.trialanderror.org}
\articletype{Untangling Strings - Editorial}
\paperdoi{10.36850/tegd-e423}
\paperpublished{November 30, 2023}
\paperissued{November 30, 2023}
% TODO: ADD ME
\paperpublisheddate{2023-11-30}
\jvolume{3}
\jissue{2}
\jpages{1--4}

\specialissue{Untangling Strings -- Further Explorations of Mistakes in Music Therapy}


\begin{document}

\begin{frontmatter}
  \maketitle
  \begin{abstract}
    \printabstracttext
  \end{abstract}
\end{frontmatter}






\lettrine{M}{aking} mistakes is human. The way we deal with and acknowledge mistakes is influenced by our cultural background. For some people just the shear idea of facing a mistake is met by losing dignity or denial while other people strive to understand the underlying mechanisms of a mistake. We define mistakes in different ways, we react differently to mistakes, we deal differently with mistakes. But if we can collectively learn from the mistakes we or others make, we can sometimes prevent future missteps, misunderstandings, miscommunications, mis-attunements, or misconceptions.



In 2022 we completed our book “Breaking Strings; Exploration of Mistakes in Music Therapy”. The book was the paramount of our completion of a 6-year period in which we continuously primed, debated, discussed, and investigated mistakes as they occur in music therapy. Along the way, we met hesitant people who debated whether the profession was ready for sharing its mistakes; we met people who perceived the sharing of mistakes solely as a supervisor's expertise; and we met people who denied that mistakes could be made in music therapy. But the largest group we met were music therapists that were relieved, happy, eager, and thankful to read about, learn, and understand different mistakes that could occur during music therapy. The reviews of this book were positive, underlining the importance to continue sharing mistakes to develop professional standards of music therapy (e.g., Nye, 2023; Ubbels, 2023; Wheeler, 2023).



In our conclusion chapter, which we called “Final Chord” we speculated on what the next “chord” should be, on how to continue to support the music therapy community to address mistakes that occurred during their practice. After the book was released, we received reactions of recognition, shared experiences and better understanding and frameworks of how to deal with mistakes of many colleagues. The notion that music can not only help, but under certain circumstances, also harm, urged us to continue the movement started by the release of the book Breaking strings: Explorations of mistakes in music therapy. New examples of mistakes we did not address were shared with us, as well as reactions of music therapists who did not agree with the content of our book at all. Not only music therapists were interested, also professionals from adjacent fields such as music educators and musicians reached out to us, sharing their visions on mistakes that they encountered in their professional lives.



One extraordinary reaction came from Dr. Stefan Gaillard, Editor-in-Chief of the Journal of Trial and Error (JOTE). Our book drew his attention and the novelty of the topic met with what was so dear to his heart, sharing information about and experience of trial and error. The journal he edits has been focusing on error, failure, mistake, and similar phenomena, as they occur in the different realms of science. In JOTE, mistakes are referred to with respect and the curiosity and wonder that we so much connected to in our book. Since we suggested that it would be fruitful in the future to “(s)hare the findings from the research and practical initiatives with other related professions” (Gilboa \& Hakvoort, 2022, p. 368), he suggested us to pitch a proposal on a special issue on “mistakes in music (therapy)”. Since we understood that there were still topics we had not covered in the book, and that we could now address not only music therapists, but also professionals from adjacent professions, we were happy to take the challenge and to open up a call- for a special issue about mistakes in music and music therapy. We titled the special issue “Untangling strings -- Further explorations of mistakes in music therapy” so that it resonates with the “breaking strings” metaphor that we presented in our book. We felt that the book was just a beginning, and that we should continue the momentum that it and previous seminars and workshops started. More attention and awareness to mistakes would enable music-users to untangle those strings the book pointed at as broken.



We invited music therapists, but also practitioners from other music related professions to contribute to the special issue. We looked for diverse angles and perspectives on mistakes in music (therapy), different contents, and different forms of (academic) writing. “Breaking strings” includes different types of chapters, reflective, theoretical, and case chapters, clinical vignettes linked to theoretical chapters, and chapters covering mistake related topics from different angles. These different types and topics formed the point of departure for the journal. We invited authors to expand on the subjects of “Breaking strings” or write on topics that the book did not yet cover that they perceived as essential. We did invite them to choose from a variety of article types (e.g., Case studies, clinical vignettes, empirical research, theoretical articles, opinion pieces). We were happy to receive a considerate number of submissions to this special issue, mostly from music therapists but also from adjacent professions. Submissions fell into two main formats: ‘opinion pieces' (Wiess and de Ruijter) and ‘case studies' (Jurkiewicz and Fiers) including the shorter form of ‘clinical vignettes' (Dassa, Subiantoro and Hadar).



The first section of the journal covers two opinion articles from two different disciplines, music therapy and conservatory level classical music education. In music therapy, mistakes can be found not only in the content of the session we offer, but also in our choices on how to structure those sessions. In her opinion article ‘\emph{To structure or not to structure, that is the question: Mistakes made in music therapy in light of the dilemma of whether or not therapy sessions should be structured'} Chava Wiess provides four different examples of what could happen when the music in a session is structured when it should have been unstructured, and vice versa. She shows how her encounters with such mistakes and her observations upon them developed her awareness and eventually supported her and her supervisees' theoretical and practical skills and knowledge. The reflection on the under- or overstructured music therapeutic interventions made her focus in more detail on the needs of clients and how that should be the main factor influencing the choice of interventions. In a second opinion article Sophie de Ruijter, a Western-classical trained flutist, focuses on the educational perspective of classical music. In ‘\emph{Finding harmony: Embracing the “Wrong” in classical music interpretation. About finding balance between tradition and creativity in classical music performance}' she describes how (classical) music education might limit musical interpretation and expression. She dives into the interpretation of what is ‘right' and what is ‘wrong' in musical performances and how music education at conservatory of music level might restrain young musicians from meeting their own creativity and expression. She suggests new educational concepts to open the classical music dogma to a new era.



The second section of the journal covers case studies and clinical vignettes featuring short examples of clinical work that went astray in one respect or another. In her case study article ‘\emph{Towards a perfect tune- navigating the notions of failure, mistake and competence in }\emph{Nordoff}\emph{ and Robbins music therapy with }\emph{marginalised}\emph{ mothers' }Afra Jurkiewicz discusses how understanding musical values or norms like right and wrong has a direct impact on the music therapists when working with mothers whose children have been taken away. The notion that these clients might have, of failing their most basic responsibility as mothers, can have a direct effect on their values as human beings and can impact the music therapist and the treatment they are offered. How can a music therapist ensure that failing again in life is minimized, while still meeting the challenges of making music. A music therapist might feel trapped from the beginning in the emotional rollercoaster of ‘perfect' musical songs and inability to play them. Finding a way out has been a challenge that Jurkiewicz shares and discusses. The bitter notion that this project was discontinued further adds another layer of mistakes in this article: mistakes made by society. Nele Fiers provides in a second case study of a mistakes made due to countertransference and how it was played out in the music of a group music therapy. In ‘\emph{“Not again”: when the therapist resists'} she takes the readers along into her musical journey with a group of clients who reacted to a group member who was living with obsessive compulsive disorder. The endless musical improvisations triggered severe reactions in the music therapist, who felt unable to contain all the needs of the clients, secluding herself from the music. Realizing her mistake enabled her to take new steps and support her client(s) on a different level.



The three shorter case studies are presented as clinical vignettes. Ayelet Dassa, in her clinical vignette, directs her attention to the termination of music therapy with one of her resistive clients, a termination that she eventually perceived as a mistake. In her article \emph{‘The music must play on -- The music therapy sessions that should not have stopped' }she raises the important consideration of how the therapist might not be aware of the impact that music therapy might have on what seems like a resistive client. In her article, she asks questions such as what happens if we make presumptions about our client's (limited) progress or their musical choices? How could we assess that in more depth and realize how important the musical bridge might be that we perceive as uneventful? In another clinical vignette Monica Subiantoro confronts us as musicians with the fact that our clients might not be as music-focused or music-centered as we are. Musician's expectations and ideas of the importance of music might be parts of a cultural inheritance that might be bestowed upon us, without us realizing it. Her unique clientele of highly verbal women on the autism spectrum confronted Monica with her own biases and ‘western cultural' perspectives on the importance of music. Because the participants of an online music therapy support group did not share this, she started considering the values and norms that could be valuable for each musician to consider. Tamar Hadar provides us with another culturally oriented clinical vignette ‘\emph{“But I didn't understand your handwriting”! Uncovering the significance of therapy progress notes for parents in music therapy'.} She describes how the notes she wrote to her client's father led to his unexpected reaction, simply because he could not read and understand them. The case makes us realize how often as therapists or music educators we presume we are clear, and others will clearly understand what we are doing musically . Yet, our ‘notes' might be incomprehensible for others and how we communicate about our work might be hard to understand for non-musicians and non-music therapists. This case stretches further than music therapy alone and can be generalized to other professions as well.



The authors provide us with considerations we did not address in “Breaking strings”. The articles of this journal covered more general mistakes, not linked to music alone. Mistakes in terminating therapy too early, realizing that clients are not that involved in music, or even hardly understand the music component. Inability of clients to understand what a practitioner has been writing down for them; mistakes that could occur in music education, but also with a general practitioner. Mistakes made in society that reflect in music therapy was not a topic in the book, yet it is addressed in this special issue. And of course, articles about mistakes in musical phenomena were included. Considering how rigid musical structures can be(come) and what negative impact this might have on either the client, musician, supervisee, music therapist or student is a topic that we started to scrutinize only now. We hope this will expand in the future. Mistakes due to personal feelings of a music therapist towards their client(s) were discussed in the book but got some strong additional examples in this journal.



Though this issue exposed us to new directions of looking at mistakes, there are still many more to be explored in the future. In her review for the Journal of Music Therapy on “breaking strings” Wheeler (2023), for instance, has emphasized that more should be said about supervision as a tool to guide music therapists through their mistakes. We agree with this and would like to expand her suggestion. Not only should we use supervision more as a tool for monitoring and dealing with mistakes, but we should also find the ways to share our mistakes in broader contexts so that others can hear, empathize, and learn that such mistakes exist behind the doors of music therapy rooms. In conferences for instance, where we are used to hearing cases that end in triumph and success, we should also be exposed to cases in which mistakes occurred, and cases that went astray. Realistic examples of what is feasible or not within music therapy might only stress where the strengths of our profession are centered. In his review in the British Journal of Music Therapy on “breaking strings” Nye (2023) added another topic that would be valuable to consider in the future. He points at “…training courses [and how they] might need to adapt, with particular reference to our own cultural biases within music itself” (p. 3). How do we train and what might need to change, so students can continue to learn from mistakes others made, due to cultural or musical biases.



We are thankful that the community of music therapists, music educators, and musicians interested in the topic of mistakes is expanding. We would like to thank the reviewers who took time to read, consider and provide feedback to the authors from their own unique perspectives on mistakes and how that can support professions in their growth and development. Thanks to each one of them, Tania Balil, David Grüning, Carlijn van der Eng, Connie Isenberg-Grzeda, Ai Nakatsuka, Helen Oosthuizen, David Schwartz and Henrike Vonk. So let this special issue be a further encouragement to the music and music therapy community to share, discuss, and publish those mistakes that can support our colleagues to open their eyes and understand what they might overlook in their clinical work. We wish that each of us keeps on being aware of their mistakes, and that we have growing courage to deal with them emotionally and professionally and to share them with others to the benefit of our profession. We would like to thank all authors of this special issue for their courage, and for their willingness to provide a peek into their music (therapy) practice in those moments where mistakes occurred. We hope this will further optimize a safe and effective music therapy practice and music (therapy) education for all our unique service users.








\newpage
\section{References}



Nye, I. (2023). Book Review: Avi Gilboa and Laurien Hakvoort (eds), Breaking Strings: Explorations of Mistakes in Music Therapy. \emph{British Journal of Music Therapy. }\url{https://doi.org/10.1177/13594575231179191}



Ubbels, L. (2023). Boekbespreking [Book review]. Breaking Strings: Explorations of Mistakes in Music Therapy. Avi Gilboa and Laurien Hakvoort. \emph{Tijdschrift}\emph{ }\emph{voor}\emph{ }\emph{VakTherapie}\emph{ 19}(2), 38-39.



Wheeler, B. L. (2023). Book review. Breaking Strings: Explorations of Mistakes in Music Therapy. \emph{Journal of Music Therapy. }\url{https://doi.org/10.1093/jmt/thad023}


\end{document}