\documentclass[authordate, empirical, issue]{jote-new-article}

\usepackage{caption}

\usepackage{tabularx}

\usepackage{graphicx}

\usepackage{hyperref}

\usepackage[backend=biber,style=apa]{biblatex}


\jotetitle{"Towards a Perfect Tune": Navigating the Notions of Failure, Mistake and Competence in Nordoff-Robbins Music Therapy with Marginalized Mothers}
\footertitle{"Towards a Perfect Tune": Navigating the Notions of Failure, Mistake and Competence in Nordoff-Robbins Music Therapy with Marginalized Mothers}
\keywordsabstract{marginalized women, failure, stigma, mistakes, music therapy}
\abstracttext{Women who have experienced their children being permanently removed from their care due to challenging life circumstances often live on the margins of society. The majority have experienced mental and physical health challenges and face many other intersecting issues. The stigma associated with losing children, coupled with a lack of support, means that these women are at risk of serious social exclusion, further exacerbating their feelings of failure, grief, and loss. The existing music therapy literature on women mostly focuses on their experiences in the context of intimate partner violence; music therapy with marginalized mothers appears hitherto unreported. This article aims to explore how notions of musical right and wrong often played into my work with the women as a Nordoff Robbins music therapist. Most of these women come from disadvantaged backgrounds and have received little or no education: involvement in music-making often evoked perceptions of correctness and progress. Women often requested that I taught them songs they liked. This could enhance their sense of failure if they were unable to play the songs as they knew them. The question of how not to perpetuate harm, whilst acknowledging my clients’ needs, became my dilemma as a therapist. The case studies discussed below highlight how, in the context of these women’s lives, experiencing doing something “right” and proving their own capacity to learn through music, could become key themes in their pathway towards recovery. This process is not, however, a straightforward endeavour and involves negotiation and commitment on the part of both therapist and client. }
\runningauthor{Jurkiewicz}
\jname{Journal of Trial \& Error}
\jyear{2023}
\paperdoi{10.36850/4jnj-aa33}
\paperreceived{May 29, 2023}
\paperaccepted{October 22, 2023}
\paperpublished{December 22, 2023}
\paperpublisheddate{2023-12-22}
\paperissued{December 22, 2023}
\paperdoi{10.36850/e21}
\author[1]{\mbox{Afra Jurkiewicz\orcid{0009-0004-6138-8392}}}
\affil[1]{Nordoff Robbins}
\corremail{\href{mailto:afraanna.jurkiewicz@nordoff-robbins.org.uk}{afraanna.jurkiewicz@nordoff-robbins.org.uk}}
\corraddress{Nordoff Robbins}
\runningauthor{Jurkiewicz}
\jwebsite{https://journal.trialanderror.org}
\articletype{Untangling Strings - Case Study}
\jvolume{3}
\jissue{2}
\jpages{22--28}


\specialissue{Untangling Strings -- Further Explorations of Mistakes in Music Therapy}

\begin{document}


\pdfbookmark[0]{Afra Jurkiewicz (2023). "Towards a Perfect Tune": Navigating the Notions of Failure, Mistake and Competence in Nordoff-Robbins Music Therapy with Marginalized Mothers}{afra}
\begin{frontmatter}
  \maketitle
  \begin{abstract}
    \printabstracttext
  \end{abstract}
\end{frontmatter}

\setcounter{page}{22}

\section{“Set up to fail”: a brief profile of marginalized mothers}







During the past decade, over 95,000 mothers were involved in care proceedings in England and Wales (Alrouh et al., 2022, p.3). It is common that parents who have had a child removed once, will go on to experience repeated removals of children from their care. Approximately 1 in 4 mothers who have had a child removed once, are at risk of returning to court for subsequent care proceedings (Alrouh et al., 2022, p.1). Until recently, birth mothers who experienced a permanent child removal received no follow-up care in relation to their own unmet needs (Cox, 2012). There have been growing attempts to understand the challenges that this particular group of women face in relation to both pre- and post-removal circumstances (Broadhurst \& Mason, 2013, 2017, 2022). A large percentage of women who find themselves in this situation were brought up in care themselves and have experienced an intersection of challenging life circumstances including mental health difficulties, intimate partner abuse, homelessness, and substance addiction. As a result of child removal, women experience further adversities including long lasting grief and loss, social and professional stigmatization, and reductions in welfare entitlements. Moreover, women suffer from the loss of a mothering role -- a role which, in the context of women's “fragile and restrictive social statuses” is perceived to hold meaning, purpose and structure (Broadhurst \& Mason, 2020, p.26). Women's experiences of “failure” are complex and multi-layered.







“How does a parent explain absence of children to other parents whom he or she previously met at the school gates, or to neighbors? As the “failed” parent looks in on the routine family life of others, this also serves as a daily and painful status of the parent's own children -- stigma intersects with loss” (Broadhurst \& Mason, 2017, p. 48).







According to Pause, an organization which supports marginalized mothers across the UK, women experience painful psychosocial consequences and are provided with scarce and inadequate support after the child removal proceedings. Women are “set up to fail” even further: the psychological and psychiatric assessments carried out by expert witnesses during care proceedings give recommendations for treatments that are often unavailable locally. This disconnect is “harmful to women and deprives them of the chance to make a positive change” (Pause, 2022, p.9). It is clear that women who have experienced child loss face multi-layered challenges that continue to be poorly understood.







\section{Music therapy with marginalized \mbox{mothers}}







Music therapy work with marginalized mothers in the UK appears severely underreported which prompts me to suspect that music therapy is not typically offered to women who have experienced the loss of their children. This could be due to scarcity of services offering structured support to marginalized mothers in general. Music therapists often work with children who have been adopted or who are in foster care (Zanders, 2015; Drake \& Edwards, 2011) and the existing literature on music therapy work with women focuses mostly on the contexts of domestic violence and intimate partner abuse (Curtis, 2016; York \& Curtis, 2015; Hernandez-Ruiz, 2020). The voices of women who have permanently lost their children are rarely heard or explored.







I started working with an organization that supports women who have experienced repeat removals of children from their care in the North-East of England during the 2020 Covid lockdown. We ran it as a pilot project that aimed to understand how music therapy could serve women in such unique circumstances. The organization mostly focused on 1-1 support in the community, offering occasional group experiences like days out to the beach or craft afternoons. Music therapy became an opportunity for weekly 1-1 and group sessions at one of their local hubs, slowly building a space where women could make appointments to experience music and spend time with each other. As a Nordoff Robbins therapist, I focused on building relationships and connections through collaborative music-making with women through activities such as instrumental and vocal improvisation, singing their favorite songs or writing new ones.







The staff members who worked closely with individual women played a crucial role during the initial stages of our project. Without their encouragement and support, not many women would have had the confidence and trust to independently come and explore active music-making. However, once a few women had experienced music therapy, the word started spreading and it wasn't long before Thursdays had famously become the “music” days.







Most of the women who came to music therapy sessions had not participated in any music-making activities since school. Some of the women talked about growing up with music or having fond memories of making music with their family members in the past. In most cases, however, ties with music had been broken as a result of certain choices (i.e., going out with friends rather than practicing guitar), lack of parental support, care responsibilities towards their siblings, and other challenges. Their engagement with music had become limited to listening to music on YouTube or TikTok, mostly alone and at home. Music therapy sessions became a chance to explore being together musically in ways that felt comfortable for them. Whilst for some, improvisation was the most immediate way into making music, for others, singing their favorite songs was a much more accessible start to our musical relationship.







In those initial phases, it was noticeable how the women's awareness of their limited education linked to their perceived lack of musical competence. Some women, after a few beats of playing on the drum, for example, would put the beaters away, saying they “couldn't do it”, describing themselves as “thick” or “dumb”. Whilst some women never came back to therapy after the first session, others only needed time and continuity to start shifting these internalized and harmful self-perceptions.







For me, as a music therapist inspired by feminist and anti-oppressive perspectives to music therapy (Hadley, 2006; Baines, 2013; Seabrook, 2019), I hoped that our work would have the potential to change women's self-perceptions in relation to their musical abilities. It feels relevant to mention that I am a classically trained pianist from Poland, who was raised in a working-class family and had the opportunity to access free musical education provided by the state. The privilege to learn did not come without its own challenges. On many occasions, throughout my educational journey, I was told that I wasn't a “good enough” pianist and would never make it to the land of musical excellence: the prominent Academy of Music. Whilst there was a clear difference in our backgrounds, histories, and access to education, I could empathize with these women's feelings of inferiority when it came to playing music.







Music therapy became a space to gently restore their bond with music and build new connections with themselves and other women through music. Group music therapy sessions became the ground “for the collective sharing of experience” (Broadhurst \& Mason, 2020, p.33), through song-writing, creating a Podcast series, and taking part in various musical events. The group consisted mostly of women who had completed an 18-month programme and were taking part in the organization's “Next Steps” initiative allowing them to take part in some activities while receiving continuous support.



Individual music therapy sessions, on the other hand, were offered mostly to women at the beginning of the programme, who often struggled in group settings and needed a more personalized approach and individualized attention. With these women, therapeutic aims were negotiated in each session; their lives outside of music were often chaotic due to housing issues, court appointments or poor mental health. Their attendance often wasn't consistent so our aims were focused on the here-and-now.







In the following case studies, I will introduce readers to two women with whom I had the privilege to work with on an individual basis. Despite their complicated and fragile life circumstances, they both became regular music therapy attendees. I will discuss how the notions of right and wrong, learning and mistakes, informed and shaped our work together.



Both women agreed to have their stories shared for the purpose of this article.







\section{Experiencing musical competency through therapeutic learning}







M, a 39-year-old woman from Eastern Europe who had all of her children removed as a result of domestic violence, was one of the few women who regularly attended music therapy sessions. She didn't have a social circle around her and lived alone with her cat, surrounded by a collection of images of her children which constantly reminded her of what she had lost. M had never been to school back in her home country and she was raised in care herself. Her motivation in therapy was to play the piano: in our first sessions, she told me that if I taught her how to play, she could “\emph{make a little money}” on TikTok. Whilst I explained we wouldn't make videos for sharing during therapy sessions, I would be happy to teach her the song she showed me on YouTube. I simplified the tune and presented it to her in small chunks, but I quickly understood that it would be hard for her to feel that she was playing the “actual” song; in such a slow tempo, these fragments were hardly recognizable. As we worked through the tune, she looked puzzled, struggling to, in her own words, “\emph{get it right}”. I eventually led her into an improvisation based on the song's harmony and she sang her version of the song while adding piano solos loosely based on what I had previously shown her. Through improvisation, I wanted her to experience musical freedom without worrying about the correct notes and fluency. On the other hand, I was acutely aware that by accompanying her in a way which would be musically satisfying, I was exposing my own proficiency on the piano and potentially undermining M's hopes and dreams in relation to learning. She said to me at one point: “\emph{I will never be able to play like you, I never played the piano}”\emph{.}







The shift from coming to sessions to achieve a very practical result (having something to show on TikTok) to finding a much wider meaning in music-making became apparent in our sessions later on, when M shared:







\begin{quote}
  “Oh, when I was smaller, I like piano. To play piano, but never did have this chance to have one. People who can help me, like to teach me, how you teach me, like a little bit […] When I am boring, I like to put in my brain what I learned from you, what I play here. I got home and no, 100 \% I don't remember. But when I put my finger, I remember how you put the finger, how you teach me”.
\end{quote}







M was aware of the limitations to her learning, but she was no longer striving for perfection. I think it was precisely the thought that she \emph{could learn }something if she had the \emph{right help} that made her come back every week. I was lucky to find a donated keyboard which I offered to her to take home; this object became emblematic, reflecting the person-centered care we were able to offer to M.







Musical improvisation was of particular importance to allow M to experience her musicality beyond “right” and “wrong” (Ansdell, 1995). Whilst learning songs can be a valid aim in music therapy, as music-centered music therapists, we search for the opportunities to open up the world of musical affordances through improvisation (Procter, 2017; Ansdell, 2014). In the context of her life, M deserved to experience flow, musical satisfaction and meaningfulness to counter her continuous struggle, grief and loss.







With time, improvisation also allowed more playfulness in our music-making, helping her to find her own way of playing and experimenting musically. As her confidence strengthened, the differences in our musical proficiency became less relevant; what mattered was what we could create, communicate, and express musically together.







M told me that she felt good in sessions but that as soon as she left the room, different thoughts (including ones about her children) would cast a shadow over the positive feelings she had had in music therapy:







\begin{quote}
  “But when I go home, I see the piano on the table, I remember. Oh, I come from the music. Let's play. Let's do again. Yes, I'm like that ...... You help me here. Because if I didn't come here to learn something, I didn't learn in my house what I do, I'd just play like I was a child you know, a little bit”.
\end{quote}







Music therapy became a space for musical and personal growth for M. Whilst achieving a particular standard as such would not be an aim of music therapy, experiencing learning and experiencing competency,\emph{ not playing like a child }but playing playfully, became a particularly poignant therapeutic outcome.







M's motivation to learn the piano also became a steppingstone towards changing other professionals' perceptions of her, as we hear from D, her support worker:







\begin{quote}
  "It's like I said to you, it's amazing when I fed back to a social worker…I recently gave a bit of an update and said that she was learning to play the keyboard and they couldn't believe it. I was like, wow, that is like absolutely amazing. You know, completely the opposite to the outlook they have about certain women. I think it's quite surprising”.
\end{quote}







I believe that the surprised reaction of the social worker could have been caused by the unhopeful outlook she had had on M before referring her to the organization. Many women are caught in negative cycles of destructive behavior and without the appropriate support systems in place, it becomes hard for them to make a positive change. Perhaps, for a social worker whose job is often focused on practical problem-solving, it came as a surprise that music and creativity could become a vessel for change in M's case.







\section{“Doing something right” -- music-making as a reparative experience of a fragile “self”}







I worked with a woman in her mid 20s, who has a diagnosis of both personality disorder and post-traumatic stress disorder (PTSD), and who lost her three children to adoption. C's fragile mental health, as well as her partner's involvement with the justice system, led to the court's decision to permanently remove the children from her care. C came to music therapy as she had always enjoyed music but lacked confidence to sing. She shared how she felt judged by others and how her attempts to make music outside of our sessions had been \emph{“a bit full on”}. C used to learn guitar at school, but her connection to music was interrupted early on by caring responsibilities for her siblings.







As we began the session, I asked what she would like to play and she pointed to the xylophone, adding a playful: \emph{“I don't know what I'm gonna make up today, you know”. }



She picked up the stick, played an “A” three times and said: “\emph{Sound like Jingle Bells, didn't we?”. }She continued to play while I set up the keyboard. “\emph{I'm trying to get the tune, trying to do Jingle Bells, but I can only get that. Jin-gle-bells and what's that? What would the rest of [it] be?”.}







I taught her the tune from my keyboard, note by note, and then in longer chunks. She was able to repeat the first phrase and exclaimed a triumphant: “\emph{I've done it, I have done that}!”. I asked her to repeat it so I could accompany her to enhance her achievement and offer her the experience of “soloist”. We went on to the next part and tried to join the two phrases together, but it proved more complicated; she quickly forgot the second phrase of the tune and looked up at me, appearing lost. I started to feel in doubt: this was not a music lesson; this was music therapy! I was not sure how to teach her: I wanted to give her what she wanted, but I would have hated for her to feel undermined if she made continuous “mistakes”.\footnote{ To maintain authenticity, I used the quotation marks when referring to mistakes as this was the language C. used in sessions.} We were already in the process, so I decided to go through the chorus with her a few more times, leading C with my voice while underpinning her melody harmonically. It sounded messy and fragmented, as every time she faltered in the second part, she would return instead to the bit she knew. When we played through the tune which sounded (more or less) like Jingle Bell's chorus, she laughed. She said: \emph{“I missed the tone a little bit”}, to which I didn't respond verbally (not wanting to draw attention to her “mistake”) but continued playing the upbeat chord progression leading her into the beginning of the tune. I didn't cue her in verbally this time; she quietly revised the notes as she played, and I underpinned whatever she was playing with harmony. Our intentions in that moment seemed mismatched: I wanted to help her flow, and she wanted to do it “right”.







I feel that my well-intentioned move away from teaching towards improvisation and flow seemed, at that stage, mismatched with C's way of wanting to connect with me and with music in that moment. What seemed to matter most for her was trying\emph{ }and her perception of getting better. She helped me to recognize how being supported in the process of learning was helping her to feel safe with me and in the music:







\begin{quote}
  “Like I feel comfortable that like, you're constantly repeating it and you're not just doing it one time and leaving me to try and do it on my own. Yet you're constantly helping us”.
\end{quote}



In the context of C's traumatic experiences in childhood and later experiences in life, it became clear how experiencing learning in a safe environment was pivotal to her self-esteem and confidence. Music therapy became a space to make mistakes “safely”, in a way which would not carry the same terrifying consequences as it had done in her past. Even later, in the context of improvisation, the concept of mistake continued to permeate through her thinking about our music-making together:


\begin{quote}
  “I like the fact that we'll make me own tunes up and your concentration on your face where you're concentrating on your piano. But then you're also watching me. So, if I was to make a mistake, you could easily correct us”.
\end{quote}







I would never think of “correcting” anything in our improvisations, as from my experience as a music therapist, any “mistake” or unwanted or jarring note could lead to interesting and playful musical territories. I think, however, that it was extremely important for her to feel that I continued to be there in music with her\emph{ }and despite\emph{ }her perceived mistakes. That resulted in C's growing feelings of acceptance and freedom:\emph{ }







\begin{quote}
  “Well, I just feel like I can come out of my shell, like a part of us feels comfortable where I can just...I'd probably be able to sing anything in any tone and you wouldn't judge us for it, even if […] the tone wasn't right. Because then you could help us in that direction where I could get the right tone and build it up from there. I can, I don't know, like, I can just be me. I feel free. Like, I can, I could even be silly if I wanted to…”.
\end{quote}







The playfulness and creativity that improvisation and playing \emph{“our own tunes”} afford reveal a new way of being herself for C. There is a newly discovered sense of ownership, personhood and meaning that C reflects on, sharing that:







{\emph{“To me, in the music, like all I want to do is focus on that and nothing else. And it makes us feel good. It makes us know I'm doing something right. And as I'm making my own tunes, I know that it's me doing that, not somebody else”. }}











\section{Reflections}







In the case of M, whilst I tried to create an environment that wasn't “threatening, abusive or exploitative” (Pavlicevic, 1997, p.134), I found myself in a contradictory situation. By exposing my musical proficiency (in order to offer her a satisfying musical experience), I was initially undermining her hopes for achieving fluency on the piano. It clearly points towards the inevitable power dynamics in a therapeutic relationship, where a therapist is both typically more highly trained and afforded more power than the client.







With the aid of musical improvisation, M's openness and willingness to learn, and my commitment and care, we managed to work with this initial tension. After months of individual music therapy sessions, her experiences of learning-by-improvising, helped to strengthen her confidence. M experienced herself as musically competent (Rolvsjord, 2014) and playing piano became a resource that she could use at home to feel less isolated and lonely.







Feminist approaches to music therapy call for music therapists to “work to heal the harm created for individuals by an unjust society, and at the same time they must also work to transform that society” (Curtis, 2012, p.211). The motivation and seriousness that M displayed in relation to learning music, led to a change in how she was perceived by her social worker and contributed to reducing the complex stigma of a “failed” mother. In the second case study, I discussed how C and I negotiated a space in which she could experience making “safe” mistakes, without worrying about the consequences. The consistency of weekly sessions also meant she could always return and try playing again. C had an opportunity to learn music without being judged, which for her meant that she could feel “herself” and “free” to do what she wanted for the first time in her life. Personally, working with C allowed me to understand that making mistakes in music can be a part of therapeutic work.







\section{Conclusion}







Both of the case studies presented in this article suggest that in the context of working with marginalized women and “failed” motherhood, music therapy can be a unique way of supporting the process of recovery through reparative experiences of musical competency, learning and making mistakes. It isn't, however, a straightforward, linear, and unchallenging endeavor. For women who do not tend to have many opportunities to experience the benefits of music-making in their everyday lives, music therapy can fill that gap and become a space for connection, playfulness, and growth.







\section{Implications: whose failure?}







The organization where I provided music therapy experienced severe loss of funding which resulted in significant cuts to the services provided. The organization previously worked with 60 women across the North East and employed several support workers who provided support and care for the women. Currently, the local authorities offer funding which only covers work with 12 women in the region and we are not able to offer music therapy as a regular service. It is clear that without other professionals and organizations who reach out and support marginalized mothers, music therapists will struggle to be accessible. After all, bringing about change in any circumstances has often been about a collective struggle.







As a society, we continuously fail to change the conditions of the most excluded and marginalized people. It would be unjust to write about “failed” mothers without at least mentioning the extent of failure of the socio-political system in which we work and live.



I hope that this article will open up a conversation about music therapy with marginalized mothers, and will inspire other music therapists to seek opportunities for similar work in other parts of the world.

























































\vspace*{\baselineskip}

\section{References}







\hspace*{\parindent}Ansdell, G. (2014). \emph{How music helps in music therapy and everyday life}. Ashgate.







Alrouh, B., Abouelenin, M., Broadhurst, K., Cowley, L., Doebler, S., Farr, I., Cusworth, L., North, L., Hargreaves, C., Akbari, A., Griffights, L. J., \& Ford, D. (2022). \emph{Mothers in recurrent care proceedings: new evidence for England and Wales. Nuffield Family Justice Observatory.} https://www.nuffieldfjo.org.uk/resource/mothers-in-recurrent-careproceedings-new-evidence-for-england-and-wales







Baines, S. (2013). Music therapy as an anti-oppressive practice. \emph{Arts Psychotherapies}, \emph{40} (1) 1-5.







Broadhurst, K. \& Mason, C. (2020). Child removal as the gateway to further adversity: Birth mother accounts of the immediate and enduring collateral consequences of child removal. \emph{Qualitative Social Work}, \emph{19}(1), 15-37.







Broadhurst, K. \& Mason, C. (2013). Maternal outcasts: Raising the profile of women who are vulnerable to successive, compulsory removals of their children -- A plea for preventative action. \emph{Journal of Social Welfare and Family Law,} \emph{35 }(3), 291--304.







Broadhurst, K. \& Mason, C. (2017). Birth parents and the collateral consequences of child removal: Towards a comprehensive framework\emph{. International Journal of Law, Policy and the Family},\emph{ 31}(1), 41--59.







Cox, P. (2012). Marginalized mothers, reproductive autonomy, and “repeat losses to care”. \emph{Journal of Law and Society,} 39, 541--561.







Curtis, S. L. (2016). Music therapy for women who have experienced domestic violence. In J. Edwards (Ed.), \emph{The Oxford handbook of music therapy,} 289--298. Oxford University Press.







Curtis, S. (2007). Claiming voice: Music therapy for childhood sexual abuse survivors. In S. Brooke (Ed.), \emph{The use of creative arts therapies with sexual abuse survivors}, pp. 196--206. Charles C Thomas Publisher.







Drake, T. \& Edwards, J. (2011). Becoming in tune: The use of music therapy to assist the developing bond between traumatised children and their new adoptive parents, \emph{Music therapy and parent-infant bonding}, 101-114.







Hadley, S. (Ed.). (2006). \emph{Feminist perspectives in music therapy}. Barcelona Publishers.







Hernández-Ruiz, E. (2020). Empowering women survivors of domestic violence. \emph{Music Therapy Perspectives}, \emph{38}(1), 3--6.







Pause (2022). \emph{Set up to fail. }Available at: \href{https://www.pause.org.uk/expert-witness-assessments/}{https://www.pause.org.uk/expert-witness-assessments/}







Pavlicevic, M. (1997). \emph{Music therapy in context: music, meaning and relationships}. Jessica Kingsley.













Rolvsjord, R. (2014) The competent client and the complexity of dis-ability. Voices: A World Forum for Music Therapy, \emph{14}(3).











Seabrook, D. (2017). Performing wellness: Playing in the spaces between music therapy and music performance improvisation practices. \emph{Voices: A World Forum for Music Therapy}, \emph{17}(3).







Seabrook, D. (2019). Toward a radical practice: A recuperative critique of improvisation in music therapy using intersectional feminist theory. \emph{Arts in Psychotherapy}, \emph{63}, 1--8.







Whipple, J., \& Lindsey, R. S (1999). Music for the soul: A music therapy program for battered women. \emph{Music Therapy Perspectives}, \emph{17} (2), 61--68.







York, E. \& Curtis, S. L. (2015). Music therapy with women survivors of domestic violence. In: B. Wheeler (Ed.), \emph{Music Therapy Handbook}, 379--389. Guilford Press.







Zanders, M.L. (2015) Music therapy practices and processes with foster-care youth: Formulating an approach to clinical work, \emph{Music Therapy Perspectives}, \emph{33} (2), 97--107.










\end{document}