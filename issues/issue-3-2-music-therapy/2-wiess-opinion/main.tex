\documentclass[authordate, empirical, issue]{jote-new-article}

\jotetitle{To Structure or not to Structure, That is the Question: Mistakes Made in Music Therapy in Light of the Dilemma Whether or not Therapy Sessions Should be Structured}
\keywordsabstract{music therapy, mistakes, structured and unstructured}
\abstracttext{Throughout my professional career as a music therapist, I have always wondered whether music therapy sessions should be structured, and if so, how and when? I had been taught to treat clients dynamically, that is, by musical and verbal responses and interventions as the session unfolded, without prior planning. Would structuring sessions stall the process and will avoid clients from expressing themselves, and going through with the process, or would structuring the sessions will benefit clients by reducing anxiety, creating a holding space, increasing a sense of calm, and advancing the process? In the earlier years of my professional career, I treated clients when there were no professional articles to follow. I had only my intuition to work with, and as a result, I made mistakes! In this article, I will share some clinical vignettes of mistakes that I made as a music therapist, that my student, and a music therapist that I gave him supervision, in light of the dilemma I have mentioned above. It may be that as a result of copping with this dilemma and reflecting on past mistakes, my theoretical orientation solidified. I have come to the conclusion that making mistakes has taught me a lot, and that those mistakes have shaped my therapeutic path.}
\runningauthor{Wiess}
\jname{Journal of Trial \& Error}
\jyear{2023}
\paperdoi{10.36850/qjak-1382}
\jvolume{3}
\jissue{2}
\paperreceived{May 8, 2023}
\author[1]{\mbox{Chava Wiess\orcid{0000-0002-3396-8750}}}
\affil[1]{David Yellin Collage}
\corremail{\href{mailto:chavaw@dyellin.ac.il}{chavaw@dyellin.ac.il}}
\corraddress{David Yellin Collage}
\runningauthor{Wiess}
\paperaccepted{October 9, 2023}
\paperpublished{December 5, 2023}
\paperpublisheddate{2023-12-05}
\jpages{5--15}
\jwebsite{https://journal.trialanderror.org}
\articletype{Entangled Strings - Opinion}

\specialissue{Untangling Strings -- Further Explorations of Mistakes in Music Therapy}

\begin{document}
\begin{frontmatter}
  \maketitle
  \begin{abstract}
    \printabstracttext
  \end{abstract}
\end{frontmatter}

\setcounter{page}{5}

\lettrine{W}{hether} or not to structure sessions and interventions is an issue that has often preoccupied me in my years as a music therapist, supervisor and student mentor in music therapy (Wiess \& Bensimon, 2020). How and when should therapy sessions be structured? What factors should we take into consideration when entering a therapeutic journey with a particular population? How do we attune to clients' emotional states, and how might this attunement inform our therapeutic choices?



There are approaches in psychology that promote structured therapy, especially for children (Drost \& Bailey, 2004; Whitaker, 1985), who need a structured process. Likewise, when therapy is short-term, such as the 12-step program for working with addicts (Kelly et al., 2020), a fixed structure helps to accelerate and focus the process of recovery. Structured sessions are often used for working with trauma, as the structure can help clients process the trauma in a safe and holding environment (Kuhfuss et al., 2021; Peterson et al., 2019; Shapiro, 2017). Structured therapy sessions are also used in war zones, where there is great emotional distress but little time for long processes. In such contexts, it is also important to reduce clients' anxiety and help them cope with their difficulties in a swift, focused way (Mishchykha et.al., 2023). In group therapy for individuals coping with chronic pain and complex diseases, structured sessions are used to offer focused help in reducing pain (Gamsa et al., 1985; Geue et.al., 2010). There are also types of structured psychotherapy that are used to alleviate feelings of loneliness, anxiety and depression (Soloshower et al., 2020; Tarugu et al., 2019). Structured sessions and interventions are also used in cognitive-behavioral psychotherapy, which offer short-term help focused on the symptoms of the problem; it is believed that structuring may help to reduce symptoms more quickly (Sumathipala et al., 2008; Wilson et al., 2005). There are also approaches that advocate a dynamic process, in which sessions are not structured and therapists work with whatever information emerges from clients in the “here and now” (De Maat et al., 2013; Yalom, 2017; 2018).



In this article, I will present four vignettes from my work as a music therapist and supervisor, in which mistakes occurred when choosing if and how to structure sessions and interventions. Some of the mistakes are ones I made myself; others were made by my students or by therapists I supervised. These errors of judgement have been instrumental in shaping my perspective on when and how to use structured therapy sessions and interventions as part of the therapeutic process.



Gilboa (2022) notes that there are five steps a music therapist goes through after identifying an error in the therapy session: (1) identifying the moment when the mistake occurred; (2) observing the emotional response to the mistake; (3) sharing the mistake with others; (4) analyzing the mistake; and (5) preventing the mistake from recurring. I will analyze the vignettes according to these steps.







\section{Vignette I: First Steps as a Music Therapist: From Unstructured to Structured}



In the first vignette, we will see that all the stages occurred as Gilboa describes them, and the mistake led immediately to learning.



At the beginning of my professional journey as a music therapist 32 years ago, I worked at a special education school for children with intellectual disabilities. While there I worked with an eight-year-old boy I'll call David, who was diagnosed with an intellectual disability and Attention-Deficit Hyperactivity Disorder (ADHD). David's parents were divorced, and David was experiencing emotional distress as a result. He suffered from a speech delay. He also had difficulty in maintaining boundaries, for example, in his drawing on all the walls of his home as high up far as his hands could reach. The therapeutic goals I set for David were to allow him to express himself through music as a creative way of bypassing his challenges with verbal communication. The therapy room had a snare drum, a cymbal, and a keyboard. When entering the therapy room, David immediately went to the instruments and started playing non-stop, laughing hysterically. I interpreted his laughter as pleasure and encouraged his playing. In the ensuing sessions, David played throughout each session, which was 30 minutes long. After about two months, his teacher met me and asked me what we were doing in the music therapy room. I told her I was giving David a place to express himself. She then told me that for the last month, whenever David came back from music therapy, he had become incontinent. At that moment, I realized that I had done something wrong. While I thought I was helping David, he was actually emotionally overwhelmed during the therapy session, and his distress manifested itself in physical symptoms when he returned to class. I was embarrassed by my own mistake and even angry with myself.



After analyzing the situation and recounting it to my supervisor, I realized that David needed a stronger sense of holding and that I had to help him by structuring the sessions. The next time we met I put all the musical instruments in the closet, except for the keyboard. David did not understand what was happening and was very angry with me. I had composed a song that spoke of playing and stopping and played it to David. However, for quite a while, David had a hard time stopping his playing on the keyboard, and the only way to make him stop was to unplug the keyboard from the socket, which I had to do several times. All through the session, I gave David positive feedback in form of improvised singing. About two months later, I added the drum to the keyboard and about three months later, I added the cymbal. David stopped being incontinent in class.



In this case, I realized that I had ignored David's difficulties with boundaries. The process unfolded largely as Gilboa (2022) describes: I became aware of my mistake immediately when David's teacher told me about what happened to him in class after the therapy sessions. I identified my own shame, frustration and anger. I shared the incident with my supervisor: together, we analyzed the mistake and realized that David needed holding. Afterwards, I avoided repeating my mistake by putting away all of the instruments except for the keyboard and returning them only gradually. Structuring the sessions using a song, the repeated experience of playing, and the elimination of certain instruments gave David a feeling of holding and stability. Since then, whenever I have a client who has difficulty in maintaining boundaries, I start by working in a structured way, and only after the client feels more stable emotionally, I gradually let go of the structure.



As a look at the professional literature reveals, some of the music therapists who work with children and adolescents diagnosed with Attention-Deficit Hyperactivity Disorder (ADHD) seem to agree that they need structured music therapy interventions and sessions, in order to maintain stability and a sense of holding (Jackson, 2003; Rickson \& Watkins, 2003; Liu et al., 2017). Structured music therapy activities may decrease the frequency of maladaptive behaviors caused by ADHD (Doolan, 2023; Jackson 2003; Nindya \& Wimbarti, 2019; Rickson \& Watkins, 2003).







\section{Vignette II: Youth that Have Undergone an Experience of Trauma: from Unstructured to Structured}



In this vignette, as we will see, not all of the stages of identifying and coping with a mistake in music therapy, as surveyed by Gilboa (2022), occurred in order.



\footnote{ In August 2005, Israel implemented its plan of disengagement from the Gaza Strip, in which 8,600 Jewish inhabitants of the area were uprooted from their homes. The Gaza Strip was considered an area with many violent incidents, some of them claiming civilian lives. The disengagement plan met with strong opposition from the uprooted residents, who did not want to leave their homes, where some of the families had lived for more than 30 years. From the perspective of the uprooted residents, they felt like refugees within their own country. The uprooted residents were not a homogeneous group in terms of their economic situation. Most of them worked in agriculture and suffered massive losses as a result of the disengagement. Some lost their source of livelihood entirely. They were relocated by the state to other parts of southern Israel, and many had great difficulty acclimating to their new surroundings. In many instances, relationships between teenagers and their parents were hindered. While the parents were in survival mode, trying to rebuild their lives, their teenage children felt abandoned and resisted their parents. }In 2005 I worked with groups of teenagers who had experienced significant trauma during the disengagment, which had cost them their homes and communities. At that time, there was very little published research on using music therapy with people who had lost their homes -- a traumatic event in itself, followed by the further trauma of feeling uprooted within their own country. I had to follow my intuition, which unfortunately was not always beneficial to my clients. During the therapy sessions with one group of teenage girls, after I asked the girls to express their feelings through musical improvisations, they played very loudly, using a lot of force, and broke some of the musical instruments; possibly, the music therapy was a trigger that brought on emotional outbursts. After repeated incidents in which instruments were destroyed, I began to think about the disruption that the traumatic experience of forced relocation caused in the continuity of the girls' lives. Having unpacked with the girls what was going on for them, I understood that everything in their lives had been turned upside down: as a result, they felt a lack of control over life, helplessness and a constant sense of being threatened. Sensing that they needed a holding environment and a safe place, I started structuring the sessions, and gradually saw a decrease in the post-traumatic symptoms.



In this vignette, in keeping with the way Gilboa (2022) describes the steps that occur after a mistake in therapy, it took me some time to realize that I had indeed made a mistake, while instruments continued to be broken in this and other groups I was working with. My feelings about my own mistake came in waves, even before I had fully realized that an error had been made. I felt frustrated and disappointed in myself for doing something wrong: I was an experienced therapist, and yet I could not understand what the problem was. At first I did not share this with anyone, in part because I did not yet understand what was happening, but also because I was embarrassed: after all, I was supposed to know. Only after becoming aware of my mistake did I share it with my supervisor. We then analyzed the error together, and I made the necessary changes to my session structure, as noted above.



The theoretical conclusion I then reached is that since the trauma occurred within the community and unsettled every aspect of these teenage girls' lives (community, family and society), they needed a sense of holding and safety during the sessions. This could be provided by structuring the sessions and using structuring musical activities, such as working with songs and drumming. This conclusion was later supported by a study I conducted with a similar group of teenage girls (Wiess \& Bensimon, 2022). In the study, some of the musical activities I devised were repeated in each session. For example, I started each session with the same warm-up exercise: while playing the same tune, a cardboard strip with a list of feelings and sensations was passed around, and each girl in the group marked how she was feeling that day. The goal was to begin the sessions in a controlled manner and let me, the therapist, gauge the emotional state of each girl and the overall atmosphere in the group before starting therapy. Other repeated activities included choosing, singing and writing songs. One of the conclusions of my study was that the repeated activities appeared to have served as the central mechanism for therapeutic change. The structured sequence of activities during each session resembled a ritual, which has been found to create predictability and provide a safe place where individuals may feel a strong sense of belonging to the group, leading eventually to their spontaneous self-expression (Wiess \& Bensimon, 2020).



My understanding of the importance of structure when working with clients who have undergone trauma was only strengthened in subsequent years by the intense traumatic reactions I have witnessed -- reactions which were a revelation to me, despite my considerable experience with this kind of therapy. Over the years, I have learned that clients who have experienced trauma benefit most from semi-structured sessions that are planned around a structured framework, but also include some degree of freedom. The musical activities that I devise as a therapist form the structured framework, while clients are given the freedom to choose the music. For example, I may decide on working with songs, but the clients will be the ones to choose the songs. The songs themselves are structured (verse + chorus) and provide an anchor for the clients (Amir, 1998; Wiess \& Maor, 2022). Each session opens and ends with a musical activity that frames the session and provides a sense of holding for the participants.



I've also learned how to change the degree of structuring over time, in accordance with the needs of the clients. In my work with similar groups the following years, I started out with structured sessions to create a sense of holding that would help the participants feel safe and relaxed. Later, when I felt that the group was well held, I stopped structuring sessions in order to allow traumatic content to come up and be processed. Even when I did not structure the entire sessions, I usually opened and closed them in a structured way, such as by using a song.



There are therapeutic approaches for working with trauma (e.g., EMDR,\footnote{Eye Movement Desensitization and Reprocessing is a form of psychotherapy developed by Dr. Francine for clients who are coping with trauma and various life crises.} SE,\footnote{ Somatic Experiencing (SE™) is a body-oriented therapeutic model applied in multiple professional settings such as psychotherapy, medicine, coaching, teaching, and physical therapy while working with trauma and other stress disorders.} PE\footnote{ Prolonged exposure therapy (PE) is a cognitive behavior therapy that is highly efficacious for working with chronic post-traumatic stress disorders (PTSD) and related depression, anxiety, and anger.}; see Brom et al., 2017; Gilboa- Schechetman et al., 2010; Kuhfuss et al., 2021; Peterson et al., 2019; Shapiro, 2014, 2017). They share the view that structured protocols are needed for holding and stabilizing clients who have undergone trauma (Brom et al., 2017; Kuhfuss et al., 2021; Shapiro, 2014). In group therapy with children and teenagers, structured therapy is recommended at the beginning, as it helps to create a sense of belonging and group cohesion, characteristics that are a prerequisite for successful therapeutic group interventions (Drost \& Bailly, 2004). Structured sessions have also been found to reduce post-traumatic symptoms and depression and significantly increase the level of functioning (Gilboa-Schechtman et al., 2010). Similar findings exist in music therapy literature on working with refugees and uprooted population (Felsenstein, 2013; Wiess \& Bensimon, 2020).



There are also several approaches for using music therapy specifically to help clients with post-traumatic symptoms using non-structured sessions and interventions. Bensimon used an unstructured approach when working with a group of combat soldiers suffering from post-traumatic stress disorder (PTSD) and found that it helped the soldiers open up to express their feelings (Bensimon et al., 2008). Hunt (2005) also used an unstructured approach with teenagers. She claimed that musical activities chosen by clients should determine the structure of the sessions, and that there was no need for therapists to impose an external structure. In her opinion, letting the teenagers decide which therapy techniques to use could allow them to regain control of the situation and improve their emotional health.



My own experience in this area has taught me the importance of paying close attention to the differences between individual and group interventions. A group functions as a kind of musical amplifier; we therefore have to be attuned to the force of the emotions that come up in group musical activities. These emotions can empower the members of the group, but they can also generate disquiet. When working with an individual, we adapt the activity to suit the specific client; in group therapy, attention must be paid to the group as a whole.







\section{Vignette III: Music Therapy Student: From Structured to Unstructured}



In this vignette, the stages of the process surrounding the mistake were different from those identified by Gilboa (2022).



Since 2001 I have been a teacher in music therapy training programs in Israel. During those years I have supervised many students. In the first and second years of their studies, students do an internship alongside a music therapist at their workplace, while also taking part in group supervision sessions at the college. I've accompanied my students in the course of their learning, knowing well that some processes take time and are part of a new therapist's development. It has been clear to me that students will make mistakes with clients during their training. I think it is important to call such errors “mistakes,” while simultaneously acknowledging that they are part of the learning process and helping students to understand that mistakes are not only human, but a source of much important learning (Hakvoort, 2022).



A music therapy student I'll call Rachel was very anxious when she began her studies. As part of her clinical internship she worked with a 10-year-old girl who was struggling with a low self-image and social difficulties among her peers. Rachel structured her sessions with the girl in advance, carefully planning the musical activities she would use. She reported that the therapy was going well, that the client was enjoying it very much, and that she was enjoying herself too. I asked her what she meant by enjoying herself, and she explained: “I'm calm because I know exactly what will happen.” In the group supervision meetings, other students raised the question of whether Rachel was teaching, or working like a music therapist, and what the difference between those two options. We talked about the issue, but didn't arrive at any conclusions.







Rachel's therapeutic goals were to help the patient express her emotions and improve her self-image, to support and strengthen her, and to help her cope with her social challenges. I was aware that Rachel needed time to feel more confident in her music therapy skills, and her skills as a therapist, since she was herself in the midst of a learning process. Part of this process takes into account the anxiety of the therapist (Hakvoort, 2022), with the supervisor there to ensure that for the time being, the client feels secure and safe and is not adversely affected by the situation. I therefore did not push Rachel to relinquish her need to structure the sessions. After two months had passed, I realized, through the supervision, that the client's social situation had not changed, and neither had Rachel's methods: she was continuing to use the same structured musical activities in her therapy sessions. As the supervisor, I began to feel at this point that the structured sessions were detrimental to the girl she worked with, for example, at one of the sessions the girl began to describe how children in her class had begun to bully her. Rachel, however, told her that they were in the middle of the activity and that they could speak about what was happening in class after the activity. When the activity ended, Rachel did not ask the girl about her situation in class, fearing that she would not know what to do with this information. The girl, for her part, did not raise the issue again, perhaps sensing Rachel's anxiety. Rachel's need to have the activity she had designed unfold as planned to its conclusion meant that the girl had no space in which to present her difficulty. The structured sessions thus did not allow the girl to raise the problems that were bothering her, and Rachel was unable to help the girl process her difficulties. Moreover, this limitation harmed her own development as a music therapist.







Viewing this vignette through Gilboa's discussion (2022), we might say that as Rachel's teacher and supervisor I identified the mistake long before she did. As far as she was concerned, the therapy she was providing was enjoyable. Rachel was angry with me and with the other students in her supervision group for challenging the way she handled the therapy. The therapy group and I were empathetic and compassionate towards Rachel. She was only able to recognize the mistake a few months later. When she became aware of it, she felt relief because she sensed that the supervision group was supporting her. Her sharing of her process with me and the group unfolded over an entire year, beginning before she realized a mistake had been made. As the supervisor and teacher, I have no way of knowing whether or not she repeated the mistake later; what I can say is that Rachel realized where she had erred.







After analyzing the situation with Rachel during group supervision, we discovered that Rachel was feeling anxious and insecure about her music therapy skills and her skills as a therapist, and that she was using structured sessions because they gave her a sense of control. We also understood from Rachel's reports that structuring the therapy sessions hindered her client's social and emotional progress and limited her ability to express her feelings. Following the discussion in the supervision group, Rachel started easing up on the inflexible structure she had used in her sessions, and noticed that once she became more flexible, the girl she worked with, began expressing herself, allowing the therapeutic process to address both her social problems and her low self-image. Through her error, Rachel understood the significance of a therapeutic process in which she identified the needs and difficulties of the client, allowed the client to express herself, and helped her to create positive change in her self-image and social relationships. Some may say that mistakes made by students should not be considered mistakes, since students are still in the process of learning; I believe, however, that student mistakes are natural and human, and that acknowledging them as mistakes allows students to learn from them for their future professional practice.



The intense anxiety that Rachel experienced is common among music therapy students and novice music therapists (Hakvoort, 2022). Using structured sessions decreases this anxiety; supervisors should therefore be sensitive and empathetic and allow students and new therapists the time they need to shift gradually to a flexible structure once they have more confidence in their own skills (Shamoom et al., 2017; Coale, 2020). Highly structured sessions may prevent clients from undergoing a significant emotional process in which they can work with their own difficulties, express their feelings, become more self-aware and gain the ability to make positive change in their lives. An inflexible structure may end up dictating what clients feel, thus curbing their emotional reactions and not allowing anger, pain and conflict to surface and be addresed (Wheeler, 2002). It is thus important that music therapy students and novice therapists engage in supervision, so they can identify how their own levels of anxiety may be affecting the therapeutic process (Coale, 2020) and challenge themselves to move outside their comfort zone in a gradual and yet sustained manner, so that they can eventually become more effective therapists.







\section{Vignette IV: Music Therapy Novice: From Structured to Unstructured}



In this vignette, the stages of identifying, responding to and dealing with a mistake did not unfold in the manner described by Gilboa (2022). This vignette is different from vignette III due to personality differences between the music therapists. In vignette III the music therapist dealt with lack of confidence issues, while here, as we will see, the music therapist acted as he did because he was not able to face anger.



The Israeli Association for Creative Arts Therapies (YAHAT) brings together therapists working with music, visual arts, movement and dance, bibliotherapy, drama and psychodrama. YAHAT is the body which certifies therapists, who go through different stages of training as therapists and supervisors before becoming qualified supervisors themselves. I myself was certified as a qualified supervisor many years ago, and since then I have supervised music therapists who come to my clinic. I also supervise therapists in all the creative fields mentioned above working with clients who have endured trauma.



A music therapist I'll call Jacob came to me for weekly supervision after having worked as a therapist for about two years. At the beginning of each of our sessions, Jacob described the therapeutic sessions he had with his clients. At my request, he also brought recordings of the sessions to our weekly meetings. At some point, I noticed that Jacob was structuring his sessions by choosing songs and using them to work with his clients, who were children of elementary-school age. Jacob told me that he had constructed his therapy sessions in the same way since his graduation, and that the method suited him. He also mentioned that if clients requested a specific song, he politely refused.



We analyzed the situation and saw that Jacob's inflexible structuring stemmed from his fear of dealing with anger and conflict in the music therapy sessions. Jacob said that since childhood, it had been difficult for him to contain his own anger, and he therefore tried to avoid it. I asked what would happen if clients got angry at him, and he answered that he was afraid it would break him emotionally. That was why, over the last two years, Jacob had persisted in structuring sessions in a way that allowed him to control both the activities and the content. He did not allow anger and difficulty to arise while working with children. He also did not understand why some of the children he worked with didn't want to come to the therapy sessions. One of the children had a violent father; as a result, he was restless, and perhaps some anger had built up inside him. The boy was drawn to the drums and seemed to want to release his anger in the sessions by drumming very loudly. Jacob, however, could not stand it, and therefore didn't allow drumming in his sessions. After his first year as a music therapist, the principal of the school where he worked dismissed him from his job because he could not see any improvement in the children's social, emotional and behavioral states. Moreover, some of the children didn't want to come to the music therapy room, and parents were dissatisfied with Jacob because he avoided meeting with them and talking to them. I felt empathy and compassion for Jacob.







When we analyzed the events, Jacob himself identified his fear of confronting anger as the root of the problem: this fear, he recognized, was preventing him from seeing the needs of the children, the school and the parents. He felt frustrated and hurt. After several months of supervision, we finally saw the beginnings of change, and Jacob agreed to seek therapy for himself. A year later, Jacob was able to talk freely about the mistake he had made as a therapist due to his personal issues, and said he regretted that the change had not happened sooner.



Through Gilboa's discussion (2022), Jacob was well aware of his own difficulty in dealing with charged, conflictual feelings, but it took time before he connected his inner struggle with the mistake he made while working with his clients. I, as Jacob's supervisor, saw the mistake long before he could admit to it. Once he became aware of the mistake, he grew angry with himself and discussed this with me, but was unable to make the necessary change until several more months had passed. This demonstrates that therapists differ from one another in the amount of time it takes them to correct a mistake. This duration may depend on whether the mistake was made as a result of the therapist's own personal history, or of his or her insufficient professional experience.



There have been other cases like Jacob's in which therapists, mainly novice ones, were afraid to deal with unpredictable or difficult emotions (Coale, 2020) and avoid them by inflexibly structuring their therapeutic sessions. Such conduct reduces the efficacy of the therapy and the therapist's ability to guide clients through a beneficial therapeutic process (Coren \& Farber, 2017; Wheeler, 2002).



\section{Conclusion}



When examining the issue of structured versus unstructured sessions and interventions in music therapy, there are many parameters to consider. Because the issue is so complex, with many different factors at play, there is no one right path to take. Mistakes occur in this context for many reasons: a lack of awareness, inexperience, a therapist who is still in a process of learning, personal issues or a dearth of relevant professional literature about the kind of population we are working with. But mistakes in music therapy are human, and they are nothing to panic about, as long as we learn from them.



I consider a mistake in music therapy when the client, the therapist or both are adversely affected by them, and when the therapy does not help the client go through a beneficial process. In this article I have shown how mistakes adversely affected the clients and the therapists as well. I've examined my own shift from using unstructured musical therapy sessions and interventions to my gradual understanding, through supervision and research, that structuring can help create the holding effect that helps clients to confront their difficulties. I have also described the opposite shift, away from structured sessions and interventions towards unstructured ones, due to the realization (again reached through supervision) that structuring in some cases does not allow patients to express their full range of feelings and prevents complicated matters from coming up. In the first two examples, those of shifting from unstructured to structured therapy, I myself was the therapist, and the mistakes were my own. The third and fourth examples involved a student and a novice therapist under supervision, both of whom needed to move from structured interventions to unstructured ones.



The movement between these two different approches of therapeutic pratice, in both directions, is intriguing and sometimes challenging. It can happen in any therapeutic encounter, such as when the therapist decides that it is appropriate to open a session in a structured way, such as by singing a greeting song with the clients, and then to let the client lead the rest of the session in order to help him or her confront their difficulties. The shift from unstructured to structured, conversely, may happen when a therapist begins the session in an unstructured way, but senses during the session that the client is overwhelmed and restless, and that more structure might help him or her feel more calm and secure. Similar shifts -- from unstructured to structured, or vice versa -- can happen across the whole of the therapeutic process, with the therapist changing from one way of managing sessions to another. Deciding what to do is not always easy, but it is fascinating.



Music therapists should constantly assess and evaluate such factors as shifts in the needs of the populations they work with, the goals of the therapy, the stage of the therapeutic process that the client has reached, and the duration of the process (whether it is short- or long-term). At the beginning of the process, when clients' anxiety levels are high, therapists should rely more on structured sessions. However, once a “safe place” is established, it may be necessary to let go and allow an unstructured process to take place. Before deciding which kind of session or process to offer, music therapists should also assess their own emotional state and anxiety levels, and ask themselves if they are structuring the therapeutic process because it is in the client's best interest, or because of their own needs as therapists. This may be ascertained through their own reflective practice, or with the help of supervision or, at times, of therapeutic intervention, if more personal triggers are identified. Only after answering these questions can the music therapist decide on the right kind of intervention for their specific clients.



In my early years as a therapist my dynamic training prompted me to choose an unstructured approach to therapy but my work with trauma victims led me to learn new methods which made me see structuring in a new light -- not as a default choice, but as a way to address the vital needs of certain clients. Structuring was not always easy to do, but my research and work with my supervisor gradually showed me the importance of structuring and the many different forms it could take. However, even when I do structure the sessions, I keep them semi-structured, that is, I give room for self-expression to enable clients to find their voice. There are also situations in which I start therapy using one approach (with or without structured sessions and interventions) and change modalities later on.







\section{References}



\hspace*{\parindent}Amir, D. (1998). The use of Israeli folksongs in dealing with women's bereavement and loss in music therapy. In D. Dokter (Ed.), \emph{Art therapists, refugees and migrants} (pp. 217--235). Jessica Kingsley Publishers



Baker, F., \& Jones, C. (2005). Holding a steady beat: The effects of a music therapy program on stabilizing behaviors of newly arrived refugee students. \emph{British Journal of Music Therapy, 19, }67‒74.



Bensimon, M., Amir, D., \& Wolf, Y. (2008). Drumming through trauma: Music therapy with post-traumatic soldiers. \emph{The Arts in Psychotherapy, 35}(1), 34--48.



Brom, D., Stoker, Y., Lawi, C., Nuriel- Porat, V., Ziv, Y., Lerner, K., \& Ross, G. (2017). Somatic experiencing for posttraumatic stress disorder: A randomized controlled outcome study. \emph{Journal of Traumatic Stress, 30}(3), 304-312.



Coale, H. W. (2020). \emph{The vulnerable therapist practicing psychotherapy in an age of anxiety}. Routledge Publishers. \href{https://DOI.org/10.4324/9781315809922}{https://doi.org/10.4324/9781315809922}



Coren, S., \& Farber, B. A. (2017). A qualitative investigation of the nature of “informal supervision” among therapists in training. \emph{Psychotherapy}\emph{ Research}\emph{, 29}(5), 679-690. \href{https://doi.org/10.1080/10503307.2017.1408974}{https://doi.org/10.1080/10503307.2017.1408974}



De Maat, S., De Jonghe, F., De Kraker, R., Leichsenring, F., Abbass, A., Luyten, P., Barber, J. P., Van, R., \& Dekker, J. (2013). The current state of the empirical evidence for psychoanalysis: a meta-analytic approach. \emph{Harvard Review of Psychiatry, 21}(3), 107--137. https://doi.org/ \href{https://doi.org/10.1097%252FHRP.0b013e318294f5fd}{10.1097/HRP.0b013e318294f5fd}



Doolan, A. (2023) Music therapy as treatment for children with attention deficit hyperactivity disorder (ADHD): Which methods are most effective? In J. Williams, M. Stead \& J. Leonard (Eds.), \emph{Music talks: Proceedings of the Western Sydney University Undergraduate Musicology Conference 2022} (pp. 45-58). Western Sydney University. https://doi.org/10.26183/x0ry-nf79



Drost, G., \& Bailly, S. (2004). \emph{Group therapy with children.} Ach Publishers.



Felsenstein, R. (2013). From uprooting to replanting: On post-trauma group music therapy for pre-school children. \emph{Nordic Journal of Music Therapy, 22}(1), 69--85. \href{https://doi.org/10.1080/08098131.2012.667824}{https://doi.org/10.1080/08098131.2012.667824}



Gamsa, A., Braha, R. E. D., \& Catchlove, R. F. H. (1985). The use of structured group therapy sessions in the treatment of chronic pain patients. \emph{Pain}\emph{,} \emph{22(1}), 91-96. \href{https://doi.org/10.1016/0304-3959(85)90151-4}{https://doi.org/10.1016/0304-3959(85)90151-4}



Geue, K., Goetze, H., Buttstaedt, M., Kleinert, E., Richter, D., \& Singer, S. (2010). An overview of art therapy interventions for cancer patients and the results of research. \emph{Complementary Therapies in Medicine}\emph{,}18(3-4), 160-170. \href{https://doi.org/10.1016/j.ctim.2010.04.001}{https://doi.org/10.1016/j.ctim.2010.04.001}



Gilboa-Schechtman, E., Foa, E. B., Shafran, N., Aderka, I. M., Powers, M. B., Rachamim, L., Rosenbach, L., Yadin, E., \& Apter, A. (2010). Prolonged exposure versus dynamic therapy for adolescent PTSD: A pilot randomized controlled trial. \emph{Journal of the American Academy of Child \& Adolescent Psychiatry, 49}(10), 1034-1042. https://doi.org/\href{https://doi.org/10.1016/j.jaac.2010.07.014}{10.1016/j.jaac.2010.07.014}



Gilboa, A. (2022). Five practical steps for dealing with mistakes in music therapy. In: A. Gilboa and L. Hakvoot (Eds.), \emph{Breaking strings: Mistakes in music therapy,} pp (271-289). ArtEZ Press.



Nindya, B., \& Wimbarti, S. S. (2019). Program intervensi musik terhadap hiperaktivitas anak attention deficit hyperactivity disorder (ADHD). \emph{Gadjah }\emph{Mada}\emph{ Journal of Professional Psychology}, 5(1) 15-25. \href{https://doi.org/10.22146/gamajpp.48584}{https://doi.org/10.22146/gamajpp.48584}



Hakvoort, L. (2022). Some mistakes are made: Mistakes \& the novice music therapy. In: A. Gilboa and L. Hakvoot (Ed.). \emph{Breaking strings: Mistakes in music therapy} (pp. 165-180). ArtEZ Press.



Hunt, M. (2005). Action research and music therapy: Group music therapy with young refugees in a school community. \emph{Voices: A World Forum for Music Therapy, 5}(2). \href{https://doi.org/10.15845/voices.v5i2.223}{https://doi.org/10.15845/voices.v5i2.223}



Jackson, N. A. (2003). A survey of music therapy methods and their role in the treatment of early elementary school children with ADHD. \emph{Journal of Music Therapy, 40}(4), 302--323, \href{https://doi.org/10.1093/jmt/40.4.302}{https://doi.org/10.1093/jmt/40.4.302}



Kelly, J. F., Abry, A., Ferri, M., \& Humphreys, K. (2020). Alcoholics anonymous and 12-Step facilitation treatments for alcohol use disorder: A distillation of a 2020 Cochrane review for clinicians and policy makers. \emph{Alcohol and Alcoholism, 55}(6), 641-651. https://doi.org/10.1093/alcalc/agaa050



Kuhfuss, M., Maldei,T., Hetmanek, A., \& Baumann, N. (2021). Somatic experiencing -- effectiveness and key factors of a body-oriented trauma therapy: a scoping literature review. \emph{European Journal of Psychotraumatology\emph{,}\emph{ }}\emph{12}(1), Article 1929023. \href{https://doi.org/10.1080/20008198.2021.1929023}{https://doi.org/10.1080/20008198.2021.1929023}



Liu, C., Zhang, L., Liu, H., \& Cheng, K. (2017). Delivery strategies of the CRISPR-Cas9 gene-editing system for therapeutic applications. \emph{Journal of Controlled Release}\emph{,}\emph{ 266}, 17-26. \href{https://doi.org/10.1016/j.jconrel.2017.09.012}{https://doi.org/10.1016/j.jconrel.2017.09.012}



Mishchykha, L., Cherniavska, N., Kravchenko, V., Vityuk, N., Kulesha-Liubinets, M., \& Khrushch, O. (2023). Application of mindfulness practices in work on stress reduction during the war. \emph{Revista}\emph{ de }\emph{Cercetare}\emph{ }\emph{şi}\emph{ }\emph{Intervenţie}\emph{ }\emph{Socială}\emph{,} \emph{81}, 25-38. Retrieved from \href{https://www.ceeol.com/search/article-detail?id=1135951}{https://www.ceeol.com/search/article-detail?id=1135951}



Orth, J. (2005). Music therapy with traumatized refugees in a clinical setting. \emph{Voices: A World Forum for Music Therapy, 5}(2). \href{https://doi.org/10.15845/voices.v5i2.227}{https://doi.org/10.15845/voices.v5i2.227}



Peterson, A. L., Foa, E. B., \& Riggs, D. S. (2019). Prolonged exposure therapy. In B. A. Moore \& W. E. Penk (Eds.), \emph{Treating PTSD in military personnel: A clinical handbook} (pp. 46--62). The Guilford Press.



Rickson, D. J., \& Watkins, G. W. (2003). Music therapy to promote prosocial behaviors in aggressive adolescent boys - a pilot study. \emph{Journal of Music Therapy, 40}(4), 283-301.  doi: 10.1093/jmt/40.4.283



Shamoon, Z. A. Lappan, S. \& Blow, A. J. (2017). Managing anxiety: A therapist common factor. \emph{Contemporary Family Therapy}\emph{ 39, }43--53.  https://doi.org/10.1007/s10591-016-9399-1



Shapiro, F. (2017). \emph{Eye movement desensitization and reprocessing (EMDR) therapy: Basic principles, protocols, and procedures} (3\textsuperscript{rd} ed.). Guilford Publications.



Shapiro, F. (2014). The role of eye movement desensitization and reprocessing (EMDR) therapy in medicine: Addressing the psychological and physical symptoms stemming from adverse life experiences. \emph{The Permanente Journal ,18}(1), 71-77. \href{https://doi.org/10.7812/TPP/13-098}{https://doi.org/10.7812/TPP/13-098}



Sloshower, J., Guss, J., Krause, R., Wallace, R. M., Williams, M.T., Reed, S., \& Skinta, M. D. (2020). Psilocybin-assisted therapy of major depressive disorder using acceptance and commitment therapy as a therapeutic frame. \emph{Journal of Contextual Behavioral Science}\emph{, 15}, 12-19. \href{https://doi.org/10.1016/j.jcbs.2019.11.002}{https://doi.org/10.1016/j.jcbs.2019.11.002}



Tarugu, J., Pavithra1, R., Vinothchandar, S., Basu, A., Chaudhuri, S., \& John, K. R. (2019). Effectiveness of structured group reminiscence therapy in decreasing the feelings of loneliness, depressive symptoms and anxiety among inmates of a residential home for the elderly in Chittoor district. \emph{International Journal of Community Medicine and Public Health, 6}(2), 847-854. http://dx.doi.org/10.18203/2394-6040.ijcmph20190218



Wheeler, B. L. (2002). Experiences and concerns of students during music therapy practica. \emph{Journal of Music Therapy, 39}(4), 274--304. \href{https://doi.org/10.1093/jmt/39.4.274}{https://doi.org/10.1093/jmt/39.4.274}



Whitaker, D. S. (1985). \emph{Using groups to help people}. Routledge \& Kegan.



Wiess, C., \& Maor, R. (2022). Harmonizing hearts with many voices - analysis of the unique phenomenon of mass singing and its contribution to individuals' resilience. \emph{Voices: A World Forum for Music Therapy, 22}(2). https://doi.org/10.15845/voices.v22i2.3295



Wiess, C., \& Bensimon, M. (2020). Group music therapy with uprooted teenagers: The importance of structure. \emph{Nordic Journal of Music Therapy, 29}(2), 174-189.



Yalom, I. D. (2017). \emph{The gift of therapy: An open letter to a new generation of therapists and their patient's paperback}. Basic Books.



Yalom, I. D., \& Crouch E. C. (2018). \emph{The theory and practice of group psychotherapy}. Cambridge University Press.


\end{document}