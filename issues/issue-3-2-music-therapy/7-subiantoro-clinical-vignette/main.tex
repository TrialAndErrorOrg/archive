\documentclass[authordate, empirical, issue]{jote-new-article}

\usepackage{caption}

\usepackage{tabularx}

\usepackage{graphicx}

\usepackage{hyperref}

\usepackage[backend=biber,style=apa]{biblatex}

\addbibresource{bibliography.bib}

\jotetitle{“I guess I just like talking”: \\ A Reflection From an Online Music Group with Six Young Autistic Women in Indonesia}
\footertitle{“I guess I just like talking”: A Reflection From an Online Music Group with Six Young Autistic Women in Indonesia}
\keywordsabstract{autistic women, music group, online, reflection, talking}
\abstracttext{This article reflects my experience facilitating an online music group session with six young autistic women in Indonesia. A participant linked her inclination to talk to what she enjoyed the most from the sessions. As her comment struck me, I pondered upon my music therapy training background and worldview that might have shaped my expectations of the sessions. What I initially perceived as a “mistake” turned into a learning. Defining and redefining the definition and role of music in the shared space should be allowed and nurtured in whichever shape and trajectory contextually fits the experiencer.}
\runningauthor{Subiantoro}
\jname{Journal of Trial \& Error}
\jyear{2023}
\paperdoi{10.36850/xd47-yz27}
\jvolume{3}
\jissue{2}
\paperreceived{May 31, 2023}
\author[1]{Monica Subiantoro\orcid{0000-0003-0024-0410}}
\affil[1]{The University of Melbourne}
\corremail{\href{mailto:monica.subiantoro@gmail.com}{monica.subiantoro@gmail.com}}
\corraddress{The University of Melbourne}
\runningauthor{Subiantoro}
\paperaccepted{October 17, 2023}
\paperpublished{December 22, 2023}
\paperpublisheddate{2023-12-22}
\paperissued{December 22, 2023}
\jwebsite{https://journal.trialanderror.org}

\specialissue{Untangling Strings -- Further Explorations of Mistakes in Music Therapy}



\jvolume{3}
\jissue{2}
\jpages{41--43}
\begin{document}
\begin{frontmatter}
  \maketitle
  \begin{abstract}
    \printabstracttext
  \end{abstract}
\end{frontmatter}

\setcounter{page}{41}




\begin{quote}
  “I guess I just liked the part where you just talk about the lyrics, I guess I just like talking when it comes with other people, but when it comes to music, I just like to listen to it alone. Because I just like the nature of like having a topic be established and just talk around it.” (Lala)\
\end{quote}







This extract was taken from a focus group discussion with a group of young autistic Indonesian women, and in which Lala was a participant. This focus group was part of my PhD research project. As the project took place during the COVID-19 pandemic where social distancing and international travel restrictions were imposed, I conducted the sessions via Zoom. Having been informed by the scarce literature on music therapy and autistic adults (e.g., Ee et al., 2019; Mazurek, 2014), my qualitative inquiry explored the young autistic Indonesian adults' experiences of an online music group. As recent autism literature highlighted the social isolation and loneliness experienced by autistic people, I embarked on the project with a focus on the members' social connectedness within the online music group sessions. I recruited the participants through social media, schools, communities, and local networks. Six young autistic women volunteered to participate in 12 weekly group sessions. I felt uneasy when I first heard Lala's statement about what she really enjoyed in the sessions. I could sense my disappointment in realising that not everyone in the group was equally inclined to engage in the musical experiences, leading to wondering if I had made mistakes in facilitating the sessions and if so, what these mistakes were.



I later considered that my uneasiness may have been provoked by an assumption that the group members would develop connections through their collective enjoyment of making music together. Therefore, the members' shared level of engagement and joy during the musical experiences would be essential in achieving this outcome. At the same time, I recalled Thompson and Elefant's (2019) reflections on working with highly verbal neurodiverse participants, where they suggested the potential roles of music in fostering relationships despite the client's seeming inclination to talk rather than engage in music.



As a music therapist, I have always been intrigued by the discourse around the role of music in or as therapy. Although I firmly believe both could work in different situations, hearing immediately from the participant themselves about the supplementary role of music was a new experience. Lala was the first participant who explicitly confessed that she would rather experience music individually. I suppose my music therapist identity has built upon the concept of nurturing therapeutic relationships through musical interaction. As a “good enough” music therapist, I have held the responsibility to create and nurture this interaction by utilising various musical experiences in the sessions.



I then recalled the questions I had asked participants at the beginning of the project before commencing the group sessions. Some participants revealed right from the start that the musical experiences being offered were not their primary reason for joining the group. On different occasions, participants also conveyed further meanings of musical sharing. Even though the themes that emerged varied, some eminent ones included: “a way to express emotions” and “prompts people to connect.” I considered that the roles of musical sharing in the group, therefore, take forms in different layers, from the intrapersonal to interpersonal processes, embodied in the participants' responses towards different experiences, including musical and verbal sharing. While not everyone in the group enjoyed the group musical experiences, or they preferred to engage with music individually, they collectively valued the role of shared music in supporting the development of connections.



There are diverse understandings of autism in Indonesia due to traditional and religious beliefs (Riany et al., 2016). While Indonesian women are often expected to fulfill their traditional roles, they proved to demonstrate their agency for social change (Rinaldo, 2013). In this online music group, these six women displayed their actions supporting each other as sharing their music and life stories became prominent during the co-designed sessions. Due to most participants' experiences of being misunderstood in the past and their common experience of hierarchical society in Indonesia, I find it crucial for the participants to feel empowered to communicate their opinions. I was delighted to discover some value in the musical experience despite the differing musical preferences among the participants. Despite Lala's carefulness in expressing her preference, her honest confession may have indicated her expression of autistic culture and the developing sense of trust between us. Therefore, acknowledging their preference to talk as a resource in the session may allow them to create a meaningful experience, either immediately within or extending from the musical experiences.



Autistic people, especially women, are often underrepresented in research studies. In Indonesia, autistic people's voices are still dominated by parents and professionals. Through this “hidden” online platform, I attempted to empower them to raise their voices. However, despite my attempt to facilitate the session with a resource-oriented approach and alignment with the neurodiversity spirit (Singer, 1999, 2016), my focus on the outcome of the group sessions was often influenced by my assumptions drawn from music therapy and psychology theories learnt in my training. Instead of listening to and understanding what they really needed, I tended to anticipate outcomes often described in the music therapy literature, particularly in the settings where music was centred. By doing so, I had overlooked my participants' needs and strengths. Reflecting on my personal reaction towards the group dynamics and acknowledging my participants' culture, strengths and preferences could become a first step toward an empowering and anti-oppressive practice (Baines, 2013, 2021).



Indonesia was a Dutch colony for 350 years. Upon writing this reflection, I was appalled to realise how the residue of colonisation slipped into the way I introduced the music therapy profession in the country and positioned myself in this project. As opposed to arriving to this collaborative space with an open mind, I was unconsciously holding my preconception of music and its definition, its role and place in this group space. As a Western classically trained musician, I had positioned different musical experiences in hierarchy and meaning based on my own experiences of privilege. I had unconsciously envisaged that music needed to be present in a physical form whose presence was required by the group to hold and to thrive.



Lala's desire for talking was rarely opportune in her life but within this lay the potential for growth of authenticity: “I myself have the potential to be talkative. I want to be chatty but don't seem to know what to say, what is the precise word.” This expressed desire reminds me of the anti-oppressive practice I had aimed for during the conception of my project and how music should not be imposed into the session. Instead, it should be allowed to take different shapes in this realm. In my learning, that was where the collaborative elements shone, not only in co-designing the session format and musical experiences, but also in defining and redefining the definition and role of music in the shared space. These definitions could be either old or new, individual or collective, yet should be meaningful for each individual to hold and apply in their lives beyond this online music group. It should be introduced and nurtured, yet allowed to grow in whichever shape and trajectory that contextually fits the experiencer.











\newpage
\section{References}



\hspace*{\parindent}Baines, S. (2013). Music therapy as an anti-oppressive practice. \emph{The Arts in Psychotherapy, 40}(1), 1-5. \href{https://doi.org/10.1016/j.aip.2012.09.003}{https://doi.org/10.1016/j.aip.2012.09.003}







Baines, S. (2021). Anti-oppressive music therapy: Updates and future considerations. \emph{The Arts in Psychotherapy, 75}, Article 101828. \href{https://doi.org/10.1016/j.aip.2021.101828}{https://doi.org/10.1016/j.aip.2021.101828}







Ee, D., Hwang, Y. I. J., Reppermund, S., Srasuebkul, P., Trollor, J. N., Foley, K.-R., \& Arnold, S. R. C. (2019). Loneliness in adults on the autism spectrum. \emph{Autism in Adulthood, 1}(3), 182-193. \href{https://doi.org/10.1089/aut.2018.0038}{https://doi.org/10.1089/aut.2018.0038}







Mazurek, M. O. (2014). Loneliness, friendship, and well-being in adults with autism spectrum disorders. \emph{Autism, 18}(3), 223-232. \href{https://doi.org/10.1177/1362361312474121}{https://doi.org/10.1177/1362361312474121}







Riany, Y. E., Cuskelly, M., \& Meredith, P. (2016). Cultural beliefs about autism in Indonesia. \emph{International Journal of Disability, Development and Education, 63}(6), 623-640. \href{https://doi.org/10.1080/1034912X.2016.1142069}{https://doi.org/10.1080/1034912X.2016.1142069}







Rinaldo, R. (2013). \emph{Mobilizing piety: Islam and feminism in Indonesia}. Oxford University Press.







Singer, J. (1999, 2016). \emph{Neurodiversity: The birth of an idea} \href{https://www.amazon.com}{https://www.amazon.com}







Thompson, G., \& Elefant, C. (2019). “But I want to talk to you!” Perspectives on music therapy practice with highly verbal children on the spectrum. \emph{Nordic Journal of Music Therapy, 28}(4), 347-359. \href{https://doi.org/10.1080/08098131.2019.1605616}{https://doi.org/10.1080/08098131.2019.1605616}






\end{document}
